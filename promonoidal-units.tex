%!TEX TS-program = latex

% This document contains material that will form part of the chapters
% _Pseudomonoids_ and _Cayley's theorem for pseudomonoids_. It is not
% itself a chapter of the thesis.

\documentclass{robinminion}
\usepackage[round]{natbib}
\usepackage{robincs}
\bibliographystyle{plainnat}

\newtheorem{conj}[propn]{Conjecture}

\newcommand\orp{\ensuremath{\mathrel{\shortleftarrow\!\!\!\!\!+\!\!\!-}}}
\newarrow{Pro}--+->

\newcommand\Arr[2]{{}#1{\hbox to 0pt{\mathsurround=0pt$\!#2$\hss}}}

\newcommand\V{\mathscr{V}}
\newcommand\B{\mathcal{B}}
\newcommand\Cat{\mathrm{Cat}}
\newcommand\Prof{\mathrm{Prof}}
\newcommand\Lin{\mathrm{Lin}}

\newcommand\I{\mathbb{I}}

\renewcommand\aa{\mathfrak{a}}
\renewcommand\ll{\mathfrak{l}}
\newcommand\rr{\mathfrak{r}}
\renewcommand\ss{\mathfrak{s}}

\renewcommand\e{\hat{\mathbf{e}}}

% Doesn't work!
%\newarrow{Eq}===={}

\title{Units in Promonoidal Categories}
\begin{document}
\maketitle

\noindent \citet{BTC} show how to construct, for any monoidal category $\C$, a strict monoidal
category $\mathbf{e}(\C)$ that is monoidally equivalent to $\C$. This construction is
a crucial step in their proof of the coherence theorem for monoidal categories.
%
Here we generalise this construction to pseudomonoids in any monoidal bicategory.
Applying the theorem to the case of promonoidal categories (i.e.\ pseudomonoids in
$\Prof\op$)  allows us to derive a useful characterisation of the
units in a promonoidal category.

Recall the coherence theorem for monoidal bicategories \citep{GPS, MonBicat},
which shows that every monoidal bicategory is monoidally biequivalent to a
Gray-category. It therefore suffices here to work in a Gray-category, since the
coherence theorem will allow our result to be transferred to arbitrary monoidal
bicategories. This is very useful in practical terms, since it allows us to suppress
some of the routine notational bureaucracy.

Therefore let $\B$ be a Gray-category, and $\C$ be a pseudomonoid in $\B$,
with unit $J: \I\to \C$ and tensor $P: \C\times\C\to\C$. We denote the tensor product
of $\B$ by $\times$, which should not be taken to imply that it is a cartesian product.
(We also write $\C^2$ as an abbreviation for $\C\times\C$, etc.)
The unit of $\B$ shall be denoted $\I$.
The associativity and unit isomorphisms of the pseudomonoid will be written
$\aa$, $\ll$, and $\rr$:
\[
	\begin{diagram}
		\I\times\C &\rTo^{J\times\C}&\C\times\C\\
		&\rdTo[snake=-1ex](1,2)<{1}
			\raise1ex\hbox{$\begin{array}c\Rightarrow\\[-5pt]\ll\end{array}$}%
			\ldTo[snake=1ex](1,2)>{P}\\
		&\C
	\end{diagram}
	\quad
	\begin{diagram}
		\C\times\C\times\C & \rTo^{P\times\C} & \C\times \C\\
		\dTo<{\C\times P} &\Nearrow \aa& \dTo>{P}\\
		\C\times\C & \rTo_P & \C
	\end{diagram}
	\quad
	\begin{diagram}
		\C\times \I &\rTo^{\C\times J}&\C\times\C\\
		&\rdTo[snake=-1ex](1,2)<{1}
			\raise1ex\hbox{$\begin{array}c\Rightarrow\\[-5pt]\rr\end{array}$}%
			\ldTo[snake=1ex](1,2)>{P}\\
		&\C
	\end{diagram}
\]
Since these 2-cells are assumed to be invertible, we shall permit ourselves to omit the
arrow below.

In our Gray monoid setting, the definition of pseudomonoid requires that
\begin{equation}\label{eq-lra}
\begin{diagram}
	\C\times\I\times\C\\
	\dTo<1 &\hbox to0pt{\hss$\C\times\ll$}\rdTo(2,1)^{\C\times J\times\C} & \C\times\C\times\C\\
	\C\times\C & \ldTo(2,1)_{\C\times P} &\dTo>{P\times\C}\\
	\dTo<P & \raise1.5em\hbox{$\aa$}& \C\times\C\\
	\C&\ldTo(2,1)_P
\end{diagram}
\qquad=\qquad
\begin{diagram}
	\C\times\I\times\C\\
	\dTo<1 &\hbox to0pt{\hss$\rr\times\C$}\rdTo(2,1)^{\C\times J\times\C} & \C\times\C\times\C\\
	\C\times\C & \ldTo(2,1)_{P\times\C}\\
	\dTo<P\\
	\C
\end{diagram}
\end{equation}
and
\begin{equation}\label{eq-aa}
	\begin{diagram}[s=2.2em,labelstyle=\scriptstyle,tight]
		&&\C^3\\
		&\ruTo^{\C^2\times P}&\dTo[snake=-5pt]<{P\times\C}&\rdTo^{\C\times P}\\
		\C^4 &\sim& \C^2 &\mathop{\Leftarrow}\limits_{\;\;\;\aa}& \C^2\\
		\dTo<{P\times\C^2}&\ruTo_{\C\times P} && \rdTo_P & \dTo>{P}\\
		\C^3 && \Arr\Downarrow\aa && \C\\
		&\rdTo_{P\times\C}&&\ruTo>{P}\\
		&&\C^2
	\end{diagram}
	\qquad=\qquad
	\begin{diagram}[s=2.2em,labelstyle=\scriptstyle,tight]
		&&\C^3\\
		&\ruTo^{\C^2\times P}&&\rdTo^{\C\times P}\\
		\C^4 && \Arr\Downarrow{\C\times \aa} && \C^2\\
		\dTo<{P\times\C^2}&\rdTo^{\C\times P\times\C} && \ruTo^{\C\times P} & \dTo>{P}\\
		\C^3 &\mathop{\Leftarrow}\limits_{\;\;\;\aa\times\C}& \C^3 &\Arr\Swarrow{\scriptstyle\!\!\!\aa}& \C\\
		&\rdTo_{P\times\C}&\dTo[snake=5pt]>{\!\!P\times\C}&\ruTo>{P}\\
		&&\C^2
	\end{diagram}
\end{equation}
Note that equation (\ref{eq-lra}) is equivalent to
\[
\begin{diagram}
	\C\times\I\times\C\\
	\dTo<1 &\hbox to0pt{\hss$\C\times\ll$}\rdTo(2,1)^{\C\times J\times\C} & \C\times\C\times\C\\
	\C\times\C & \ldTo(2,1)_{\C\times P}\\
	\dTo<P\\
	\C
\end{diagram}
\qquad=\qquad
\begin{diagram}
	\C\times\I\times\C\\
	\dTo<1 &\hbox to0pt{\hss$\rr\times\C$}\rdTo(2,1)^{\C\times J\times\C} & \C\times\C\times\C\\
	\C\times\C & \ldTo(2,1)_{P\times\C} &\dTo>{\C\times P}\\
	\dTo<P & \raise1.5em\hbox{$\aa$}& \C\times\C\\
	\C&\ldTo(2,1)_P
\end{diagram}
\]
by composing with the inverse of $\aa$. We shall often use it in this form, without
further remark. Similar variations of other equations may also be used without
drawing attention to the fact.

\section{Some facts about pseudomonoids}
This section is essentially a translation of \cite{KellyML} into the language
of general pseudomonoids. In particular, we shall show that the following
three equations hold of any pseudomonoid $\C$.
\begin{equation}\label{eq-lla}
\begin{diagram}
	\I\times\C^2 & \rTo^{J\times\C^2} &\C^3& \rTo^{\C\times P} & \C^2\\
	&\rdTo(2,2)<1\raise1em\hbox to0pt{$\ll\times\C$\hss} & \dTo[snake=.5em]>{P\times\C} &\aa&\dTo>P\\
	&&\C^2 &\rTo_P &\C
\end{diagram}
\quad=\quad
\begin{diagram}
	\I\times\C^2 &\rTo^{J\times\C^2} & \C^3\\
	\dTo<{\I\times P} &\sim & \dTo>{\C\times P}\\
	\I\times\C &\rTo^{J\times\C} & \C^2\\
	&\rdTo(2,2)_1\raise0.5em\hbox to0pt{\hskip0.5em$\ll$\hss} &\dTo>P\\
	&&\C
\end{diagram}
\end{equation}
\begin{equation}\label{eq-rra}
\begin{diagram}
	\C^2\times\I & \rTo^{\C^2\times J} &\C^3& \rTo^{P\times\C} & \C^2\\
	&\rdTo(2,2)<1\raise1em\hbox to0pt{$\C\times\rr$\hss} & \dTo[snake=.5em]>{\C\times P} &\aa&\dTo>P\\
	&&\C^2 &\rTo_P &\C
\end{diagram}
\quad=\quad
\begin{diagram}
	\C^2\times\I &\rTo^{\C^2\times J} & \C^3\\
	\dTo<{P\times\I} &\sim & \dTo>{P\times\C}\\
	\C\times\I &\rTo^{\C\times J} & \C^2\\
	&\rdTo(2,2)_1\raise0.5em\hbox to0pt{\hskip0.5em$\rr$\hss} &\dTo>P\\
	&&\C
\end{diagram}
\end{equation}
\begin{equation}\label{eq-lr}
\begin{diagram}
	\I\times\I\\
	\dTo<{J\times\I} & \rdTo(2,1)^{\I\times J} & \I\times\C\\
	\C\times\I &\raise1em\hbox{$\sim$}&\dTo>{J\times\C}\\
	\dTo<1 & \hbox to0pt{\hss$\rr$\hskip0.5em}\rdTo(2,1)^{\C\times J}& \C\times\C\\
	\C&\ldTo(2,1)_P
\end{diagram}
\quad=\quad
\begin{diagram}
	\I\times\I\\
	\dTo<{J\times\I} & \rdTo(2,1)^{\I\times J} & \I\times\C\\
	\C\times\I &\ldTo(2,3)^1&\dTo>{J\times\C}\\
	\dTo<1 & \raise1em\hbox to0pt{\hskip1em$\ll$\hss}& \C\times\C\\
	\C&\ldTo(2,1)_P
\end{diagram}
\end{equation}
%
In fact, it turns out that only equation (\ref{eq-lla}) is needed to prove the embedding theorem.
For the sake of completeness, however, we shall prove all three.
%
The following lemma will be very useful. It corresponds to the fact that, in a monoidal category,
the functors $I\tn-$ and $-\tn I$ are faithful.
\begin{lemma}\label{lemma-faithful}
	Let $A$ be some object of $\B$, let $f$, $g: A\to\C$
	and let $\gamma$, $\delta: f\To g$. If
	\[\hbox{\vrule height 3em depth 2em width 0pt}
	\begin{diagram}
		\rnode{CA}{\C\times A} & \Downarrow{\scriptstyle\C\times\gamma} & \rnode{CC}{\C\times\C} &\rTo^P &\C
		\ncarc{->}{CA}{CC}\Aput{\C\times f}
		\ncarc{<-}{CC}{CA}\Aput{\C\times g}
	\end{diagram}
	\quad=\quad
	\begin{diagram}
		\rnode{CA}{\C\times A} & \Downarrow{\scriptstyle\C\times\delta} & \rnode{CC}{\C\times\C} &\rTo^P &\C
		\ncarc{->}{CA}{CC}\Aput{\C\times f}
		\ncarc{<-}{CC}{CA}\Aput{\C\times g}
	\end{diagram}
	\]
	then $\gamma = \delta$.
	%
	Dually, if
	\[\hbox{\vrule height 3em depth 2em width 0pt}
	\begin{diagram}
		\rnode{AC}{A\times\C} & \Downarrow{\scriptstyle\gamma\times\C} & \rnode{CC}{\C\times\C} &\rTo^P &\C
		\ncarc{->}{AC}{CC}\Aput{f\times\C}
		\ncarc{<-}{CC}{AC}\Aput{g\times\C}
	\end{diagram}
	\quad=\quad
	\begin{diagram}
		\rnode{AC}{A\times\C} & \Downarrow{\scriptstyle\delta\times\C} & \rnode{CC}{\C\times\C} &\rTo^P &\C
		\ncarc{->}{AC}{CC}\Aput{f\times\C}
		\ncarc{<-}{CC}{AC}\Aput{g\times\C}
	\end{diagram}
	\]
	then $\gamma=\delta$.
\end{lemma}
\begin{proof}
	For the first part, suppose
	\[\hbox{\vrule height 3em depth 2em width 0pt}
	\begin{diagram}
		\rnode{CA}{\C\times A} & \Downarrow{\scriptstyle\C\times\gamma} & \rnode{CC}{\C\times\C} &\rTo^P &\C
		\ncarc{->}{CA}{CC}\Aput{\C\times f}
		\ncarc{<-}{CC}{CA}\Aput{\C\times g}
	\end{diagram}
	\quad=\quad
	\begin{diagram}
		\rnode{CA}{\C\times A} & \Downarrow{\scriptstyle\C\times\delta} & \rnode{CC}{\C\times\C} &\rTo^P &\C.
		\ncarc{->}{CA}{CC}\Aput{\C\times f}
		\ncarc{<-}{CC}{CA}\Aput{\C\times g}
	\end{diagram}
	\]
	Then
	\[\begin{array}{rcl}
		\gamma &=&
			\begin{diagram}
			&&\rnode{top}{\I\times\C}\\
			&&\sim\\
			\rnode{IA}{\I\times A} & \rTo^{J\times A} & \rnode{CA}{\C\times A}
				& \Downarrow{\scriptstyle\C\times\gamma} & \rnode{CC}{\C\times\C}
				\raise2em\hbox to0pt{\hskip.5em$\ll$\hss}
				\raise-2em\hbox to0pt{\hskip.5em$\ll$\hss}
				&\rTo^P & \rnode{C}{\C}\\
			&&\sim\\
			&&\rnode{bot}{\I\times\C}
			\ncarc{->}{CA}{CC}\Aput{\C\times f}
			\ncarc{<-}{CC}{CA}\Aput{\C\times g}
			\ncarc{->}{IA}{top}\Aput{\I\times f}
			\ncarc{->}{top}{CC}\aput{-53}(0.6){J\times\C}
			\ncarc{->}{top}{C}\Aput{1}
			\ncarc{<-}{C}{bot}\Aput{1}
			\ncarc{<-}{CC}{bot}\aput{53}(0.4){J\times\C}
			\ncarc{<-}{bot}{IA}\Aput{\I\times g}
			\end{diagram}
		\\[7em]
		&=&
			\begin{diagram}
			&&\rnode{top}{\I\times\C}\\
			&&\sim\\
			\rnode{IA}{\I\times A} & \rTo^{J\times A} & \rnode{CA}{\C\times A}
				& \Downarrow{\scriptstyle\C\times\delta} & \rnode{CC}{\C\times\C}
				\raise2em\hbox to0pt{\hskip.5em$\ll$\hss}
				\raise-2em\hbox to0pt{\hskip.5em$\ll$\hss}
				&\rTo^P & \rnode{C}{\C}\\
			&&\sim\\
			&&\rnode{bot}{\I\times\C}
			\ncarc{->}{CA}{CC}\Aput{\C\times f}
			\ncarc{<-}{CC}{CA}\Aput{\C\times g}
			\ncarc{->}{IA}{top}\Aput{\I\times f}
			\ncarc{->}{top}{CC}\aput{-53}(0.6){J\times\C}
			\ncarc{->}{top}{C}\Aput{1}
			\ncarc{<-}{C}{bot}\Aput{1}
			\ncarc{<-}{CC}{bot}\aput{53}(0.4){J\times\C}
			\ncarc{<-}{bot}{IA}\Aput{\I\times g}
			\end{diagram}
		\\
		&=&\delta.
	\end{array}\]
	The second part is proved similarly.
\end{proof}
%
\begin{propn}
	Equations (\ref{eq-lla}) and (\ref{eq-rra}) hold, of any pseudomonoid $\C$.
\end{propn}
\begin{proof}
	Consider the 2-cell
	\begin{diagram}
	\C\times\I\times\C^2 & \rTo^{\C\times J\times\C^2} &\C^4& \rTo^{\C^2\times P} & \C^3\\
	&\rdTo(2,2)<1\raise1em\hbox to0pt{$\C\times\ll\times\C$\hss}
		& \dTo[snake=.5em]>{\C\times P\times\C} &\C\times\aa&\dTo>{\C\times P}\\
	&&\C^3 &\rTo_{\C\times P} &\C^2\\
	&&\dTo<{P\times\C} & \aa & \dTo>P\\
	&&\C^2 & \rTo_P & \C
	\end{diagram}
	%
	By equation (\ref{eq-lra}), this is equal to
	\begin{diagram}
	\C\times\I\times\C^2 & \rTo^{\C\times J\times\C^2} &\C^4& \rTo^{\C^2\times P} & \C^3\\
	\dTo<1&\raise1em\hbox to0pt{\hss$\rr\times\C^2$}\ldTo[hug]_{P\times\C^2}
		& \dTo[snake=.8em]>{\C\times P\times\C} &\C\times\aa&\dTo>{\C\times P}\\
	\C^3&&\C^3 &\rTo_{\C\times P} &\C^2\\
	&\rdTo_{P\times\C}&\dTo<{P\times\C} & \aa & \dTo>P\\
	&&\C^2 & \rTo_P & \C
	\end{diagram}
	%
	which, by (\ref{eq-aa}), is equal to
	\begin{diagram}[tight,w=3.5em]
	\C\times\I\times\C^2 & \rTo^{\C\times J\times\C^2} &\C^4& \rTo^{\C^2\times P} & \C^3\\
	\dTo<1 &\ldTo(2,2)_{P\times\C^2}\raise1em\hbox to0pt{\hss$\rr\times\C^2$}&\sim
		&\ldTo^{P\times\C} &\dTo>{\C\times P}\\
	\C^3 &\rTo_{\C\times P} & \C^2 &\aa &\C^2\\
	&\rdTo_{P\times\C} &\aa&\rdTo_P&\dTo>P\\
	&&\C^2 & \rTo_P & \C
	\end{diagram}
	%
	Pasting on a structural 2-cell and its inverse gives
	\begin{diagram}[tight,w=4em]
	&&\C^4\\
	&\ruTo^{\C\times J\times\C^2} &\sim&\rdTo^{\C^2\times P}\\
	\C\times\I\times\C^2&\rTo^{\C\times\I\times P} &\C\times\I\times\C &\rTo_{\C\times J\times\C} &\rnode{x}{\C^3}\\
	&\rdTo[snake=1em]^{\C\times J\times\C^2} &&\sim\\
	\dTo<1 &\rr&\C^4& \rTo^{\C^2\times P} & \rnode{y}{\C^3}\\
	&\ldTo(2,2)_{P\times\C^2}&\sim
		&\ldTo^{P\times\C} &\dTo>{\C\times P}\\
	\C^3 &\rTo_{\C\times P} & \C^2 &\aa &\C^2\\
	&\rdTo_{P\times\C} &\aa&\rdTo_P&\dTo>P\\
	&&\C^2 & \rTo_P & \C
	\ncline[doubleline=true,doublesep=2pt]{-}xy
	\end{diagram}
	which, since the structural 2-cells behave naturally, is equal to
	\begin{diagram}[tight,w=4em]
	&&\C^4\\
	&\ruTo^{\C\times J\times\C^2} &\sim&\rdTo^{\C^2\times P}\\
	\C\times\I\times\C^2&\rTo^{\C\times\I\times P} &\C\times\I\times\C &\rTo_{\C\times J\times\C} &\rnode{x}{\C^3}\\
	&&&\raise-1em\hbox{$\rr\times\C$}\\
	\dTo<1 &&\dTo<1&& \rnode{y}{\C^3}\\
	&&&\ldTo^{P\times\C} &\dTo>{\C\times P}\\
	\C^3 &\rTo_{\C\times P} & \C^2 &\aa &\C^2\\
	&\rdTo_{P\times\C} &\aa&\rdTo_P&\dTo>P\\
	&&\C^2 & \rTo_P & \C\hbox{\hskip1pt.\hss}
	\ncline[doubleline=true,doublesep=2pt]{-}xy
	\end{diagram}
	%
	This diagram can be redrawn as
	\begin{diagram}
	&&\rnode{CICC}{\C\times\I\times\C^2} &\rTo^{\C\times J\times\C^2} &\C^4\\
	&&\dTo<{\C\times\I\times P} &\sim& \dTo>{\C^2\times P}\\
	&&\C\times\I\times\C &\rTo^{\C\times J\times\C} &\rnode{x}{\C^3}\\
	&&\dTo<1&\rr\times\C\\
	\rnode{CCC}{\C^3}&\rTo^{\C\times P}&\C^2 &\lTo^{P\times\C} &\rnode{y}{\C^3}\\
	\dTo<{P\times\C} &\aa & \dTo>P & \aa & \dTo>{\C\times P}\\
	\C^2 &\rTo_P & \C &\lTo_P & \C^2
	\ncline[doubleline=true,doublesep=2pt]{-}xy
	\nccurve[angleA=180,angleB=90]{->}{CICC}{CCC}\Bput{1}
	\end{diagram}
	%
	which by equation (\ref{eq-lra}) is equal to
	\begin{diagram}
	&&\rnode{CICC}{\C\times\I\times\C^2} &\rTo^{\C\times J\times\C^2} &\C^4\\
	&&\dTo<{\C\times\I\times P} &\sim& \dTo>{\C^2\times P}\\
	&&\C\times\I\times\C &\rTo^{\C\times J\times\C} &\rnode{x}{\C^3}\\
	&&\dTo<1&\C\times\ll\\
	\rnode{CCC}{\C^3}&\rTo^{\C\times P}&\C^2 &\lTo^{\C\times P} &\rnode{y}{\C^3}\\
	\dTo<{P\times\C} &\aa & \dTo>P\\
	\C^2 &\rTo_P & \C\hbox to0pt{\hskip1pt.\hss}
	\ncline[doubleline=true,doublesep=2pt]{-}xy
	\nccurve[angleA=180,angleB=90]{->}{CICC}{CCC}\Bput{1}
	\end{diagram}
	%
	Comparing this with the diagram we started with, and cancelling $\aa$, we have
	\[
	\begin{diagram}
	\C\times\I\times\C^2 & \rTo^{\C\times J\times\C^2} &\C^4& \rTo^{\C^2\times P} & \C^3\\
	&\rdTo(2,2)<1\raise1em\hbox to0pt{$\C\times\ll\times\C$\hss}
		& \dTo[snake=.5em]>{\C\times P\times\C} &\C\times\aa&\dTo>{\C\times P}\\
	&&\C^3 &\rTo_{\C\times P} &\C^2\\
	&&&& \dTo>P\\
	&&&& \C
	\end{diagram}
	\quad=\quad
	\begin{diagram}
	\C\times\I\times\C^2 &\rTo^{\C\times J\times\C^2} &\C^4\\
	\dTo<{\C\times\I\times P} &\sim& \dTo>{\C^2\times P}\\
	\C\times\I\times\C &\rTo^{\C\times J\times\C} &\rnode{x}{\C^3}\\
	\dTo<1&\C\times\ll\\
	\C^2 &\lTo^{\C\times P} &\rnode{y}{\C^3}\\
	\dTo>P\\
	\C
	\ncline[doubleline=true,doublesep=2pt]{-}xy
	\end{diagram}
	\]
	whence Lemma~\ref{lemma-faithful} allows us to deduce equation (\ref{eq-lla}), as required.
	Equation (\ref{eq-rra}) follows by symmetry.
\end{proof}
%
The next lemma is the general form of the fact that $I\tn\lambda_A = \lambda_{I\tn A}$
in a monoidal category.
\begin{lemma}\label{lemma-lambda}
The following equation holds of any pseudomonoid $\C$:
\begin{equation}\label{eq-lambda}
\hskip-2em
\begin{diagram}
	\I\times\I\times\C & \rTo^{J\times\I\times\C} & \rnode{CIC}{\C\times\I\times\C}\\
	&&\dTo>{\C\times J\times\C}\\
	&\hbox to0pt{\hskip1em$\C\times\ll$\hss}&\C^3\\
	&&\dTo>{\C\times P}\\
	&&\rnode{CC}{\C^2}\\
	&&\dTo>P\\
	&&\C
	\ncarc[arcangle=-70]{->}{CIC}{CC}\Bput{1}
\end{diagram}
=
\begin{diagram}
	\I\times\I\times\C & \rTo^{J\times\I\times\C} & \C\times\I\times\C\\
	\dTo<{\I\times J\times\C} & \sim & \dTo>{\hbox to0pt{$\C\times J\times\C$\hss}}\\
	\I\times\C^2 & \rTo_{J\times\C^2}& \C^3\\
	\dTo<{\I\times P} & \sim & \dTo>{\C\times P}\\
	\I\times\C & \rTo^{J\times \C} & \C^2\\
	&\rdTo(2,2)_1\raise1em\hbox to 0pt{$\ll$\hss} & \dTo>P\\
	&&\C
\end{diagram}
\end{equation}
\end{lemma}
\begin{proof}
	Since the structural 2-cells behave naturally,
	\begin{diagram}
		\I\times\I\times\C & \rTo^{J\times\I\times\C} & \rnode{CIC}{\C\times\I\times\C}\\
		&&\dTo>{\C\times J\times\C}\\
		\dTo<1&\hbox to0pt{\hskip1em$\C\times\ll$\hss}&\C^3\\
		&&\dTo>{\C\times P}\\
		\I\times\C&\rTo^{J\times\C}&\rnode{CC}{\C^2}\\
		&\rdTo(2,2)_1\raise1em\hbox to0pt{\hskip1em$\ll$\hss}&\dTo>P\\
		&&\C
		\ncarc[arcangle=-70]{->}{CIC}{CC}\Bput{1}
	\end{diagram}
	is equal to
	\begin{diagram}
		&\rnode{IIC}{\I\times\I\times\C} & \rTo^{J\times\I\times\C} & \C\times\I\times\C\\
		&\dTo<{\I\times J\times\C} & \sim & \dTo>{\C\times J\times\C}\\
		\ll&\rnode{IC}{\I\times\C^2} & \rTo_{J\times\C^2}& \C^3\\
		&\dTo<{\I\times P} & \sim & \dTo>{\C\times P}\\
		&\rnode{IC}{\I\times\C} & \rTo^{J\times \C} & \C^2\\
		&&\rdTo(2,2)_1\raise1em\hbox to 0pt{\hskip1em$\ll$\hss} & \dTo>P\\
		&&&\C
		\ncarc[arcangle=-70,ncurv=1]{->}{IIC}{IC}\Bput{1}
	\end{diagram}
	Vertically composing with the inverse of $\ll$ gives the claimed equation.
\end{proof}
%
\begin{propn}
	Equation (\ref{eq-lr}) holds of any pseudomonoid $\C$.
\end{propn}
\begin{proof}
	\[\begin{array}{rcl}
	\begin{diagram}
		&\I\times\I\times\C \\
		&\dTo<{J\times\I\times\C} \\
		&\rnode{CIC}{\C\times\I\times\C}\\
		&\dTo>{\C\times J\times\C} \\
		\rr\times\C & \C^3 &\rTo^{\C\times P} & \C^2 \\
		&\dTo>{P\times\C} &\aa& \dTo>P \\
		&\rnode{CC}{\C^2} &\rTo_{P} & \C
		\ncarc[arcangle=-70,ncurv=1]{->}{CIC}{CC}\Bput{1}
	\end{diagram}
	&\quad=\quad&
	\begin{diagram}
		&\I\times\I\times\C \\
		&\dTo<{J\times\I\times\C} \\
		&\rnode{CIC}{\C\times\I\times\C}\\
		&\dTo>{\C\times J\times\C} \\
		\C\times\ll & \C^3 \\
		&\dTo>{\C\times P} \\
		&\rnode{CC}{\C^2} &\rTo_{P} & \C
		\ncarc[arcangle=-70,ncurv=1]{->}{CIC}{CC}\Bput{1}
	\end{diagram}
	\\[11em]
	&=&
	\begin{diagram}
		\I\times\I\times\C & \rTo^{J\times\I\times\C} & \C\times\I\times\C\\
		\dTo<{\I\times J\times\C} & \sim & \dTo>{\C\times J\times\C}\\
		\I\times\C^2 & \rTo_{J\times\C^2}& \C^3\\
		\dTo<{\I\times P} & \sim & \dTo>{\C\times P}\\
		\I\times\C & \rTo^{J\times \C} & \C^2\\
		&\rdTo(2,2)_1\raise1em\hbox to 0pt{$\ll$\hss} & \dTo>P\\
		&&\C
	\end{diagram}
	\\[10em]
	&=&
	\begin{diagram}
		\I\times\I\times\C & \rTo^{J\times\I\times\C} & \C\times\I\times\C\\
		\dTo<{\I\times J\times\C} & \sim & \dTo>{\C\times J\times\C} \\
		\I\times\C^2 & \rTo^{J\times\C^2} & \rnode{x}{\C^3}\\
		\dTo<1 & \ll\times\C \\
		\C^2 & \lTo^{P\times\C} & \rnode{y}{\C^3} \\
		\dTo<P & \aa & \dTo>{\C\times P} \\
		\C & \lTo_P & \C^2
		\ncline[doubleline=true,doublesep=2pt]{-}xy
	\end{diagram}
	\end{array}\]
	These equalities hold by, respectively, equations (\ref{eq-lra}), (\ref{eq-lambda}), and (\ref{eq-lla}).
	Now vertically compose with the inverse of $\aa$, and apply Lemma~\ref{lemma-faithful}
	to deduce the claim.
\end{proof}





















\section{Equivalence theorem for pseudomonoids}

Let $\C_0$ denote the monoidal category underlying the pseudomonoid.
Precisely, $\C_0$ is the category $\B(\I, \C)$, equipped with the following
monoidal structure. The unit is $J$, and the tensor product $A\tn B$ of
two objects $A$,$B$ is the composite
\[
	\I=\I\times \I\rTo^{\I\times B}\I\times\C\rTo^{A\times\C}\C\times\C\rTo^P\C.
\]
The associativity and unit isomorphisms are derived in the obvious way
from those of $\C$, and inherit their coherence properties.

(Note that if we take $\B$ to be a Gray-category equivalent to $\Prof\op$,
then a pseudomonoid in $\B$ is a promonoidal category, and $\C_0$ is
the functor category $[\C,\Set]$ equipped with Day's convolution tensor.)

The main theorem of this section is that $\C_0$ is monoidally equivalent to
a certain strict monoidal category,
which we denote $\e(\C)$ and define as follows. An object is a 1-cell
\[
	\C \rTo^F \C
\]
together with an invertible 2-cell
\begin{diagram}
	\C\times\C & \rTo^P & \C\\
	\dTo<{F\times\C}&\Arr\Swarrow{\phi^F}&\dTo>F\\
	\C\times\C&\rTo_P & \C
\end{diagram}
such that
\[
	\begin{diagram}[s=2.2em,labelstyle=\scriptstyle,tight]
		&&\C^2\\
		&\ruTo^{\C\times P}&\dTo[snake=-5pt]<{F\times\C}&\rdTo^P\\
		\C^3 &\sim& \C^2 &\mathop{\Leftarrow}\limits_{\;\;\;\phi^F}& \C\\
		\dTo<{F\times\C^2}&\ruTo_{\C\times P} && \rdTo_P & \dTo>{F}\\
		\C^3 && \Arr\Downarrow\aa && \C\\
		&\rdTo_{P\times\C}&&\ruTo>{P}\\
		&&\C^2
	\end{diagram}
	\qquad=\qquad
	\begin{diagram}[s=2.2em,labelstyle=\scriptstyle,tight]
		&&\C^2\\
		&\ruTo^{\C\times P}&&\rdTo^P\\
		\C^3 && \Arr\Downarrow\aa && \C\\
		\dTo<{F\times\C^2}&\rdTo^{P\times\C} && \ruTo^P & \dTo>{F}\\
		\C^3 &\mathop{\Leftarrow}\limits_{\;\;\;\phi^F\times\C}& \C^2 &\Arr\Swarrow{\scriptstyle\!\!\!\phi^F}& \C\hbox to 0pt{\hskip1pt.\hss}\\
		&\rdTo_{P\times\C}&\dTo[snake=5pt]>{\!\!F\times\C}&\ruTo>{P}\\
		&&\C^2
	\end{diagram}
\]
%
In one place below -- the proof of Proposition~\ref{prop-i-equiv} -- it is convenient
to use the following rotated view of this cube:
\[
	\begin{diagram}[s=2.2em,labelstyle=\scriptstyle,tight]
		&&\C^3\\
		&\ldTo^{\C\times P}&\dTo[snake=-5pt]<{F\times\C^2}&\rdTo^{P\times\C}\\
		\C^2 &\sim& \C^3 &\phi^F\!\!\times\!\C& \C^2\\
		\dTo<{F\times\C}&\ldTo_{\C\times P} && \rdTo_{P\times\C} & \dTo>{F\times\C}\\
		\C^2 && \aa && \C^2\\
		&\rdTo_{P}&&\ldTo>{P}\\
		&&\C
	\end{diagram}
	\qquad=\qquad
	\begin{diagram}[s=2.2em,labelstyle=\scriptstyle,tight]
		&&\C^3\\
		&\ldTo^{\C\times P}&&\rdTo^{P\times\C}\\
		\C^2 && \aa && \C^2\\
		\dTo<{F\times\C}&\rdTo^{P} && \ldTo^P & \dTo>{F\times\C}\\
		\C^2 &\phi^F& \C &\phi^F& \C\hbox to 0pt{\hskip1pt.\hss}\\
		&\rdTo_{P}&\dTo[snake=5pt]>{F}&\ldTo>{P}\\
		&&\C
	\end{diagram}
\]
%
The tensor product $F\tn G$ of two objects is their composite $FG$ in $\B$, with $\phi^{FG}$
being the pasting
\begin{diagram}
	\rnode{b}{\C\times\C} & \rTo^P & \C\\
	\dTo<{G\times\C}&\Arr\Swarrow{\phi^G}&\dTo>G\\
	\C\times\C&\rTo_P & \C\\
	\dTo<{F\times\C}&\Arr\Swarrow{\phi^F}&\dTo>F\\
	\rnode{a}{\C\times\C}&\rTo_P & \C
%	This is an equality in a Gray-category...
%	\ncarc[arcangle=90,linestyle=none]{->}ab
%	\lput*{:U}{\shortmid}
%	\ncarc[arcangle=90]{->}ab
%	\Aput{(GF)\times\C}\Bput{\hskip 1.3em\cong}
\end{diagram}
The tensor unit $I\in\e(\C)$ is the identity 1-cell, with the identity 2-cell.

A morphism $F\to G$ is a natural transformation $\gamma: F\To G$ such that
\[\psset{ncurv=2}
	\begin{diagram}
		\rnode{b}{\C\times\C} & \rTo^P & \C\\
		\dTo<{F\times\C}&\Arr\Swarrow{\phi^F}&\dTo>F\\
		\rnode{a}{\C\times\C} & \rTo_P & \C
		\ncarc[arcangle=90]{<-}ab
		\Aput{G\times\C}\Bput{\hskip .7em\mathop{\Leftarrow}\limits_{\gamma\times\C}}
	\end{diagram}
	\quad=\qquad
	\begin{diagram}
		\C\times\C & \rTo^P & \rnode{b}{\C}\\
		\dTo<{G\times\C}&\Arr\Swarrow{\phi^G}&\dTo>G\\
		\C\times\C & \rTo_P & \rnode{a}{\C}
		\ncarc[arcangle=-90]{<-}ab
		\Bput{F}\Aput{\mathop{\Leftarrow}\limits_{\textstyle\gamma}\hskip 1.3em}
	\end{diagram}
\]
\begin{remark}
	Note that $\e(\C)$ is only strict monoidal because we are taking $\B$
	to be a Gray monoid. When $\B$ is a general bicategory, $\e(\C)$
	will still be monoidal, but not usually strict.
\end{remark}
Our object is to show that $\e(\C)$ is monoidally equivalent to $\C_0$.
Thus we define a functor
\[
	i_\C: \C_0 \to \e(\C)
\]
as follows. For an object $M\in\C_0$, we let $i_\C(M)$ be the composite
\[
	\I\times\C \rTo^{M\times\C} \C\times\C\rTo^P \C.
\]
The associated natural isomorphism $\phi^{i_\C(M)}$ is
\begin{diagram}[h=2em]
	\I\times\C\times\C & \rTo^{\I\times P}& \I\times \C\\
	\dTo<{M\times\C\times\C} & \raise6pt\hbox{$\sim$} & \dTo>{M\times\C}\\
	\C\times\C\times\C & \rTo^{\C\times P} & \C\times\C\\
	\dTo<{P\times\C} & {\aa} & \dTo>P\\
	\C\times\C & \rTo_P & \C
\end{diagram}
thus a morphism $\gamma: i_\C(M)\to i_\C(N)$ is a natural transformation that
satisfies the equation
\begin{equation}\label{eq-box}
	\begin{diagram}[w=2.8em,h=2em,labelstyle=\scriptstyle,tight]
		&&\C\times\C\\
		&\ruTo^{M\times\C}&\dTo[snake=-5pt]<{\C\times P}&\rdTo^P\\
		\I\times\C &\sim& \C\times\C &\aa& \C\\
		\uTo<{\I\times P}&\ruTo_{M\times\C\times\C} && \rdTo_{P\times\C} & \uTo>{P}\\
		\I\times\C\times\C && \Arr\Downarrow{\gamma\times\C} && \C\times\C\\
		&\rdTo_{N\times\C\times\C}&&\ruTo>{P\times\C}\\
		&&\C\times\C\times\C
	\end{diagram}
	=
	\begin{diagram}[w=2.8em,h=2em,labelstyle=\scriptstyle,tight]
		&&\C\times\C\\
		&\ruTo^{M\times\C}&&\rdTo^P\\
		\I\times\C && \Arr\Downarrow\gamma && \C\\
		\uTo<{\I\times P}&\rdTo^{N\times\C} && \ruTo^P & \uTo>{P}\\
		\I\times\C\times\C &\sim& \C\times\C &\aa & \C\times\C\\
		&\rdTo_{N\times\C\times\C}&\uTo[snake=5pt]>{\!\!\C\times P}&\ruTo>{P\times\C}\\
		&&\C\times\C\times\C
	\end{diagram}
\end{equation}
%
Given a natural transformation $\delta: M\To N$, define $i_\C(\delta)$ to be the whiskering
\begin{diagram}[h=4em]\\
\rnode{1C}{\I\times \C} & \Downarrow {\scriptstyle\delta\times\C} &
\rnode{CC}{\C\times\C}&\rTo^P&\C\\
	\ncarc[arcangle=90]{->}{1C}{CC}
	\Aput{M\times\C}
	%
	\ncarc[arcangle=-90]{->}{1C}{CC}
	\Bput{N\times\C}
\end{diagram}
which clearly satisfies the equation above.

\begin{propn}\label{prop-i-ff}
	The functor $i_\C$ is full and faithful.
\end{propn}
\begin{proof}
Fix some $M$ and $N$ in $\B(\I,\C)$.
Given a map $\gamma: i_\C(M)\to i_\C(N)$, define the map $p(\gamma)\in\B(\I,\C)(M,N)$ to
be the pasting
\begin{diagram}
	&&&&\rnode{top}{\C\times \I}\\
	&&&&\dTo>{\C\times J}\\
	&&\raise2em\hbox{$\sim$}&&\C\times\C&\raise1em\hbox{$\rr$}\\
	\rnode{II}{\I\times \I} & \rTo_{\I\times J} & \I\times\C&\ruTo(2,1)^{M\times\C}&\Arr\Downarrow\gamma
		&\rdTo(2,1)^P &\rnode{C}{\C}\\
	&&\raise-1.5em\hbox{$\sim$}&\rdTo(2,1)_{N\times\C}&\C\times\C&\ruTo(2,1)_P\raise-1em\hbox{$\rr$}\\
	&&&&\uTo>{\C\times J}\\
	&&&&\rnode{bot}{\C\times \I}
	\ncarc{->}{II}{top}\Aput{M\times \I}
	\ncarc{->}{top}{C}\Aput{1}
	\ncarc{<-}{bot}{II}\Aput{N\times \I}
	\ncarc{<-}{C}{bot}\Aput{1}
\end{diagram}
%
We shall show that this $p$ is inverse to the action of $i_\C$ on the homset
$\B(\I,\C)(M,N)$, hence that $i_\C$ is full and faithful. So first we need to show
that for each $\delta\in [\C,\Set](M,N)$, we have $p(i_\C(\delta)) = \delta$.
Diagramatically, $p(i_\C(\delta))$ is the pasting
\begin{diagram}
	&&&&\rnode{top}{\C\times \I}\\
	&&\sim&&\dTo>{\C\times J}\\
	\rnode{II}{\I\times \I} & \rTo_{\I\times J} & \rnode{IC}{\I\times\C}&\Downarrow{\scriptstyle\delta\times\C}
		&\rnode{CC}{\C\times\C} &\rTo^P &\rnode{C}{\C}\\
	&&\sim&&\uTo>{\C\times J}\\
	&&&&\rnode{bot}{\C\times \I}
	\ncarc{->}{II}{top}\Aput{M\times \I}
	\ncarc{->}{top}{C}\Aput{1}
	\ncarc{<-}{bot}{II}\Aput{N\times \I}
	\ncarc{<-}{C}{bot}\Aput{1}
	\ncarc{->}{IC}{CC}\Aput{M\times\C}
	\ncarc{<-}{CC}{IC}\Aput{N\times\C}
\end{diagram}
By definition of a Gray monoid, we know that
\[
	\begin{diagram}
	\\
	\rnode{IC}{\I\times\C} & \Downarrow{\scriptstyle\delta\times\C} & \rnode{CC}{\C\times\C}\\
	\uTo<{\I\times J} & \raise-1em\hbox{$\sim$} & \uTo>{\C\times J}\\
	\I\times \I & \rTo_{N\times \I} & \C\times \I\\
	\ncarc{->}{IC}{CC}\Aput{M\times\C}
	\ncarc{<-}{CC}{IC}\Aput{N\times\C}
	\end{diagram}
	\qquad=\qquad
	\begin{diagram}
	\\
	\I\times \C & \rTo_{M\times \C} & \C\times\C\\
	\uTo<{\I\times J} & \raise1em\hbox{$\sim$} & \uTo>{\C\times J}\\
	\rnode{II}{\I\times \I} & \Downarrow{\scriptstyle\delta\times \I} & \rnode{CI}{\C\times \I}\\
	\ncarc{->}{II}{CI}\Aput{M\times \I}
	\ncarc{<-}{CI}{II}\Aput{N\times \I}
	\end{diagram}
\]
from which it easily follows that $p(i_\C(\delta))=\delta$, as required.

For the other direction, consider some $\gamma: i_\C(M)\to i_\C(N)$ in $\e(\C)$.
Diagramatically, $i_\C(p(\gamma))$ is the pasting
\begin{diagram}[w=4em,labelstyle=\scriptstyle]
	&&&&\rnode{top}{\C\times \I\times\C}\\
	&&&&\dTo<{\C\times J\times\C}\\
	&&\raise1em\hbox{$\sim$}&&\C\times\C\times\C&\raise1em\hbox{$\rr\times\C$}\\
	\rnode{IIC}{\I\times \I\times\C}&\rTo^{\I\times J\times\C}&\I\times\C\times\C&\ruTo(2,1)^{M\times\C\times\C}
		&\Arr\Downarrow{\gamma\times\C}&\rdTo(2,1)^{P\times\C}&\rnode{CC}{\C\times\C}
		&\rTo^{\textstyle P}&\C\\
	&&\raise-1em\hbox{$\sim$}&\rdTo(2,1)_{N\times\C\times\C}&\C\times\C\times\C&\ruTo(2,1)_{P\times\C}
		\raise-1em\hbox{$\rr\times\C$}\\
	&&&&\uTo<{\C\times J\times\C}\\
	&&&&\rnode{bot}{\C\times \I\times\C}
	\ncarc{->}{IIC}{top}\Aput{M\times \I\times\C}
	\ncarc{->}{top}{CC}\Aput{1}
	\ncarc{<-}{bot}{IIC}\Aput{N\times \I\times\C}
	\ncarc{<-}{CC}{bot}\Aput{1}
\end{diagram}
We want to show that this is equal to $\gamma$, for which we use an easy lemma:
\begin{lemma}\label{lemma-gamma}
\begin{diagram}[labelstyle=\scriptstyle]
	&&\C\times\C\times\C\\
	\I\times\C\times\C&\ruTo(2,1)^{M\times\C\times\C}&\Downarrow{\gamma\times\C}
		&\rdTo(2,1)^{P\times\C}&\rnode{CC}{\C\times\C}&\rTo^{\textstyle P}&\C\\
	&\rdTo(2,1)_{N\times\C\times\C}&\C\times\C\times\C&\ruTo(2,1)_{P\times\C}
\end{diagram}
is equal to
\begin{diagram}[w=4em,labelstyle=\scriptstyle]
	&&&&\rnode{top}{\C\times\C\times\C}\\
	&&&&\dTo<{\C\times P}&\rdTo(2,1)^{P\times\C}&\C\times\C\\
	&&\raise3em\hbox{$\sim$}&&\C\times\C&\raise1em\hbox{$\aa$}&\dTo>{P}\\
	\rnode{ICC}{\I\times\C\times\C}&\rTo^{\I\times P}&\C\times\C&\ruTo(2,1)^{M\times\C}
		&\Arr\Downarrow{\gamma}&\rdTo(2,1)^{P}&\C\\
	&&\raise-1em\hbox{$\sim$}&\rdTo(2,1)_{N\times\C}&\C\times\C&\ruTo(2,1)_{P}
		\raise-2em\hbox{$\aa$}&\uTo>P\\\
	&&&&\uTo<{\C\times P} &&\C\times\C.\\
	&&&&\rnode{bot}{\C\times\C\times\C}&\ruTo(2,1)_{P\times\C}
	\ncarc{->}{ICC}{top}\Aput{M\times \I\times\C}
	\ncarc{<-}{bot}{ICC}\Aput{N\times \I\times\C}
\end{diagram}
\end{lemma}
\begin{proof}
	Clearly the identity 2-cell on
	\[
	\I\times\C\times\C \rTo^{M\times\C\times\C} \C\times\C\times\C
		\rTo^{P\times\C}\C\times\C\rTo^P\C
	\]
	is equal to
	\begin{diagram}[tight,w=4em]
	&&\C\times\C\times\C&\rTo^{P\times\C}&\C\times\C\\
	&\ruTo^{M\times\C\times\C}&\sim&\rdTo^{\C\times P}&\aa&\rdTo^P\\
	\I\times\C\times\C&\rTo^{\I\times P}&\I\times\C&\rTo^{M\times\C}&\C\times\C&\rTo^P&\C.\\
	&\rdTo_{M\times\C\times\C}&\sim&\ruTo_{\C\times P} & \aa &\ruTo_P\\
	&&\C\times\C\times\C&\rTo_{P\times\C\times\C}&\C\times\C
	\end{diagram}
	Vertically compose this with
	\begin{diagram}[labelstyle=\scriptstyle]
	&&\C\times\C\times\C\\
	\I\times\C\times\C&\ruTo(2,1)^{M\times\C\times\C}&\Downarrow{\gamma\times\C}
		&\rdTo(2,1)^{P\times\C}&\rnode{CC}{\C\times\C}&\rTo^{\textstyle P}&\C,\\
	&\rdTo(2,1)_{N\times\C\times\C}&\C\times\C\times\C&\ruTo(2,1)_{P\times\C}
	\end{diagram}
	then equation (\ref{eq-box}) gives the desired result.
\end{proof}
%
So, by Lemma~\ref{lemma-gamma}, $i_\C(p(\gamma))$ is equal to
\begin{diagram}[w=4em,labelstyle=\scriptstyle]
	&&&&&&\rnode{topCIC}{\C\times \I\times\C}\\
	&&&&&&\dTo<{\C\times J\times\C}&\raise-2em\hbox{$\rr\times\C$}\\
	&&&&\raise2em\hbox{$\sim$}&&\rnode{topCCC}{\C\times\C\times\C}\\
	&&&&&&\dTo<{\C\times P}&\rdTo(2,1)^{P\times\C}&\rnode{topCC}{\C\times\C}\\
	&&&&\raise1em\hbox{$\sim$}&&\C\times\C&\raise1em\hbox{$\aa$}&\dTo>{P}\\
	\rnode{IIC}{\I\times \I\times\C}&\rTo^{\I\times J\times\C}&\rnode{ICC}{\I\times\C\times\C}
		&\rTo^{\I\times P}&\C\times\C&\ruTo(2,1)^{M\times\C}
		&\Arr\Downarrow{\gamma}&\rdTo(2,1)^{P}&\C\\
	&&&&\raise-1em\hbox{$\sim$}&\rdTo(2,1)_{N\times\C}&\C\times\C&\ruTo(2,1)_{P}
		\raise-2em\hbox{$\aa$}&\uTo>P\\\
	&&&&&&\uTo<{\C\times P} &&\rnode{botCC}{\C\times\C}\\
	&&&&\raise-2em\hbox{$\sim$}&&\rnode{botCCC}{\C\times\C\times\C}&\ruTo(2,1)_{P\times\C}\\
	&&&&&&\uTo<{\C\times J\times\C}&\raise2em\hbox{$\rr\times\C$}\\
	&&&&&&\rnode{botCIC}{\C\times \I\times\C}
	\ncarc{->}{ICC}{topCCC}\Aput{M\times \I\times\C}
	\ncarc{->}{IIC}{topCIC}\Aput{M\times \I\times\C}
	\ncarc{->}{topCIC}{topCC}\Aput{1}
	\ncarc{<-}{botCCC}{ICC}\Aput{N\times \I\times\C}
	\ncarc{<-}{botCIC}{IIC}\Aput{N\times \I\times\C}
	\ncarc{<-}{botCC}{botCIC}\Aput{1}
\end{diagram}
%
which, by definition of a pseudomonoid, is equal to
\begin{diagram}[w=4em,labelstyle=\scriptstyle]
	&&&&&&\rnode{topCIC}{\C\times \I\times\C}\\
	&&&&&&\dTo<{\C\times J\times\C}\\
	&&&&\raise2em\hbox{$\sim$}&&\rnode{topCCC}{\C\times\C\times\C}&\C\times\ll\\
	&&&&&&\dTo<{\C\times P}\\
	&&&&\raise1em\hbox{$\sim$}&&\rnode{topCC}{\C\times\C}\\
	\rnode{IIC}{\I\times \I\times\C}&\rTo^{\I\times J\times\C}&\rnode{ICC}{\I\times\C\times\C}
		&\rTo^{\I\times P}&\C\times\C&\ruTo(2,1)^{M\times\C}
		&\Arr\Downarrow{\gamma}&\rdTo(2,1)^{P}&\C\\
	&&&&\raise-1em\hbox{$\sim$}&\rdTo(2,1)_{N\times\C}&\rnode{botCC}{\C\times\C}&\ruTo(2,1)_{P}\\
	&&&&&&\uTo<{\C\times P} &&\\
	&&&&\raise-2em\hbox{$\sim$}&&\rnode{botCCC}{\C\times\C\times\C}&\C\times\ll\\
	&&&&&&\uTo<{\C\times J\times\C}\\
	&&&&&&\rnode{botCIC}{\C\times \I\times\C}
	\ncarc{->}{ICC}{topCCC}\Aput{M\times \I\times\C}
	\ncarc{->}{IIC}{topCIC}\Aput{M\times \I\times\C}
	\ncarc[arcangle=90,ncurv=1.5]{->}{topCIC}{topCC}\Aput{1}
	\ncarc{<-}{botCCC}{ICC}\Aput{N\times \I\times\C}
	\ncarc{<-}{botCIC}{IIC}\Aput{N\times \I\times\C}
	\ncarc[arcangle=90,ncurv=1.5]{<-}{botCC}{botCIC}\Aput{1}
\end{diagram}
which is equal to
\begin{diagram}
	&&&&&&\rnode{top}{\C\times \I\times\C}\\
	&&&&&&\dTo>1\\
	&&\ll&&&&\C\times\C\\
	\rnode{IIC}{\I\times \I\times\C} &\rTo^{\I\times J\times\C} & \I\times\C\times\C &\rTo^{\I\times P}
		&\rnode{IC}{\I\times C} &\ruTo(2,1)^{M\times\C}&\Arr\Downarrow\gamma&\rdTo(2,1)^P&\C\\
	&&\ll&&&\rdTo(2,1)_{N\times\C}&\C\times\C&\ruTo(2,1)_P\\
	&&&&&&\uTo>1\\
	&&&&&&\rnode{bot}{\C\times \I\times\C}
	\psset{ncurv=1,arcangle=60}
	\nccurve[angleA=90,angleB=180]{->}{IIC}{top}\Aput{M\times \I\times\C}
	\ncarc{->}{IIC}{IC}\Aput{1}
	\nccurve[angleA=180,angleB=-90]{<-}{bot}{IIC}\Aput{N\times \I\times\C}
	\ncarc{<-}{IC}{IIC}\Aput{1}
\end{diagram}
which is clearly equal to $\gamma$, as required. So $i_\C$ is indeed a full and
faithful functor.
\end{proof}
%
\begin{propn}\label{prop-i-monoidal}
	The functor $i_\C$ is strong monoidal, when equipped with the
	isomorphism $m_I: I\to i_\C(J)$:
	\begin{diagram}
		\I\times\C &\rTo^{J\times\C}&\C\times\C\\
		&\rdTo[snake=-1ex](1,2)<{1}
			\raise1ex\hbox{$\begin{array}c\Rightarrow\\[-5pt]\ll\end{array}$}%
			\ldTo[snake=1ex](1,2)>{P}\\
		&\C,
	\end{diagram}
	and the natural family of isomorphisms $m_{M,N}: i_\C(M)\tn i_\C(N)\to i_\C(M\tn N)$:
	\begin{diagram}
	&&\I\times\C &\rTo^{M\times\C} & \C\times\C & \rTo^P & \C\\
	&&\uTo<{\I\times P} & \sim & \uTo<{\C\times P} & \Arr\Searrow\aa &\uTo>P\\
	\I\times\I\times\C &\rTo_{\I\times N\times\C} & \I\times\C\times\C & \rTo_{M\times\C\times\C}
		&\C\times\C\times\C & \rTo_{P\times\C} & \C\times\C.
	\end{diagram}
\end{propn}
\begin{proof}
	First we must verify that the natural transformations $m_I$ and $m_{M,N}$ are indeed
	arrows of $\e(\C)$. For $m_I$, this means that
	\[
	\begin{diagram}
		&&\I\times\C^2 & \rTo^{\I\times P} & \I\times \C \\
		&\ldTo^{J\times\C^2} \\
		\C^3 & \ll\times\C & \dTo>1 && \dTo>1 \\
		&\rdTo_{P\times\C} \\
		&&\C^2 &\rTo_P & \C
	\end{diagram}
	\quad=\quad
	\begin{diagram}
		&&\I\times\C^2 & \rTo^{\I\times P} & \I\times \C \\
		&\ldTo^{J\times\C^2} &\sim & \ldTo^{J\times \C} \\
		\C^3 & \rTo^{\C\times P} & \C^2 &\ll& \dTo>1 \\
		&\rdTo_{P\times\C} & \aa & \rdTo_P \\
		&&\C^2 &\rTo_P & \C
	\end{diagram}
	\]
	which follows easily from (\ref{eq-lla}). For $m_{M,N}$, we have
	\[\hskip-4em
	\begin{diagram}[hug]
	&&\I^2\times\C \\
	&\ruTo[nohug]^{\I^2\times P} &\dTo>{\I\times N\times\C} \\
	\I^2\times\C^2 & \sim & \I\times\C^2 \\
	\dTo<{\I\times N \times \C^2} & \ruTo^{\I\times \C\times P} &&\rdTo[nohug]^{\I\times P} \\
	\I\times\C^3 && \I\times\aa &&\I\times\C \\
	\dTo<{M\times\C^3} & \rdTo^{\I\times P\times\C} &&\ruTo^{\I\times P} &\dTo>{M\times\C} \\
	\C^4 &\sim & \I\times\C^2 & \sim & \C^2 \\
	\dTo<{P\times\C^2} & \rdTo^{\C\times P\times\C} &\dTo[snake=1em]>{M\times\C^2} & \ruTo_{\C\times P} & \dTo>P \\
	\C^3 & \aa\times\C & \C^3 & \aa & \C \\
	&\rdTo[nohug]_{P\times\C} & \dTo[snake=1em]>{P\times\C} & \ruTo[nohug]_{P} \\
	&&\C^2
	\end{diagram}
	=
	\begin{diagram}[hug]
	&&\I^2\times\C \\
	&\ruTo[nohug]^{\I^2\times P} &\dTo>{\I\times N\times\C} \\
	\I^2\times\C^2 & \sim & \I\times\C^2 \\
	\dTo<{\I\times N \times \C^2} & \ruTo^{\I\times \C\times P} &\dTo[snake=-1em]>{M\times\C^2}&\rdTo[nohug]^{\I\times P} \\
	\I\times\C^3 &\sim& \C^3 &\sim&\I\times\C \\
	\dTo<{M\times\C^3} & \ruTo_{\C^2\times P} &&\rdTo_{\C\times P} &\dTo>{M\times\C} \\
	\C^4 && \C\times\aa && \C^2 \\
	\dTo<{P\times\C^2} & \rdTo^{\C\times P\times\C} && \ruTo_{\C\times P} & \dTo>P \\
	\C^3 & \aa\times\C & \C^3 & \aa & \C \\
	&\rdTo[nohug]_{P\times\C} & \dTo[snake=1em]>{P\times\C} & \ruTo[nohug]_{P} \\
	&&\C^2
	\end{diagram}
	\]
	\[
	=
	\begin{diagram}[hug]
	&&\I^2\times\C \\
	&\ruTo[nohug]^{\I^2\times P} &\dTo>{\I\times N\times\C} \\
	\I^2\times\C^2 & \sim & \I\times\C^2 \\
	\dTo<{\I\times N \times \C^2} & \ruTo^{\I\times \C\times P} &\dTo[snake=-1em]>{M\times\C^2}&\rdTo[nohug]^{\I\times P} \\
	\I\times\C^3 &\sim& \C^3 &\sim&\I\times\C \\
	\dTo<{M\times\C^3} & \ruTo^{\C^2\times P} &\dTo[snake=-1em]>{P\times\C}&\rdTo^{\C\times P} &\dTo>{M\times\C} \\
	\C^4 &\sim& \C^2 &\aa& \C^2 \\
	\dTo<{P\times\C^2} & \ruTo^{\C\times P} && \rdTo^{P} & \dTo>P \\
	\C^3 && \aa && \C \\
	&\rdTo[nohug]_{P\times\C} && \ruTo[nohug]_{P} \\
	&&\C^2
	\end{diagram}
	\]
	showing that $m_{M,N}$ is indeed a morphism of $\e(\C)$.
	
	We also need to check that these definitions make $i_\C$ into a monoidal
	functor, i.e. that the following diagrams commute:
	\[\begin{array}{cc}
		\begin{diagram}
			i_\C(M)\tn I & \rTo^{i_\C(M)\tn m_I} & i_\C(M)\tn i_\C(I) \\
			\dTo<{\rho_{i_\C(M)}} && \dTo>{m_{M,I}} \\
			i_\C(M) & \lTo_{i_\C(\rho_M)} & i_\C(M\tn I)
		\end{diagram}
		&
		\begin{diagram}
			I\tn i_\C(M) & \rTo^{m_I \tn i_\C(M)} & i_\C(I)\tn i_\C(M) \\
			\dTo<{\lambda_{i_\C(M)}} && \dTo>{m_{I,M}} \\
			i_\C(M) & \lTo_{i_\C(\lambda_M)} & i_\C(I\tn M)
		\end{diagram}
		\\
		\multicolumn 2c{\begin{diagram}
			& & i_\C(M\tn N)\tn i_\C(Q) & \rTo^{m_{M,N\tn Q}} & i_\C(M\tn(N\tn Q)) \\
			& \ruTo^{i_\C(M)\tn m_{N,Q}} \\
			i_\C(M)\tn i_\C(N)\tn i_\C(Q) &&&& \dTo>{i_\C(\alpha_{M,N,Q})} \\
			& \rdTo_{m_{M,N}\tn i_\C(Q)} \\
			& & i_\C(M)\tn i_\C(N\tn Q) & \rTo_{m_{M\tn N, Q}} & i_\C((M\tn N)\tn Q) 
		\end{diagram}}
	\end{array}\]
	%
	The first of these diagrams corresponds to the claim that
	\begin{diagram}
		& \rnode{IC}{\I\times\C} & \rTo^{M\times\C} & \C^2 & \rTo^P & \C \\
		& \uTo<{\I\times P} & \sim & \uTo<{\C\times P} & \aa & \uTo>P \\
		\I\times\ll& \I\times\C^2 & \rTo_{M\times\C^2} & \C^3 & \rTo_{P\times\C} & \rnode{CC}{\C^2} \\
		& \uTo<{\I\times J\times\C} & \sim & \uTo<{\C\times J\times\C} & \rr\times\C \\
		& \rnode{IIC}{\I^2\times\C} & \rTo_{M\times\I\times\C} & \rnode{CIC}{\C\times\I\times\C}
		\ncarc[arcangle=70,ncurv=1]{->}{IIC}{IC}\Aput{1}
		\nccurve[angleA=0,angleB=-90]{->}{CIC}{CC}\Bput{1}
	\end{diagram}
	is equal to the identity 2-cell, which follows from equation~(\ref{eq-lra}) and the naturality of
	the $\sim$~cells. The second corresponds to the claim that
	\begin{diagram}
		& \I^2\times\C \\
		& \dTo<{\I\times M\times\C} \\
		& \rnode{ICC}{\I\times\C^2} & \rTo^{\I\times P} & \rnode{IC}{\I\times\C} \\
		& \dTo<{J\times\C^2} & \sim & \dTo>{J\times\C} \\
		\ll\times\C & \C^3 & \rTo^{\C\times P} & \C^2 & \ll \\
		& \dTo<{P\times\C} & \aa & \dTo>{P} \\
		& \rnode{CC}{\C^2} & \rTo_P & \rnode{C}{\C}
		\ncarc[arcangle=-70,ncurv=1]{->}{ICC}{CC}\Bput{1}
		\ncarc[arcangle=70,ncurv=1]{->}{IC}{C}\Aput{1}
	\end{diagram}
	is the identity, which follows from (\ref{eq-lla}).
	
	The final, associativity, diagram
	amounts to
	\[\hskip-4em
	\begin{diagram}[hug]
	I^3\times\C&&\I^2\times\C \\
	\dTo<{I^2\times Q\times\C}&\ruTo^{\I^2\times P} &\dTo>{\I\times N\times\C} \\
	\I^2\times\C^2 & \sim & \I\times\C^2 \\
	\dTo<{\I\times N \times \C^2} & \ruTo^{\I\times \C\times P} &&\rdTo[nohug]^{\I\times P} \\
	\I\times\C^3 && \I\times\aa &&\I\times\C \\
	\dTo<{M\times\C^3} & \rdTo^{\I\times P\times\C} &&\ruTo^{\I\times P} &\dTo>{M\times\C} \\
	\C^4 &\sim & \I\times\C^2 & \sim & \C^2 \\
	\dTo<{P\times\C^2} & \rdTo^{\C\times P\times\C} &\dTo[snake=1em]>{M\times\C^2} & \ruTo_{\C\times P} & \dTo>P \\
	\C^3 & \aa\times\C & \C^3 & \aa & \C \\
	&\rdTo[nohug]_{P\times\C} & \dTo[snake=1em]>{P\times\C} & \ruTo[nohug]_{P} \\
	&&\C^2
	\end{diagram}
	=
	\begin{diagram}[hug]
	\I^3\times\C&&\I^2\times\C \\
	\dTo<{I^2\times Q\times\C}&\ruTo^{\I^2\times P} &\dTo>{\I\times N\times\C} \\
	\I^2\times\C^2 & \sim & \I\times\C^2 \\
	\dTo<{\I\times N \times \C^2} & \ruTo^{\I\times \C\times P} &\dTo[snake=-1em]>{M\times\C^2}&\rdTo[nohug]^{\I\times P} \\
	\I\times\C^3 &\sim& \C^3 &\sim&\I\times\C \\
	\dTo<{M\times\C^3} & \ruTo^{\C^2\times P} &\dTo[snake=-1em]>{P\times\C}&\rdTo^{\C\times P} &\dTo>{M\times\C} \\
	\C^4 &\sim& \C^2 &\aa& \C^2 \\
	\dTo<{P\times\C^2} & \ruTo^{\C\times P} && \rdTo^{P} & \dTo>P \\
	\C^3 && \aa && \C \\
	&\rdTo[nohug]_{P\times\C} && \ruTo[nohug]_{P} \\
	&&\C^2
	\end{diagram}
	\]
	which is true by the argument used above to show that $m_{M,N}$ is
	a morphism of $\e(\C)$.
\end{proof}
%
\begin{propn}\label{prop-i-equiv}
	$i_\C$ is a monoidal equivalence.
\end{propn}
\begin{proof}
	By Propositions~\ref{prop-i-ff} and~\ref{prop-i-monoidal}, it suffices to show that
	$i_\C$ is essentially surjective on objects. Therefore let $(F, \phi^F)$ be an
	object of $\e(\C)$. We claim that this object is isomorphic to $i_\C(FJ)$,
%	applied to
%	\[
%		I\rTo^J\C\rTo^F\C,
%	\]
	by the isomorphism
	\begin{diagram}
		\rnode{IC}{\I\times\C} & \rTo^{J\times\C} & \C\times\C & \rTo^{F\times\C} & \C\times\C\\
		&\ll & \dTo>P & \phi^F & \dTo>P \\
		&&\rnode{C}{\C} & \rTo_F & \C.
		\nccurve[angleA=270,angleB=180]{->}{IC}{C}\Bput{1}
	\end{diagram}
	It remains only to show that this isomorphism is indeed a morphism of $\e(\C)$,
	i.e. that
	\[
	\begin{diagram}
	\I\times\C^2 & \rTo^{\I\times P} & \rnode x{\I\times\C} & & \rnode y{\I\times\C} \\
	\dTo<1 && \dTo<1 & \ll & \dTo>{J\times\C} \\
	\C^2 & \rTo^P & \C & \lTo^P & \C^2 \\
	\dTo<{F\times\C} & \phi^F & \dTo<F & \phi^F & \dTo>{F\times\C} \\
	\C^2 & \rTo_{P} & \C & \lTo_P & \C^2
	\ncline[doubleline=true,doublesep=2pt]{-}xy
	\end{diagram}
	\quad=\quad
	\begin{diagram}
	\rnode x{\I\times\C^2} && \rnode y{\I\times\C^2} & \rTo^{\I\times P} & \I \times\C \\
	\dTo<1 & \ll\times\C & \dTo[snake=1em]>{J\times\C^2} & \sim & \dTo>{J\times\C} \\
	\C^2 & \lTo^{P\times\C} & \C^3 & \rTo^{\C\times P} & \C^2 \\
	\dTo<{F\times\C} & \phi^F\!\!\times\C & \dTo[snake=1em]>{F\times\C^2} & \sim & \dTo>{F\times\C} \\
	\C^2 & \lTo^{P\times\C} & \C^3 & \rTo^{\C\times P} & \C^2 \\
	&\rdTo_P & \raise.7em\hbox{$\aa$} & \ldTo_P \\
	&& \C
	\ncline[doubleline=true,doublesep=2pt]{-}xy
	\end{diagram}
	\]
	By the rotated version of the cube that defines the objects of $\e(\C)$, the
	right-hand side is equal to
	\begin{diagram}
	\rnode x{\I\times\C^2} && \rnode y{\I\times\C^2} & \rTo^{\I\times P} & \I \times\C \\
	\dTo<1 & \ll\times\C & \dTo[snake=1em]>{J\times\C^2} & \sim & \dTo>{J\times\C} \\
	\C^2 & \lTo^{P\times\C} & \C^3 & \rTo^{\C\times P} & \C^2 \\
	\dTo<{F\times\C} & \rdTo^P & \raise.7em\hbox{$\aa$} & \ldTo^P & \dTo>{F\times\C} \\
	\C^2 & \phi^F & \C & \phi^F & \C^2 \\
	&\rdTo_P & \dTo>F & \ldTo_P \\
	&& \C
	\ncline[doubleline=true,doublesep=2pt]{-}xy
	\end{diagram}
	which, by (\ref{eq-lla}), is equal to the left-hand side, as required.
\end{proof}

\iffalse
\section{Braided and symmetric pseudomonoids}
So far, we have assumed only that $\B$ is a monoidal bicategory. However, many of the
monoidal bicategories we're interested in -- especially $\Cat$ and $\Prof$ -- are
actually \emph{symmetric} monoidal. The definition of symmetric monoidal bicategory
is inevitably fairly complicated: we have a pseudonatural equivalence $\sigma$ with
object components \[
	\sigma_{A,B}: A\times B\to B\times A,
\]called the braiding, and invertible modifications
\begin{diagram}
	A\times B\times C\times D & \rTo^{A\times B \times\sigma_{C,D}} & A\times B\times D\times C \\
	\dTo<{\sigma_{A,B}\times C\times D} & \Arr\Nearrow{\varpi_{A,B,C,D}} & \dTo>{\sigma_{A,B\times D}\times C} \\
	B\times A\times C\times D & \rTo_{B\times\sigma_{A\times C,D}} & B\times D\times A\times C
\end{diagram}
and \begin{diagram}[h=1.5em]
	A\times B \\
	&\rdTo^{\sigma_{A,B}} \\
	\dTo<1 & \hbox to0pt{\hss$\mathop\Rightarrow\limits_{\displaystyle\nu_{A,B}}$} & B\times A \\
	&\ldTo_{\sigma_{B,A}} \\
	A\times B
\end{diagram}
subject to various axioms: the reader is referred to \citet[sections~4 and~5]{MonBicat}
for the details of these. Note that this definition takes $\B$ to be a Gray monoid.
Generalising to an arbitrary monoidal bicategory $\B$ is routine in principle, though
in practice very complicated.

Now, a braided pseudomonoid in the symmetric Gray monoid $\B$ is a pseudomonoid
$\C$ equipped with an invertible 2-cell $\ss$:
\begin{diagram}
	\rnode{left}{\C\times\C} & \rTo^{\sigma_{\C,\C}} & \rnode{right}{\C\times\C} \\
	&\raise1em\hbox{$\mathop\Rightarrow\limits_{\displaystyle\ss}$} \\
	&\rnode{C}{\C}
	\nccurve[angleA=-90,angleB=135]{->}{left}{C}\Bput{P}
	\nccurve[angleA=-90,angleB=45]{->}{right}{C}\Aput{P}
\end{diagram}
\fi

\section{The promonoidal unit}
In the case $\B=\mathbf{Cat}$, the embedding theorem shows that every monoidal category is
monoidally equivalent to a \emph{strict} monoidal category. In the case $\B=\mathbf{Prof}$,
which is the case we're interested in here, the embedding allows us to construct a canonical
representation for the promonoidal unit. (This turns out to be particularly powerful in the
braided, or symmetric, case.)

In summary, the story goes as follows.
It is a familiar fact that, in a semigroup, all units must be equal: if $i$ and $j$ are
both units, then $i=ij=j$. The existence or otherwise of a unit is really a \emph{property}
of a semigroup; there is no choice of how to define the unit. By a similar argument, in a
pseudomonoid (or `pseudosemigroup', a pseudomonoid without the unit structure) all units
are isomorphic. Being good category theorists, we don't care about the difference
between isomorphic structures; so we have no real choice in how to define the unit of
a pseudomonoid: given the tensor structure ($P$ and $\aa$), either it is possible to
define a unit, or it isn't. We might say that the unit is `essentially property-like'
\citep[cf.][]{proplike}. This point may seem rather irrelevant, since the easiest way
to demonstrate the existence of a unit is often to exhibit one. However, it turns out
that in the case of promonoidal categories that is not necessarily true. There is a
canonical representative for the unit -- the functor denoted $\Lin$ below -- defined
solely in terms of the tensor structure ($P$ and $\alpha$) that is isomorphic to the
unit whenever there is a unit to be isomorphic to. Furthermore, in the case of braided
or symmetric promonoidal categories, we can define a simple test of whether or not
there is a unit. So we may \emph{define} a braided promonoidal category purely in
terms of the tensor, associator and braiding, subject to a condition that determines
the existence of a unit. Should an actual unit be required, we are free to use the
canonical unit $\Lin$.

Now for the details. Let $\C$ be a promonoidal category. The monoidal category $\mathbf{Prof}(1, \C)$
is $[\C, \Set]$, equipped with Day's convolution tensor. The monoidal category
$\e(\C)$ does not really have a simpler description than the general one given above:
an object is a profunctor $\C\rPro\C$ together with a natural transformation
\begin{diagram}
	\C\times\C & \rPro^P & \C\\
	\dPro<{F\times\C}&\Arr\Swarrow{\phi^F}&\dTo>F\\
	\C\times\C&\rPro_P & \C
\end{diagram}
So $\phi^F$ is a natural isomorphism with components
\[
	\phi^F_{A,B,C}: \int^X P(A,B;X)\times F(X;C) \to \int^X F(A;X)\times P(X,B;C).
\]
Now let $A\in\C$. We have the sequence of natural isomorphisms
\[\begin{array}{rclp{5cm}}
	JA &\cong& [\C,\Set](\C(A,-), J) & {by Yoneda} \\
	&\cong& \e(\C)(i_\C(\C(A,-)), i_\C(J)) & {since $i_\C$ is full and faithful} \\
	&\cong& \e(\C)(i_\C(\C(A,-)), I) & {applying $m_I$}
\end{array}\]
Define the functor $\Lin: \C\to\Set$ as $Lin(A) := \e(\C)(i_\C(\C(A,-)), I)$.
Thus we have exhibited a natural isomorphism $\theta: J \To \Lin$.
An element of the set $\Lin(A)$ is a natural transformation $\gamma$ with components
\[
	\gamma_{X,Y}: P(A,X;Y) \to \C(X,Y)
\]
such that the diagram
\begin{equation}\label{diag-linel}
\begin{diagram}
	\int^X P(L,M;X)\times P(A,X;N) & \rTo^{\int^X P(L,M;X)\times\gamma_{X,N}}
		& \int^X P(L,M;X)\times\C(X,N) \\
	&&\dTo>\cong \\
	\dTo<{\alpha_{A,L,M}} && P(L,M;N) \\
	&&\uTo>\cong \\
	\int^X P(A,L;X)\times P(X,M;N) & \rTo_{\int^X \gamma_{L,X}\times P(X,M;N)}
		& \int^X\C(L,X)\times P(X,M;N)
\end{diagram}
\hskip-2em % Tuck the equation number in
\end{equation}
commutes for all $L$, $M$ and $N$ in $\C$. We shall refer to the elements
of this set as the \defns{linear element} of $A$.

For the purposes of this thesis, of course, we're particularly interested
in the case where the profunctor $P$ is represented by a functor $\tn: \C\times\C\to\C$.
In this case, a linear element is a natural transformation with components
\[
	\gamma_{X,Y}: \C(A\tn X, Y) \to \C(X,Y)
\]
which, by Yoneda, can be represented by a natural transformation with components
\[
	\gamma_X: X \to A\tn X.
\]
With this representation, the condition boils down to the requirement that
the diagram
\begin{diagram}
	A &\rTo^{\gamma_Y} & A\tn Y \\
	\dTo<{\gamma_{X\tn Y}} && \dTo>{\gamma_X\tn Y} \\
	A\tn(X\tn Y) & \rTo_{\alpha_{A,X,Y}} & (A\tn X)\tn Y
\end{diagram}
should commute, for all $X$, $Y\in\C$.

Since an element of $\Lin(A)$ is a natural transformation 
\[
	\gamma_{X,Y}: P(A,X;Y) \to \C(X,Y)
\]
for every $A$ there is an obvious natural transformation $\lambda'^A$
with components
\[
	\lambda'^A_{X,Y}: Lin(A)\times P(A,X;Y) \to \C(X,Y),
\]
dinatural in $A$.

\begin{propn}
	The diagram
	\begin{diagram}[h=2em]
		JA\times P(A,X;Y) \\
		&\rdTo^{\lambda^A_{X,Y}}\\
		\dTo<{\theta_A\times P(A,X;Y)} && \C(X,Y) \\
		&\ruTo_{\lambda'^A_{X,Y}} \\
		\Lin A \times P(A,X;Y)
	\end{diagram}
	commutes, for all $A$, $X$ and $Y\in\C$.
\end{propn}
\begin{proof}
	This is a direct consequence of the definition of $\theta$: any
	apparent complexity here is notational rather than mathematical.
	First we shall calculate the effect of $\theta_A: JA\to\Lin A$
	on an element $\e\in JA$. We'll consider in sequence the three
	isomorphisms that define $\theta$. The first takes $e$ to the
	natural transformation with $X$-component
	\[
		(f:A\to X) \mapsto J(f)(e);
	\]
	this must then be mapped by the second to a natural transformation
	\[
		P(A,X;Y) \to \int^Z JZ\times P(Z,X;Y)
	\]
	natural in $X$ and $Y$.
	The elements of $\int^Z JZ\times P(Z,X;Y)$ are equivalence classes
	of pairs $\langle j, p\rangle$, with $j\in JZ$ and $p\in P(Z,X;Y)$
	for some $Z\in\C$. Our element is mapped to the $A$-indexed natural
	family of functions
	\[
		(p\in P(A,X;Y)) \mapsto [\langle e,p\rangle].
	\]
	where $[\langle e,p\rangle]$ denotes the equivalence class containing
	$\langle e,p\rangle$. The final natural isomorphism takes this element
	to the $A$-indexed natural family of functions
	\[
		(p\in P(A,X;Y)) \mapsto \lambda_{X,Y}([\langle e,p\rangle]).
	\]
	Now it's easy to show that the diagram commutes: let $\langle e,p\rangle$
	be an element of $JA\times P(A,X;Y)$. The vertical arrow maps it to
	the pair $\langle f, p\rangle$, where $f$ is the linear element
	displayed above. $\lambda'$ then maps this to $\lambda_{X,Y}^A(e,p)$,
	as required.
\end{proof}
%
There is an apparent asymmetry here: although it was easy to define
$\lambda'$, there is no obvious way to define a corresponding $\rho'$
-- unless of course our promonoidal category is braided, of which more
below. This asymmetry derives from the asymmetric definition of $\e(\C)$,
and of course it would be possible to define a dual version $\mbox{\textbf{\v e}}(\C)$,
which would also be monoidally isomorphic to $\Prof(1,\C)$, hence to $\e(\C)$.
Using this, we could define a `co-linear elements' functor $\Lin': \C\to\Set$,
with a canonical natural isomorphism
\[
	\rho'^A_{X,Y}: \Lin'A\times P(X,A;Y) \to \C(X,Y).
\]
However, in general there is no canonical natural isomorphism between
$\Lin$ and $\Lin'$. In the braided case, there is. So in that case we
may simply take $\Lin$ (or equivalently $\Lin'$) to be the unit, which
role it is able to fulfil if and only if $\lambda'$ (equivalently $\rho'$)
is invertible. More formally we have:
\begin{propn}
	Let $\C$ be a category, and $P: \C\times\C\rPro\C$ a profunctor.
	Let $\alpha$ be an associator satisfying the pentagon
	condition, and let $\sigma$ be a braiding satisfying the
	hexagon conditions. There exists a unit $J:1\rPro\C$ (with
	coherent unit isomorphisms $\lambda$ and $\rho$) if and only
	if the natural transformation
	\[
		\lambda': \int^A \Lin A\times P(A,X;Y) \to \C(X,Y),
	\]
	defined above, is invertible.
\end{propn}
\begin{proof}
	We have already established the `only if' direction, so let
	$\C$, $P$, $\alpha$ and $\sigma$ be given, and define $\Lin: \C\to\Set$
	and $\lambda'$. Suppose that $\lambda'$ is invertible.
	
	By a general fact about braided pseudomonoids that I haven't written
	up yet, it suffices to show that
	\[
		\begin{diagram}
			1\times\C^2 & \rPro^{\Lin\times\C^2} & \C^3 \\
			\dPro<{1\times P} & \sim & \dPro>{\C\times P} \\
			\rnode{IC}{1\times\C} & \rPro^{\Lin\times\C} & \C^2 \\
			&\lambda' & \dPro>P \\
			&&\rnode{C}{\C}
			\nccurve[angleA=270,angleB=180,ncurv=1]{->}{IC}{C}\Bput1
		\end{diagram}
		=
		\begin{diagram}
			\rnode{ICC}{1\times\C^2} & \rPro^{\Lin\times\C^2} & \C^3 & \rPro^{\C\times P} & \C^2 \\
			&\lambda'\times\C &\dPro[snake=1em]>{P\times\C} & \alpha & \dPro>P \\
			&&\rnode{CC}{\C^2} & \rPro_P & \C
			\nccurve[angleA=270,angleB=180,ncurv=1]{->}{ICC}{CC}\Bput1
		\end{diagram}
	\]
	Concretely, this amounts to showing that the diagram
	\begin{diagram}[labelstyle=\scriptstyle]
		\int^{A,X} P(A,X;N)\times\Lin A\times P(L,M;X)
		& \rTo^{\int^X \lambda'_{X,N}\times P(L,M;X)}
		& \int^X \C(X,N)\times P(L,M;X) \\
		&&\dTo>\cong \\
		\dTo<{\int^A\Lin A\times\alpha_{A,L,M,N}}
		&& P(L,M;N) \\
		&&\uTo>\cong \\
		\int^{A,X}\Lin A\times P(A,L;X)\times P(X,M;N)
		& \rTo_{\int^X\lambda'_{L,X}\times P(X,M;N)}
		& \int^X\C(L,X)\times P(X,M;N)
	\end{diagram}
	commutes, which is an immediate consequence of (\ref{diag-linel}).
\end{proof}
%It has been observed by Gordon and Power that
%\citeauthor{BTC}'s theorem is essentially an application of the bicategorical Yoneda
%lemma \citep{FIB} to the suspension of a monoidal category (i.e. to a monoidal
%category \emph{qua} one-object bicategory). There is a general notion of
%\emph{enriched bicategory}, whose hom-objects are drawn from a monoidal bicategory.
%Given this idea, the suspension of a promonoidal category may be defined: it is
%a $\Prof$-enriched category with one object. It seems almost inevitable that something
%like the following must be true.
%\begin{conj}[weak Yoneda for enriched bicategories]
%	For any $\V$-bicategory $\B$, the obvious `Yoneda' pseudo-functor
%	\[
%		\B_0 \to [\B\op,\V]_0
%	\]
%	is locally an equivalence.
%\end{conj}
%The only seeming difficulty with proving this conjecture is one that is all too familiar in
%higher-dimensional category theory: the notions involved require a vast number of
%coherence conditions to be satisfied. (Even the general definition of monoidal bicategory
%is not for the faint of heart.) Therefore we have satisfied ourselves, for the moment, with
%proving just the needed case of this general conjecture. Our theorem corresponds (very
%nearly) to the case where $\V=\Prof$ and $\B$ has just one object.

%One final remark is in order. As indicated by the vague qualifier `very nearly' in the preceding
%sentence, our theorem as stated is a little stronger than would be obtained from the Yoneda
%conjecture. Specialising the general conjecture yields a construction that is identical to
%our $\e(\C)$ \emph{except} that the definition of object in $\e(\C)$ is subject to an extra condition,
%involving the unit of $\C$. This matters to us, since it is significant in our application that the
%definition of $\e(\C)$ does not involve the unit of $\C$ in any way.



\bibliography{cs}
\end{document}
  