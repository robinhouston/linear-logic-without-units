 \begin{propn}\label{prop-phi}
	There is a natural isomorphism $\phi: \C(A, B^*) \cong \C(B, A^*)$.
\end{propn}
\begin{proof}
	Given $f: A\to B^*$, define $\phi(f)$ to be
	\begin{center}\cdiag{d-phif}\end{center}
	To see that this assignment is natural in $A$ and $B$, suppose we have	$g: A_1\to A_2$ and $h: B_1 \to B_2$. Then naturality amounts to the
	claim that, for every $f: A_2\to B_2^*$,
	\vskip\abovedisplayskip
	\hbox to \columnwidth{\hss%
		$\cdiag{d-phi-nat-lhs} = \cdiag{d-phi-nat-rhs}$
	\hss}
	\vskip\belowdisplayskip\noindent which is clearly true, by the naturality and
	dinaturality of $\e$ and $\eta$.
	
	To see that $\phi$ is invertible, we shall show that it is its own inverse.%
	\footnote{%
		In the braided case, $\phi^{-1}$ differs from $\phi$ by the direction
		of the braiding, as the diagrams show.%
	}
	For any $f: A\to B^*$, $\phi(\phi(f))$ is
	the morphism
	\vskip\abovedisplayskip
	\hbox to \columnwidth{\hss%
		\cdiag{d-phiphi-rhs}
	\hss}
	See Fig.~\ref{fig-phiphi} for a proof that $\phi(\phi(f)) = f$. Notice that the fourth
	step of the proof uses the $f=1$ case of Lemma~\ref{sec}.
\end{proof}
\begin{figure}
\hbox to \columnwidth{\hss$\begin{array}{rlcl}
	& \cdiag{d-phiphi-rhs} &=&  \cdiag{d-phiphi1}\\[4em]
	=& \cdiag{d-phiphi2} &=&  \cdiag{d-phiphi3}\\[4em]
	=& \cdiag{d-phiphi4} &=&  \cdiag{d-phiphi5}\\[4em]
	=& \cdiag{d-phiphi-lhs}
\end{array}$\hss}
\caption{Proof that $\phi(\phi(f)) = f$ (Prop.~\ref{prop-phi})}\label{fig-phiphi}
\end{figure}
