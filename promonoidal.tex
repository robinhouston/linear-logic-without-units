%!TEX TS-program = latex
\documentclass{robinthesis}

\begin{thesischapter}{Promon}{Linear Logic without Units}
In this, the title chapter, we at last address the question
that we set out to answer at the outset. We consider the notion
of model described in the introduction, and bring to bear all the
machinery of earlier chapters to find a simple formulation of it.

\section{Promonoidal categories}
As we have foreshadowed, we are mainly interested in the monoidal
bicategory $\Prof$ of categories and profunctors. Its objects are
categories, and a 1-cell $\C\rPro\D$ is a profunctor from $\C$ to $\D$;
which is to say, a functor $\C\op\times\D\to\Set$. (Note that some
authors use the converse direction, so what we call a profunctor
from $\C$ to $\D$ they would call a profunctor from $\D$ to $\C$.
There are good arguments in favour of both choices, and we have
simply chosen the one that is easiest for our purposes.)

Profunctors are composed by a convolution operation, where the
composite
\[
	\CB \rPro^{F} \C \rPro^{G} \D
\]
is defined by the coend formula
\[
	GF(B,D) = \int^{C} F(B,C) \times G(C,D).
\]
The 2-cells of $\Prof$ are just ordinary natural transformations.
It is easy enough to check that $\Prof$ is a bicategory. The
monoidal structure is more subtle, and although it has long been
assumed that $\Prof$ is a symmetric monoidal bicategory, the first
rigorous published proof was given very recently by \citet{CartesianBicatsII}.
The monoidal structure is given by the ordinary cartesian product
of categories.

A \emph{promonoidal category} is a pseudomonoid in $\Prof$. More
concretely, a promonoidal category consists of a category $\C$
equipped with profunctors
\[
	P: \C\times\C\rPro\C
\]
and
\[
	J: 1\rPro\C
\]
together with natural isomorphisms with components
\[
	\aa_{A,B,C,D}: \int^{X} P(A,X;D)\times P(B,C;X)
		\to \int^{X} P(A,B;X)\times P(X,C;D),
\]
\[
	\ll_{A,B}: \int^{X} J(X)\times P(X,A;B) \to \C(A,B),
\]
and
\[
	\rr_{A,B}: \int^{X} J(X)\times P(A,X;B) \to \C(A,B),
\]
satisfying the pseudomonoid axioms.

A functor $F: \C\to \D$ induces profunctors $\C(-,F=): \C\rPro \D$
and $\C(F-,=): \D\rPro\C$. These extend in an obvious way to
bijective-on-objects embeddings
\[
	\Cat \to\Prof
\]
and \[
	\Cat\op\to\Prof,
\]
which (by Yoneda) are locally full and faithful. In particular,
the covariant embedding gives every monoidal category the
structure of a promonoidal category in a natural way. Concretely,
one takes $P(A,B;C) = \C(A\tn B,C)$ and $J(A) = \C(I,A)$.

\subsection{Braiding and symmetry}
Clearly a braided promonoidal category must be a braided pseudomonoid
in $\Prof$: that is, a promonoidal category $\C$ additionally equipped
with a natural isomorphism with components
\[
	\sigma_{A,B,C}: P(A,B;C) \to P(B,A;C)
\]
satisfying the braiding axioms.

We have not given a general definition of symmetric pseudomonoid,
because to do so would require venturing into the deep waters of
\emph{sylleptic} monoidal bicategories. However we can easily say
what it means for a braided promonoidal category to be symmetric;
we simply demand that\[
	\sigma_{A,B,C} = \sigma_{B,A,C}^{-1}.
\]
It is clear that, in the symmetric case, each of the two braiding
axioms implies the other.

\section{Modelling linear logic without units}
As explained in the introduction, the appropriate structures for
modelling the unitless fragment of multiplicative intuitionistic
linear logic are promonoidal categories where the multiplication
$P$ is represented by a functor, but the unit $J$ generally is not.
\begin{definition}
	A semi symmetric monoidal category (semi SMC) consists
	of a category $\C$ equipped with functors
	\[
		\tn: \C\times\C\to \C
	\]
	and
	\[
		J:\C\to\Set
	\]
	and natural isomorphisms with components
	\[
		\alpha_{A,B,C}: A\tn(B\tn C),
	\]
	\[
		\sigma_{A,B}: A\tn B\to B\tn A,
	\]
	and
	\[
		\lambda_{A,B}: \int^{X} J(X)\times\C(X\tn A,B) \to \C(A,B)
	\]
	satisfying the axioms for a symmetric promonoidal category.
	The associativity (pentagon) and symmetry (hexagon) axioms do not involve the unit,
	so are the same as for an ordinary symmetric monoidal category.
	In detail, the diagrams
	\begin{diagram}
	 A\tensor \bigl(B\tensor (C\tensor D)\bigr)
	 &\rTo^\alpha&(A\tensor B)\tensor (C\tensor D)
	 &\rTo^\alpha & \bigl((A\tensor B)\tensor C\bigl)\tensor D
	 \\
	 \dTo<{A\tn\alpha} &&&& \uTo>{\alpha\tn D}
	 \\
	 A\tensor\big((B\tensor C)\tensor D\big)
	 && \rTo_\alpha && \bigl(A\tensor(B\tensor C)\bigr)\tensor D
	\end{diagram}
	and
	\begin{diagram}
	 A\tn(B\tn C) & \rTo^{\alpha} & (A\tn B)\tn C & \rTo^{\sigma} & C\tn(A\tn B) \\
	 \dTo<{A\tn\sigma} &&&& \dTo>{\alpha} \\
	 A\tn(C\tn B) & \rTo_{\alpha} & (A\tn C)\tn B  & \rTo_{\sigma\tn B} & (C\tn A)\tn B
	\end{diagram}
	must commute. The unit axiom says that
	the diagram
	\begin{diagram}%\dlabel{bigone}
        \int^X J(X)\times\C((X\tn B)\tn C, Z) & \rTo^{\int^X J(X)\times\C(\alfa XBC, Z)}
                & \int^X J(X)\times\C(X\tn (B\tn C), Z)
        \\
        \dTo<{\cong}
        \\
        \int^{X,Y} J(X)\times\C(X\tn B, Y)\times \C(Y\tn C, Z)&
                &\dTo>{\lambda_{B\tn C,Z}}
        \\
        \dTo<{\int^Y\lambda_{B,Y}\times\C(Y\tn C,Z)}
        \\
        \int^Y\C(B,Y)\times\C(Y\tn C,Z) &\rTo_\cong&\C(B\tn C,Z)
	\end{diagram}
	must commute for all $B$,$C$,$Z\in\C$.
\end{definition}
\begin{definition}
	A semi symmetric monoidal closed category (semi SMCC) is a semi SMC
	$\C$ together with a functor
	\[
		\lolli: \C\op\times\C\to\C
	\]
	and a natural isomorphism with components
	\[
		\curry_{A,B,C}: \C(A\tn B,C)\to \C(A,B\lolli C).
	\]
\end{definition}

The purpose of this chapter is to find a simpler formulation of
the notion of semi SMCC. We should like to formulate the notion
in a way that does not involve coends, for one thing. In fact we
give two such formulations, which are superficially very different.
Before delving into the details, we give a summary of our two
presentations.

\subsection{The presentation via linear elements}\label{s-linel}
There is an obvious notion of \emph{unitless SMCC}, essentially
obtained  taking the ordinary definition of SMCC
and erasing those parts that mention the unit object.
In the introduction, we saw that unitless SMCCs are not
adequate to model the unitless fragment of IMLL, because of
the need to represent sequents whose left-hand side is empty.
However, it turns out that the gap between unitless SMCCs and
semi SMCCs is smaller than it seems: surprisingly, no additional
structure is actually required! In fact a semi SMCC can be described
as a unitless SMCC satisfying a certain \emph{property}. Here we
briefly describe the property in question.

Given a unitless SMCC $\C$, and an object $A\in\C$, let us define
a \emph{linear element} of $A$ to be a natural transformation with
components
\[
	\gamma_{X}: X \to A\tn X
\]
such that, for all $X$ and $Y\in\C$, the diagram
\begin{diagram}
	& X\tn Y \\
	\ldTo(1,2)<{\gamma_{X\tn Y}} && \rdTo(1,2)>{\gamma_{X}\tn Y} \\
	A\tn(X\tn Y) & \rTo_{\alpha_{A,X,Y}} & (A\tn X)\tn Y
\end{diagram}
% \begin{diagram}
% 	A &\rTo^{\gamma_Y} & A\tn Y \\
% 	\dTo<{\gamma_{X\tn Y}} && \dTo>{\gamma_X\tn Y} \\
% 	A\tn(X\tn Y) & \rTo_{\alpha_{A,X,Y}} & (A\tn X)\tn Y
% \end{diagram}
commutes. It's easy to see that, in an ordinary monoidal category,
the linear elements of $A$ are in bijective correspondence with
morphisms $I\to A$. Now, there is an obvious functor
\[
	\Lin: \C\to \Set
\]
that takes each object $A\in\C$ to the set of linear elements on
$\C$. (In an ordinary monoidal category, it is isomorphic to the
hom functor $\C(I,-)$.) Furthermore, there is a canonical natural
transformation with components
\[
	\ell_{A,B}: \Lin(A\lolli B) \to \C(A,B);
\]
if $\gamma$ is a linear element of $A\lolli B$, then
$\ell_{A,B}(\gamma)$ is just the composite
\[
	A \rTo^{\gamma_{A}} (A\lolli B)\tn A \rTo^{\ev^{A}_{B}} B
\]
where $\ev^{A}$ is the counit of the adjunction $-\tn A \dashv A\lolli-$.

Our additional condition, then, is as follows: a unitless SMCC
$\C$ can be regarded as a semi SMCC just when the natural transformation
$\ell$ is invertible, so that every arrow $A\to B$ comes from a
unique linear element of $A\lolli B$. It is easy to see that this
condition is satisfied when $\C$ has a unit object, and in
Section~\ref{s-promon-unit} we shall prove that in fact it
is satisfied if and only if $\C$ has a promonoidal unit.

\subsection{The `$\psi$' presentation}\label{s-psi-presentation}
Our second presentation is given in terms of structure rather
than properties. We show that to give a semi SMCC is to give
a category $\C$ equipped with an associative, symmetric\footnote{
      \emph{I.e.}, equipped with the standard associativity and symmetry 
      isomorphisms and coherence laws of a symmetric monoidal category.}
functor
$\tn:\C\times\C\to\C$, a functor $\lolli:\C\op\times\C\to\C$
%       (\emph{internal hom}) 
through which hom factors up to isomorphism,
\begin{diagram}
       \C\op\times\C &\rTo^\lolli& \C\\
       &\rdTo_{\mathrm{hom}}\raise1em\hbox{\qquad$\cong$}&\dTo.>{\exists J}\\
       &&\Set
\end{diagram}
and a natural isomorphism
\[
       \psi_{A,B,C}\;\;:\;\; (A\tn B)\lolli C \;\;\to\;\;
       A\lolli(B\lolli C)
\]
such that the diagram
\begin{diagram}[h=1.5em] \bigl(A\tn(B\tn C)\bigr)\lolli D
&\rTo^{\alpha_{A,B,C}\lolli D} &\bigl((A\tn B)\tn
C\bigr)\lolli D\\ &&\dTo>{\psi_{A\tn B,C,D}}\\
\dTo<{\psi_{A,B\tn C,D}}&&(A\tn B)\lolli(C\lolli D)\\
&&\dTo>{\psi_{A,B,C\lolli D}}\\ A\lolli\bigl((B\tn C)\lolli
D\bigr)&\rTo_{A\lolli\psi_{B,C,D}}
&A\lolli\bigl(B\lolli(C\lolli D)\bigr) \end{diagram} commutes.

%XXXX - incoherent presentation
% One can go further: although the condition above is not
% redundant, it is nevertheless the case that, given a natural isomorphism
% $\psi$ not necessarily satisfying the condition, one can find another
% natural isomorphism $\psi'$ that does satisfy it. Therefore, to
% give a semi SMCC it is enough to give a unitless SMCC $\C$ equipped with
% a functor $J:\C\to\Set$ such that $J(A\lolli B)$ is naturally isomorphic
% to $\C(A,B)$.

\section{The promonoidal unit}\label{s-promon-unit}
Although we are primarily interested in semi SMCCs, the basic argument
here works in any braided promonoidal category, so we shall work at
that level of generality and specialise to our intended application
at the end.

The essential tool here is the Cayley Theorem for pseudomonoids:
in the case $\B=\mathbf{Cat}$, it shows that every monoidal category is
monoidally equivalent to a \emph{strict} monoidal category. In the case $\B=\mathbf{Prof}$,
which is the case we're interested in here, it allows us to construct a canonical
representation for the promonoidal unit. (This turns out to be particularly powerful in the
braided, or symmetric, case.)

In summary, the story goes as follows.
It is a familiar fact that, in a semigroup, all units must be equal: if $i$ and $j$ are
both units, then $i=ij=j$. The existence or otherwise of a unit is really a \emph{property}
of a semigroup; there is no choice of how to define the unit. By a similar argument, in a
pseudomonoid (or `pseudosemigroup', a pseudomonoid without the unit structure) all units
are isomorphic. Being good category theorists, we don't care about the difference
between isomorphic structures; so we have no real choice in how to define the unit of
a pseudomonoid: given the tensor structure ($P$ and $\aa$), either it is possible to
define a unit, or it isn't. We might say that the unit is `essentially property-like'
\citep[cf.][]{proplike}. This point may seem rather irrelevant, since the easiest way
to demonstrate the existence of a unit is often to exhibit one. However, it turns out
that in the case of promonoidal categories that is not necessarily true. There is a
canonical representative for the unit -- the functor here denoted $\Lin$ -- defined
solely in terms of the tensor structure ($P$ and $\alpha$) that is isomorphic to the
unit whenever there is a unit to be isomorphic to. Furthermore, in the case of braided
or symmetric promonoidal categories, we can define a simple test of whether or not
there is a unit. So we may \emph{define} a braided promonoidal category purely in
terms of the tensor, associator and braiding, subject to a condition that determines
the existence of a unit. Should an actual unit be required, we are free to use the
canonical unit $\Lin$.

Now for the details. Let $\C$ be a promonoidal category. The monoidal category $\mathbf{Prof}(1, \C)$
is $[\C, \Set]$, equipped with Day's convolution tensor. The monoidal category
$\Mod_{\C}(\C,\C)$ does not really have a simpler description than the general one given above:
an object is a profunctor $\C\rPro\C$ together with a natural transformation
\begin{diagram}
	\C\times\C & \rPro^P & \C\\
	\dPro<{F\times\C}&\Arr\Swarrow{\phi^F}&\dTo>F\\
	\C\times\C&\rPro_P & \C
\end{diagram}
satisfying the conditions of Definition~\chref{Cayley}{def-module-map}.
So $\phi^F$ is a natural isomorphism with components
\[
	\phi^F_{A,B,C}: \int^X P(A,B;X)\times F(X;C) \to \int^X F(A;X)\times P(X,B;C).
\]
Now let us write $\E$ as an abbreviation for the monoidal category $\Mod_{\C}(\C,\C)$,
and let $A\in\C$. We have a sequence of natural isomorphisms
\[\begin{array}{rclp{5cm}}
	JA &\cong& [\C,\Set](\C(A,-), J) & {by Yoneda} \\
	&\cong& \E(\phi'_\C(\C(A,-)), \phi'_\C(J)) & {since $i_\C$ is full and faithful} \\
	&\cong& \E(\phi'_\C(\C(A,-)), I) & {applying $m_I$}
\end{array}\]
where $\phi'_{\C}$ is the monoidal equivalence $[\C,\Set]\to\E$
described by Proposition~\chref{Cayley}{prop-psmoncayley}.
If we define the functor $\Lin: \C\to\Set$ as $\Lin(A) := \E(\phi'_\C(\C(A,-)), I)$,
then we have exhibited a natural isomorphism $\theta: J \To \Lin$.
An element of the set $\Lin(A)$ is a natural transformation $\gamma$ with components
\[
	\gamma_{X,Y}: P(A,X;Y) \to \C(X,Y)
\]
such that the diagram
\begin{equation}\label{diag-linel}
\begin{diagram}[h=2em]
	\int^X P(L,M;X)\times P(A,X;N) & \rTo^{\int^X P(L,M;X)\times\gamma_{X,N}}
		& \int^X P(L,M;X)\times\C(X,N) \\
	&&\dTo>\cong \\
	\dTo<{\alpha_{A,L,M}} && P(L,M;N) \\
	&&\uTo>\cong \\
	\int^X P(A,L;X)\times P(X,M;N) & \rTo_{\int^X \gamma_{L,X}\times P(X,M;N)}
		& \int^X\C(L,X)\times P(X,M;N)
\end{diagram}
\hskip-2em % Tuck the equation number in
\end{equation}
commutes for all $L$, $M$ and $N$ in $\C$. We shall refer to the elements
of this set as the \defns{linear element} of $A$.

For the purposes of this thesis, of course, we're particularly interested
in the case where the profunctor $P$ is represented by a functor $\tn: \C\times\C\to\C$.
In this case, a linear element is a natural transformation with components
\[
	\gamma_{X,Y}: \C(A\tn X, Y) \to \C(X,Y)
\]
which, by Yoneda, can be represented by a natural transformation with components
\[
	\gamma_X: X \to A\tn X.
\]
With this representation, the condition boils down to the requirement that
the diagram
\begin{diagram}
	& X\tn Y \\
	\ldTo(1,2)<{\gamma_{X\tn Y}} && \rdTo(1,2)>{\gamma_{X}\tn Y} \\
	A\tn(X\tn Y) & \rTo_{\alpha_{A,X,Y}} & (A\tn X)\tn Y
\end{diagram}
should commute, for all $X$, $Y\in\C$.

Since an element of $\Lin(A)$ is a natural transformation 
\[
	\gamma_{X,Y}: P(A,X;Y) \to \C(X,Y),
\]
for every $A$ there is an obvious natural transformation $\lambda'^A$
with components
\[
	\lambda'^A_{X,Y}: \Lin(A)\times P(A,X;Y) \to \C(X,Y),
\]
dinatural in $A$.

\begin{propn}
	The diagram
	\begin{diagram}[h=2em]
		JA\times P(A,X;Y) \\
		&\rdTo^{\lambda^A_{X,Y}}\\
		\dTo<{\theta_A\times P(A,X;Y)} && \C(X,Y) \\
		&\ruTo_{\lambda'^A_{X,Y}} \\
		\Lin A \times P(A,X;Y)
	\end{diagram}
	commutes, for all $A$, $X$ and $Y\in\C$.
\end{propn}
\begin{proof}
	This is a direct consequence of the definition of $\theta$: any
	apparent complexity here is notational rather than mathematical.
	First we shall calculate the effect of $\theta_A: JA\to\Lin A$
	on an element $e\in JA$. We'll consider in sequence the three
	isomorphisms that define $\theta$. The first takes $e$ to the
	natural transformation with $X$-component
	\[
		(f:A\to X) \mapsto J(f)(e);
	\]
	this must then be mapped by the second to a natural transformation
	\[
		P(A,X;Y) \to \int^Z JZ\times P(Z,X;Y)
	\]
	natural in $X$ and $Y$.
	The elements of $\int^Z JZ\times P(Z,X;Y)$ are equivalence classes
	of pairs $\langle j, p\rangle$, with $j\in JZ$ and $p\in P(Z,X;Y)$
	for some $Z\in\C$. Our element is mapped to the $A$-indexed natural
	family of functions
	\[
		(p\in P(A,X;Y)) \mapsto [\langle e,p\rangle].
	\]
	where $[\langle e,p\rangle]$ denotes the equivalence class containing
	$\langle e,p\rangle$. The final natural isomorphism takes this element
	to the $A$-indexed natural family of functions
	\[
		(p\in P(A,X;Y)) \mapsto \lambda_{X,Y}([\langle e,p\rangle]).
	\]
	Now it's easy to show that the diagram commutes: let $\langle e,p\rangle$
	be an element of $JA\times P(A,X;Y)$. The vertical arrow maps it to
	the pair $\langle f, p\rangle$, where $f$ is the linear element
	displayed above. $\lambda'$ then maps this to $\lambda_{X,Y}^A(e,p)$,
	as required.
\end{proof}
%
There is an apparent asymmetry here: although it was easy to define
$\lambda'$, there is no obvious way to define a corresponding $\rho'$
-- unless of course our promonoidal category is braided, of which more
below. This asymmetry derives from the fact that $\E$ is defined using
\emph{right} $\C$-modules,
and of course it would be possible to define a dual version using left
modules, which would also be monoidally isomorphic to $\Prof(1,\C)$,
hence to $\E$. Using this, we could define a `co-linear elements' functor
$\Lin': \C\to\Set$, with a canonical natural isomorphism
\[
	\rho'^A_{X,Y}: \Lin'A\times P(X,A;Y) \to \C(X,Y).
\]
However, in general there is no canonical natural isomorphism between
$\Lin$ and $\Lin'$. In the braided case, there is. So in that case we
may simply take $\Lin$ (or equivalently $\Lin'$) to be the unit, which
role it is able to fulfil if and only if $\lambda'$ (equivalently $\rho'$)
is invertible. More formally we have:
\begin{propn}\label{prop-promonunit}
	Let $\C$ be a category, and $P: \C\times\C\rPro\C$ a profunctor.
	Let $\alpha$ be an associator satisfying the pentagon
	condition, and let $\sigma$ be a braiding satisfying the
	hexagon conditions. There exists a unit $J:1\rPro\C$ (with
	coherent unit isomorphisms $\lambda$ and $\rho$) if and only
	if the natural transformation
	\[
		\lambda': \int^A \Lin A\times P(A,X;Y) \to \C(X,Y),
	\]
	defined above, is invertible.
\end{propn}
\begin{proof}
	We have already established the `only if' direction, so let
	$\C$, $P$, $\alpha$ and $\sigma$ be given, and define $\Lin: \C\to\Set$
	and $\lambda'$. Suppose that $\lambda'$ is invertible.
	
	By the presentation of braided pseudomonoids given in Section~\chref{Psmon}{s-braided-facts},
	it suffices to show that
	\[
		\begin{diagram}
			1\times\C^2 & \rPro^{\Lin\times\C^2} & \C^3 \\
			\dPro<{1\times P} & \sim & \dPro>{\C\times P} \\
			\rnode{IC}{1\times\C} & \rPro^{\Lin\times\C} & \C^2 \\
			&\lambda' & \dPro>P \\
			&&\rnode{C}{\C}
			\nccurve[angleA=270,angleB=180,ncurv=1]{->}{IC}{C}\Bput1
		\end{diagram}
		=
		\begin{diagram}
			\rnode{ICC}{1\times\C^2} & \rPro^{\Lin\times\C^2} & \C^3 & \rPro^{\C\times P} & \C^2 \\
			&\lambda'\times\C &\dPro[snake=1em]>{P\times\C} & \alpha & \dPro>P \\
			&&\rnode{CC}{\C^2} & \rPro_P & \C
			\nccurve[angleA=270,angleB=180,ncurv=1]{->}{ICC}{CC}\Bput1
		\end{diagram}
	\]
	Concretely, this amounts to showing that the diagram
	\begin{diagram}[h=2em,labelstyle=\scriptstyle]
		\int^{A,X} P(A,X;N)\times\Lin A\times P(L,M;X)
		& \rTo^{\int^X \lambda'_{X,N}\times P(L,M;X)}
		& \int^X \C(X,N)\times P(L,M;X) \\
		&&\dTo>\cong \\
		\dTo<{\int^A\Lin A\times\alpha_{A,L,M,N}}
		&& P(L,M;N) \\
		&&\uTo>\cong \\
		\int^{A,X}\Lin A\times P(A,L;X)\times P(X,M;N)
		& \rTo_{\int^X\lambda'_{L,X}\times P(X,M;N)}
		& \int^X\C(L,X)\times P(X,M;N)
	\end{diagram}
	commutes, which is an immediate consequence of (\ref{diag-linel}).
\end{proof}

\subsection{Application to semi SMCCs}
In the case where $\C$ is a semi SMCC, we have the isomorphism
\[
	\int^A \Lin A\times P(A,X;Y) \cong \int^{A} \Lin(A)\times\C(A,X\lolli Y)
	\cong \Lin(X\lolli Y),
\]
and the natural transformation $\lambda'_{X,Y}: \Lin(X\lolli Y)\to\C(X,Y)$
is precisely the natural transformation that we called $\ell$ in Section~\ref{s-linel}.
Therefore Proposition~\ref{prop-promonunit} does indeed justify the presentation
described in Section~\ref{s-linel}.

\section{The $\psi$ presentation}
In the closed case we have a functor $\lolli$ such that
\[
	P(A,B;C) \cong \C(A, B\lolli C),
\]
The left-unit isomorphism of a promonoidal category has components
\[
	\lambda_{A,B}: \int^{X} J(X)\times P(X,A;B) \to \C(A,B),
\]
the left-hand side of which is isomorphic to
\[
	\int^{X} J(X)\times \C(X, A\lolli B)
\]
which in turn is isomorphic to $J(A\lolli B)$. So we can take
$\lambda$ to be an isomorphism
\[
	\lambda_{A,B}: J(A\lolli B) \to \C(A,B),
\]
thereby eliminating the need to mention coends. The problem
(if we want a fully coherent presentation) is to reconcile the
fact that the associativity and symmetry are defined using $\tn$,
whereas the unit is defined using $\lolli$. Abstractly, we may
consider that we have two isomorphic multiplication profunctors,
say $P$ and $Q$ where
\[
	P(A,B;C) = \C(A\tn B,C)
\]
and
\[
	Q(A,B;C) = \C(A, B\lolli C),
\]
with the associativity and symmetry isomorphisms defined on $P$,
and the unit isomorphism defined on $Q$. Of course a unit isomorphism
may be defined on $P$ by using the isomorphism with $Q$, and the
coherence condition for the unit expressed in terms of this composite.
Here it simplifies matters to back up
and address the question at the level of structure internal to
a general monoidal bicategory. (The symmetry does not play an
essential role in this part of the argument, so there is no need
to assume a braiding here.) We shall use the language of
components, and to make the notation more familiar we shall
write $A\tn B$ to mean $P(A,B)$ and $A\odot B$ to mean $Q(A,B)$.
Also we'll write $I$ to mean $J()$. So (symmetry aside)
we have invertible 2-cells with components
\[
	\alpha_{A,B,C}: A\tn(B\tn C) \to (A\tn B)\tn C,
\]
\[
	\lambda_{A}: I\odot A\to A,
\]
and
\[
	\chi_{A,B}: A\tn B \to A\odot B,
\]
this last corresponding to the currying isomorphism.
We assume that $\alpha$ satisfies the pentagon condition, and that the
diagram of components
\begin{diagram}\dlabel[$\boldsymbol{\lambda\alpha\chi}$]{diag-lac}
	I\tn(A\tn B) && \rTo^{\alpha_{I,A,B}} && (I\tn A)\tn B \\
	\dTo<{\chi_{I,A\tn B}} && \dnum && \dTo>{\chi_{I,A}\tn B} \\
	I\odot(A\tn B) & \rTo_{\lambda_{A\tn B}} & A\tn B
		& \lTo_{\lambda_{A}\tn B} & (I\odot A)\tn B
\end{diagram}
commutes. Now, define $\psi$ to be the unique invertible 2-cell
with components
\[
	\psi_{A,B,C}: A\odot(B\tn C)\to (A\odot B)\odot C,
\]
such that the diagram
\begin{diagram}\dlabel[$\boldsymbol{\alpha\chi\psi}$]{diag-acp}
	A\tn(B\tn C) && \rTo^{\alpha_{A,B,C}} && (A\tn B)\tn C \\
	\dTo<{\chi_{A,B\tn C}} &&\dnum&& \dTo>{\chi_{A,B}\tn C} \\
	A\odot(B\tn C) & \rTo_{\psi_{A,B,C}} & (A\odot B)\odot C
		& \lTo_{\chi_{A\odot B, C}} & (A\odot B)\tn C
\end{diagram}
commutes. In the abstract this seems a rather odd thing to construct,
but in our concrete example it corresponds (via Yoneda) to a natural
isomorphism $(X\tn Y)\lolli Z \cong X\lolli(Y\lolli Z)$, an internal
analogue of currying. We shall consider the relationship between
$\chi$ and $\psi$, with the aim of replacing the former by the latter.
\begin{lemma}\label{lemma-chipsiunit}
	If diagram \pref{diag-acp} commutes, then diagram
	\pref{diag-lac} commutes if and only if the following does.
	\begin{diagram}\dlabel[$\boldsymbol{\lambda\chi\psi}$]{diag-lcp}
		I\odot(A\tn B) & \rTo^{\psi_{I,A,B}} & (I\odot A)\odot B \\
		\dTo<{\lambda_{A\tn B}} &\dnum& \dTo>{\lambda_{A}\odot B} \\
		A\tn B & \rTo_{\chi_{A,B}} & A\odot B
	\end{diagram}
\end{lemma}
\begin{proof}
	Consider the diagram
	\begin{diagram}[h=1.5em]
		I\tn(A\tn B) && \rTo^{\alpha_{I,A,B}} && (I\tn A)\tn B \\
		\\
		\dTo<{\chi_{I,A\tn B}} && (I\odot A)\odot B && \dTo>{\chi_{I,A}\tn B} \\
		&\ruTo^{\psi_{I,A,B}} & \dTo>{\scriptstyle\lambda_{A}\odot B} & \luTo^{\chi_{I\odot A,B}}\\
		I\odot(A\tn B) && A\odot B && (I\odot A)\tn B \\
		& \rdTo_{\lambda_{A\tn B}} & \uTo>{\scriptstyle\chi_{A,B}} & \ldTo_{\lambda_{A}\tn B} \\
		&&A\tn B
	\end{diagram}
	The upper region is an instance of \pref{diag-acp}, and the quadrilateral
	at lower-right commutes by naturality. Since all the components are invertible,
	it follows that the outside \pref{diag-lac} commutes if and only if the
	left-hand quadrilateral \pref{diag-lcp} does.
\end{proof}
\begin{lemma}\label{lemma-pentagons}
	If diagram \pref{diag-acp} commutes, then so does
	\begin{diagram}\dlabel[$\boldsymbol{\alpha\psi}$]{diag-ap}
	 A\odot \bigl(B\tensor (C\tensor D)\bigr)
	 &\rTo^\psi&(A\odot B)\odot (C\tensor D)
	 &\rTo^\psi & \bigl((A\odot B)\odot C\bigl)\odot D
	 \\
	 \dTo<{A\tn\alpha} &&\dnum&& \uTo>{\psi\odot D}
	 \\
	 A\odot\big((B\tensor C)\tensor D\big)
	 && \rTo_\psi && \bigl(A\odot(B\tensor C)\bigr)\odot D
	\end{diagram}
\end{lemma}
\begin{proof}
	We can use \pref{diag-acp} to show that \pref{diag-ap} is equivalent
	to the pentagon condition.
	\begin{sidewaysfigure}
	\begin{diagram}
	 A\odot \bigl(B\tensor (C\tensor D)\bigr)
	 &&\rTo^\psi&&(A\odot B)\odot (C\tensor D)
	 &&\rTo^\psi && \bigl((A\odot B)\odot C\bigl)\odot D
	 \\
	 &\luTo &&& \uTo &&& \ruTo
	 \\
	 &&A\tensor \bigl(B\tensor (C\tensor D)\bigr)
	 &\rTo^\alpha&(A\tensor B)\tensor (C\tensor D)
	 &\rTo^\alpha & \bigl((A\tensor B)\tensor C\bigl)\tensor D
	 \\
	 \dTo<{A\odot\alpha} &&\dTo<{A\tn\alpha} &&&& \uTo>{\alpha\tn D} && \uTo>{\psi\odot D}
	 \\
	 && A\tensor\big((B\tensor C)\tensor D\big)
	 && \rTo_\alpha && \bigl(A\tensor(B\tensor C)\bigr)\tensor D
	 \\
	 &\ldTo &&&&&& \rdTo
	 \\
	 A\odot\big((B\tensor C)\tensor D\big)
	 &&&& \rTo_\psi &&&& \bigl(A\odot(B\tensor C)\bigr)\odot D
	\end{diagram}
	\caption{Diagram used in the proof of Lemma~\ref{lemma-pentagons}}
	\label{fig-pentagons}
	\end{sidewaysfigure}
	Consider the diagram shown in Figure~\ref{fig-pentagons}. The
	unlabelled arrows are constructed using repeated instances of $\chi$:
	for example the vertical arrow
	\[
		(A\tn B)\tn(C\tn D) \to (A\odot B)\odot(C\tn D)
	\]
	is equal to the diagonal of the commutative square
	\begin{diagram}
		(A\tn B)\tn(C\tn D) & \rTo^{\chi_{A,B}\tn(C\tn D)} & (A\odot B)\tn(C\tn D) \\
		\dTo<{\chi_{A\tn B,C\tn D}} && \dTo>{\chi_{A\odot B,C\tn D}} \\
		(A\tn B)\odot(C\tn D) & \rTo_{\chi_{A,B}\odot(C\tn D)} & (A\odot B)\odot(C\tn D)
	\end{diagram}
	The cells that involve these arrows thus commute by a combination of
	naturality and condition~\pref{diag-acp}. So the central pentagon
	commutes if and only if the outside does.
\end{proof}
\begin{lemma}\label{lemma-acp}
	If \pref{diag-lcp} and \pref{diag-ap} commute, then so does \pref{diag-acp}.
\end{lemma}
\begin{proof}
	Consider the diagram shown in Figure~\ref{fig-acp}.
	\begin{sidewaysfigure}
	\begin{diagram}[hug]
	 I\odot \bigl(B\tensor (C\tensor D)\bigr)
	 &&\rTo^{\psi_{I,B,C\tn D}}&&(I\odot B)\odot (C\tensor D)
	 &&\rTo^{\psi_{I\odot B,C,D}} && \bigl((I\odot B)\odot C\bigl)\odot D
	 \\
	 &\rdTo^{\lambda_{B\tn(C\tn D)}} &&& \dTo<{\lambda_{B}\odot(C\tn D)}
		&\natural_{\psi}&& \ldTo^{(\lambda_{B}\odot C)\odot D}
	 \\
	 &&B\tensor (C\tensor D)
	 &\rTo^{\chi_{B,C\tn D}}&B\odot (C\tensor D)
	 &\rTo^{\psi_{B,C,D}} & (B\odot C)\odot D
	 \\
	 \dTo<{I\odot\alpha_{B,C,D}} &\natural_{\lambda}&\dTo<{\alpha} &&&& \uTo>{\chi_{B,C}\odot D}
		&& \uTo>{\psi\odot D}
	 \\
	 && (B\tensor C)\tensor D
	 && \rTo_{\chi_{B\tn C,D}} && (B\tensor C)\odot D
	 \\
	 &\ruTo^{\lambda_{(B\tn C)\tn D}} &&&&&& \luTo^{\lambda_{B\tn C}\odot D}
	 \\
	 I\odot\big((B\tensor C)\tensor D\big)
	 &&&& \rTo_{\psi_{I,B\tn C,D}} &&&& \bigl(I\odot(B\tensor C)\bigr)\odot D
	\end{diagram}
	\caption{Diagram used in the proof of Lemma~\ref{lemma-acp}}
	\label{fig-acp}
	\end{sidewaysfigure}
	The regions marked with the symbol $\natural$ commute by
	naturality, the other three quadrilaterals commute by~\pref{diag-lcp},
	and the outside is an instance of~\pref{diag-ap}. Thus the inner
	pentagonal region commutes, which is precisely~\pref{diag-acp}.
\end{proof}
So $\chi$ and $\psi$ are interdefinable: given $\chi$, we can define
$\psi$ using diagram~\pref{diag-acp}, and alternatively given $\psi$
we can define $\chi$ using diagram~\pref{diag-lcp}. If we take $\psi$
rather than $\chi$ to be part of our defining data, then it suffices
to take~\pref{diag-ap} as an additional axiom (in addition to the
pentagon condition). Diagram \pref{diag-lcp} commutes by construction
so, by Lemma~\ref{lemma-acp}, condition~\pref{diag-acp} holds;
then by Lemma~\ref{lemma-chipsiunit} condition~\pref{diag-lac}
holds too.

In the concrete case we're considering, this justifies the
presentation of Section~\ref{s-psi-presentation}.

\section{The star-autonomous Case}
Finally, we consider full (non-intuitionistic) multiplicative
linear logic. The appropriate notion of model (for the unitless
fragment) is easy to find:
\begin{definition}
	A \emph{semi star-autonomous category} is a semi SMC $\C$
	equipped with an equivalence $-^{*}: \C\to\C\op$ and a natural
	isomorphism with components $\C(A\tn B, C)\cong\C(A, (B\tn\C^{*})^{*})$.
\end{definition}
We shall write $B\lolli C$ as an abbreviation for $(B\tn C^{*})^{*}$.
The $\psi$ presentation becomes remarkably simple in the
star-autonomous case:
\begin{propn}
	To give a semi star-autonomous category is to give a
	category $\C$ equipped with an associative, symmetric
	functor $\tn: \C\times\C\to\C$, an equivalence
	$-^{*}: \C\to\C\op$, and a functor $J: \C\to\Set$
	with a natural isomorphism
	\[
		J((A\tn B^{*})^{*}) \cong \C(A,B).
	\]
\end{propn}
\begin{proof}
	Define $\psi_{A,B,C}$ to be the composite
	\[
		\bigl((A\tn B)\tn C^{*}\bigr)^{*}
		\rTo^{(\alpha_{A,B,C^{*}})^{*}}
		\bigl(A\tn (B\tn C^{*})\bigr)^{*}
		\rTo^{(A\tn n_{B\tn C^{*}})^{*}}
		\Bigl( A\tn \bigl( (B\tn C^{*})^{*} \bigr)^{*} \Bigr)^{*}.
	\]
	With this definition, the diagram~\pref{diag-ap} commutes
	as a consequence of the pentagon condition.
\end{proof}

\end{thesischapter}