%!TEX TS-program = pdflatex
\documentclass{robinthesis}

\begin{thesischapter}{SemiCC}{Compact Closed Categories without Units}
It is, of course, a routine matter to specialise
this definition of semi star-aut\-on\-om\-ous category to the compact closed case.
This chapter gives an elementary axiomatisation of semi compact closure, and shows its
equivalence to the `abstract' notion. The present definition is completely algebraic,
and is perhaps easier to understand and use. The definition itself is not
really new: \citet[][\S3.5]{HinesSS} has a similar-looking definition, which
seems to be strictly weaker than the present one, and \citet{ProofNetCats}
give a more general version.\footnote{
	The Do{\v s}en-Petri{\'c} axioms are intended to define a
	semi star-autonomous category, therefore assume two tensors $\tn$ and $\parr$,
	related by a linear distributivity. If one takes the two tensors to be equal, and the
	linear distributivity to be the ordinary associativity, then one recovers the present
	definition with some redundancy.}

\begin{definition}
A \emph{category with tensor} $\C$ is a category equipped with a tensor product
\[
	\tn: \C\times\C\to\C,
\]
together with natural isomorphisms having components
\[\begin{array}{l}
	\alpha_{A,B,C}: A\tn(B\tn C) \to (A\tn B)\tn C,\\
	\sigma_{A,B}: A\tn B \to B\tn A
\end{array}\]
such that $\sigma_{B,A}^{-1} = \sigma_{A,B}$, and subject to the pentagon and hexagon conditions
found in the usual definition of symmetric monoidal category.
\end{definition}
%
Although the development is stated in terms of a symmetric tensor, it is perfectly
possible -- with only a little more work -- to carry it through when the tensor is
merely braided. The string diagrams, in particular, should make it clear which
direction of braiding is required in any particular definition. Note that an additional
axiom is needed in the braided case, specifically the braid dual of the second
cancellation condition, and braid-dual versions of the lemmas need to be proved.
We also introduce an abbreviation that will be useful in the next definition:
let $\theta$ denote the unique canonical natural isomorphism with components
\[
	\theta_{A,B,C}: A\tn(B\tn C) \to (C\tn A)\tn B.
\]
(This may be defined as either
$\alpha_{C,A,B}.\sigma_{A\tn B,C}.\alpha_{A,B,C}$
or
$(\sigma_{A,C}\tn B)\alpha_{A,C,B}(A\tn\sigma_{B,C})$;
the hexagon condition says precisely that these must be equal.)
\begin{definition}
A \emph{semi compact closed category} is a category $\C$ with tensor, equip\-ped
with:
for every object $A\in\C$, a \emph{dual object} $A^*$, and
natural transformations $\eta^A$ and $\e^A$ with components
\[\begin{array}l
	\eta^A_X: X \to X\tn(A^*\tn A)\\
	\e^A_X: (A\tn A^*)\tn X \to A
\end{array}\]
These natural transformations are called the \emph{unit} and \emph{counit}
of $A$, and are required to satisfy the four axioms shown in Fig.~\ref{fig-ax}.
\end{definition}
\begin{figure}
	\[\begin{array}{cc}
	\hskip-3em
	\begin{diagram}
		X\tn Y &\rTo^{X\tn \eta^A_Y} & X\tn(Y\tn(A^*\tn A))\\
		&\rdTo_{\eta^A_{X\tn Y}}&\dTo>{\alpha_{X,Y,A^*\tn A}}\\
		&&(X\tn Y)\tn(A^*\tn A)
	\end{diagram}
	&
	\hskip 2em
	\begin{diagram}
		(A\tn A^*)\tn(X\tn Y)\\
		\dTo<{\alpha_{A^*\tn A,X,Y}}&\rdTo^{\e^A_{X\tn Y}}\\
		((A^*\tn A)\tn X)\tn Y &\rTo_{\e^A_X\tn Y}&X\tn Y
	\end{diagram}
	\\[5em]
	\hskip-3em
	\begin{diagram}
		A &\rTo^1& A\\
		\dTo<{\eta^A_A}&&\uTo>{\e^A_A}\\
		A\tn(A^*\tn A) &\rTo_{\alpha_{A,A^*,A}}&(A\tn A^*)\tn A
	\end{diagram}
	&
	\hskip 2em
	\begin{diagram}
		A^* &\rTo^1& A^*\\
		\dTo<{\eta^A_{A^*}}&&\uTo>{\e^A_{A^*}}\\
		A^*\tn(A^*\tn A) & \rTo_{\theta_{A^*,A^*,A}}&(A\tn A^*)\tn A^*
	\end{diagram}
	\end{array}\]
	\caption{Coherence conditions for a semi compact closed category}
	\label{fig-ax}
\end{figure}%
The plan for the rest of this chapter is as follows. In \S\ref{s-direct} we develop the
theory of semi compact closed categories directly from the axioms, since it is instructive
to see how readily this may be done, and how similar it is to the ordinary theory of
compact closure. (But see later for an alternative, indirect, approach.) \S\ref{s-hhs}
then shows that every semi compact closed category is (degenerately) semi star-%
autonomous in the sense of Chapter~\refchapter{Promon}.

\S\ref{s-embed} is independent of the previous sections, and shows
how an arbitrary semi compact closed category may be fully and faithfully embedded in an
ordinary compact closed category (which has one additional object playing the role of
the unit). This embedding preserves the tensor and duality on the nose, which makes it
possible to transfer most of our knowledge about compact closed categories to the
unitless situation, and in particular to deduce the main results of \S\ref{s-direct}.

\section{Direct development}\label{s-direct}
We shall use string diagrams \citep{GTC}, to make the calculations easier
to follow. Our diagrams are to be read from left to right, and we notate $\eta$
and $\e$ as in Fig.~\ref{fig-diagdef}.
\begin{figure}
	\hbox to\columnwidth\bgroup\hss
		\begin{tabular}{c@{\hskip 3em}c}
		\diag{d-eta} & \diag{d-eps}
		\end{tabular}
	\hss\egroup
	\caption{Diagrammatic notation for $\eta$ and $\e$}\label{fig-diagdef}
\end{figure}
Diagrammatic forms of the axioms are shown in Figs.~\ref{fig-diagnat}--\ref{fig-diagcanc}.
	\begin{figure}
	\hbox to\columnwidth\bgroup\hss
		\begin{tabular}{c}
		\diag{d-eta-nat}
		\\[1em]
		\diag{d-eps-nat}
		\end{tabular}
	\hss\egroup
	\caption{Diagrammatic form of the naturality conditions}\label{fig-diagnat}
\end{figure}
\begin{figure}
	\hbox to\columnwidth\bgroup\hss
		\begin{tabular}{c}
		\diag{d-eta-alph}
		\\[1em]
		\diag{d-eps-alph}
		\end{tabular}
	\hss\egroup
	\caption{Diagrammatic form of the associativity conditions}\label{fig-diagass}
\end{figure}
\begin{figure}
	\hbox to\columnwidth\bgroup\hss
		\begin{tabular}{c}
		\diag{d-ax1}
		\\[1em]
		\diag{d-ax2}
		\end{tabular}
	\hss\egroup
	\caption{Diagrammatic form of the cancellation conditions}\label{fig-diagcanc}
\end{figure}
The first task is to show how the duality operation can be extended to a contravariant
functor, in such a way that $\eta$ and $\e$ are both dinatural in $A$. Given an arrow
$f: A\to B$, we define $f^*: B^*\to A^*$ as shown in Fig.~\ref{fig-fstar}.
\begin{figure}
	\hbox to\columnwidth{\hss\diag{d-fstar}\hss}
	\caption{Given $f: A\to B$, we define $f^*: B^*\to A^*$ using this diagram}\label{fig-fstar}
\end{figure}
Note that, directly from the second cancellation axiom, we have $1_A^* = 1_{A^*}$
for all $A\in\C$, thus our putative functor preserves identities (which is a good start).
It is surprisingly complicated to prove directly that it also preserves composition, but
it will be easy once we have the right lemmas.
%$\eta$ and $\e$ are dinatural.
%In fact we shall show that $\e$ is dinatural: the case
%of $\eta$ is symmetrical.
\begin{lemma}\label{lemma}
	For all $X$, $A$ and $Y\in\C$, the following diagrams commute.
	\begin{diagram}
		(X\tn(A\tn A^*))\tn Y &\rTo^{\alpha_{X,A\tn A^*, Y}^{-1}}& X\tn((A\tn A^*)\tn Y)\\
		\dTo<{\sigma_{A\tn A^*,X}\tn Y}&&\dTo>{X\tn\e^A_Y}\\
		((A\tn A^*)\tn X)\tn Y &\rTo_{\e^A_X\tn Y}&X\tn Y
	\end{diagram}
	\begin{diagram}
		X\tn Y &\rTo^{X\tn\eta^A_Y}&X\tn(Y\tn(A^*\tn A))\\
		\dTo<{\eta^A_X\tn Y}&&\dTo>{X\tn\sigma_{A^*\tn A, Y}}\\
		(X\tn(A^*\tn A))\tn Y &\rTo_{\alpha_{X,A^*\tn A,Y}^{-1}}&X\tn((A^*\tn A)\tn Y)
	\end{diagram}
\end{lemma}
\begin{proof}
	The proof of the first diagram, by string diagram manipulation, is shown in Fig.~\ref{fig-lemma}.
	(Perhaps the least obvious step is the penultimate one, which uses the
	naturality of $\sigma$.) The second is proved by a symmetrical argument:
	Fig.~\ref{fig-lemma'} shows the diagrammatic form of its statement.
\end{proof}
\begin{figure}
\[\begin{array}{rlcl}
	& \cdiag{d-l1} &=&  \cdiag{d-l2}\\[4em]
	=& \cdiag{d-l3} &=&  \cdiag{d-l4}\\[4em]
	=& \cdiag{d-l5} &=&  \cdiag{d-l6}\\[4em]
	=& \cdiag{d-l7}
\end{array}\]
\caption{A diagrammatic proof of Lemma~\ref{lemma}}\label{fig-lemma}
\end{figure}
\begin{figure}
	\hbox to\columnwidth\bgroup\hss
		\cdiag{d-l-eta-lhs} = \cdiag{d-l-eta-rhs}
	\hss\egroup
	\caption{The second part of Lemma~\ref{lemma}}\label{fig-lemma'}
\end{figure}
\begin{lemma}\label{sec}
For any objects $X$,$A$,$B$,$Y$, and arrow $f:A\to B$, the following diagram
commutes. (The associativities have been suppressed to make it more comprehensible.)
\begin{diagram}
	X\tn B^*\tn Y & \rTo^{X\tn f^*\tn Y} & X\tn A^*\tn Y\\
	\dTo<{\eta^A_X\tn B^*\tn Y} && \uTo>{X\tn A^*\tn\e^B_Y}\\
	X\tn A^*\tn A \tn B^*\tn Y &\rTo_{X\tn A^*\tn f \tn B^*\tn Y}& X\tn A^*\tn B\tn B^*\tn Y
\end{diagram}
In string diagram terms, this says
\[
	\cdiag{d-sec-rhs} = \cdiag{d-sec-lhs}
\]
\end{lemma}
\begin{proof}
	The proof is again by string diagram manipulation, shown in Fig.~\ref{fig-sec}.
	Both parts of Lemma~\ref{lemma} are used.
\end{proof}
\begin{figure}
\hbox to \columnwidth{\hss$\begin{array}{rlcl}
	& \cdiag{d-sec-rhs} &=&  \cdiag{d-sec1}\\[4em]
	=& \cdiag{d-sec2} &=&  \cdiag{d-sec3}\\[4em]
	=& \cdiag{d-sec4} &=&  \cdiag{d-sec5}\\[4em]
	=& \cdiag{d-sec-lhs}
\end{array}$\hss}
\caption{A diagrammatic proof of Lemma~\ref{sec}}\label{fig-sec}
\end{figure}%
% The final lemma is not needed immediately, but this seems the right place
% to prove it. [Erm, it's needed pretty damn soon! ;-)]
\begin{lemma}\label{third}
	For all $X$, $A$, $Y\in\C$, the following diagram commutes.
%	X\tn A\tn Y    X\tn A^*\tn A \tn A\tn Y    X\tn A\tn A\tn A^*\tn Y    X \tn A\tn Y 
	\begin{diagram}
	X\tn A\tn Y & \rTo^{1} & X\tn A\tn Y\\
	\dTo<{\eta^A_X\tn A\tn Y} && \uTo>{X\tn A\tn\e^A_Y}\\
	X\tn A^*\tn A \tn A\tn Y & \rTo_{X\tn\sigma_{A^*,A\tn A, Y}} & X\tn A\tn A\tn A^*\tn Y 
	\end{diagram}
	(The associativities have again been suppressed.)
\end{lemma}
\begin{proof}
s	See Fig.~\ref{fig-third}. The first step uses both parts of Lemma~\ref{lemma}.%
	\footnote{In the braided case, it uses the braid-dual analogue of that lemma.}
\end{proof}
\begin{figure}
\hbox to \columnwidth{\hss$\begin{array}{rlcl}
	& \cdiag{d-third1} &=&  \cdiag{d-third2}\\[4em]
	=& \cdiag{d-third3} &=&  \cdiag{d-third4}\\[4em]
\end{array}$\hss}
\caption{Proof of Lemma~\ref{third}}\label{fig-third}
\end{figure}%
All the hard work was in the lemmas: everything else is comparatively straightforward.
\begin{propn}\label{prop-dinat}
	The natural transformations $\eta$ and $\e$ are also dinatural in
	the superscript variable.
\end{propn}
\begin{proof}
	See Fig.~\ref{fig-dinat} for a proof that $\e$ is dinatural.
	The proof for $\eta$ may be obtained by turning the string diagrams upside down. 
\end{proof}
\begin{figure}
\hbox to \columnwidth{\hss$\begin{array}{rlcl}
	& \cdiag{d-dinat-rhs} &=&  \cdiag{d-dinat1}\\[4em]
	=& \cdiag{d-dinat2} &=&  \cdiag{d-dinat3}\\[4em]
	=& \cdiag{d-dinat-lhs}
\end{array}$\hss}
\caption{Proof that $\e$ is dinatural (Prop.~\ref{prop-dinat})}\label{fig-dinat}
\end{figure}%
\begin{propn}\label{prop-comp}
	The duality preserves composition, i.e.\ given $f:A\to B$ and $g: B\to C$,
	we have $(gf)^* = f^*g^*$.
\end{propn}
\begin{proof}
	See Fig.~\ref{fig-comp}. The third equality uses the dinaturality of $\eta$.
\end{proof}
\begin{figure}
\hbox to \columnwidth{\hss$\begin{array}{rlcl}
	& \cdiag{d-comp1} &=&  \cdiag{d-comp2}\\[4em]
	=& \cdiag{d-comp3} &=&  \cdiag{d-comp4}\\[4em]
	=& \cdiag{d-comp5}
\end{array}$\hss}
\caption{Proof of Prop.~\ref{prop-comp}}\label{fig-comp}
\end{figure}%
\begin{propn}\label{prop-starstar}
	There is a natural isomorphism $A\cong A^{**}$.
\end{propn}
\begin{proof}
Fig.~\ref{fig-starstar} shows how to construct a natural isomorphism $A\cong A^{**}$.
\begin{figure}
	\hbox to \columnwidth\bgroup\hss
	\begin{tabular}{c@{\qquad}c}
	\diag{d-starstar1} & \diag{d-starstar2}
	\end{tabular}
	\hss\egroup
	\caption{How to construct a natural isomorphism $A\cong A^{**}$}\label{fig-starstar}
\end{figure}%
Naturality is immediate from the naturality and dinaturality of $\eta$ and $\e$,
and the naturality of $\sigma$. Figs.~\ref{fig-starstar-inva} and~\ref{fig-starstar-invb}
show that these maps are indeed mutually inverse, hence determine an isomorphism.
Notice that the second step in Fig.~\ref{fig-starstar-inva} uses Lemma~\ref{third}.
\begin{figure}
	\hbox to \columnwidth{\hss$\begin{array}{rlcl}
		& \cdiag{d-starinv-a1} &=&  \cdiag{d-starinv-a2}\\[4em]
		=& \cdiag{d-starinv-a3} &=&  \cdiag{d-starinv-a4}
	\end{array}$\hss}
	\caption{The maps from Fig.~\ref{fig-starstar} compose to give the identity on $A$}
	\label{fig-starstar-inva}
\end{figure}%
\begin{figure}
	\hbox to \columnwidth{\hss$\begin{array}{rlcl}
		& \cdiag{d-starinv-b1} &=&  \cdiag{d-starinv-b2}\\[4em]
		=& \cdiag{d-starinv-b3} &=&  \cdiag{d-starinv-b4}\\[4em]
		=& \cdiag{d-starinv-b5} &=&  \cdiag{d-starinv-b6}
	\end{array}$\hss}
	\caption{The maps from Fig.~\ref{fig-starstar} compose to give the identity on $A^{**}$}
	\label{fig-starstar-invb}
\end{figure}
\end{proof}
\begin{propn}\label{prop-adj}
	For each object $A$, there is an adjunction $A\tn- \dashv A^*\tn-$, which
	determines a natural isomorphism
	\[
		\C(B\tn A, C)\cong\C(A, B^*\tn C)
	\]
\end{propn}
\begin{proof}
There are obvious natural transformations
\[\begin{array}l
	\gamma_{A,B,C}: \C(A\tn B^*, C) \to \C(A, C\tn B),\\
	\delta_{A,B,C}: \C(A, B^*\tn C) \to \C(B\tn A, C)
\end{array}\]
illustrated in Figs.~\ref{fig-gamma}--\ref{fig-delta}.
\begin{figure}
\[
	\cdiag{d-f} \To \cdiag{d-gamma-f}
\]
\caption{The natural transformation $\gamma$}\label{fig-gamma}
\end{figure}%
\begin{figure}
\[
	\cdiag{d-g} \To \cdiag{d-delta-g}
\]
\caption{The natural transformation $\delta$}\label{fig-delta}
\end{figure}%
We need to show that one of these natural transformations is invertible, which
we shall do by showing that they are in some sense mutually inverse. We begin
by showing that the composite
\vskip\abovedisplayskip\vbox{%
	\hbox to \columnwidth{$
		\C(A\tn B^*, C) \rTo^\gamma \C(A, C\tn B)
			\rTo^\cong \C(A, C\tn B^{**})$\hss}%
	\vskip\baselineskip
	\hbox to \columnwidth{\hss$
		\rTo^\cong \C(A, B^{**}\tn C)
			\rTo^\delta \C(B^*\tn A, C) \rTo^\cong \C(A\tn B^*, C)
$}}\vskip\belowdisplayskip\noindent
is the identity (where the unlabelled isomorphisms are symmetry or involution maps).
Consider some $f: A\tn B^*\to C$: the result of applying this composite to $f$ is
shown in Fig.~\ref{fig-dg}.
\begin{figure}
\hbox to \columnwidth{\hss\diag{d-delta-gamma-f}\hss}
\caption{The result of applying $\gamma$ and then $\delta$ to some $f: A\tn B^*\to C$}
\label{fig-dg}
\end{figure}%
Fig.~\ref{fig-dg-proof} shows that this is equal to $f$. (The first step combines
several uses of naturality and associativity conditions.)
\begin{figure}
	\hbox to \columnwidth{\hss$\begin{array}{rlcl}
		& \multicolumn 3l{\cdiag{d-delta-gamma-f}}\\[4em]
		=& \cdiag{d-dg-1} &=&  \cdiag{d-dg-2}\\[4em]
		=& \cdiag{d-dg-3} &=&  \cdiag{d-dg-4}\\[4em]
		=&\cdiag{d-f}
	\end{array}$\hss}
	\caption{The map shown in Fig.~\ref{fig-dg} is equal to $f$}
	\label{fig-dg-proof}
\end{figure}%
Therefore $\gamma$ has a post-inverse and $\delta$ a pre-inverse.

Similarly one may take a map $g: A\to B^*\tn C$, and apply to it the composite
\vskip\abovedisplayskip\vbox{%
	\hbox to \columnwidth{$
	\C(A, B^*\tn C) \rTo^\delta \C(B\tn A, C) \rTo^\cong \C(B^{**}\tn A, C)
	$\hss}%
	\vskip\baselineskip
	\hbox to \columnwidth{\hss$
	\rTo^\cong \C(A\tn B^{**}, C)
		\rTo^\gamma \C(A, C\tn B^*) \rTo^\cong \C(A, B^*\tn C)
$\hss}}\vskip\belowdisplayskip\noindent
as shown in Fig.~\ref{fig-gd}.
\begin{figure}
	\hbox to \columnwidth{\hss\diag{d-gamma-delta-g}\hss}
	\caption{The result of applying $\delta$ and then $\gamma$ to some $g: A\to B^*\tn C$,}
	\label{fig-gd}
\end{figure}%
This is equal to $g$ -- the proof is obtained by turning all the diagrams in Fig.~\ref{fig-dg-proof}
upside down -- hence $\gamma$ also has a pre-inverse and $\delta$ a post-inverse.
Therefore both are invertible, as claimed.
\end{proof}

\section{The promonoidal structure}\label{s-hhs}
This section shows that a semi compact closed category is semi star-autonomous
in the sense of Chapter~\refchapter{Promon}. The proof relies on the characterisation
of semi SMC categories via linear elements, as given in Section~\chref{Promon}{s-linel}.
With this machinery available it is easy to prove the main result of this section:
\begin{propn}\label{prop-ssa}
	A semi compact closed category is semi star-autonomous.
\end{propn} 
\begin{proof}
	Let $\C$ be a semi compact closed category. By assumption it
	is equipped with a symmetric tensor, and if we define
	\[
		A\lolli B := A^*\tn B
	\]
	then Prop.\ref{prop-adj} shows that we have a natural isomorphism
	\[
		\delta_{A,B,C}:\C(B\tn A,C) \lTo^\cong \C(A, B\lolli C).
	\]
	It remains only to construct an inverse to the function
	$\ell_{A,B}: \Lin(A^*\tn B) \to \C(A,B)$.
	
	If we represent a linear element $x\in\Lin(A^*\tn B)$ as shown in Fig.~\ref{fig-x},
	\begin{figure}
		\hbox to \columnwidth{\hss\cdiag{d-lin4}\hss}
		\caption{The diagrammatic representation of a linear element $x\in\Lin(A^*\tn B)$}
		\label{fig-x}
	\end{figure}%
	note that, by the definition of $\delta$, the arrow $\ell_{A,B}(x)$ is as shown
	in Fig.~\ref{fig-lx}.
	\begin{figure}
		\hbox to \columnwidth{\hss\cdiag{d-lx}\hss}
		\caption{The diagrammatic representation of $\ell_{A,B}(x)$}
		\label{fig-lx}
	\end{figure}%
		
	Given a map $f: A\to B$,
	define $\ell_{A,B}^{-1}(f)$ to be the natural transformation whose
	component at $X$ is
	\[
		X \rTo^{\eta^A_X} X\tn(A^*\tn A) \rTo^{X\tn(A^*\tn f)} X\tn(A^*\tn B).
	\]
	For any $f:A\to B$, the arrow $\ell_{A,B}(l_{A,B}^{-1}(f))$ is the composite
	\[
		A \rTo^{\eta^A_A} A\tn(A^*\tn A) \rTo^{A\tn(A^*\tn f)} A\tn(A^*\tn B)
			\rTo^{\alpha_{A,A^*,B}} (A\tn A^*)\tn B \rTo^{\e^A_B} B.
	\]
	By the naturality of $\alpha$ and $\epsilon$, and the first cancellation
	condition, this is indeed equal to $f$. Conversely suppose we have a
	linear element $x\in\Lin(A^*\tn B)$. Since it will be convenient to use
	a string diagram calculation here, we introduce a diagrammatic notation
	for this linear element, shown in Fig.~\ref{fig-x}.
	Now Fig.~\ref{fig-lin} shows a proof that $\ell_{A,B}^{-1}(l_{A,B}(x))$ is equal to~$x$.
%	\[\hskip -5em
%		X \rTo^{\eta^A_X} X\tn(A^*\tn A)
%		\rTo^{X\tn(A^*\tn x_A)} X\tn(A^*\tn (A\tn(A^*\tn B))
%		\rTo^{X\tn(A^*\tn \alpha_{A,A^*,B})} X\tn(A^*\tn ((A\tn A^*)\tn B))
%		\rTo^{X\tn(A^*\tn \e^A_B)} X\tn(A^*\tn B)
%	\]
	\begin{figure}
	\hbox to \columnwidth{\hss$\begin{array}{rlcl}
		& \cdiag{d-lin1} &=&  \cdiag{d-lin2}\\[4em]
		=& \cdiag{d-lin3} &=&  \cdiag{d-lin4}
	\end{array}$\hss}
		\caption{Proof that $\ell_{A,B}^{-1}(\ell_{A,B}(x)) = x$. (See Prop.~\ref{prop-ssa})}
		\label{fig-lin}
	\end{figure}%
\end{proof}

\section{Embedding theorem}\label{s-embed}
If we wanted to add a unit object $I$ to a semi star-autonomous category $\C$, we would
also have to add an infinite family of other objects such as $I^*$, $I^*\tn A$ for $A\in\C$,
and so on. In the compact closed case, there is no such obstacle, since $I^*$ is always
isomorphic to $I$, and we may take $I^* = I$ without essential loss of generality. This
raises the hope that it may always be possible to fully embed any semi compact closed
category $\C$ into a compact closed category $\C'$, in such a way the objects of $\C'$
are essentially just the objects of $\C$ plus a unit object.
%
It turns out that such an embedding is possible, as this section shows.

Recall the `$\mathbf e$' construction of \cite{BTC}, there used to prove
the case $\B=\Cat$ of the Cayley Theorem (our Proposition~\chref{Cayley}{prop-psmoncayley}).
Given a monoidal category $\C$,
the category $\mathbf{e}(\C)$ is defined as follows. An object of $\mathbf{e}(\C)$ is
a pair $(F, \gamma^F)$ of a functor $F:\C\to\C$ and a natural isomorphism with
components
\[
	\gamma^F_{A,B}: F(A\tn B) \to F(A)\tn B.
\]
A morphism $\delta: (F, \gamma^F) \to(G,\gamma^G)$ is a natural transformation
$F\To G$ such that the diagram
\begin{diagram}
	F(A\tn B) & \rTo^{\gamma^F_{A,B}} & F(A)\tn B
	\\
	\dTo<{\delta_{A\tn B}} && \dTo>{\delta_A\tn B}
	\\
	G(A\tn B) & \rTo^{\gamma^G_{A,B}} & G(A)\tn B
\end{diagram}
commutes for all $A$ and $B\in \C$.

The tensor product $(F, \gamma^F)\tn (G, \gamma^G)$ is defined to be $(FG, \gamma^{FG})$,
where $\gamma^{FG}_{A,B}$ is the composite
\[
	FG(A\tn B) \rTo^{F\gamma^G_{A,B}} F(GA \tn B) \rTo^{\gamma^F_{GA,B}} FGA\tn B.
\]
The tensor product of two arrows is their horizontal composite as natural transformations.
The tensor unit $I$ is simply the identity functor, with the identity natural transformation.

There is an functor $L: \C\to\mathbf{e}(\C)$, where $L(A) := (A\tn-, \alpha_{A,-,-})$
for objects $A\in \C$, and $L(f) := f\tn-$ for arrows $f$. Note that the unit object of $\C$ plays no part in the construction of the category $\mathbf{e}(\C)$ or the functor $L$, so that everything so far makes sense for a semi compact closed category. Furthermore:

\begin{propn}\label{prop-L-ff}
	When $\C$ is a semi compact closed category, thr functor $L$ is full and faithful.
\end{propn}
\begin{proof}
	\citeauthor{BTC}'s proof of this claim (for $\C$ a monoidal category) uses the tensor unit in an essential way, so we
	need to find a new proof that uses semi compact closure instead. The functor $L$ induces,
	for every $X$ and $Y\in\C$, a function $\C(X,Y) \to \mathbf{e}(\C)(LX, LY)$. We'll
	describe an inverse to this function, showing that it is invertible and hence that $L$ is
	full and faithful.
	
	Let $\delta$ be a natural transformation $LX\To LY$.
	Thus $\delta$ consists of components $\delta_A: X\tn A \to Y\tn A$, natural in $A$
	and such that the diagram
	\begin{diagram}
		X\tn (A\tn B) & \rTo^{\alpha_{X,A,B}} & (X\tn A)\tn B\\
		\dTo<{\delta_{A\tn B}} && \dTo>{\delta_A\tn B}\\
		Y\tn (A\tn B) & \rTo^{\alpha_{Y,A,B}} & (Y\tn A)\tn B
	\end{diagram}
	commutes for all $A$, $B\in\C$.
	
	It will be convenient to use string diagrams in the proof: we'll picture
	$\delta_A$ as
	\[
		\cdiag{d-L-delta}.
	\]
	In string diagram terms, the commutative square above is a rewiring condition
	of the sort we have seen above:
	\[\begin{array}[c]{lcr}
		\cdiag{d-L-delta-rewiring-lhs} &=& \cdiag{d-L-delta-rewiring-rhs}.
	\end{array}\]
	The naturality of $\delta$ means that functions can pass through the loop:
	\[\begin{array}[c]{lcr}
		\cdiag{d-L-delta-nat-lhs} &=& \cdiag{d-L-delta-nat-rhs}.
	\end{array}\]
	%
	Now we can define our inverse to the action of L, to take $\delta$ to
	the following arrow $f: X\to Y$:
	\[
		\cdiag{d-L-f}
	\]
	We must show that $f\tn A = \delta_A$, for any $A\in \C$. The proof is
	a routine string diagram manipulation, shown in Fig.~\ref{fig-ftnA}.
	\begin{figure}
		\[\begin{array}[c]{rcl}
			\cdiag{d-ftnA-1}	
			&=&\cdiag{d-ftnA-2}	\\ \strut\\
			&=&\cdiag{d-ftnA-3}	\\ \strut\\
			&=&\cdiag{d-ftnA-4}	\\ \strut\\
			&=&\cdiag{d-ftnA-5}	\\ \strut\\
			&=&\cdiag{d-ftnA-6}	\\ \strut\\
			&=&\cdiag{d-ftnA-7}	\\ \strut\\
			&=&\cdiag{d-ftnA-8}	\\ \strut\\
			&=&\cdiag{d-ftnA-9}	\\ \strut\\
			&=&\cdiag{d-ftnA-10}
		\end{array}\]
		\caption{Proof that $f\tn A = \delta_A$, used in Prop.~\ref{prop-L-ff}.}
		\label{fig-ftnA}
	\end{figure}
	This shows that the passage  $\mathbf{e}(\C)(LX, LY)\to \C(X,Y)\to\mathbf{e}(\C)(LX, LY)$
	is the identity. For the other direction, we need to show that
	\[\begin{array}[c]{c}
		\cdiag{d-L-other-lhs} = \cdiag{d-L-other-rhs}
	\end{array}\]
	which is immediate.
\end{proof}
%
Now define $\E$ to be the full subcategory of $\mathbf{e}(\C)$ determined by the
objects that have adjoints. This is clearly a compact closed category.
\begin{propn}
	The image of $L$ is contained in $\E$. Specifically, for every $X\in\C$, the object
	$L(X)$ is adjoint to $L(X^*)$ in $\mathbf{e}(\C)$.
\end{propn}
\begin{proof}
	Define $\eta^{LX}: I \to L(X^*)L(X)$ to have components
	\[
		\cdiag{d-eta-LX}
	\]
	and $\e^{LX}: L(X)L(X^*)\to I$ to have components
	\[
		\cdiag{d-eps-LX}
	\]
	These are clearly natural in $A$, and it's easy to verify that they satisfy the
	condition making them maps of $\mathbf{e}(\C)$.
	To show that they really do define an adjunction between $L(X)$ and $L(X^*)$,
	we need to show that
	\[
		\cdiag{d-Ladj-X}
		\qquad \mbox{and} \qquad
		\cdiag{d-Ladj-Xstar}
	\]
	are both equal to the identity. This is an easy exercise in manipulations of the
	sort that are by now routine.
\end{proof}
%
Finally, let $\C'$ be the full subcategory of $\E$ determined by those objects that are
either isomorphic to $L(X)$ for some $X$, or isomorphic to $I$. This subcategory is
closed under the tensor and adjoint operations, so it's compact closed. The image
of the (full and faithful) functor $L$ is contained in $\C'$ by definition. Thus $\C$ is
embedded, in a structure-preserving fashion, in a compact closed category that
has essentially only one extra object, the unit object. (If in fact $\C$ had a unit object
all along, this functor will be an equivalence.)


%***************************************************************************************************

%Given a semi star-autonomous category, it is certainly not possible to turn it into a
%star-autonomous category proper simply by adding a unit object $I$. 

%It is well-known that every monoidal category $\C$ is monoidally equivalent to a strict
%monoidal category $\mathbf{e}(\C)$.
%The construction was first given by \citet{BTC}, and it was subsequently observed
%by Gordon and Power that this construction can be regarded as an application of the
%bicategorical Yoneda lemma to the suspension of the monoidal category in question.
%(The suspension of a monoidal category $\C$ is the one-object bicategory whose
%1-cells are the objects of $\C$ -- with composition being the tensor product -- and
%whose 2-cells are the arrows of $\C$.)

%This latter point of view provides a very easy proof that every compact closed category $\C$
%is equivalent, in the appropriate sense, to a strict monoidal one. To say that a symmetric
%monoidal category is compact closed is to say that every 1-cell of its suspension has a
%left (equivalently right) adjoint. Since pseudo-functors preserve adjointness, it must be the case
%that $\mathbf{e}(\C)$ is compact closed, and that the equivalence with $\C$ preserves
%the duality.

%This section shows how it is possible to use \citeauthor{BTC}'s constuction to embed
%any semi compact closed category $\C$ into a strict monoidal compact closed category
%$\bar{\mathbf{e}}(\C)$, in such a way that the embedding is full and faithful, and every
%object of \dots

\end{thesischapter}