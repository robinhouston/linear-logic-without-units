%!TEX TS-program = latex
\documentclass{robinthesisdraft}
\usepackage{robincs,thesisdefs,xr}

\title{Cayley's Theorem for Pseudomonoids}
\begin{document}
\maketitle

In its traditional form, Cayley's theorem says that every finite group
is (isomorphic to) a subgroup of one of the symmetric groups. What the
proof actually shows is that every monoid $M$ is a submonoid of the monoid
of endofunctions $|M|\to|M|$, via the function that maps the element $x\in M$
to the function $x\o-$.

A `Cayley Theorem for monoidal categories' has been used by
\citet[Proposition~1.3]{BTC} to show that every monoidal category
is monoidally equivalent to a \emph{strict} monoidal category.
But it turns out that we need the corresponding theorem for
promonoidal categories; and rather than just prove another special
case, it seems wise to generalise. So the purpose of this
chapter is to state and prove a Cayley Theorem for pseudomonoids.

As motivation, we begin by reviewing the one-dimensional case,
of monoids in a monoidal category.

\section{Cayley's theorem for monoids}

I'm not aware of any literature that specifically addresses the
question of giving an analogue of Cayley's theorem for monoids
in a monoidal category, but it is a special case of a standard
construction. A monoid in the monoidal category $\V$ can be
regarded as a one-object $\V$-enriched category.\footnote{This
operation -- reimagining a monoid as a one-object category -- is often
called \emph{suspension}.} Then we can apply the enriched Yoneda
Lemmas of \citet[sections~1.9 and~2.4]{KellyEnriched} to this
$\V$-category. Since it requires some effort to extract the
specifics of the one-object case from Kelly's constructions,
we give an overview here.

Let $\V$ be a monoidal category, and let $M$ be a monoid in $\V$,
with unit $e: I\to M$ and multiplication $m: M\tn M\to M$.
%
% In the pseudomonoid case, we shall work in a Gray-monoid and use
% coherence to transfer the result to an arbitrary monoidal bicategory.
% Since this process is not entirely trivial, we shall illustrate
% the process in the one-dimensional case, where the analogous
% strategy would be to work in a \emph{strict} monoidal category $\V$.
% Thus we shall assume, for the moment, that $\V$ is strictly
% associative. (We could also take the unit to be strict, of course,
% but we do not, since it doesn't particularly help.)

The Cayley-Yoneda theorems are best stated in terms of modules:
\begin{definition}\label{def-module}
	A \emph{right $M$-module} consists of an object $X\in\V$ and a map
	$x: X\tn A\to X$ such that the diagrams
	\[
	\begin{diagram}
		X\tn M\tn M & \rTo^{x\tn M} & X\tn M \\
		\dTo<{X\tn m} && \dTo>{x} \\
		X\tn M & \rTo_{x} & X
	\end{diagram}
	\quad\mbox{and}\quad
	\begin{diagram}
		X &\rTo^{\cong} & X\tn I \\
		\dTo<1 && \dTo>{X\tn e} \\
		X & \lTo_{x} & X\tn M
	\end{diagram}
	\]
	commute.
\end{definition}
\begin{remark}
	Note that $(M,m)$ is a right $M$-module.
\end{remark}
\begin{definition}\label{def-module-map}
	Let $(X,x)$ and $(Y,y)$ be right M-modules.
	A \emph{morphism of modules} from $X$ to $Y$ is
	a map $f: X\to Y$ for which the diagram
	\begin{diagram}
		X\tn M & \rTo^{x} & X \\
		\dTo<{f\tn M} && \dTo>f \\
		Y\tn M & \rTo_{y} & Y
	\end{diagram}
	commutes. The set of such maps will be denoted
	$\Mod_{M}(X,Y)$.
\end{definition}
Kelly proves two versions of his Yoneda Lemma, which he refers
to as `weak' and `strong'. We shall call our corresponding theorems
the `external' and `internal' Cayley Theorem, respectively. The
internal theorem requires some additional properties of $\V$, but
the external one is perfectly general.

\subsection{The external Cayley Theorem}
\begin{thm}[External Cayley]\label{thm-1d-external}
	For every right $M$-module $X$, the natural transformation
	\[
		\phi_{X}: \Mod_{M}(M, X) \to \V(I, X),
	\]
	defined as $\phi_{X}(f) = f\cdot e$, is invertible with
	$\phi_{X}^{-1}(z)$ equal to the composite
	\[
		M \rTo^{\cong} I\tn M \rTo^{z\tn M} X\tn M \rTo^{x} X.
	\]
\end{thm}
\begin{proof}
	Let $f: M\to X$ be a map of modules. We'll first show that
	$\phi_{X}^{-1}(\phi_{X}(f)) = f$: consider the diagram
	\begin{diagram}
		M & \rTo^{\cong} & I\tn M & \rTo^{e\tn M} & M\tn M & \rTo^{f\tn M} & X\tn M \\
		&&& \rdTo_{\cong} & \dTo>m && \dTo>x \\
		&&&& M & \rTo_{f} & X
	\end{diagram}
	The triangle commutes by the left-unit law for the monoid $M$,
	and the square because $f$ is a map of modules. The upper edge
	is $\phi_{X}^{-1}(\phi_{X}(f))$ by definition, and the lower edge is
	equal to $f$.
	
	Now let $z: I\to X$, and consider the diagram
	\begin{diagram}
		&&X \\
		&\ruTo^{z} && \rdTo^{\cong} \\
		I & \rTo_{\cong} & I\tn I & \rTo_{z\tn I}& X\tn I \\
		\dTo<e && \dTo<{I\tn e} && \dTo>{X\tn e} \\
		M & \rTo_{\cong} & I\tn M & \rTo_{z\tn M} & X\tn M & \rTo_{x} & X
	\end{diagram}
	The cells commute by naturality and functoriality of tensor,
	and by the unit condition in the definition of module the upper
	edge is equal to $z$. The lower edge is $\phi_{X}(\phi_{X}^{-1}(z))$
	by definition. Thus $\phi_{X}$ and $\phi_{X}^{-1}$ are indeed
	mutually inverse.
	
	We must also show that $\phi_{X}^{-1}(z)$ is a map of modules,
	so let $z:I\to X$ and consider the diagram
	\begin{diagram}
		M\tn M & \rTo^{\cong} & I\tn M\tn M & \rTo^{z\tn M\tn M} & X\tn M\tn M
			& \rTo^{x\tn M} & X\tn M \\
		\dTo<m &&\dTo>{I\tn m} && \dTo>{X\tn m} && \dTo>m \\
		M & \rTo_{\cong} & I\tn M & \rTo_{z\tn M} & X\tn M & \rTo_{x} & X
	\end{diagram}
	The squares commute, from left to right, by the naturality of the
	left-unit isomorphism, the functoriality of tensor, and the fact
	that $X$ is a module. Thus the outer edge commutes, showing that
	$\phi_{X}^{-1}(z)$ is a map of modules.
\end{proof}
\begin{remark}
	More can be said about $\phi_{X}$, as follows. Each of the sets
	$\Mod_{M}(M, X)$ and $\V(I,X)$ is a right $\V(I,M)$-module
	in a natural way, using the $M$-module structure of $X$.
	Then $\phi_{X}$ is a map of modules with respect to these
	module structures. We omit\footnote{An earlier version of
	this chapter included these details: I removed them, as an
	unneeded distraction.} the details of this, but they are
	quite routine.
\end{remark}
\begin{remark} % \V(I,M) is a monoid
	The set $\V(I, M)$ inherits the monoid structure from $M$:
	its unit is $u:I\to M$, and the product of $a$, $b:I\to M$
	is the composite
	\[
		I \rTo^{\cong} I\tn I \rTo^{a\tn b} M\tn M \rTo^{m} M.
	\]
	This is the \emph{underlying ordinary monoid}
	of the abstract monoid $M$.
\end{remark}

To relate Theorem~\ref{thm-1d-external} to the ordinary
Cayley Theorem, take the module $X$ to be $M$ itself.
Then we have that $\V(I, M)$, the underlying ordinary
monoid of $M$, is isomorphic to $\Mod_{M}(M,M)$, which
is a submonoid of $\V(M,M)$.
%
We should like to be able to say that this $\phi_{M}$
is an isomorphism of monoids, i.e.\ that it preserves
the monoid structure.
\begin{propn}
	The isomorphism
	\[
		\phi_{M}: \Mod_{M}(M,M) \to \V(I, M)
	\]
	is a map of monoids.
\end{propn}
\begin{proof}
	For the unit, $\phi_{M}(1_{M}) = 1_{M}\cdot e = e$.
	For the multiplication, let $f$, $g: M\to M$ be right
	module morphisms: we must show that $\phi_{M}(f\cdot g)$:
	\[
		I \rTo^{e} M \rTo^{g} M \rTo^{f} M
	\]
	is equal to $\phi_{M}(f)\tn\phi_{M}(g)$:
	\[
		I \rTo^{\cong} I\tn I \rTo^{(f\cdot e)\tn(g\cdot e)} M\tn M \rTo^{m} M.
	\]
	Consider the diagram
	\begin{diagram}
		I\tn I & \rTo^{I\tn (g\cdot e)} & I\tn M  & \rTo^{e\tn M} & M\tn M
			 & \rTo^{f\tn M} & M\tn M \\
		\uTo<\cong && \uTo<\cong &\rdTo^{\cong} & \dTo>m && \dTo>m \\
		I  & \rTo_{g\cdot e} & M & \rTo_{1} & M & \rTo_{f} & M
	\end{diagram}
	in which the right-hand square commutes because $f$ is a map of right $M$-modules,
	and the upper triangle by the right-unit law for $M$.
	By the functoriality of tensor, the top edge is equal to $(f\cdot e)\tn(g\cdot e)$,
	so the upper outer edge is $\phi_{M}(f)\tn\phi_{M}(g)$ and the lower
	edge is $\phi_{M}(f\cdot g)$, which are therefore equal as required.
\end{proof}

% \subsection{Coherence issues}
% We have shown that the external Cayley Theorem is true for strict
% monoidal categories, and would like to conclude that it is therefore
% true for arbitrary monoidal categories. This is, of course, overkill
% in the present situation, where it would be trivial to modify our
% proofs to take account of XXXX
% 
\subsection{The internal Cayley Theorem}
The theorem above is external in the sense that it essentially
concerns $\V(I, M)$, the underlying ordinary monoid of $M$. If
the monoidal category $\V$ is closed, then we can state an
internal version, purely in terms of arrows in $\V$ itself.
%
So let $\V$ be biclosed, i.e.\ suppose we are given
functors
\[\begin{array}{l}
	\lolli: \V\op\times\V\to\V \\
	\illol: \V\times\V\op\to\V,
\end{array}\]
and natural isomorphisms with components
\[\begin{array}{l}
	\curry_{A,B,C}:  \V(A\tn B, C) \cong \V(A, B\lolli C), \\
	\curry'_{A,B,C}: \V(A\tn B, C) \cong \V(B, C\illol A).
\end{array}\]
Then the theorem is:
\begin{thm}[Internal Cayley]\label{thm-1d-internal}
	Let $(X,x)$ be a right $M$-module. Then the diagram
	\begin{diagram}
		X & \rTo^{\curry_{X,M,X}(x)} & M\lolli X & \pile{\rTo^{h} \\ \rTo_{k}}
		 & (M\tn M)\lolli X
	\end{diagram}
	is a coequaliser diagram, where the maps $h$ and $k$ are defined as
	follows. The map $h$ is obtained by currying
	\[
		(M\lolli X)\tn M\tn M \rTo^{\ev^{M}_{X}\tn M} X\tn M \rTo^{x} X,
	\]
	and $k$ is obtained by currying
	\[
		(M\lolli X)\tn M\tn M \rTo^{(M\lolli X)\tn m} (M\lolli X)\tn M \rTo^{\ev^{M}_{X}} X.
	\]
\end{thm}
%
Although the statement of the theorem uses only the right-closed
structure $\lolli$, the left-closed structure $\illol$ plays a
crucial role in the proof. We will not give a detailed proof here,
since this is something of a digression, but explain the outline.
The idea is that for every object $X\in\C$, the object $X\illol X$
is a monoid in a canonical way, via the definable `internal unit'
and `internal composition' operations. Then a right $M$-module is
precisely a map of monoids $M\to X\illol X$ for some $X$. \foo


If the proof is to be organised in a sensible way, we need to call
upon some elementary properties of (bi)closed monoidal categories
\citep[Chapter~1]{KellyEnriched}. We shall state these with respect
to the left internal hom $\illol$, though corresponding things are
of course also true of the right internal hom. There is an internal
composition operation
\[
	\comp^{B}_{A,C}: (B\illol A)\tn(C\illol B) \to C\illol A
\]
obtained by currying the composite
\[
	A\tn(B\illol A)\tn(C\illol B) \rTo^{{ev'}^{A}_{B}}
		B\tn(C\illol B) \rTo^{{ev'}^{B}_{C}} C,
\]
and a family of internal unit maps
\[
	u_{A}: I \to A\illol A
\]
obtained by currying the right-unit isomorphism $A\tn I\rTo^{\cong}A$.
The internal composition is associative, and the internal units are
units for internal composition, in the obvious sense.
%
In particular the object $A\illol A$ is a monoid
with multiplication given by internal composition, and unit $u_{A}$.
Below we shall denote this monoid $A^{A}$ for concision.)

Furthermore, for each object $A\in\V$ there is a natural transformation
\[
	t^{A}_{X,Y}: Y\illol X \to (A\tn Y)\illol(A\tn X)
\]
obtained by currying the maps
\[
	A\tn X\tn(Y\illol X) \rTo^{A\tn{\ev'}}^{X}_{Y} A\tn Y
\]
(which is dinatural in $A$, in addition to being natural in $X$ and $Y$).
We call this the \emph{internal tensor}, and it is preserves internal
identities and composition.

For each object $A\in\V$ there is a natural transformation
\[
	h^{A}_{X,Y}: Y\illol X \to (Y\illol A)\illol(X\illol A)
\]
obtained by currying the internal composition, which is again
dinatural in $A$ and preserves internal identities and composition.
Also, there is a natural transformation with components
\[
	d_{A,B,C}: (C\illol B)\tn A \to (C\tn A)\illol B
\]
obtained by currying the map
\[
	B\tn(C\illol B)\tn A \rTo^{{\ev'}^{B}_{C}\tn A} C\tn A.
\]

\begin{lemma}\label{lemma-module-as-monoid-map}
	A map $x: X\tn M\to X$ makes $(X,x)$ into a right $M$-module
	if and only if
	\[
		\curry'_{X,M,X}(x): M\to X\illol X
	\]
	is a map of monoids.
\end{lemma}
\begin{proof}
	Curry the diagrams in Definition~\ref{def-module}.
\end{proof}
%
In conjunction with the observations above, this lemma implies
that, for every $A\in\V$, if $(X,x)$ is a right $M$-module, there
is a natural way to make both $A\tn X$ and $X\illol A$ into right
$M$-modules too. For, since the $t^{A}$ and $H^{A}$ maps preserve
internal identities and composition, in particular
\[
	t^{A}_{X,X}: X\illol X \to (A\tn X)\illol(A\tn X)
\]
and
\[
	h^{A}_{X,X}: X\illol X \to (X\illol A)\illol(X\illol A)
\]
are maps of monoids, which may be composed with $\curry'_{X,M,X}(x)$.
Explicitly, the module structure of $A\tn X$ is given by the map
\[
	A\tn X\tn M \rTo^{A\tn x} A\tn X,
\]
and the module structure of $X\illol A$ is given by
\[
	(X\illol A)\tn M \rTo^{d_{M,A,X}} (A\tn M)\illol A \rTo^{x\illol A} X\illol A.
\]
\begin{lemma}\label{lemma-mod-map-curry}
	Let $(X,x)$ and $(Y,y)$ be right $M$-modules.
	If $f:A\tn X\to Y$ is a map of modules, then so is
	\[
		\curry'(f): X\to Y\illol A.
	\]
\end{lemma}
\begin{proof}
	Consider the diagram
	\begin{diagram}
		&& (A\tn X\illol A)\tn M & \rTo^{(f\illol A)\tn M} & (Y\illol A)\tn M\\
		&\ruTo^{{\coev'}^{A}_{X}\tn M} & \dTo>{d_{M,A,A\tn X}}
			& & \dTo>{d_{M,A,Y}} \\
		X\tn M & \rTo_{{\coev'}^{A}_{X,M}} & A\tn X\tn M \illol A
			& \rTo_{f\tn M\illol A} & Y\tn M\illol A \\
		\dTo<x && \dTo>{A\tn x\illol A} && \dTo>{x\illol A} \\
		X & \rTo_{{\coev'}^{A}_{X}} & A\tn X\illol A
			& \rTo_{f\illol A} & Y\illol A
	\end{diagram}
	The triangle commutes by definition of $d$, the upper square by
	naturality of $d$, the lower-left square by naturality of $\coev'$,
	and the lower-right square since $f$ is a map of modules. Hence
	the outside commutes, as required.
\end{proof}

With this background in place, it is easy to deduce the internal Cayley
Theorem from the external one:
\begin{proof}[Proof of Theorem~\ref{thm-1d-internal}]
	Let $A\in\V$, and $g: A\to M\lolli X$ be such that $hg=kg$.
	If we uncurry $g$, and recurry the resulting $g'$ with respect
	to the left internal hom, we obtain maps
	\[\begin{array}{l}
		g'\phantom{{}'}: A\tn M\to X, \\
		g'': M\to X\illol A.
	\end{array}\]
	The first thing to observe is that $g'$ is a map of modules, for
	we have \begin{diagram}[midvshaft,hug,w=5em]
		& & (M\lolli X)\tn M\tn M \\
		A\tn M\tn M & \ruTo[snake=-1em](2,1)^{f\tn M\tn M}
			&& \rdTo[snake=1em](2,1)^{\ev^{M}_{X}\tn M}
			& X\tn M \\
		& \rdTo[snake=-1em](2,1)_{f\tn M\tn M} & (M\lolli X)\tn M\tn M \\
		\dTo<{A\tn m} && \dTo>{(M\lolli X)\tn M} && \dTo>x \\
		A\tn M & \rTo_{f\tn M} & (M\lolli X)\tn M & \rTo_{\ev^{M}_{X}} & X
	\end{diagram}
	where the quadrilateral commutes by functoriality of tensor, and the
	hexagon commutes since $hg=kg$. Thus, by Lemma~\ref{lemma-mod-map-curry},
	$g''$ is also a map of modules.
	
	By the external Cayley Theorem, $g''$ is thus equal to
	\[
		M \rTo^{\cong} I\tn M \rTo^{f\tn M} (X\illol A)\tn M
			\rTo^{d_{M,A,X}} (X\tn M)\illol A \rTo^{x\illol A} X\illol A
	\]
	for some unique $f: I \to X\illol A$.
	%
	Hence $g'$ is equal to
	\[
		A\tn M \rTo^{f'\tn M} X\tn M \rTo^{x} X
	\]
	for some unique $f': A\to X$ (obtained by uncurrying $f$).
	%
	Thus $g$ is equal to
	\[
		A \rTo^{f'} X \rTo^{\curry_{X,M,X}(x)} M\lolli X
	\]
	for this unique $f'$, which is precisely the universal property of
	the equaliser diagram.
\end{proof}
\begin{remark}
	The external theorem can be obtained as a corollary of the internal
	one, under an additional assumption on $\V$. The additional assumption is
	that the functor $\V(I,-); \V\to\Set$ should preserve equalisers. In
	particular, this is always so if $\V$ has small coproducts, because in that
	case $\V$ is \emph{tensored} \citep[in the sense of][section~2.7]{KellyEnriched}
	as a $\Set$-category, and so the functor $\V(I,-)$ has a left adjoint.
	Then we can obtain the external theorem simply by applying the functor
	$\V(I,-)$ to the equaliser diagram of the internal theorem.
\end{remark}

\bibliography{cs}
\end{document}