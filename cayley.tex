%!TEX TS-program = latex
\documentclass{robinthesisdraft}
\usepackage{robincs,thesisdefs,xr}

\title{Cayley's Theorem for Pseudomonoids}
\begin{document}
\maketitle

In its traditional form, Cayley's theorem says that every finite group
is (isomorphic to) a subgroup of one of the symmetric groups. What the
proof actually shows is that every monoid $M$ is a submonoid of the monoid
of endofunctions $|M|\to|M|$, via the function that maps the element $x\in M$
to the function $x\o-$.

A `Cayley Theorem for monoidal categories' has been used by
\citet[Proposition~1.3]{BTC} to show that every monoidal category
is monoidally equivalent to a \emph{strict} monoidal category.
But it turns out that we need the corresponding theorem for
promonoidal categories; and rather than just prove another special
case, it seems wise to generalise. So the purpose of this
chapter is to state and prove a Cayley Theorem for pseudomonoids.

As motivation, we begin by reviewing the one-dimensional case,
of monoids in a monoidal category.

\section{Cayley's theorem for monoids}

I'm not aware of any literature that specifically addresses the
question of giving an analogue of Cayley's theorem for monoids
in a monoidal category, but it is a special case of a standard
construction. A monoid in the monoidal category $\V$ can be
regarded as a one-object $\V$-enriched category.\footnote{This
operation -- reimagining a monoid as a one-object category -- is often
called \emph{suspension}.} Then we can apply the enriched Yoneda
Lemmas of \citet[sections~1.9 and~2.4]{KellyEnriched} to this
$\V$-category. Since it requires some effort to extract the
specifics of the one-object case from Kelly's constructions,
we give an overview here.

Let $\V$ be a monoidal category, and let $M$ be a monoid in $\V$,
with unit $e: I\to M$ and multiplication $m: M\tn M\to M$.
%
% In the pseudomonoid case, we shall work in a Gray-monoid and use
% coherence to transfer the result to an arbitrary monoidal bicategory.
% Since this process is not entirely trivial, we shall illustrate
% the process in the one-dimensional case, where the analogous
% strategy would be to work in a \emph{strict} monoidal category $\V$.
% Thus we shall assume, for the moment, that $\V$ is strictly
% associative. (We could also take the unit to be strict, of course,
% but we do not, since it doesn't particularly help.)

The Cayley-Yoneda theorems are best stated in terms of modules:
\begin{definition}\label{def-module}
	A \emph{right $M$-module} consists of an object $X\in\V$ and a map
	$x: X\tn M\to X$ such that the diagrams
	\[
	\begin{diagram}
		X\tn M\tn M & \rTo^{x\tn M} & X\tn M \\
		\dTo<{X\tn m} && \dTo>{x} \\
		X\tn M & \rTo_{x} & X
	\end{diagram}
	\quad\mbox{and}\quad
	\begin{diagram}
		X &\rTo^{\cong} & X\tn I \\
		\dTo<1 && \dTo>{X\tn e} \\
		X & \lTo_{x} & X\tn M
	\end{diagram}
	\]
	commute.
\end{definition}
\begin{remark}
	Note that $(M,m)$ is a right $M$-module.
\end{remark}
\begin{definition}\label{def-module-map}
	Let $(X,x)$ and $(Y,y)$ be right M-modules.
	A \emph{morphism of modules} from $X$ to $Y$ is
	a map $f: X\to Y$ for which the diagram
	\begin{diagram}
		X\tn M & \rTo^{x} & X \\
		\dTo<{f\tn M} && \dTo>f \\
		Y\tn M & \rTo_{y} & Y
	\end{diagram}
	commutes. The set of such maps will be denoted
	$\Mod_{M}(X,Y)$.
\end{definition}
Kelly proves two versions of his Yoneda Lemma, which he refers
to as `weak' and `strong'. We shall call our corresponding theorems
the `external' and `internal' Cayley Theorem, respectively. The
internal theorem requires some additional properties of $\V$, but
the external one is perfectly general.

\subsection{The external Cayley Theorem}
\begin{thm}[External Cayley]\label{thm-1d-external}
	For every right $M$-module $X$, the natural transformation
	\[
		\phi_{X}: \Mod_{M}(M, X) \to \V(I, X),
	\]
	defined as $\phi_{X}(f) = f\cdot e$, is invertible with
	$\phi_{X}^{-1}(z)$ equal to the composite
	\[
		M \rTo^{\cong} I\tn M \rTo^{z\tn M} X\tn M \rTo^{x} X.
	\]
\end{thm}
\begin{proof}
	Let $f: M\to X$ be a map of modules. We'll first show that
	$\phi_{X}^{-1}(\phi_{X}(f)) = f$: consider the diagram
	\begin{diagram}
		M & \rTo^{\cong} & I\tn M & \rTo^{e\tn M} & M\tn M & \rTo^{f\tn M} & X\tn M \\
		&&& \rdTo_{\cong} & \dTo>m && \dTo>x \\
		&&&& M & \rTo_{f} & X
	\end{diagram}
	The triangle commutes by the left-unit law for the monoid $M$,
	and the square because $f$ is a map of modules. The upper edge
	is $\phi_{X}^{-1}(\phi_{X}(f))$ by definition, and the lower edge is
	equal to $f$.
	
	Now let $z: I\to X$, and consider the diagram
	\begin{diagram}
		&&X \\
		&\ruTo^{z} && \rdTo^{\cong} \\
		I & \rTo_{\cong} & I\tn I & \rTo_{z\tn I}& X\tn I \\
		\dTo<e && \dTo<{I\tn e} && \dTo>{X\tn e} \\
		M & \rTo_{\cong} & I\tn M & \rTo_{z\tn M} & X\tn M & \rTo_{x} & X
	\end{diagram}
	The cells commute by naturality and functoriality of tensor,
	and by the unit condition in the definition of module the upper
	edge is equal to $z$. The lower edge is $\phi_{X}(\phi_{X}^{-1}(z))$
	by definition. Thus $\phi_{X}$ and $\phi_{X}^{-1}$ are indeed
	mutually inverse.
	
	We must also show that $\phi_{X}^{-1}(z)$ is a map of modules,
	so let $z:I\to X$ and consider the diagram
	\begin{diagram}
		M\tn M & \rTo^{\cong} & I\tn M\tn M & \rTo^{z\tn M\tn M} & X\tn M\tn M
			& \rTo^{x\tn M} & X\tn M \\
		\dTo<m &&\dTo>{I\tn m} && \dTo>{X\tn m} && \dTo>m \\
		M & \rTo_{\cong} & I\tn M & \rTo_{z\tn M} & X\tn M & \rTo_{x} & X
	\end{diagram}
	The squares commute, from left to right, by the naturality of the
	left-unit isomorphism, the functoriality of tensor, and the fact
	that $X$ is a module. Thus the outer edge commutes, showing that
	$\phi_{X}^{-1}(z)$ is a map of modules.
\end{proof}
\begin{remark}
	More can be said about $\phi_{X}$, as follows. Each of the sets
	$\Mod_{M}(M, X)$ and $\V(I,X)$ is a right $\V(I,M)$-module
	in a natural way, using the $M$-module structure of $X$.
	Then $\phi_{X}$ is a map of modules with respect to these
	module structures. We omit\footnote{An earlier version of
	this chapter included these details: I removed them, as an
	unneeded distraction.} the details of this, but they are
	quite routine.
\end{remark}
\begin{remark} % \V(I,M) is a monoid
	The set $\V(I, M)$ inherits the monoid structure from $M$:
	its unit is $u:I\to M$, and the product of $a$, $b:I\to M$
	is the composite
	\[
		I \rTo^{\cong} I\tn I \rTo^{a\tn b} M\tn M \rTo^{m} M.
	\]
	This is the \emph{underlying ordinary monoid}
	of the abstract monoid $M$.
\end{remark}

To relate Theorem~\ref{thm-1d-external} to the ordinary
Cayley Theorem, take the module $X$ to be $M$ itself.
Then we have that $\V(I, M)$, the underlying ordinary
monoid of $M$, is isomorphic to $\Mod_{M}(M,M)$, which
is a submonoid of $\V(M,M)$.
%
We should like to be able to say that this $\phi_{M}$
is an isomorphism of monoids, i.e.\ that it preserves
the monoid structure.
\begin{propn}
	The isomorphism
	\[
		\phi_{M}: \Mod_{M}(M,M) \to \V(I, M)
	\]
	is a map of monoids.
\end{propn}
\begin{proof}
	For the unit, $\phi_{M}(1_{M}) = 1_{M}\cdot e = e$.
	For the multiplication, let $f$, $g: M\to M$ be right
	module morphisms: we must show that $\phi_{M}(f\cdot g)$:
	\[
		I \rTo^{e} M \rTo^{g} M \rTo^{f} M
	\]
	is equal to $\phi_{M}(f)\tn\phi_{M}(g)$:
	\[
		I \rTo^{\cong} I\tn I \rTo^{(f\cdot e)\tn(g\cdot e)} M\tn M \rTo^{m} M.
	\]
	Consider the diagram
	\begin{diagram}
		I\tn I & \rTo^{I\tn (g\cdot e)} & I\tn M  & \rTo^{e\tn M} & M\tn M
			 & \rTo^{f\tn M} & M\tn M \\
		\uTo<\cong && \uTo<\cong &\rdTo^{\cong} & \dTo>m && \dTo>m \\
		I  & \rTo_{g\cdot e} & M & \rTo_{1} & M & \rTo_{f} & M
	\end{diagram}
	in which the right-hand square commutes because $f$ is a map of right $M$-modules,
	and the upper triangle by the right-unit law for $M$.
	By the functoriality of tensor, the top edge is equal to $(f\cdot e)\tn(g\cdot e)$,
	so the upper outer edge is $\phi_{M}(f)\tn\phi_{M}(g)$ and the lower
	edge is $\phi_{M}(f\cdot g)$, which are therefore equal as required.
\end{proof}

% \subsection{Coherence issues}
% We have shown that the external Cayley Theorem is true for strict
% monoidal categories, and would like to conclude that it is therefore
% true for arbitrary monoidal categories. This is, of course, overkill
% in the present situation, where it would be trivial to modify our
% proofs to take account of XXXX
% 
\subsection{The internal Cayley Theorem}
The theorem above is external in the sense that it essentially
concerns $\V(I, M)$, the underlying ordinary monoid of $M$. If
the monoidal category $\V$ is closed, then we can state an
internal version, purely in terms of arrows in $\V$ itself.
%
So let $\V$ be biclosed, i.e.\ suppose we are given
functors
\[\begin{array}{l}
	\lolli: \V\op\times\V\to\V \\
	\illol: \V\times\V\op\to\V,
\end{array}\]
and natural isomorphisms with components
\[\begin{array}{l}
	\curry_{A,B,C}:  \V(A\tn B, C) \cong \V(A, B\lolli C), \\
	\curry'_{A,B,C}: \V(A\tn B, C) \cong \V(B, C\illol A).
\end{array}\]
Then the theorem is:
\begin{thm}[Internal Cayley]\label{thm-1d-internal}
	Let $(X,x)$ be a right $M$-module. Then the diagram
	\begin{diagram}
		X & \rTo^{\curry_{X,M,X}(x)} & M\lolli X & \pile{\rTo^{h} \\ \rTo_{k}}
		 & (M\tn M)\lolli X
	\end{diagram}
	is a coequaliser diagram, where the maps $h$ and $k$ are defined as
	follows. The map $h$ is obtained by currying
	\[
		(M\lolli X)\tn M\tn M \rTo^{\ev^{M}_{X}\tn M} X\tn M \rTo^{x} X,
	\]
	and $k$ is obtained by currying
	\[
		(M\lolli X)\tn M\tn M \rTo^{(M\lolli X)\tn m} (M\lolli X)\tn M \rTo^{\ev^{M}_{X}} X,
	\]
	where the map $\ev^{M}_{X}$ is the result of uncurrying the identity
	on $M\lolli X$.
\end{thm}
%
Although the statement of the theorem uses only the right-closed
structure $\lolli$, the left-closed structure $\illol$ plays a
crucial role in the proof. We will not give a detailed proof here%
\footnote{Again, I did type out a detailed proof. So this is not
	just an excuse for laziness!},
since this is something of a digression, but explain the outline.
The preliminary observations are that:
\begin{enumerate}
	\item For every object $X\in\C$, the object $X\illol X$
	is a monoid in a canonical way, via the definable `internal unit'
	and `internal composition' maps:
	\[\begin{array}{l}
		u: I \to X\illol X, \\
		m: (X\illol X)\tn(X\illol X) \to X \illol X.
	\end{array}\]
	\item A right $M$-module is
	precisely a map of monoids $M\to X\illol X$ for some $X$.
	\item It follows
	from this that a right $M$-module structure on $X$ induces a canonical
	right $M$-module structure on $A\tn X$ for each object $A\in \V$, because
	there is an `internal tensor' map
	\[
		X\illol X \rTo (A\tn X)\illol(A\tn X),
	\]
	which is compatible with the internal unit and compositions.
	\item In a similar
	way, a right $M$-module structure on $X$ induces a canonical
	right $M$-module structure on $X\illol A$, for each object $A$.
\end{enumerate}
Our second observation is trivial to prove: simply curry the diagrams
in Definition~\ref{def-module}. The others are proved in \citet[Section~1.6]{KellyEnriched}.
%
Now for the proof itself: suppose we are given an object $A\in\V$ and
a map $g: A\to M\lolli X$ such that $hg=kg$. We can curry $g$ to get a map
\[
	g': A\tn M \to X,
\]
and it is not hard to check that, since $hg=kg$, this $g'$ is a
map of modules from $A\tn M$ to $X$; where the module structure on
$A\tn M$ is obtained from that of $M$ by point (3) above. Then we
curry $g'$ with respect to the left-closure, to obtain a map
\[
	g'': M \to X\illol A,
\]
which is a map of modules with respect to the module
structure on $X\illol A$ obtained from (4). By the
external Cayley Theorem, $g''$ is thus equal to
\[
	M \rTo^{\cong} I\tn M \rTo^{f\tn M} (X\illol A)\tn M
		\rTo X\illol A
\]
for some unique $f: I \to X\illol A$, where the unlabelled
arrow above comes from the $M$-module structure on $X\illol A$.
From this it can be deduced that $g'$ is equal to
\[
	A\tn M \rTo^{f'\tn M} X\tn M \rTo^{x} X
\]
for some unique $f': A\to X$ (obtained by uncurrying $f$).
%
Thus $g$ is equal to
\[
	A \rTo^{f'} X \rTo^{\curry_{X,M,X}(x)} M\lolli X
\]
for this unique $f'$, which is precisely the universal property of
the equaliser diagram.
\begin{remark}
	Above we have deduced the internal theorem from the external one.
	Conversely, the external theorem can be obtained as a corollary of the internal
	one, under an additional assumption on $\V$. The additional assumption is
	that the functor $\V(I,-); \V\to\Set$ should preserve equalisers. In
	particular, this is always so if $\V$ has small coproducts, because in that
	case $\V$ is \emph{tensored} \citep[in the sense of][Section~2.7]{KellyEnriched}
	as a $\Set$-category, and so the functor $\V(I,-)$ has a left adjoint.
	Then we can obtain the external theorem simply by applying the functor
	$\V(I,-)$ to the equaliser diagram of the internal theorem.
\end{remark}

\section{Cayley's Theorem for Pseudomonoids}
Both these theorems, internal and external, admit generalisation to
the higher-dimensional setting. But since it is sufficient for our
applications, we shall largely confine ourselves to the external version,
with just some brief remarks about the internal one.

\subsection{Modules for pseudomonoids}
First we must give a suitable definition of $\C$-module, where $\C$ is
a pseudomonoid. In fact, we want to define a bicategory $\Mod_{\C}$ of
right $\C$-modules.
\begin{definition}\label{def-psmod} % Right \C-module
	A \emph{right $\C$-module} $(X,x)$ for the pseudomonoid $\C$
	consists of a 1-cell $x: X\tn\C\to X$ together with invertible
	2-cells
	\begin{diagram}[h=2em]
		&&X\tn\C\\
		&\ruTo^{X\tn J}\\
		X\tn\I & \Arr\Nearrow{\chi^{J}_{X}} & \dTo>x \\
		&\rdTo_{r_{X}} \\
		&&X
	\end{diagram}
	and
	\begin{diagram}
		X\tn(\C\tn\C) & \rTo^{a_{X,\C,\C}} & (X\tn\C)\tn\C
			& \rTo^{x\tn\C} & X\tn\C \\
		\dTo<{X\tn P} && \Nearrow{\chi^{P}_{X}} && \dTo>x\\
		X\tn\C && \rTo_{x} && X
	\end{diagram}
	such that the following two equations hold:
	\[
		\begin{diagram}[tight,vtrianglewidth=1.7em,labelstyle=\scriptstyle]
			&&X\tn\C^{3} && \rTo^{x\tn\C^{2}} && X\tn\C^{2} \\
			&\ldTo^{X\tn\C\tn P} && \rdTo^{X\tn P\tn\C}
			 	&&\Arr\Nearrow{\scriptstyle\chi^{P}_{X}\tn\C} && \rdTo^{x\tn\C} \\
			X\tn\C^{2} && \Right_{\displaystyle X\tn\aa} && X\tn\C^{2}
				&& \rTo^{x\tn\C} && X\tn\C \\
			&\rdTo_{X\tn P} && \ldTo_{X\tn P}
				&& \Arr\Nearrow{\chi^{P}_{X}} && \ldTo_{x} \\
			&&X\tn\C && \rTo_{x} && X
		\end{diagram}
		=
		\begin{diagram}[tight,vtrianglewidth=1.7em,labelstyle=\scriptstyle]
			&&X\tn\C^{3} && \rTo^{x\tn\C^{2}} && X\tn\C^{2} \\
			& \ldTo^{X\tn\C\tn P} &&\sim&&\ldTo^{X\tn P}
				&& \rdTo^{x\tn\C} \\
			X\tn\C^{2} && \rTo^{x\tn\C} && X\tn\C
				&& \Right_{\displaystyle\chi^{P}_{X}} && X\tn\C \\
			&\rdTo_{X\tn P} && \Arr\Nearrow{\chi^{P}_{X}}
				&& \rdTo_{x} && \ldTo_{x} \\
			&&X\tn\C && \rTo_{x} && X
		\end{diagram}
	\]
	and
	\[
		\begin{diagram}[w=4em]
			X\tn \I\tn \C & \rTo^{X\tn J\tn \C} & X\tn \C^{2}
				& \rTo^{x\tn \C} & X\tn \C \\
			&\rdTo_{1}\raise 1em\hbox to1em{$\Arr\Nearrow{X\tn\ll}$\hss}
				& \dTo>{X\tn P} & \Arr\Nearrow{\chi^{P}_{X}} & \dTo>x \\
			&&X\tn \C & \rTo_{x} & X
		\end{diagram}
		=
		\begin{diagram}[w=4em]
			X\tn \I\tn \C & \rTo^{X\tn J\tn \C} & X\tn \C^{2} \\
			&\rdTo_{1} \raise 1em\hbox to 1em{$\Arr\Nearrow{\chi^{J}_{X}\tn\C}$\hss}
				& \dTo>{x\tn \C} \\
			&&X\tn \C\\
			&&\dTo>x \\
			&&X
		\end{diagram}
	\]
\end{definition}
\begin{definition}\label{def-psmod-map} % Map of modules
	Given right $\C$-modules $(X,x)$ and $(Y,y)$, a
	\emph{map of $\C$-modules} from $X$ to $Y$ consists
	of a 1-cell $f: X\to Y$ together with an invertible 2-cell
	\begin{diagram}
		X\tn \C & \rTo^{f\tn \C} & Y\tn \C \\
		\dTo<x & \Arr\Nearrow{\phi^{f}} & \dTo>y \\
		X & \rTo_{f} & Y
	\end{diagram}
	such that
	\[
		\begin{diagram}[tight,vtrianglewidth=1.7em,labelstyle=\scriptstyle]
			&&X\tn\C^{2} && \rTo^{f\tn\C^{2}} && Y\tn\C^{2} \\
			&\ldTo^{X\tn P} && \rdTo^{X\tn\C}
			 	&&\Arr\Nearrow{\scriptstyle\phi^{f}\tn\C} && \rdTo^{y\tn\C} \\
			X\tn\C && \Right_{\displaystyle \chi^{P}_{X}} && X\tn\C
				&& \rTo^{f\tn\C} && Y\tn\C \\
			&\rdTo_{x} && \ldTo_{x}
				&& \Arr\Nearrow{\phi^{f}} && \ldTo_{y} \\
			&&X && \rTo_{f} && X
		\end{diagram}
		=
		\begin{diagram}[tight,vtrianglewidth=1.7em,labelstyle=\scriptstyle]
			&&X\tn\C^{2} && \rTo^{f\tn\C^{2}} && Y\tn\C^{2} \\
			& \ldTo^{X\tn P} &&\sim&&\ldTo^{Y\tn P}
				&& \rdTo^{y\tn\C} \\
			X\tn\C && \rTo^{f\tn\C} && Y\tn\C
				&& \Right_{\displaystyle\chi^{P}_{Y}} && X\tn\C \\
			&\rdTo_{x} && \Arr\Nearrow{\phi^{f}}
				&& \rdTo_{y} && \ldTo_{y} \\
			&&X && \rTo_{f} && Y
		\end{diagram}
	\]
	and
	\[
		\begin{diagram}[h=2em]
			&\rnode{XI}{X\tn \I} & \rTo^{f\tn\I} & Y\tn \I \\
			&\dTo<{X\tn J} & \sim & \dTo>{Y\tn J} \\
			\Right_{\chi^{J}_{X}} & X\tn \C & \rTo^{f\tn \C} & Y\tn \C \\
			&\dTo<x & \Arr\Nearrow{\phi^{f}} & \dTo>y \\
			&\rnode{X}{X} & \rTo_{f} & Y
			%
			\ncarc[arcangle=-90,ncurv=1.3]{->}{XI}{X}\Bput{1}
		\end{diagram}
		=
		\begin{diagram}[h=2em]
			X\tn \I & \rTo^{f\tn \I} & Y\tn \I \\
			&&&\rdTo^{Y\tn J} \\
			\dTo<1 & & \dTo<1 & \Right_{\chi^{J}_{Y}} & Y\tn \C \\
			&&&\ldTo_{y} \\
			X &\rTo_{f} & Y
		\end{diagram}
	\]
\end{definition}
\begin{definition}\label{def-psmod-bicat} % The bicategory Mod_\C
	The bicategory $\Mod_{\C}$ of right $\C$-modules is defined
	as follows. An object is a right $\C$-module, a 1-cell is a
	map of right $\C$-modules, and a 2-cell $\gamma: f\To g: X\to Y$
	is a 2-cell $f\To g$ in $\B$ such that
	\[
		\begin{diagram}[w=4em]
			& \raise -2em\hbox to 0pt{\hss$\Arr\Uparrow{\gamma\tn\C}$} \\
			\rnode{XC}{X\tn \C} & \rTo_{f\tn\C} & \rnode{YC}{Y\tn \C} \\
			\dTo<x & \Arr\Nearrow{\phi^{f}} & \dTo>y \\
			X & \rTo_{f} & Y \\
			%
			\ncarc[arcangle=50]{->}{XC}{YC}\Aput{g\tn\C}
		\end{diagram}
		\qquad=\qquad
		\begin{diagram}[w=4em]
			\\
			\rnode{XC}{X\tn \C} & \rTo^{g\tn\C} & \rnode{YC}{Y\tn \C} \\
			\dTo<x & \Arr\Nearrow{\phi^{g}} & \dTo>y \\
			\rnode{X}{X} & \rTo^{g} & \rnode{Y}{Y} \\
			& \raise 2em\hbox{$\Arr\Uparrow{\gamma}$}
			%
			\ncarc[arcangle=-60]{->}{X}{Y}\Bput{f\tn\C}
		\end{diagram}
	\]
	Composition is defined as in $\B$, and given $f: X\to Y$
	and $g: Y\to Z$, the 2-cell $\phi^{gf}$ is the pasting
	\begin{diagram}
		X\tn \C & \rTo^{f\tn \C} & Y\tn \C
			& \rTo^{g\tn \C} & Z\tn \C\\
		\dTo<x & \Arr\Nearrow{\phi^{f}} & \dTo>y 
			& \Arr\Nearrow{\phi^{g}} & \dTo>z \\
		X & \rTo_{f} & Y & \rTo_{g} & Z
	\end{diagram}
	which is easily verified to satisfy the necessary equations.
\end{definition}

\subsection{The case $\B=\Cat$}
Since these higher modules are perhaps unfamiliar, we shall
unpack the definitions in the familiar case $\B=\Cat$.
Let $\C$ be a monoidal category; then a right $\C$-module
is a category $\X$ together with a functor
\[
	\bullet: \X\times\C\to\X,
\]
and natural isomorphisms
\[
	\alpha^{\bullet}_{X,A,B}: X\bullet(A\tn B) \to (X\bullet A)\bullet B
\]
and
\[
	\rho^{\bullet}_{X}: X\bullet I \to X
\]
where we take $X$ to be a (generic) element of $\X$,
and $A$, $B$ to be elements of $\C$.
%
The coherence conditions say that the diagrams
\begin{diagram}
	X\bullet(A\tn(B\tn C))
		& \rTo^{\alpha^{\bullet}} & (X\bullet A)\bullet(B\tn C)
		& \rTo^{\alpha^{\bullet}} & ((X\bullet A)\bullet B)\bullet C \\
	\dTo<{X\bullet\alpha} &&&& \uTo>{\alpha^{\bullet}\bullet C} \\
	X\bullet((A\tn B)\tn C) && \rTo_{\alpha^{\bullet}}
		&& (X\bullet(A\tn B))\bullet C
\end{diagram}
and
\begin{diagram}[vtriangleheight=3em]
	X\bullet(I\tn A) && \rTo^{\alpha^{\bullet}} && (X\bullet I)\bullet A \\
	& \rdTo[snake=-1ex]_{X\bullet\lambda_{A}}
		&& \ldTo[snake=1ex]_{\rho^{\bullet}_{X}\bullet A} \\
	&& X\bullet A
\end{diagram}
must commute, for all $X\in\X$ and $A$, $B$, $C\in\C$.
%
It is clear that, in particular, $\C$ itself is a right
$\C$-module, when equipped in the obvious way; a fact which
is furthermore easily seen to be true in the general setting.

Now, if we have two right $\C$-modules $\X$ and $\Y$, a
map of modules $F: \X\to\Y$ is a functor $F$ equipped with
a natural isomorphism
\[
	F^{\bullet}_{X,A}: F(X\bullet A)\to FX\bullet A
\]
such that the diagrams
\begin{diagram}
	F(X\bullet(A\tn B)) && \rTo^{F^{\bullet}_{X,A\tn B}}
		&& FX\bullet(A\tn B) \\
	\dTo<{F(\alpha^{\bullet}_{X,A,B})}
		&&&& \dTo>{\alpha^{\bullet}_{FX,A,B}} \\
	F((X\bullet A)\bullet B) & \rTo_{F^{\bullet}_{X\bullet A,B}}
		& F(X\bullet A)\bullet B & \rTo_{F^{\bullet}_{X,A}\bullet B}
		& (FX\bullet A)\bullet B
\end{diagram}
and
\begin{diagram}[vtriangleheight=2.5em]
	F(X\bullet I) && \rTo^{F^{\bullet}_{X,I}} && FX\bullet I\\
	&\rdTo[snake=-1ex]_{F(\rho^{\bullet}_{X})}
		&& \ldTo[snake=1ex]_{\rho^{\bullet}_{FX}} \\
	&&FX
\end{diagram}
commute.

\subsection{Remarks on the unit conditions}\label{s-unit-remarks}
The definitions above are conventional in style. However,
we have the same sort of redundancy that we saw, for pseudofunctors
and pseudonatural transformations, in Section~\chref{Bicats}{s-identities}.
This time, the redundancy turns out to be essential to some of our later
results.

For a start, we have:
\begin{propn}
	The unit condition of Definition~\ref{def-psmod-map}
	is redundant.
\end{propn}
\begin{proof}
	In the situation $\B=\Cat$, the proof is via the diagram
	\begin{diagram}
	F(X\bullet(I\tn A)) && \rTo^{F^{\bullet}_{X,I\tn A}}
		&& FX\bullet(I\tn A) \\
	& \rdTo(0,4)<{F(\alpha^{\bullet}_{X,I,A})}
		\rdTo[snake=1em](1,2)^{F(X\bullet\lambda_{A})}
		& & \ldTo[snake=-1em](1,2)^{FX\bullet\lambda_{A}}
		\ruTo(0,4)<{\alpha^{\bullet}_{FX,I,A}} \\
		& F(X\bullet A) & \rTo^{F^{\bullet}_{X,A}} & FX\bullet A \\
	 \ruTo[snake=.5em](1,2)_{F(\rho^{\bullet}_{X}\bullet A)}
		&& \ruTo[snake=-.5em](1,2)<{F(\rho^{\bullet}_{X})\bullet A}
		&& \luTo(1,2)<{\rho^{\bullet}_{FX}\bullet A} \\
	F((X\bullet I)\bullet A) & \rTo_{F^{\bullet}_{X\bullet I,A}}
		& F(X\bullet I)\bullet A & \rTo_{F^{\bullet}_{X,I}\bullet A}
		& (FX\bullet I)\bullet A
	\end{diagram}
	We observe that, since the outside commutes and all the
	other cells commute, and since the arrows are all invertible,
	the lower triangle must also commute. Then let $A=I$ and use
	the fact that the functor $-\bullet I$ is full and faithful.
	
	The same proof works in the general setting: 
\end{proof}

\bibliography{cs}
\end{document}