%!TEX TS-program = latex
\documentclass{robincs}
\usepackage{prooftree,rotating,amsbsy}
\bibliographystyle{plainnat}

\makeatletter

%% Patch up a minor bug in the 'rotating' package:
\let\orig@end@rotfloat = \end@rotfloat
\def\end@rotfloat{\orig@end@rotfloat\ignorespacesafterend}
\let\endsidewaysfigure\end@rotfloat
\let\endsidewaystable\end@rotfloat

%% Just emphasise definitions, don't do any fancy stuff.
\let\defn = \emph

%% Augment the \using command in prooftree.sty with a slightly more
%  helpful version. For example, \usingrule \tn r for the `tensor right' rule.
\def\usingrule#1#2{\proofusing{\mathord{#1}\mkern.5mu\textsc{#2}}}

%% This allows you to write $\alfa ABC$ rather than $\alpha_{A,B,C}$,
%  which saves typing and makes the code more readable.
\newcommand\alfa[3]{\alpha_{#1,#2,#3}}

%% Use small integral signs, even in displayed equations:
\let\origint = \int
\def\int{{\textstyle\origint}}

%% Curry
\newcommand\curry{\mathrm{curry}}

%% Lin  (Should we do something fancy to make $\Lin_\D$ DTRT?)
\newcommand\Lin[1][\C]{\mathrm{Lin}_{#1}}

%% Proposition 1.2, etc.
\renewcommand\thepropn{\thesection.\arabic{propn}}
\@addtoreset{propn}{section}

%% Parenthesised reference.
\newcommand\pref[1]{\textup(\ref{#1}\textup)}

%% For labelling diagrams.
%  (The two-part definition of \dnum means that the first token
%  of its expansion is \hbox, so \raise1em\dnum will work.)
\newcommand\dlabel[2][]{\def\arg{#1}%
  \ifx\arg\@empty
    \refstepcounter{equation}%
        \global\let\robin@dnum = \theequation
  \else
    \def\@currentlabel{#1}%
        \gdef\robin@dnum{#1}%
  \fi
  \label{#2}}
\newcommand\dnum{\hbox\bgroup\morednum}
\newcommand\morednum[1][]{\def\arg{#1}\textup(%
        \ifx\arg\@empty%
                \robin@dnum%
        \else
                \ref{#1}%
        \fi
        \textup)\egroup}

%% For making diagrams with overlapping columns.
%  (used in (pentagon))
\def\spleft#1{\hbox to 2em{\hss$#1$}}
\def\spright#1{\hbox to 2em{$#1$\hss}}

%% Centred math environment, spilling into the margins if needed
\newenvironment{mspill}{%
        \vskip\abovedisplayskip\hbox to \textwidth\bgroup\hss%
        \hbox\bgroup$\displaystyle
}{%
        $\egroup\hss\egroup\vskip\belowdisplayskip\noindent\ignorespacesafterend
}

%% Set off an important fact
\newenvironment{setoff}{\@startsection{paragraph}{4}{\z@}%
                {1ex}%
                {-.5em}%
                {\normalfont\normalsize\bfseries}*}
        {\vskip1ex\noindent\ignorespacesafterend}

%%
\renewcommand\paragraph{\@startsection{paragraph}{4}{\z@}%
                {1ex}{-.5em}{\normalfont\normalsize\bfseries}}


\makeatother

%% Custom hyphenation for obscure words :-)
\hyphenation{pro-mon-oid-al aut-on-om-ous}

%% Should we show incomplete sections?
\newif\ifincomplete
% (not for now)
\incompletefalse

%% Changeable terminology:
\newcommand\staraut{star-aut\-on\-om\-ous\xspace}
%
\newcommand\sm    {semi-mon\-oid\-al\xspace}
\newcommand\smc   {semi-mon\-oid\-al category\xspace}
\newcommand\smcs  {semi-mon\-oid\-al categories\xspace}
\newcommand\ssm   {symmetric semi-mon\-oid\-al\xspace}
\newcommand\ssmc  {symmetric semi-mon\-oid\-al category\xspace}
\newcommand\ssmcs {symmetric semi-mon\-oid\-al categories\xspace}
\newcommand\Ssmcs {Symmetric Semi-mon\-oid\-al Categories\xspace}
\newcommand\ssmcc {symmetric semi-mon\-oid\-al closed category\xspace}
\newcommand\ssmccs{symmetric semi-mon\-oid\-al closed categories\xspace}
\newcommand\Ssmccs{Symmetric Semi-mon\-oid\-al Closed Categories\xspace}

\newcommand\ssta  {semi star-aut\-on\-om\-ous\xspace}
\newcommand\stac  {star-aut\-on\-om\-ous category\xspace}
\newcommand\sstac {semi star-aut\-on\-om\-ous category\xspace}
\newcommand\sstacs{semi star-aut\-on\-om\-ous categories\xspace}

\newcommand\SEC  {\textsc{ssmc}\xspace}
\newcommand\SMCC {\textsc{smcc}\xspace}
\newcommand\SECC {\textsc{ssmcc}\xspace}
\newcommand\SECCs{\textsc{ssmcc}s\xspace}

%% added by Dominic for Intro
\newif\ifcomments\commentsfalse
\newcommand\comment[2][]{\ifcomments\marginpar{#1\raggedright\tiny#2}\fi}
\newcommand\acomment{\comment[A]}% comments by Andrea
\newcommand\dcomment{\comment[D]}% comments by Dominic
\newcommand\rcomment{\comment[R]}% comments by Robin
%
\newcommand\mll  {MLL$^-$\xspace}
%\newcommand\axiomlabel{\textsf{ax}}
\renewcommand\perp{^\bot}
\newcommand\perpp{\perp{}\perp}
\newcommand\dual[1]{#1\perp}
%\newcommand\ie{\emph{i.e.}}
%\newcommand\eg{\emph{e.g.}}

% display box
\newlength{\tw}\setlength{\tw}{\textwidth}\addtolength{\tw}{-\arrayrulewidth}
\addtolength{\tw}{-2\tabcolsep}\addtolength{\tw}{-1cm}
\newenvironment{plainbox}{\begin{center}\begin{tabular}{@{}|c@{\hspace{1ex}}|@{}}\hline
 \begin{minipage}[b]{\tw}\vspace*{.1in}
 \begingroup\small\baselineskip13pt}{\endgroup\end{minipage} \\[1ex] \hline
 \end{tabular}\end{center}}


% Preamble ends.
%%

\author{Robin Houston, Dominic Hughes and Andrea Schalk}
\title{Modelling Linear Logic without Units\\(Preliminary Results)}

\begin{document}
\maketitle

\section{Introduction}\label{intro}

Proof nets for \mll, unit-free Multiplicative Linear Logic
\citep{Girard87}, provide elegant, abstract representations of
% sequent calculus 
proofs.
% modulo rule commutations.
%
Cut-free \mll proof nets form a category, under path
composition\footnote{Path composition coincides with
normalisation by cut elimination \citep{Girard87}.}, in the manner of
Eilenberg-Kelly-Mac~Lane graphs \citep{EK66,KM71}.
%
Objects are formulas of \mll, and a morphism $A\to B$, a proof
net from $A$ to $B$, is a linking or matching between complementary
leaves (occurrences of variables).  For example, here is a morphism
from $p$ (a variable) to $p\otimes (q\tensor \dual{q})\perp$,
\[\thicklines
\begin{array}{l@{\hspace{10ex}}l}
        p\begin{picture}(0,0)\put(-3,-3){\line(0,-1){27}}\end{picture}
        \\[5ex]
        p\otimes(q%
        \begin{picture}(0,0)%
                \put(-2,9){\qbezier(0,0)(9,15)(18,0)}%
        \end{picture}%
        \otimes \dual{q})\perp
\end{array}\]
and here is an example of post-composing this proof net with another
from $p\otimes (q\tensor \dual{q})\perp$ to $\big((p\otimes
q)\perp\otimes q\big)\perp$, giving a proof net from $p$ to
$\big((p\otimes q)\perp\otimes q\big)\perp$:
%
\[\thicklines
\begin{array}{l@{\hspace{10ex}}c@{\hspace{10ex}}l}
        p\begin{picture}(0,0)\put(-3,-3){\line(0,-1){27}}\end{picture}
        &&
        p\begin{picture}(0,0)\put(-3,-3){\line(0,-1){66}}\end{picture}
        \\[5ex]
        p\begin{picture}(0,0)\put(-3,-3){\line(0,-1){27}}\end{picture}
        \otimes(q
        \begin{picture}(0,0)%
                \put(-2,9){\qbezier(0,0)(9,15)(18,0)}%
                \put(0,-5){\qbezier(0,0)(10,-12)(20,-24)}%
        \end{picture}
        \otimes
        \begin{picture}(0,0)%
                \put(0,-5){\qbezier(0,0)(-10,-12)(-20,-24)}%
        \end{picture}
        \dual{q})\perp
        & \mapsto
        \\[4.6ex]
        \makebox[0pt][r]{$\big(($}p\otimes q)\perp\otimes q\big)\perp
        &&
        \makebox[0pt][r]{$\big(($}p\otimes q%
        \begin{picture}(0,0)%
                \put(-2,10){\qbezier(0,0)(13,17)(26,0)}%
        \end{picture}%
        )\perp\otimes q\big)\perp
\end{array}\]
%
To obtain the result of composition, one simply traces paths.

The category of proof nets is almost, but not quite, a \staraut
category \citep{BarrStac}.  The mismatch is the lack of units.
This
%\dcomment{Maybe you can improve the phrasing of this sentence}
prompts the question of what categorical structure
\emph{does} match \mll.  More precisely:
\begin{quote}
\textbf{Question:}\\
        \it Can one axiomatise a categorical structure, a relaxation
        of \emph{star-autonomy}, suitable for modelling \mll?  The
        category of \mll proof nets \textup(with explicit negation\footnote{
        \emph{I.e.}, with formulas generated from variables by $\tn$
        and $(-)\perp$, rather than from literals by tensor and par.
        In other words, one drops the quotienting by de Morgan duality
        which is implicit in the usual definition of \mll formulas.}\textup)
        should be a free such category.
\end{quote}
%
Here by a \emph{categorical structure} we mean
\dcomment{Maybe you can improve the phrasing in this paragraph} 
data (functors, natural isomorphisms, and so forth) together with
coherence diagrams, akin to the axiomatisation of \staraut categories.
%

The naive proposal simply drops the units from a standard
axiomatisation of \staraut category $\C$ \citep{BarrStac}:
\begin{itemize}
\item Tensor. A functor $-\tensor-:\C\times \C\to\C$.
\item Associativity. 
%
A natural isomorphism 
\(
  \alpha_{A,B,C}:(A\tensor B)\tensor C\to A\tensor(B\tensor C)
\)
natural in objects $A,B,C\in\C$ such that the usual pentagon commutes.
%following pentagon commutes:
%\begin{equation}\label{pentagon}
%\begin{diagram}
%  \bigl((A\tensor B)\tensor C\bigl)\tensor D
%  &\rTo^\alpha&(A\tensor B)\tensor (C\tensor D)
%  &\rTo^\alpha & A\tensor \bigl(B\tensor (C\tensor D)\bigr)
%  \\
%  \dTo<{\alpha\tn D} &&&& \dTo>{A\tn\alpha}
%  \\
%  \bigl(A\tensor(B\tensor C)\bigr)\tensor D
%  && \rTo_\alpha && A\tensor\big((B\tensor C)\tensor D\big)
%\end{diagram}
%\end{equation}

\item \vspace*{-.5ex}Symmetry.
%
A natural isomorphism $\sigma_{A,B}:A\tn B\to B\tn A$ natural in
objects $A,B\in\C$ such that $\sigma_{B,A}^{-1}=\sigma_{A,B}$
and the usual hexagon commutes.
% following hexagon commutes:
%\begin{equation}\label{hexagon}
%\begin{diagram}
%  (A\tn B)\tn C &\rTo^\alpha & A\tn(B\tn C)
%    & \rTo^\sigma & (B\tn C)\tn A \\
%  \dTo<{\sigma\tn C} &&&& \dTo>{\alpha}\\
%  (B\tn A)\tn C & \rTo_\alpha & B\tn(A\tn C)
%    & \rTo_{B\tn\sigma} & B\tn(C\tn A)
%\end{diagram}
%\end{equation}

\item \vspace*{-1ex}Involution.
%
A functor $(-)\perp:\C\op\to\C$ together with a natural isomorphism
$A\to A\perpp$.
\item An isomorphism $\C(A\tensor B,C\perp)\to\C(A,(B\tn C)\perp)$
  natural in all objects $A,B,C$.
\end{itemize}
However, while there is a proof net from $p$ to $p\tn (q\tn\dual{q})\perp$
(the first proof net depicted at the beginning of the
Introduction), this axiomatisation fails to provide a corresponding
morphism from $p$ to $p\otimes (q\tensor \dual{q})\perp$ in the free
such category generated from the variables $p$ and $q$.
%
The problem of finding the right axiomatisation is
non-trivial.
%\dcomment{Following Andrea's comments, I reverted back to
%proof nets as motivation.  She's right that there's a danger otherwise
%of looking like clowns pursuing ``generalisation for
%generalisation's sake''}

\paragraph*{The solution predates the problem.} As so often, the solution 
long predates the problem.
%
\citet{DayPro} defines a \emph{promonoidal} category\footnote{
        The original paper on the subject \citep{DayPro}
        uses the term `premonoidal'. To add to the confusion,
        the word premonoidal is now used to mean something
        quite different.
} as a generalisation of a monoidal category.
%
Rather than having a functor \[\tn:\C\times\C\to\C\] and a unit object
$I\in\C$, a promonoidal category has functors
\[\begin{array}{c}
        P: \C\op\times\C\op\times\C\to\Set,\\
        J: \C \to \Set.
\end{array}\]
 This brings us to our primary definition:
%\begin{quote}%\textbf{Primary Definition:}\dcomment{Main $\mapsto$ primary, 
%             %as in ``first'' or ``primordial'', rather than ``most important''}\\
\begin{plainbox}
\it
        A \textbf{\smc}
        $\C$ is a promonoidal category 
        such that\comment[D]{rephrased (a) for more precision, and (b) for typesetting prettiness}
        $P(A,B,C) = \C(A\tn B,C)$ for some functor $\tn:\C\times\C\to\C$.
\end{plainbox}
%\end{quote}
The motivation behind the choice of terminology \emph{semi} here is
twofold.  A monoidal category is a promonoidal category
satisfying:
\begin{itemize}
   \item[(a)] $P(A,B,C) = \C(A\tn B,C)$ for some functor $\tn:\C\times\C\to\C$, and
%
   \item[(b)] $J(A) = \C(I,A)$ for some object $I$.
\end{itemize}
\emph{Semi} refers to our use of just one of the two properties.
%
Secondly,
if one views \emph{semi} as short for \emph{semigroupal}, then one has
an analogy with semigroups, which are monoids without unit.
%

Emulating the usual progression from monoidal category to
\staraut category via symmetry and closure, we progress to
a notion of semi \staraut category, our candidate for an answer
to the question posed at the beginning of the Introduction.  (This
being a preliminary report on work in progress, we have yet to
complete the proof that the category of proof nets is a free semi
\staraut category.)
%
A key step in this progression is the following:
\begin{plainbox}
\it
        A \textbf{\ssmcc} \textup(\SECC\textup) is
        a category $\C$ equipped with an associative, symmetric
%                \footnote{
%                \emph{I.e.}, equipped with the standard associativity and symmetry 
%                isomorphisms and coherence laws of a symmetric monoidal category.}
        functor
        $\tn:\C\times\C\to\C$, a functor $\lolli:\C\op\times\C\to\C$
%       (\emph{internal hom}) 
        through which hom factors up to isomorphism,
        \begin{diagram}
                \C\op\times\C &\rTo^\lolli& \C\\
                &\rdTo_{\mathrm{hom}}\raise1em\hbox{\qquad$\cong$}&\dTo.\\
                &&\Set
        \end{diagram}
        and a natural isomorphism
        \[
                \psi_{A,B,C}\;\;:\;\; (A\tn B)\lolli C \;\;\to\;\;
                A\lolli(B\lolli C)
        \]
        such that
        \begin{diagram}[h=1.5em] \bigl(A\tn(B\tn C)\bigr)\lolli D
        &\rTo^{\alpha_{A,B,C}\lolli D} &\bigl((A\tn B)\tn
        C\bigr)\lolli D\\ &&\dTo>{\psi_{A\tn B,C,D}}\\
        \dTo<{\psi_{A,B\tn C,D}}&&(A\tn B)\lolli(C\lolli D)\\
        &&\dTo>{\psi_{A,B,C\lolli D}}\\ A\lolli\bigl((B\tn C)\lolli
        D\bigr)&\rTo_{A\lolli\psi_{B,C,D}}
        &A\lolli\bigl(B\lolli(C\lolli D)\bigr) \end{diagram} commutes.
\end{plainbox}
See Section~\ref{s-sec} for details.  
%
Once again, 
%in progressing from semi-monoidal category to \ssmcc, 
the solution predates the problem: the natural isomorphism $\psi$, and
its commuting diagram are exactly as in the definition of symmetric
monoidal closed category in \citet{EKClosed}.

\paragraph*{Related work.}  
Our interest in obtaining an axiomatisation was sparked
by the desire to characterise the category of unit-free
proof nets for Multip\-lic\-at\-ive-Add\-it\-ive Linear Logic \citep{MallNets}
as a free category.

Soon after beginning to think about the problem, we came across an
interesting and informative proposal and discussion in a draft of
\citet{LSFreeBool}.
% \emph{en route} to a notion of Boolean category
%(generalising a Boolean algebra).
%
In this draft the authors define what they call \emph{(unitless)
autonomous categories}, motivated (like us) by the desire to model
unitless fragments of MLL.

The definition in the original draft was weaker than the authors
intended. The current version contains a definition that is
equivalent to ours, though it has a redundant axiom.
%
(Section~\ref{s-ls} discusses this in more detail.)
%

In our initial exploration of candidates for semi \staraut
category, we (independently) considered essentially the same
definition as in \cite{ProofNetCats}: a unitless linearly
distributive category with a suitable duality on objects.  Ultimately
we chose the approach more analogous to the standard progression from
monoidal category to \staraut category, via symmetry and
closure: it ties directly into the pioneering work of \citet{EKClosed},
with the \pref{psicoh} diagram, and also Day's promonoidal categories
\citep{DayPro}. Our definition is also substantially simpler, in
that it requires fewer coherence conditions.

%notation in original
%\cite{Girard87}; also $\dual{P}$ for dual literal.  
%We might as well stick to it.}
%\begin{displaymath}\begin{prooftree}\thickness=.08em
%\[
%  \[ 
%    \justifies \vdash P,\dual{P} \using \axiomlabel
%  \]
%  \hspace{3ex}
%  \[
%    \justifies \vdash Q,\dual{Q} \using \axiomlabel
%  \]
%  \justifies 
%  \vdash P,\: \dual{P}\otimes Q,\: \dual{Q} \using \tn
%  \hspace{3ex}
%\]
%\hspace{3ex}
%\[
%  \justifies \vdash P,\dual{P} \using \axiomlabel
%\]
%\justifies 
% \vdash P,\: \dual{P}\otimes Q,\: \dual{Q} \otimes P,\: \dual{P} \using \tn
%\end{prooftree}\end{displaymath}
%%
%translates into the following proof net, a `linking' or `matching'
%between complementary leaves of the conclusion:

%The translation to a proof net simply traces the leaves of the axioms
%down through the proof to the conclusion.  The above proof net is
%obtained thus:
%%%
%%\begin{prooftree}\thickness=.08em
%%\piproof{1}{\Gamma,A}
%%\[
%% \piproof{2}{B,\Delta,X} \piproof{3}{Y,\Sigma}
%% \justifies B,\Delta,X\tensor Y,\Sigma \using \tn
%%\]
%%\justifies
%%\Gamma,A\tensor B,\Delta,X\tensor Y,\Sigma
%%\using \tn
%%\end{prooftree}
%%&\diagconv{\tensor}&
%%\begin{prooftree}\thickness=.08em
%%\[
%% \piproof{1}{\Gamma,A} \piproof{2}{B,\Delta,X}
%% \justifies \Gamma,A\tensor B,\Delta,X \using \tn
%%\]
%%\piproof{3}{Y,\Sigma}
%%\justifies
%%\Gamma,A\tensor B,\Delta,X\tensor Y,\Sigma
%%\using \tn
%%\end{prooftree}
%The linkings between edges are a variant of Eilenberg-Kelly-MacLane
%graphs \cite{EK66,KM71}\ldots
%
%There is a well known correspondence between particular kinds of
%category and particular logical systems, e.g.
%\vskip\abovedisplayskip\noindent
%\begin{tabular}{l@{\qquad}l}
%       \qquad\emph{category}&\qquad\emph{logic}\\[1ex]
%       Cartesian closed & Intuitionistic logic\\
%       Symmetric monoidal closed & Multiplicative intuitionistic linear logic\\
%       Star-autonomous & Multiplicative linear logic (MLL)
%\end{tabular}
%\vskip\belowdisplayskip

%The unit-free fragment of MLL is well understood from the logical
%point of view, and it admits a particularly simple notion of proof net.
%The purpose of this note is to describe the corresponding categories.

%The first thing to observe is that there is a non-trivial problem to be solved.
%Consider the unit-free fragment of multiplicative intuitionistic
%linear logic (MILL${\!}^-$) and its corresponding sequent calculus.
%%
%We cannot simply drop the units from our category altogether:
%although the logic has no unit, there are still valid proofs
%that involve sequents of the form ${}\vdash A$, whose left-hand side
%is empty. Such a sequent would ordinarily be represented by an arrow
%$I\to A$ in the category. Moreover, these sequents are essential for
%certain proofs: for example, not only does the proof
%\[\begin{prooftree}\thickness=.08em
%       p\vdash p \quad \[
%               q\vdash q
%               \justifies {}\vdash q\lolli q
%               \usingrule\lolli r
%       \]
%       \justifies
%       p\vdash p\tn(q\lolli q)
%       \usingrule\tn r
%\end{prooftree}\]
%use such a sequent, but \emph{every} proof of this conclusion
%must use such a sequent.

%On the other hand, if the categorical tensor product does
%have a unit~$I$, then there is a functor taking $X$ to
%$X\lolli I$; this does not correspond to anything in the logic.
%Hence we can neither allow the unit to remain, nor remove it
%altogether. A middle way is needed.

%The solution lies in the use of \defn{promonoidal}\footnote{
%       The original paper on the subject \citep{DayPro}
%       uses the term `premonoidal'. To add to the confusion,
%       the word premonoidal is now used to mean something
%       quite different.
%} categories.
%(As so often, the solution long predates the problem.) A promonoidal
%category is a kind of generalised monoidal category. Rather than
%having a functor \[\tn:\C\times\C\to\C\] and a unit object $I\in\C$,
%a promonoidal category has functors
%\[\begin{array}{c}
%       P: \C\op\times\C\op\times\C\to\Set,\\
%       J: \C \to \Set.
%\end{array}\]
%Each ordinary monoidal category $\C$ has a natural promonoidal structure
%with $P(A,B,C) = \C(A\tn B,C)$ and $J(A) = \C(I, A)$. This brings us
%to our primary definition:

%\begin{setoff}{Main Definition:}
%       A semi monoidal category $\C$ is a promonoidal category in which
%       $P(A,B,C) = \C(A\tn B,C)$ for some functor $\tn:\C\times\C\to\C$.
%\end{setoff}
%%
%We then consider \expandafter\emph{\ssmccs} (\SECCs)
%as the natural notion of model for unit-free MILL, and give a simple
%concrete description. [Dominic is going to make this part sound
%much better and more convincing.]

\paragraph*{Structure of paper.}
Section~\ref{s-summary} gives two different (but equivalent)
elementary definitions of \SECC.  Most of the remainder of the note is
devoted to showing that the first of these is equivalent to the conceptual
definition (Definition~\ref{def-secc}).

Section~\ref{s-bg} describes the background needed to understand the
subsequent development, in what is intended to be a clear and gentle
(but not rigorous) way. In particular, the definitions of \emph{coend}
and \emph{promonoidal category} are explained.
%
Section~\ref{s-sec} describes \ssmcs, using
Appendix~\ref{app-ax} (which gives a simple but
non-standard axiomatisation of symmetric monoidal categories).
Section~\ref{s-secc} shows that the conditions of \S\ref{s-summ-1}
are necessary and sufficient.
Section~\ref{s-sstac} defines \sstacs.

Finally section~\ref{s-related} discusses the relationship between
our definitions and the recent proposals of \citet{LSFreeBool}
and \citet{ProofNetCats}.

\paragraph*{Acknowledgements} 
Many thanks to Richard Garner and Martin Hyland for their penetrating
advice about an earlier version of this work. We are also grateful to
Lutz Stra\ss burger for useful discussion by email.  We acknowledge
use of Paul Taylor's diagrams package.

% Plus star-autonomous stuff

\section{Summary of Results}\label{s-summary}\dcomment{Perhaps we can skip this section, 
given the summary of the SSMCC definition in the Intro?  Entirely your call.}
The definition of \SECC (in terms of promonoidal categories) is
given in Def.~\ref{def-secc}. The main technical contribution of this
note is to show that this conceptual definition can be recast in
more elementary terms.
\subsection{First Description}\label{s-summ-1}
%Symmetric E-monoidal closed categories admit the following
%elementary description.
\begin{propn}\label{prop-1}
        A \ssmcc can be described by the following data:
        \begin{itemize}
        \item A category $\C$,
        \item Functors $\tn: \C\times\C\to\C$ and $\lolli:\C\op\times\C\to\C$,
        \item Natural isomorphisms $\alpha$, $\sigma$ and $\psi$ with
                components
                \[\begin{array}{r@{\;}l}
                        \alpha_{A,B,C}:& (A\tn B)\tn C\to A\tn(B\tn C),\\
                        \sigma_{A,B}:  & A\tn B\to B\tn A,\\
                        \psi_{A,B,C}:  & (A\tn B)\lolli C \to A\lolli(B\lolli C),
                \end{array}\]
                such that
                \begin{starredequation}[$\boldsymbol\sigma$]\label{sigma}
                  \sigma_{A,B} = \sigma_{B,A}^{-1}
                \end{starredequation}
                  and the
                following coherence diagrams commute for all $A$,$B$,$C$,
                $D\in\C$:
                \begin{diagram}\dlabel[$\boldsymbol\alpha$]{pentagon}
                  \bigl((A\tensor B)\tensor C\bigl)\tensor D
                  &\rTo^{\alpha_{A\tn B,C,D}}&(A\tensor B)\tensor (C\tensor D)
                  &\rTo^{\alpha_{A,B,C\tn D}} & A\tensor \bigl(B\tensor (C\tensor D)\bigr)
                  \\
                  &\rdTo[snake=-1em](1,2)<{\alpha_{A,B,C}\tn D} &\dnum
                  & \ruTo[snake=1em](1,2)>{A\tn\alpha_{B,C,D}}
                  \\
                  & \spleft{\bigl(A\tensor(B\tensor C)\bigr)\tensor D}
                  & \rTo_{\alpha_{A,B\tn C,D}}
                  & \spright{A\tensor\big((B\tensor C)\tensor D\big),}
                \end{diagram}
%
                \begin{diagram}\dlabel[$\boldsymbol{\alpha\sigma}$]{hexagon}
                  (A\tn B)\tn C &\rTo^{\alfa ABC} & A\tn(B\tn C)
                    & \rTo^{\sigma_{A,B\tn C}} & (B\tn C)\tn A \\
                  \dTo<{\sigma_{A,B}\tn C} &&\dnum&& \dTo>{\alfa BCA}\\
                  (B\tn A)\tn C & \rTo_{\alpha_{B,A,C}} & B\tn(A\tn C)
                    & \rTo_{B\tn\sigma_{A,C}} & B\tn(C\tn A)
                \end{diagram}
%
                \begin{diagram}[h=1.5em]\dlabel[$\boldsymbol\psi$]{psicoh}
                \bigl(A\tn(B\tn C)\bigr)\lolli D &\rTo^{\alpha_{A,B,C}\lolli D}
                        &\bigl((A\tn B)\tn C\bigr)\lolli D\\
                &&\dTo>{\psi_{A\tn B,C,D}}\\
                \dTo<{\psi_{A,B\tn C,D}}&\dnum&(A\tn B)\lolli(C\lolli D)\\
                &&\dTo>{\psi_{A,B,C\lolli D}}\\
                A\lolli\bigl((B\tn C)\lolli D\bigr)&\rTo_{A\lolli\psi_{B,C,D}}
                        &A\lolli\bigl(B\lolli(C\lolli D)\bigr)
                \end{diagram}
        \item A functor $J:\C\to\Set$ and a natural isomorphism $e$
                with components $e_{A,B}: J(A\lolli B) \to \C(A,B)$.
        \end{itemize}
\end{propn}
%
This proposition is proved in section~\ref{s-secc} below.

\subsection{Second Description}\label{s-summ-2}
In fact there is a canonical choice for $J$ and $e$; in order to
state what it is, we need a few definitions:
\begin{definition}
  A \defn{category with tensor} is a category $\C$
  equipped with a functor $\tn:\C\times\C\to\C$ and
  a natural isomorphism $\alpha$ satisfying condition~\pref{pentagon}.

  A \defn{category with symmetric tensor} is a category $\C$
  with tensor, together with a natural isomorphism $\sigma$
  such that conditions \pref{sigma} and \pref{hexagon} hold.
\end{definition}
\begin{definition}\label{def-le}
        Let $\C$ be a category with tensor.
        A \defn{linear element} $a$ of the object $A\in\C$ is a natural transformation
        with components
        \[
                a_X: X \to A\tn X
        \]
        such that
        \begin{diagram}%\dlabel{triangle}
          X\tensor Y & \rTo^{a_X\tn Y} & (A\tn X)\tn Y\\
          &\rdTo[snake=-1ex]_{a_{X\tn Y}} & \dTo>{\alpha_{A,X,Y}}\\
          && A\tn (X\tn Y)
        \end{diagram}
        commutes for all $X$,$Y\in\C$.
\end{definition}
%
\begin{definition}\label{def-Lin}
        Given a category $\C$ with tensor, define a functor
        \[
                \Lin: \C\to\Set
        \]
        as follows. For $A\in\C$, let $\Lin(A)$ be the set of linear
        elements of $A$. For $f:A\to B$ and $a\in\Lin(A)$, let
        $\Lin(f)(a)$ be the linear element of $B$ with components
        \[
                X \rTo^{a_X} A\tn X \rTo^{f\tn X}B\tn X.
        \]
\end{definition}
It turns out that, in the situation of Prop.~\ref{prop-1}, there is a
canonical natural isomorphism between $J$ and $\Lin$. Furthermore, it
happens that this natural isomorphism takes $e$ to a particular natural
transformation $l$, which is defined as follows. 
\begin{definition}\label{def-l}
        Suppose we have $(\C,\tn,\alpha)$ as above, together with a functor
        \(
                \lolli: \C\op\times\C\to\C
        \)
        and a natural isomorphism
        \[
                \curry_{A,B,C}: \C(A\tn B,C) \to \C(A, B\lolli C)
        \]
        with counit (i.e.\ evaluation map) $\e^A_B:(A\lolli B)\tn A\to B$.
        
        Define the natural transformation $l_{A,B}: \Lin(A\lolli B) \to \C(A,B)$
        as follows: for each $x\in\Lin(A\lolli B)$, let $l_{A,B}(x)$ be the
        composite
        \[
                A \rTo^{x_{A}} (A\lolli B)\tn A  \rTo^{\e^A_B} B.
        \]
\end{definition}
Our first description (Prop.~\ref{prop-1}) is equivalent to the following.   
\begin{propn}\label{prop-2}
        An \SECC can be described by:
        \begin{itemize}
        \item A category $\C$ with symmetric tensor,
        \item A functor $\lolli:\C\op\times\C\to\C$ with a natural isomorphism
                \[
                        \curry_{A,B,C}:\C(A\tn B,C) \to \C(A, B\lolli C)
                \]
        \end{itemize}
        such that the natural transformation
        $l$ of Def.~\ref{def-l}
        %$l_{A,B}:\Lin(A\lolli B)\to\C(A,B)$
        is invertible.
\end{propn}
%
The proof of Prop.~\ref{prop-2} is still in draft form,
and is not included in this preliminary note.

\section{Technical Background}\label{s-bg}
This section gives an informal introduction to coends and
promonoidal categories.
\subsection{Coends}\label{s-coends}
The definition of promonoidal category involves coends, so we
need to understand them to some extent. Fortunately they are
quite simple: a coend is just a slightly more general version
of a colimit.

Recall that a colimit of a functor $J: \D\to\C$ is a universal natural
transformation from $J$ to some object $X\in\C$. Coends just
extend this idea to mixed-variance functors $J: \D\times \D\op\to\C$:
a coend of $J$ is a universal \emph{dinatural}\footnote{
        The dinatural transformations we need to consider are
        of the special kind called \emph{extraordinary natural
        transformations}. The distinction is important in
        enriched category theory, where extraordinary natural
        transformations can be defined but dinatural transformations
        can not in general.
}
transformation from $J$ to an object $X\in\C$.

A dinatural transformation $\gamma: J\To X$ is a family
of arrows
\[
        \gamma_A: J(A,A) \to X
\]
indexed by the objects $A\in \D$, such that for every
$f:A\to B$ in $\D$ the diagram
\begin{diagram}
        J(A,B) & \rTo^{J(A,f)}& J(A,A)\\
        \dTo<{J(f,B)}&&\dTo>{\gamma_A}\\
        J(B,B) & \rTo_{\gamma_B}& X
\end{diagram}
commutes. Such a dinatural transformation is universal if, for every
object $Y$ and dinatural transformation $\delta: J\To Y$,
there is a unique fill-in morphism $g: X\to Y$ such that
\begin{diagram}[h=1.5em]
        &&X\\
        &\ruTo^{\gamma_A}\\
        J(A,A)&&\dTo>g\\
        &\rdTo_{\delta_A}\\
        &&Y
\end{diagram}
commutes for every $A\in\D$. 

It is conventional, and very handy, to write coends using
integral notation. The coend of $J$ is written $\int^{A\in\D} J(A,A)$;
we usually omit the `$\in\D$' part when it is obvious from the context.

In the rest of this note, we only need to use coends in $\Set$ or in
functor categories of the form $[\C,\Set]$ for some $\C$. These categories
have all (small) coends, so we shall never have to worry about whether
or not a particular coend exists.\footnote{
        We are glossing over some size issues here,
        which can be dealt with in the usual way.
        The foundationally conservative reader may
        read the word `category', where it appears
        in a definition, as `small category'.
}

Here are some important properties of coends.
We use them heavily in the sequel, often without remark.
\begin{itemize}
        \item Left adjoints preserve coends. In particular, in a cartesian
                closed category $\C$ we have
                \[
                        A\times\int^{X\in\D} J(X,X) \cong \int^{X\in\D} A\times J(X,X)
                \]
                for every $A\in\C$;
        \item The `Fubini theorem': if $\int^{Y\in\D_2}J(X,X',Y,Y)$ exists for
                        all $X$, $X'\in\D_1$ then
        \[
                \int^{X\in\D_1}\int^{Y\in\D_2} J(X,X,Y,Y) \cong
                        \int^{(X,Y)\in\D_1\times\D_2}J(X,X,Y,Y)
        \]
        for $J: \D_1\times\D_1\op\times\D_2\times\D_2\op\to\C$;
        \item For every $F:\C\to\Set$ and $Y\in\C$,
        \[
                FY \cong \int^{X\in\C} FX\times\C(X,Y)
        \]
        and the coend on the right exists;
        \item The dual of the above: for every $F: \C\op\to\Set$ and $X\in\C$,
        \[
                FX \cong \int^{Y\in\C} FY\times\C(X,Y).
        \]
\end{itemize}
The latter two isomorphisms can be regarded as versions of the
Yoneda lemma. For proofs of all these facts, and a generally very
nice tutorial introduction to coends, see the lecture notes by
\citet{CatNotes}.
%
%One special case deserves remark: for every $A$,$B$,$C\in\C$
%we have
%\[
%       \C(A,C) \cong \int^B\C(A,B)\times\C(B,C).
%\]

\subsection{Promonoidal Categories}\label{s-promon}
As mentioned in the introduction, a promonoidal category is a
category $\C$ together with functors
\[\begin{array}{c}
        P: \C\op\times\C\op\times\C\to\Set,\\
        J: \C \to \Set.
\end{array}\]
and natural isomorphisms $\alpha$, $\lambda$ and $\rho$
satisfying conditions analogous\footnote{
        In fact this is no mere analogy: monoidal and promonoidal
        categories are both instances of the general notion of
        \emph{pseudomonoid} in a monoidal bicategory \citep{MonBicat}.
        A promonoidal category is a pseudomonoid in the bicategory
        whose objects are categories and whose 1-cells are modules
        (aka profunctors or distributors).
} to those in the definition of a monoidal category.

In order to understand what these conditions are, we develop
an informal procedure for translating the language
of monoidal categories into the language of promonoidal categories.
In a monoidal category $\C$ we can form various functors
$\C^n\to\C$ for some natural number $n$, using the tensor
product and unit object. For example we have the functor
$F:\C^2\to\C$ defined as
\[
        F(A_1, A_2) = (A_1 \tn I) \tn (A_2 \tn A_1).
\]
In general, such a functor is described by an expression
formed from variables $A_1, \dots, A_n$ and the constant $I$
using the binary operation $\tn$. These expressions
can be thought of as denoting the objects of the free monoidal category
on a countably infinite set of generators.

\def\qq#1{\raise1pt\hbox{$\ulcorner$}\!#1\!\raise1pt\hbox{$\urcorner$}}%
Given such an expression $S$ and promonoidal category $\C$, we can
define a corresponding expression $\qq{S}$, representing a
functor $(\C\op)^n\to[\C,\Set]$, recursively as follows:
\[\begin{array}{r@{{}={}}l}
        \qq{I}     & J,\\
        \qq{A_i}   & \C(A_i, -),\\
        \qq{S\tn T}& \int^{X,Y} \qq S(X) \times \qq T(Y) \times P(X,Y,-).
\end{array}\]
(We assume that bound variables are renamed where necessary.)

For example, we have
\[\begin{array}{r@{\;}c@{\;}l}
        \qq{A_1\tn A_2} &=& \int^{X,Y} \qq{A_1}(X)\times\qq{A_2}(Y)\times P(X,Y,-)\\
                &=&     \int^{X,Y} \C(A_1,X)\times \C(A_2,Y)\times P(X,Y,-)\\
                &\cong& \int^{X} \bigl(\C(A_1,X)\times \int^Y \C(A_2,Y)\times P(X,Y,-)\bigr)\\
                &\cong& \int^{X} \C(A_1,X)\times P(X,A_2,-)\\
                &\cong& P(A_1,A_2,-)
\end{array}\]
and
\[\begin{array}{r@{\;}c@{\;}l}
        \qq{A_1\tn I} &=& \int^{X,Y} \qq{A_1}(X)\times\qq{I}(Y)\times P(X,Y,-)\\
                &=&     \int^{X,Y} \C(A_1,X)\times J(Y)\times P(X,Y,-)\\
                &\cong& \int^Y J(Y)\times P(A_1,Y,-).
\end{array}\]
Similarly we have
\[\begin{array}{r@{{}\protect\cong{}}l}
        \qq{I\tn A_1} & \int^X J(X)\times P(X,A_1,-),\\
        \qq{(A_1\tn A_2)\tn A_3} & \int^X P(A_1,A_2,X)\times P(X,A_3,-),\\
        \qq{A_1\tn (A_2\tn A_3)} & \int^X P(A_1,X,-)\times P(A_2,A_3,X).
\end{array}\]
%
Now we can say what the types of $\alpha$, $\lambda$ and $\rho$
should be. They should have components
\[\begin{array}{r@{\;}l}
        \alfa ABC:& \int^X P(A,B,X)\times P(X,C,-) \to \int^X P(A,X,-)\times P(B,C,X)\\
        \lambda_A:& \int^X J(X)\times P(X,A,-) \to \C(A,-)\\
        \rho_A:& \int^Y J(Y)\times P(A,Y,-) \to \C(A,-)
\end{array}\]
Each component here is \emph{itself} a natural transformation, so
we can add an extra variable and ask for natural isomorphisms
\[\begin{array}{r@{\;}l}
        \alpha_{A,B,C,Z}:& \int^X P(A,B,X)\times P(X,C,Z) \to \int^X P(A,X,Z)\times P(B,C,X)\\
        \lambda_{A,Z}:& \int^X J(X)\times P(X,A,Z) \to \C(A,Z)\\
        \rho_{A,Z}:& \int^Y J(Y)\times P(A,Y,Z) \to \C(A,Z)
\end{array}\]
between functors to $\Set$.

We impose the usual coherence conditions, as described for monoidal
categories in the appendix and elaborated below in the \sm
case.
\ifincomplete
\subsection{Ends}\label{s-ends}
The notion of \defn{end} is dual to that of coend, and they enjoy
similar properties. (We use ends in \S\ref{s-uniq} below.) A
dinatural transformation from the object $X\in\C$ to the functor
$J:\D\times\D\op\to\D$ is a family of maps $\gamma_A: X\to J(A,A)$
such that for every $f:A\to B$ in $\C$, the diagram
\begin{diagram}
        X & \rTo^{\gamma_A}& J(A,A)\\
        \dTo<{\gamma_B} && \dTo>{J(f,A)}\\
        J(B,B) & \rTo_{J(B,f)}&J(B,A)
\end{diagram}
commutes.

The \defn{end of $J$} is a universal dinatural transformation
$X\To J$; if the end exists, we write this object $X$ as
$\int_{X\in\D} J(X,X)$.

Some important (and easily verified) properties of ends are:
\begin{itemize}
        \item The set of natural transformations $F\To G:\D\to\C$ is
                isomorphic to $\int_X\C(FX,GX)$;
        \item For a functor $F:\C\to\Set$, the Yoneda lemma
                gives a natural isomorphism $FA\cong \int_X\C(A,X)\To FX$,
                and the end on the right exists;
        \item The Fubini theorem for ends: given a functor
          \[J: \D_1\times\D_1\op\times\D_2\times\D_2\op\to\C,\]
                if $\int_{Y\in\D_2}J(X,X',Y,Y)$ exists for
                        all $X$, $X'\in\D_1$ then
        \[
                \int_{X\in\D_1}\int_{Y\in\D_2} J(X,X,Y,Y) \cong
                        \int_{(X,Y)\in\D_1\times\D_2}J(X,X,Y,Y).
        \]
%       for $J: \D_1\times\D_1\op\times\D_2\times\D_2\op\to\C$.
\end{itemize}
\fi % incomplete

\subsection{Notation for Diagrams}
In a diagram containing several cells which are known to commute,
we often label each such cell with the reason that it commutes,
removing the need for separate explanations that must be
cross-referenced with the diagram.
The symbol $\natural$ is used to indicate that a cell commutes
by naturality of some natural transformation.

\section{\Ssmcs}\label{s-sec}
We consider the special case of a \defn{\smc},
i.e.\ a promonoidal category $\C$ whose $P$ is represented by
a functor $\tn:\C\times\C\to\C$.
Therefore suppose we have such a functor $\tn$, and that
$P(A,B,X) = \C(A\tn B,X)$. Now we have
\[\begin{array}{r@{\;}c@{\;}l}
        \qq{(A_1\tn A_2)\tn A_3}  &\cong& \int^X \qq{A_1\tn A_2}(X)\times\C(X\tn A_3,-)\\
                &\cong& \int^X \C(A_1\tn A_2, X)\times\C(X\tn A_3,-)\\
                &\cong& \C((A_1\tn A_2)\tn A_3, -)
\end{array}\]
and similarly
\[
        \qq{A_1\tn(A_2\tn A_3)} \cong \C(A_1\tn (A_2\tn A_3), -).
\]
Thus, by Yoneda, the associativity isomorphism may be represented
by a natural isomorphism $\alpha$ with components
\[
        \alpha_{A,B,C}: (A\tn B)\tn C \to A\tn(B\tn C),
\]
just as in an ordinary monoidal category, subject to the
usual pentagon condition~\pref{pentagon}.

We are really interested in \emph{symmetric} \smcs, so suppose
that there is also a symmetry
$\sigma$ with components $\sigma_{A,B}: A\tn B \to B\tn A$
such that $\sigma_{A,B} = \sigma_{B,A}^{-1}$ for all $A$,$B\in\C$,
and satisfying the hexagon condition~\pref{hexagon}.

By the argument in the appendix -- more precisely, by reinterpreting
that argument in the promonoidal setting -- we have a \ssmc
just when there is a natural isomorphism
$\lambda$ with components
\[
        \lambda_{A,Z}: \int^X J(X)\times \C(X\tn A,Z) \to \C(A,Z)
\]
such that the diagram
\begin{diagram}\dlabel{bigone}
        \int^X J(X)\times\C((X\tn B)\tn C, Z) & \rTo^{\int^X J(X)\times\C(\alfa XBC, Z)}
                & \int^X J(X)\times\C(X\tn (B\tn C), Z)
        \\
        \dTo<{\cong}
        \\
        \int^{X,Y} J(X)\times\C(X\tn B, Y)\times \C(Y\tn C, Z)&\dnum
                &\dTo>{\lambda_{B\tn C,Z}}
        \\
        \dTo<{\int^Y\lambda_{B,Y}\times\C(Y\tn C,Z)}
        \\
        \int^Y\C(B,Y)\times\C(Y\tn C,Z) &\rTo_\cong&\C(B\tn C,Z)
\end{diagram}
commutes for all $B$,$C$,$Z\in\C$.

\section{\Ssmccs}\label{s-secc}
%There is a general notion of symmetric promonoidal closed
%category, where the closed structure is descibed by a functor
%       \[
%               L: \C\times\C\op\times\C\to\Set.
%       \]
%We will not go into the general definition here -- the interested
%reader could consult \citet{MonBicat}. We are interested in
%the following special case:
Generally speaking, a promonoidal category $\C$ is \defn{left closed}
if it has a functor $\lolli:\C\op\times\C\to\C$ with a natural
isomorphism
\[
    P(A,B,C) \cong \C(A, B\lolli C).
\]
In the case of present interest, we have:
\begin{definition}\label{def-secc}
        A \defn{\ssmcc} (\defn{\SECC})
        is a \ssmc $\C$ together with a functor
        \[
                \lolli:\C\op\times\C\to\C
        \]
        and a natural isomorphism with components
        \[
                \curry_{A,B,C}:\C(A\tn B, C)\to\C(A, B\lolli C).
        \]
\end{definition}
%
%For the remainder of this section, we work in a symmetric E-monoidal
%closed category $\C$.
%
Recall the characterisation claimed in Prop.~\ref{prop-1}.
This differs from Def.~\ref{def-secc} in the following ways:
%\begin{itemize}
instead of $\lambda$ we have $e$;
instead of $\curry$ we have $\psi$; and
instead of \pref{bigone} we have \pref{psicoh}.
%\end{itemize}
(Note that no conditions are imposed explicitly on $J$ or $e$.)
The rest of the section is devoted to proving Prop.~\ref{prop-1}.

\begin{lemma}\label{lemma-1}
        There is an isomorphism \SECC,
        $\qq{I\tn A} \cong J(A\lolli -)$, natural in $A$.
\end{lemma}
\begin{proof}
We have the chain of natural isomorphisms
\[\begin{array}{r@{\;}c@{\;}l}
        \qq{I\tn A} &=& \int^X J(X)\times \C(X\tn A,-)\\
                &\cong& \int^X J(X)\times \C(X,A\lolli -)\\
                &\cong& J(A\lolli -);
\end{array}\]
\vskip -\belowdisplayskip
\vskip -\baselinestretch\baselineskip
\end{proof}
%
\newcommand\JX{\mathchoice{\int^X JX\times}{\int^X JX\times}%
        {\int^X\!\!JX\times}{\int^X\!\! JX\times}}%
\newcommand\JXY{\mathchoice{\int^{X,Y} JX\times}{\int^{X,Y} JX\times}%
        {\int^{X,Y}\!JX\times}{\int^{X,Y}\! JX\times}}%
%
\begin{lemma}\label{lemma-2}
        To give a natural isomorphism $\lambda_A: \qq{I\tn A}\To\qq{A}$
        such that \pref{bigone} commutes
        is to give a natural isomorphism
        \[
                e_{A,Z}: J(A\lolli Z) \to \C(A, Z).
        \]
        such that the diagram
%%      \begin{diagram}[h=1.5em,labelstyle=\scriptstyle]
%%                                                      \dlabel{diag-e}
%%              \C(X\tn(B\tn C),Z) &\rTo^{\C(\alpha_{X,B,C}, Z)}        & \C((X\tn B)\tn C,Z)\\
%%                      &                                               & \dTo>{\curry_{X\tn B,C,Z}}\\
%%              \dTo<{\curry_{X,B\tn C,Z}}&     & \C(X\tn B,C\lolli Z)\\
%%                      &                                               & \dTo>{\curry_{X,B,C\lolli Z}}\\
%%              \C(X,(B\tn C)\lolli Z)&\dnum& \C(X,B\lolli(C\lolli Z))\\
%%              \dTo<{J(-)(x)}&                         & \dTo>{J(-)(x)}\\
%%              J((B\tn C)\lolli Z) &           & J(B\lolli(C\lolli Z))\\
%%              \dTo<{e_{B\tn C,Z}}&            & \dTo>{e_{B, C\lolli Z}}\\
%%              \C(B\tn C,Z)&\rTo_{\curry_{B,C,Z}}& \C(B,C\lolli Z)
%%      \end{diagram}
        \begin{diagram}[h=1.5em,labelstyle=\scriptstyle]\dlabel{diag-e}
                \JX\C(X\tn(B\tn C),Z) &\rTo^{\JX\C(\alpha_{X,B,C}, Z)} & \JX\C((X\tn B)\tn C,Z)\\
                &                                                               & \dTo>{\JX\curry_{X\tn B,C,Z}}\\
                \dTo<{\JX\curry_{X,B\tn C,Z}}&  & \JX\C(X\tn B,C\lolli Z)\\
                 &                                                              & \dTo>{\JX\curry_{X,B,C\lolli Z}}\\
                \JX\C(X,(B\tn C)\lolli Z)&\dnum & \JX\C(X,B\lolli(C\lolli Z))\\
                \dTo<{\cong}&                                   & \dTo>{\cong}\\
                J((B\tn C)\lolli Z) &                   & J(B\lolli(C\lolli Z))\\
                \dTo<{e_{B\tn C,Z}}&                    & \dTo>{e_{B, C\lolli Z}}\\
                \C(B\tn C,Z)&\rTo_{\curry_{B,C,Z}}& \C(B,C\lolli Z)
        \end{diagram}
        commutes for all $B$,$C$,$Z\in\C$.% ($X$) and $x\in JX$.
\end{lemma}
\begin{proof}
By Lemma~\ref{lemma-1} we can derive $e$ from $\lambda$ and
vice versa, and the translations are mutually inverse. So it
makes no difference whether we are given $e$ or $\lambda$,
since each can be derived from the other in a canonical way.

It remains to show that \pref{bigone} is equivalent to
\pref{diag-e}.
\begin{sidewaysfigure}
\begin{diagram}[labelstyle=\scriptstyle]
         \JX\C((X\tn B)\tn C,Z) &\rTo^{\JX\curry_{X\tn B,C,Z}}& \JX\C(X\tn B,C\lolli Z)
        \\
         \dTo<\cong&&\dTo<\cong&\rdTo^{\JX\curry_{X,B,C\lolli Z}}
        \\
         \rnode{a}{\JXY\C(X\tn B,Y)\times\C(Y\tn C,Z)}
                &\rTo_{\JXY\C(X\tn B,Y)\times\curry_{Y,C,Z}}
                & \JXY\C(X\tn B,Y)\times\C(Y,C\lolli Z)
                &\qquad&\JX\C(X,B\lolli(C\lolli Z))
        \\
         \dTo>{\JXY\curry_{X,B,Y}\times\C(Y\tn C,Z)}&
                &\dTo<{\JXY\curry_{X,B,Y}\times\C(Y, C\lolli Z)}&\ldTo_\cong
        \\
         \JXY\C(X,B\lolli Y)\times\C(Y\tn C,Z)
                &\rTo_{\JXY\C(X,B\lolli Y)\times\curry_{Y,C,Z}}
                &\JXY\C(X,B\lolli Y)\times\C(Y,C\lolli Z)
        \\
         \dTo<\cong && \dTo<\cong && \dTo>\cong
        \\
         \int^YJ(B\lolli Y)\times\C(Y\tn C,Z)
                &\rTo_{\int^Y\!\!J(B\lolli Y)\times\curry_{Y,C,Z}}
                &\int^YJ(B\lolli Y)\times\C(Y,C\lolli Z)
        \\
         \dTo>{\int^Y\!\!e_{B,Y}\times\C(Y\tn C,Z)} &
                & \dTo<{\int^Y\!\!e_{B,Y}\times\C(Y,C\lolli Z)}
                &\luTo[leftshortfall=1.5em,crab+]^\cong
                \rdTo[rightshortfall=2em,crab-]_\cong
        \\
         \rnode{b}{\int^Y\C(B,Y)\times\C(Y\tn C,Z)}
                &\rTo_{\int^Y\!\!\C(B,Y)\times\curry_{Y,C,Z}}
                &\int^Y\C(B,Y)\times\C(Y,C\lolli Z) && J(B\lolli(C\lolli Z))
        \\
         \dTo<\cong && \dTo<\cong &\ldTo_{e_{B,C\lolli Z}}
        \\
         \C(B\tn C,Z) &\lTo_{\curry^{-1}_{B,C,Z}}& \C(B, C\lolli Z)
        %
        \nccurve[angle=180]{->}ab\Aput{\scriptstyle \int^Y\!\!\lambda_{B,Y}\times\C(Y\tn C,Z)}
\end{diagram}
\caption{Diagram used in the proof of Lemma~\ref{lemma-2}}\label{fig-1}
\end{sidewaysfigure}
Consider the diagram in Fig.~\ref{fig-1}.
%
The left-hand cell commutes by the relationship between $e$ and $\lambda$.
The remaining cells commute by naturality,
or functoriality of the coend.
%
The left edge of this diagram is equal to the left and lower edge
of \pref{bigone}. Therefore condition~\pref{bigone} is
equivalent to the condition in the statement.
\end{proof}
%
We are still working in an \SECC as originally defined, so we have
a $\curry$ isomorphism. We construct a natural isomorphism $\psi'$ as
follows.
\begin{definition}
        Given a natural isomorphism
        \[
                \curry_{A,B,C}: \C(A\tn B,C)\to\C(A, B\lolli C)
        \]
        we define the natural isomorphism
        \[
                \psi'_{A,B,C}: (A\tn B)\lolli C\enskip\longrightarrow\enskip A\lolli(B\lolli C)
        \]
        to be the unique such natural transformation for which
        \begin{diagram}[h=1.5em]\dlabel{def-psi}
                \C(A,(X\tn Y)\lolli Z) &\rTo^{\C(A,\psi')}& \C(A,X\lolli(Y\lolli Z))\\
                \dTo<{\curry_{A,X\tn Y,Z}^{-1}}\\
                \C(A\tn(X\tn Y),Z)&\dnum&\uTo>{\curry_{A,X,Y\lolli Z}}\\
                \dTo<{\C(\alpha_{A,X,Y},Z)}\\
                \C((A\tn X)\tn Y, Z)&\rTo_{\curry_{A\tn X,Y,Z}}&\C(A\tn X,Y\lolli Z)
        \end{diagram}
        commutes for all $A$,$X$,$Y$,$Z\in\C$.
        (Uniqueness is a consequence of the Yoneda lemma.)
\end{definition}
Using this, we can recast condition~\pref{bigone} very simply.
\begin{lemma}\label{lemma-3}
        In an \SECC, condition~\pref{bigone} holds iff
        \begin{diagram}\dlabel{psi}
                J((A\tn B)\lolli C) &\rTo^{J(\psi'_{A,B,C})}&J(A\lolli(B\lolli C))\\
                \dTo<{e_{A\tn B,C}}&\dnum&\dTo>{e_{A,B\lolli C}}\\
                \C(A\tn B,C)&\rTo_{\curry_{A,B,C}}      &\C(A,B\lolli C)
        \end{diagram}
        commutes for all $A$,$B$,$C\in\C$.
\end{lemma}
\begin{proof}
        By Lemma~\ref{lemma-2} we know that \pref{bigone} is equivalent
        to \pref{diag-e}. Now we have
        \begin{diagram}[h=1.5em,labelstyle=\scriptstyle]
                \JX\C(X\tn(B\tn C),Z) &\rTo^{\JX\C(\alpha_{X,B,C}, Z)}  & \JX\C((X\tn B)\tn C,Z)\\
                        &                                               & \dTo>{\JX\curry_{X\tn B,C,Z}}\\
                \dTo<{\JX\curry_{X,B\tn C,Z}}&\mbox{\pref{def-psi}}& \JX\C(X\tn B,C\lolli Z)\\
                        &                                               & \dTo>{\JX\curry_{X,B,C\lolli Z}}\\
                \JX\C(X,(B\tn C)\lolli Z)&\rTo^{\JX\C(X,\psi'_{B,C,Z})}&\JX\C(X,B\lolli(C\lolli Z))\\
                \dTo<\cong&             \natural& \dTo>\cong\\
                J((B\tn C)\lolli Z) &\rTo_{J(\psi'_{B,C,Z})}& J(B\lolli(C\lolli Z))\\
                \dTo<{e_{B\tn C,Z}}&            & \dTo>{e_{B, C\lolli Z}}\\
                \C(B\tn C,Z)&\rTo_{\curry_{B,C,Z}}& \C(B,C\lolli Z)
        \end{diagram}
        The upper two regions commute for the reasons marked, and all the
        arrows are invertible, therefore the outside \pref{diag-e}
        commutes iff the lower cell \pref{psi} does. 
\end{proof}
\begin{lemma}\label{lemma-4}
        In every \SECC, diagram \pref{psicoh} commutes
        for $\psi'$ in place of $\psi$.
\end{lemma}
\begin{sidewaysfigure}
        \begin{diagram}[labelstyle=\scriptstyle]
                 \rnode{tl}{\C(X, (A\tn(B\tn C))\lolli D)}
                        &\rTo^{\C(X,\alpha_{A,B,C}\lolli D)}
                        &\rnode{tr}{\C(X,((A\tn B)\tn C)\lolli D))}
                \\
                 \uTo<{\curry_{X,A\tn(B\tn C),D}}&\natural
                        &\uTo>{\curry_{X, (A\tn B)\tn C,D}}
                \\
                 \C(X\tn(A\tn(B\tn C)),D)
                        &\rTo^{\C(X\tn\alpha_{A,B,C},D)}
                        &\C(X\tn((A\tn B)\tn C),D)
                \\
                 \dTo<{\C(\alpha_{X,A,B\tn C}, D)}&
                        \raise-1em\hbox{\pref{pentagon}}
                        &\dTo>{\C(\alpha_{X,A\tn B,C}, D)}
                        &\hbox{\pref{def-psi}}
                \\
                 \C((X\tn A)\tn(B\tn C), D)&
                        &\C((X\tn(A\tn B))\tn C,D)
                \\
                 &\rdTo(1,2)^{\C(\alpha_{X\tn A,B,C},D)}
                        \ldTo(1,2)^{\C(\alpha_{X,A,B}\tn C,D)}
                        &\dTo>{\curry_{X\tn(A\tn B),C,D}}
                \\
                 &\hbox to 10em{\hss$\C(((X\tn A)\tn B)\tn C,D)$}
                        \hbox to 2em{\hss$\natural$\enskip}
                        &\C(X\tn(A\tn B), C\lolli D)
                        &\pile{
                                \rTo^{\curry_{X,A\tn B,C\lolli D}}\\
                                \lTo_{\curry_{X,A\tn B,C\lolli D}^{-1}}}
                        &\rnode{r}{\C(X, (A\tn B)\lolli(C\lolli D))}
                \\
                 \dTo>{\curry_{X\tn A, B\tn C,D}}&
                        &\rdTo(1,2)_{\curry_{(X\tn A)\tn B,C,D}}
                        \dTo>{\C(\alpha_{X,A,B}, C\lolli D)}
                \\
                 &&\C((X\tn A)\tn B, C\lolli D)
                \\
                 &\raise 2em\hbox{\pref{def-psi}}&\dTo>{\curry_{X\tn A,B,C\lolli D}}
                 &\hbox{\pref{def-psi}}
                \\
                 \C(X\tn A, (B\tn C)\lolli D)
                        &\rTo_{\C(X\tn A,\psi'_{B,C,D})}
                        &\C(X\tn A, B\lolli(C\lolli D))
                \\
                 \dTo<{\curry_{X,A,(B\tn C)\lolli D}}&
                        \natural
                        &\dTo>{\curry_{X,A,B\lolli(C\lolli D)}}
                \\
                 \rnode{bl}{\C(X,A\lolli((B\tn C)\lolli D))}
                        &\rTo_{\C(X,A\lolli\psi'_{B,C,D})}
                        &\rnode{br}{\C(X,A\lolli(B\lolli(C\lolli D)))}
                %
                \psset{linearc=.5}
                \ncangle[angle=180,arm=30pt]{->}{tl}{bl}
                        \Bput{\scriptstyle \C(X,\psi'_{A,B\tn C,D})}
                        \Aput{\hskip 3em\hbox{\pref{def-psi}}}
                \nccurve[angleA=0,angleB=90]{->}{tr}r
                        \Aput{\scriptstyle \C(X,\psi'_{A\tn B,C,D})}
                \nccurve[angleA=-90,angleB=0]{->}r{br}
                        \Aput{\scriptstyle \C(X,\psi'_{A,B,C\lolli D})}
        \end{diagram}
        \caption{Diagram used in the proof of Lemma~\ref{lemma-4}}
        \label{fig-2}
\end{sidewaysfigure}
\begin{proof}
        Consider the diagram in Fig.~\ref{fig-2}. All the regions
        commute for the reasons marked, thus the outside commutes.
        By Yoneda, it follows that \pref{psicoh} commutes as required.
\end{proof}
%
\begin{lemma}
        Suppose we have a natural isomorphism $\psi$ with
        components
        \[
                \psi_{A,B,C}: (A\tn B)\lolli C \enskip\longrightarrow\enskip A\lolli(B\lolli C)
        \]
        such that
        \begin{diagram}\dlabel{psi-lemma}
                J((A\tn B)\lolli C) &\rTo^{J(\psi_{A,B,C})}&J(A\lolli(B\lolli C))\\
                \dTo<{e_{A\tn B,C}}&\dnum&\dTo>{e_{A,B\lolli C}}\\
                \C(A\tn B,C)&\rTo_{\curry_{A,B,C}}      &\C(A,B\lolli C)
        \end{diagram}
        commutes for all $A$,$B$,$C\in\C$.
        If $\psi$ satisfies condition~\pref{psicoh} then $\psi=\psi'$.
\end{lemma}
\begin{proof}
        Suppose $\psi$ satisfies condition~\pref{psicoh}.
        Then we have
        \begin{mspill}
        \begin{diagram}[h=1.5em,w=2em,labelstyle=\scriptstyle]
                \C(A\tn(B\tn C), D)&&&\rTo^{\C(\alpha_{A,B,C},D)}&&&\C((A\tn B)\tn C,D)\\
                &\rdTo[snake=1em]^{e^{-1}}&&\raise3pt\hbox{$\natural$}
                        &&\ruTo[crab+,leftshortfall=-.7em,rightshortfall=1.5em,snake=-1em]^{e}
                        \ldTo[crab-,leftshortfall=2em,rightshortfall=-.7em,snake=1em]_{e^{-1}}\\
                && J(\bigl(A\tn(B\tn C)\bigr)\lolli D) &\rTo^{J(\alpha\lolli D)}
                        &J(\bigl((A\tn B)\tn C\bigr)\lolli D)
                        &&\dTo>{\curry_{A\tn B,C,D}}\\
                && &&\dTo>{J(\psi_{A\tn B,C,D})}&\raise1em\hbox to 0pt{\quad\pref{psi-lemma}\hss}\\
                \dTo<{\curry_{A,B\tn C,D}}>{\mbox{\quad\enskip\pref{psi-lemma}}}&&
                                \dTo<{J(\psi_{A,B\tn C,D})}&\hbox{\pref{psicoh}}&J((A\tn B)\lolli(C\lolli D))
                                &\pile{\rTo\\\lTo}&\C(A\tn B,C\lolli D)\\
                && &&\dTo>{J(\psi_{A,B,C\lolli D})}&\raise-1em\hbox to 0pt{\quad\pref{psi-lemma}\hss}\\
                && J(A\lolli\bigl((B\tn C)\lolli D\bigr))&\rTo_{J(A\lolli\psi)}
                        &J(A\lolli\bigl(B\lolli(C\lolli D)\bigr))
                        &&\dTo>{\curry_{A,B,C\lolli D}}\\
                &\ldTo[crab+,leftshortfall=-.7em,rightshortfall=1.5em,snake=-1em]^{e}
                        \ruTo[crab-,rightshortfall=-.7em,leftshortfall=1.5em,snake=2em]_{e^{-1}}
                        &&\raise-3pt\hbox{$\natural$}&&
                        \luTo[crab+,rightshortfall=-.7em,leftshortfall=1.5em,snake=1em]^{e^{-1}}
                        \rdTo[crab-,leftshortfall=-.7em,rightshortfall=2.5em,snake=-2.5em]_{e}
                \\
                \rnode{bl}{\C(A, (B\tn C)\lolli D)}&&
                        &\rTo_{\C(A,\psi_{B,C,D})}&&
                        &\rnode{br}{\C(A,B\lolli(C\lolli D))}\\
                \ncarc[arcangle=-30]{->}{bl}{br}\Bput{\scriptstyle \C(A,\psi'_{B,C,D})}
        \end{diagram}\end{mspill}\vskip 5em\noindent
        The marked cells commute for the reasons indicated,
        and the outer edge by assumption.
        Since all arrows are invertible, it follows that the
        lower cell also commutes; hence $\psi=\psi'$ by Yoneda.
\end{proof}
%
In other words, if we are given a $\psi$ satisfying \pref{psicoh},
we can use \pref{psi-lemma} to construct a $\curry$
isomorphism such that \pref{diag-e} holds.
%
(We have already proved that, given a $\curry$ isomorphism
satisfying \pref{diag-e}, we can use \pref{def-psi} to construct
a natural isomorphism $\psi'$ such that \pref{psicoh} holds.)
%
This completes the proof of Prop.~\ref{prop-1}.

\subsection{Tensor of Elements}\label{s-tensel}
By Yoneda's lemma, an element of $JA$ corresponds to a
natural transformation $\C(A,-)\To J$. If we have
elements $a\in JA$ and $b\in JB$ then we may define a
natural transformation
\[\begin{array}{r@{}c@{}l}
        \C(A\tn B, -) &\rTo^\cong& \C(A, B\lolli -)\\
                &\rTo^a& J(B\lolli -)\\
                &\rTo^\cong&  \C(B, -)\\
                &\rTo^b&  J,
\end{array}\]
corresponding to an element of $J(A\tn B)$. We denote
this element $a\tn b$.
It is easy to check that this operation defines a natural
transformation with components $m_{A,B}: JA\times JB\to J(A\tn B)$.

\begin{propn}
This natural transformation agrees with the associativity,
in the sense that the diagram
\begin{diagram}[h=1.5em]
        \C(A\tn (B\tn C), -)\\
        &\rdTo[snake=1.3em]^{a\tn (b\tn c)}\\
        \dTo<{\C(\alfa ABC, -)} && J\\
        &\ruTo[snake=1.3em]_{(a\tn b)\tn c}\\
        \C((A\tn B)\tn C, -)
\end{diagram}
commutes for every $A$,$B$,$C\in\C$ with
$a\in JA$, $b\in JB$ and $c\in JC$.
\end{propn}
\begin{proof}
        Consider the diagram
        \begin{diagram}
        \C(A\tn(B\tn C),Z)&\rTo^\cong&\C(A,(B\tn C)\lolli Z)&\rTo^{a}&J\bigl((B\tn C)\lolli Z)
                &\rTo^\cong&\C(B\tn C,Z)\\
        && &\rdTo(2,4)_{\C(A,\psi)}& &\rdTo(2,4)_{J(\psi)}\hbox to0em{\pref{psi}}& \dTo>\cong\\
        \dTo<{\C(\alpha,Z)} &\pref{def-psi}& && \natural && \C(B,C\lolli Z)\\
        && && && \uTo>\cong\\
        \C((A\tn B)\tn C,Z)&\rTo_\cong&\C(A\tn B,C\lolli Z)&\rTo_\cong&\C\bigl(A, B\lolli(C\lolli Z)\bigr)
                &\rTo_{a}&J\bigl(B\lolli(C\lolli Z)\bigr)\\
        \end{diagram}
        Since the internal cells commute, so does the outside.
        Now observe that, by definition, the natural transformation
        $a\tn(b\tn c)$ is equal to the upper path followed by the
        composite
        \[
                \C(B,C\lolli Z) \rTo^b J(C\lolli Z) \cong \C(C,Z) \rTo^c JZ,
        \]
        while $(a\tn b)\tn c$ is equal to the lower path followed
        by this composite. Thus the claim follows.
\end{proof}

% xxx - also with symmetry, via Lamarche-Strassburger condition

\ifincomplete
\section{The Second Characterisation}\label{s-uniq}
It is not difficult to verify Prop.~\ref{prop-2} directly;
but such an approach leaves one with very little idea \emph{why}
such a thing should be true. Even the notion of `linear element'
remains mysterious and apparently ad hoc.
In this section we shall describe how Prop.~\ref{prop-2} follows
from the general ideas of enriched category theory.

In order to appreciate the general discussion, some
familiarity with enriched categories \citep[][ch.~1--2]{KellyEnriched}
is required; however, we give a self-contained proof of
Prop.~\ref{prop-2}.

Before getting down to details, let us sketch the story.
\newcommand\V{\mathcal{V}}%
It is known (at least as folklore, and presumably attributable
to the Australian school) that it is possible to define categories
enriched over a promonoidal category~$\V$. If $\V$ is \sm closed,
then $\V$ itself has the structure of a $\V$-category with hom-objects
given by $\lolli$.

Essentially all of the results that hold for symmetric monoidal
closed $\V$ continue to hold in the \ssm closed
case. In particular, $\V\times\V$ has the structure of a $\V$-%
category and $\tn:\V\times\V\to\V$ is a $\V$-functor in a natural
way.
%
Similarly $\V\op\times\V$ is a $\V$-category and $\lolli$ is
a $\V$-functor.
%
A linear element $a$ of $A\in\V$, as defined in Def.~\ref{def-le},
is exactly a $\V$-natural transformation $\mathord- \To A\tn\mathord-$.

The unit of the $\curry$ adjunction,\[
        \eta^A_B: B \to A\lolli(B\tn A),
\]
is $\V$-dinatural in $A$ for each $B$.\footnote{
        More than this, the $\curry$ adjunction extends to a $\V$-adjunction.
}
In fact it is the \emph{universal}
such $\V$-dinatural transformation, exhibiting $B$ as the \emph{$\V$-enriched
end} of $\mathord-\lolli(B\tn\mathord-)$. Let us use the symbol
$\oint_X FX$ for the $\V$-enriched end of a $\V$-functor $F$, thereby
distinguishing it from the ordinary end $\int_X FX$ of the ordinary
functor underlying $F$. Thus $\eta$ determines an isomorphism
$B\cong\oint_AA\lolli(B\tn A)$.

Given $\V$-functors $F$,$G:\V\to\V$, if the end $\oint_XFX\lolli GX$
exists then $J\oint_XFX\lolli GX$ is naturally isomorphic to the
set of $\V$-natural transformations $F\To G$. 
(This is a consequence of the enriched analogue of the characterisation
of natural transformations mentioned in \S\ref{s-ends}.)
Hence, by the previous paragraph, $JB$ is naturally isomorphic to the
set of linear elements of $B$.

This is the core of the argument. In more detail\dots

[TO BE CONTINUED]

\section{\SEC Functors and Natural Transformations}\label{s-fandnt}
There are general definitions of \emph{lax promonoidal functor}
\citep{DayMonoidalMonads} and \emph{strong promonoidal functor}
\citep{PromonKan}.

\begin{definition}
        Let there be given \SECCs $\C$ and $\D$. A \defn{lax \ssm functor}
        $F:\C\to\D$ is given by a functor $F:\C\to\D$ together
        with natural transformations
        \[\begin{array}{r@{\;}l}
                i_A:& JA \to JFA,\\
                m_{A,B}:& FA\tn FB \to F(B\tn C),
        \end{array}\]
        such that the following diagrams
        commute for all $A$,$B$,$C\in\C$:
        \begin{diagram}
                (FA\tn FB)\tn FC &\rTo^{\alpha}& FA\tn(FB\tn FC)\\
                \dTo<{m_{A,B}\tn FC} && \dTo>{FA\tn m_{B,C}}\\
                F(A\tn B)\tn FC && FA\tn F(B\tn C)\\
                \dTo<{m_{A\tn B,C}} && \dTo>{m_{A,B\tn C}}\\
                F\bigl((A\tn B)\tn C\bigr) &\rTo_{F\alpha}& F\bigl(A\tn(B\tn C)\bigr)
        \end{diagram}
        \begin{diagram}
                FA \tn FB &\rTo^\sigma & FB \tn FA\\
                \dTo<{m_{A,B}} && \dTo>{m_{B,A}}\\
                F(A\tn B) &\rTo_{F\sigma}& F(B\tn A)
        \end{diagram}
        \begin{diagram}
                J(A\lolli B) &\rTo^{i_{A\lolli B}}&JF(A\lolli B)\\
                &&\dTo>{Jn_{A,B}}\\
                \dTo<{e_{A,B}} && J(FA \lolli FB)\\
                &&\dTo>{e_{FA,FB}}\\
                \C(A,B) &\rTo_{F} & \D(FA,FB)
        \end{diagram}
        where $n_{A,B}$ is the natural transformation
        \[
                F(A\lolli B) \to FA\lolli FB
        \]
        obtained from $m$.
\end{definition}
\begin{remark}
        We are using the fact that natural transformations
        $m_{A,B}: FA\tn FB \to F(A\tn B)$ are in bijective
        correspondence with natural transformations
        $n_{A,B}: F(A\lolli B) \to FA \lolli FB$.
        There is a concise proof using the calculus of ends, viz:
        \[\begin{array}{r@{{}\protect\cong{}}l}
                \int_{A,B}\D(FA\tn FB,F(A\tn B))
                        & \int_{A,B}\D(FA, FB\lolli F(A\tn B)\\
                        & \int_{A,B}\int_C \D(A\tn B,C)\To\D(FA, FB\lolli FC)\\
                        & \int_{B,C}\int_A\D(A, B\lolli C) \To\D(FA, FB\lolli FC)\\
                        & \int_{B,C}\D(F(B\lolli C), FB\lolli FC).
        \end{array}\]
\end{remark}

\fi % incomplete

\section{The Star-autonomous Case}\label{s-sstac}
There is a general notion of promonoidal \staraut category
\citep[][\S7]{DSQuantum}. A symmetric promonoidal \staraut category
is a symmetric promonoidal category $\C$ equipped with a full and
faithful functor $-\perp: \C\to\C\op$ and a natural
isomorphism
\[
        P(A, B, C\perp) \cong P(A, C, B\perp).
\]
This specialises in the obvious way:
\begin{definition}
        A \sstac is a \ssmc $\C$ with a
        full and faithful functor $-\perp:\C\to\C\op$
        and a natural isomorphism
        \begin{equation}\label{eq-sstac}
                \C(A\tn B, C\perp) \cong \C(A\tn C, B\perp).
        \end{equation}
\end{definition}
Note that, since $-\perp$ is full and faithful, there is a natural
isomorphism $\C(A,B) \cong \C\op(A\perp,B\perp) = \C(B\perp, A\perp)$.

\begin{lemma}\label{lemma-perpp}
        There is a natural isomorphism $B\cong B\perpp$.
\end{lemma}
\begin{proof}
        There is a sequence of natural isomorphisms
        \[\begin{array}{r@{{}\protect\cong{}}lp{5cm}}
                \C(A,B) & \C(B\perp, A\perp)    &$\perp$ is full and faithful\\
                & \int^XJX\times\C(X\tn B\perp, A\perp) &using $\lambda^{-1}$\\
                & \int^XJX\times\C(X\tn A, B\perpp)         &by \pref{eq-sstac}\\
                & \C(A, B\perpp)                                &using $\lambda$.
        \end{array}\]
        Therefore, by Yoneda's lemma, it follows that
        $B$ is naturally isomorphic to $B\perpp$, as required.
\end{proof}
%
\begin{propn}
        Every \sstac is a \SECC, with $A\lolli B$ defined
        as $(A\tn B\perp)\perp$.
\end{propn}
\begin{proof}
        Clearly $\lolli$ is a functor of the correct type,
        so it remains only to establish the existence of
        a natural isomorphism $\C(A\tn B,C)\cong\C(A,B\lolli C)$.
        We have the following sequence of isomorphisms:
        \[\begin{array}{r@{\;}c@{\;}lp{5cm}}
                \C(A\tn B, C) &\cong& \C(A\tn B, C\perpp) & by Lemma~\ref{lemma-perpp}\\ 
                &\cong& \C(B\tn A, C\perpp) & by symmetry\\
                &\cong& \C(B\tn C\perp, A\perp) & by \pref{eq-sstac}\\
                &\cong& \C(A\perpp, (B\tn C\perp)\perp) &since $\perp$ is full and faithful\\
                &\cong& \C(A, (B\tn C\perp)\perp) & by Lemma~\ref{lemma-perpp}\\
                &=&     \C(A, B\lolli C) &by definition of $\lolli$.
        \end{array}\]
        \vskip-2em
\end{proof}

\section{Related Work}\label{s-related}
\subsection{The Lamarche-Stra{\ss}burger Definition}\label{s-ls}

Not long after starting work on this we came
across the draft of \citet{LSFreeBool}. The authors define what they
call \emph{(unitless) autonomous categories}, motivated (like us) by
the desire to model unitless fragments of MLL.

% I think we should define `Lamarche-Stra\ss burger category' if we're
% using that, so the term should be emphasized appropriately. I'm
% afraid your Latex is too complicated for me! (And then  just leave
% out that they call it `autonomous category'.)

Using our notation, an `autonomous category' in the sense of
\citet{LSFreeBool} consists of:
\begin{itemize}
        \item a category $\C$,
        \item functors $\tn:\C\times\C\to\C$ and $\lolli:\C\op\times\C\to\C$
                with a natural isomorphism
                \(
                \curry_{A,B,C}: \C(A\tn B,C) \rTo^\cong \C(A, B\lolli C),
                \)
        \item natural isomorphisms $\alpha$ and $\sigma$ such that
                $\sigma_{A,B} = \sigma_{B,A}^{-1}$,
                satisfying conditions \pref{pentagon} and \pref{hexagon},
        \item a functor $J:\C\to\Set$, with a natural isomorphism
                \[
                e_{A,B}: J(A\lolli B) \rTo^\cong \C(A,B),
                \]
                subject to the condition obtained by translating \pref{II}
                into the promonoidal setting.
\end{itemize}
%
(For the original definition in the draft \citep{LSFreeBool}, in terms
of \emph{virtual objects}, see Appendix~\ref{app-ls}.)
%
It is clear that every \SECC is a Lamarche-Stra\ss burger category;
however the converse appears to be false. In fact, certain properties are
desired of these categories which do not appear to be derivable from
the given conditions. In the case where the functor $J$ is representable,
the axioms do not seem to imply that the category is symmetric monoidal
closed.
%
Also it is claimed (p.~3 of op. cit.)
that the canonical natural transformation $JA\times JB \to J(A\tn B)$
`agrees well with associativity'.\footnote{
        Our definition does have this property, as
        shown in \S\ref{s-tensel}.
}
In fact, one of the motivations
behind producing this preprint is to point out that, at least, we can
provide a solution to the problem of finding a definition with the
desired properties. In correspondence Lamarche and Stra\ss
burger have indicated that they might change this definition in the
final version of their paper.

For the moment, we should like to point out that their current
definition of `autonomous functor' (their Def.~2.1.4) also does not
quite do what one might expect.
On the one hand it does not demand
(nor imply) that the functor preserves the $\curry$ isomorphism, on
the other it demands a natural \emph{isomorphism} $J\cong JF$. (This
latter condition is not satisfied, for example, by the unique functor
$\Set\to\mathbf1$.)


\subsection{The Do{\v s}en-Petri{\'c} Definition}\label{s-dp}
\citet{ProofNetCats} define what they call a \defn{proof-net category}
to be a unitless linearly distributive category in which each object
has a dual in a suitable sense. This is a reasonable approach, and
we conjecture that the resulting definition is equivalent to ours.

Early in our exploration of candidates for \sstac,
we considered essentially the same definition.
Ultimately we chose the approach more analogous to the
standard progression from monoidal category to \stac,
via symmetry and closure: it ties directly into the pioneering work
of \citet{EKClosed}, with the \pref{psicoh} diagram, and also
Day's promonoidal categories \citeyearpar{DayPro}.
Along the way it gives us a resonable notion of model for
the intuitionistic fragment of \mll.

\subsection{Eilenberg and Kelly}\label{s-ek}
The pioneering work of \citet{EKClosed} -- in which closed
categories are defined for the first time -- also deserves to be
mentioned here. The authors define `symmetric monoidal closed
category' using a large number of axioms with a great deal
of redundancy. (Our diagram \pref{psicoh} is one of them.)
It seems reasonable to conjecture that, if one were to delete
the unit and the axioms involving it from this original
definition, the structures satisfying the remaining axioms
would be just the \SECCs.

\section{Ongoing Work}
There are general definitions of \emph{lax promonoidal functor}
\citep{DayMonoidalMonads}, \emph{strong promonoidal functor}
\citep{PromonKan}, and \emph{promonoidal \staraut{} functor}
\citep{DSQuantum}.
These definitions need to be specialised to \SECCs and \sstacs in a
sensible way.

We are also working on refining the definition of \sstac, to
give an elementary description that does not rely on the
definition of \SECC.
We conjecture that our definition is equivalent to
\citeauthor{ProofNetCats}'s proof-net categories -- see \S\ref{s-dp}.

As mentioned in the introduction, we hope to show that the proof-net
category of \citet{MallNets} is the free \sstac with finite products
(free in a 2-categorical sense\footnote{Since MALL formulas are implicitly
quotiented by de Morgan duality.}).

\appendix
\section{Axioms for Monoidal Categories}\label{app-ax}
\renewcommand\theequation{\thesection.\arabic{equation}}
\setcounter{equation}{0}
This appendix describes some axioms for a
symmetric monoidal category, and shows that this axiomatisation
is equivalent to the usual one.\footnote{
        There is every chance that this axiomatisation is somewhere
        in the literature. I don't know a reference for it though.
}
%
(Of course we are really interested in promonoidal categories:
the arguments here can readily be transferred to the more
general setting.)

A monoidal category is a category $\C$ equipped with a functor
\[
        \tn:\C\times\C \to\C
\]
and an object $I\in\C$, together with natural isomorphisms
$\alpha$, $\lambda$ and $\rho$ having components
\[\begin{array}{r@{{}\to{}}l}
        \alfa ABC: (A\tn B)\tn C & A\tn(B\tn C)\\
        \lambda_A: I\tn A & A\\
        \rho_A: A\tn I & A
\end{array}\]
such that the following diagrams commute for all $A$,$B$,$C$,$D\in\C$:

\begin{mspill}\begin{diagram}%\dlabel{pentagon}
  \bigl((A\tensor B)\tensor C\bigl)\tensor D
  &\rTo^{\alpha_{A\tn B,C,D}}&(A\tensor B)\tensor (C\tensor D)
  &\rTo^{\alpha_{A,B,C\tn D}} & A\tensor \bigl(B\tensor (C\tensor D)\bigr)
  \\
  &\rdTo[snake=-1em](1,2)<{\alpha_{A,B,C}\tn D} &\dnum[pentagon]
  & \ruTo[snake=1em](1,2)>{A\tn\alpha_{B,C,D}}
  \\
  & \spleft{\bigl(A\tensor(B\tensor C)\bigr)\tensor D}
  & \rTo_{\alpha_{A,B\tn C,D}}
  & \spright{A\tensor\big((B\tensor C)\tensor D\big)}
\end{diagram}\end{mspill}

\begin{diagram}\dlabel{AIC}
        (A\tn I)\tn C &\rTo^{\alpha_{A,I,C}}&A\tn(I\tn C)\\
        &\rdTo[snake=-1ex](1,2)<{\rho_A\tn C}\raise1ex\dnum\ldTo[snake=1ex](1,2)>{A\tn\lambda_C}\\
        &A\tn C
\end{diagram}
These axioms have many interesting consequences. Most importantly,
it follows that the following three diagrams commute for all
$A$,$B$,$C\in\C$:
\[\begin{array}{c@{\qquad}c}
        \begin{diagram}\dlabel{IBC}
                (I\tn B)\tn C &\rTo^{\alfa IBC}&I\tn(B\tn C)\\
                &\rdTo[snake=-1ex](1,2)_{\lambda_B\tn C}\raise1ex\dnum
                        \ldTo[snake=1ex](1,2)_{\lambda_{B\tn C}}\\
                &B\tn C
        \end{diagram}
        &
        \begin{diagram}\dlabel{ABI}
                (A\tn B)\tn I &\rTo^{\alfa ABI}&A\tn(B\tn I)\\
                &\rdTo[snake=-1ex](1,2)_{\rho_{A\tn B}}\raise1ex\hbox{(\theequation)}\ldTo[snake=1ex](1,2)_{A\tn\rho_B}\\
                &A\tn B
        \end{diagram}
        \\
        \multicolumn 2c{
        \begin{diagram}\dlabel{II}
                \rnode{II}{I\tn I} &\dnum& \rnode{I}{I}\\
                \nccurve[angleA=60,  angleB=120] {->}{II}{I}\Aput{\lambda_I}
                \nccurve[angleA=-60, angleB=-120]{->}{II}{I}\Bput{\rho_I}
        \end{diagram}}
\end{array}\]
\citet{BTC} give a simple and elegant proof.
There are other possible axiomatisations: for example, conditions
\pref{pentagon}, \pref{IBC} and \pref{II} are also collectively
sufficient.\footnote{This can be proved by the technique of \citet{BTC}.}

A \emph{symmetric} monoidal category is a monoidal category $\C$
equipped with a natural isomorphism $\sigma$ having components
\[
        \sigma_{A,B}: A\tn B \to B\tn A,
\]
such that $\sigma_{A,B} = \sigma_{B,A}^{-1}$ and the following
commutes for all $A$,$B$,$C\in\C$:
\begin{diagram}%\dlabel{hexagon}
  (A\tn B)\tn C &\rTo^{\alfa ABC} & A\tn(B\tn C)
    & \rTo^{\sigma_{A,B\tn C}} & (B\tn C)\tn A \\
  \dTo<{\sigma_{A,B}\tn C} &&\dnum[hexagon]&& \dTo>{\alfa BCA}\\
  (B\tn A)\tn C & \rTo_{\alpha_{B,A,C}} & B\tn(A\tn C)
    & \rTo_{B\tn\sigma_{A,C}} & B\tn(C\tn A)
\end{diagram}
It then follows that
\begin{diagram}[w=4em]\dlabel{IA}
        I\tn A &\rTo^{\sigma_{I,A}}&A\tn I\\
        &\rdTo[snake=-1ex](1,2)_{\lambda_A}
        \raise1ex\dnum
        \ldTo[snake=1ex](1,2)_{\rho_A}\\
        &A
\end{diagram}
commutes for every $A\in\C$; again, see \citet{BTC} for a proof.

Conditions \pref{pentagon} and \pref{hexagon} are indispensable;
the usual definition of symmetric monoidal category requires \pref{AIC}
in addition. However, it is sometimes convenient to eliminate the
$\rho$ isomorphism from the data: that is permissible, since
\pref{IA} shows that $\rho$ may be defined in terms of $\lambda$
and $\sigma$. It turns out that, in this situation, we may require
\pref{IBC} in place of \pref{AIC}. Specifically:
\begin{propn}\label{prop-A1}
        If \pref{IBC}, \pref{hexagon} and \pref{IA} hold then so does \pref{AIC}.
\end{propn}
\begin{proof}
Consider the diagram
%\newcommand\nat{\hbox{($\natural$)}}%
\let\nat = \natural%
\begin{diagram}[labelstyle=\scriptstyle]
  &&(I\tn A)\tn C &\rput(0,-.5){\hbox{\pref{IBC}}}\rTo^{\alfa IAC}&I\tn(A\tn C)
 \\
  &\ruTo^{\sigma_{A,I}\tn C}&\dTo>{\lambda_A\tn C}
  &\ldTo_{\lambda_{A\tn C}}&\nat&\rdTo^{I\tn\sigma_{A,C}}
 \\
  (A\tn I)\tn C&\rTo_{\rho_A\tn C}&\rput(-.4,.9){\hbox{\pref{IA}}}A\tn C
  &\rTo_{\sigma_{A,C}}&C\tn A\rput(.5,-.6){\hbox{\pref{IBC}}}
  &\lTo^{\lambda_{C\tn A}}&I\tn(C\tn A)
 \\
  &\rdTo_{\alfa AIC}&\uTo>{A\tn\lambda_C}&\nat&\uTo<{\lambda_C\tn A}&\ruTo_{\alfa ICA}
 \\
  &&A\tn(I\tn C)&\rTo_{\sigma_{A,I\tn C}}&(I\tn C)\tn A
\end{diagram}
The outside is an instance of \pref{hexagon}, and the labelled
regions commute for the reasons marked.
Since all the morphisms are invertible, it follows that the
unlabelled region at lower left commutes. This region is just \pref{AIC}.
\end{proof}
%
In summary, we may define a symmetric monoidal category to be
a category $\C$ with a functor $\tn$ and a unit object $I$,
together with natural isomorphisms $\alpha$, $\lambda$ and
$\sigma$ such that $\sigma_{A,B} = \sigma_{B,A}^{-1}$ and
diagrams \pref{pentagon}, \pref{hexagon} and \pref{IBC} commute.

\section{The Lamarche-Stra\ss burger Definition}\label{app-ls}
\newcommand\A{\mathbb{A}}
\newcommand\B{\mathbb{B}}
\newcommand\I{\mathbb{I}}
\renewcommand\C{\mathcal{C}}

%% Uncomment the following line if you want
%  the labels to be smaller, as they are in
%  the Lamarche-Strassburger paper:
%
%\let\labelstyle = \scriptstyle

In their draft, \cite{LSFreeBool} give a very interesting discussion
of autonomous categories without units, and the following definition
of \emph{(unitless) autonomous category}, based on the notion of a
\emph{virtual object}.  In Section~\ref{s-ls} we presented the definition
in our own notation (i.e., promonoidal style); for the sake of
completeness, here is (a condensed presentation of) the original
definition of \cite{LSFreeBool}, in terms of virtual objects.

%\paragraph*{Definition B.1.} 
A category $\C$ \emph{has tensors} if it is
equipped with a bifunctor $-\tn-:\C\times\C\to\C$ with the usual
associativity and symmetry isomorphisms $$\begin{array}{rl}
{}\textsf{assoc}_{A,B,C}:& A \tn(B \tn C) \to (A \tn B) \tn C\\
{}\textsf{twist}_{A,B} : & A \tn B \to B \tn A
\end{array}$$
obeying the usual associated coherence laws.
%
When it exists, write $-\lolli-:\C\op\times\C\to\C$ for internal hom,
defined by adjointness to tensor as in the usual case with units.
Write $h_X=\C(-,X)$ and $h^X=\C(X,-)$ for the hom functors, and
$H^X=X\lolli-:\C\to \C$.
%
Writing a functor $\C\to\Set$ as $h^\A$ for a symbol $\A$ allows one
to write an element $s$ of the set $h^\A(s)$, corresponding (by
Yoneda's Lemma) to a natural transformation $h^X\to h^\A$, as
\[
        \A \rTo.^s X
\]
When the symbol $\A$ is an object of $\C$, the functor $h^\A$ is
representable, in the usual sense; when $\A$ is not an object of $\C$,
it is a \emph{virtual object}.
%
In general a dotted arrow will mean at least one of the source or
target is virtual, and should be interpreted as a reverse-direction
natural transformation between the corresponding functors. For
example, given $f : X \to Y$ and $t = (h^{\A}f)(s)$, one can draw the
`commutative diagram'
\begin{diagram}[h=1em,w=1.5em]
        &&\A\\
        &\ldTo.^s && \rdTo.^t\\
        X && \rTo_f && Y\;\;,
\end{diagram}
justifying the notation $t = f \circ s$, or simply $t = fs$.  Define
$\A\tn X$ in the obvious way, i.e., $h^{\A\tn X}=h^\A H^X$.  This
construction is natural in both variables: given $s:\A \rTo. \B$
(between virtual objects) and a morphism $f:X\to Y$, there is an
obvious \(s\tn f:\A\tn X \rTo. \B\tn Y\).

\begin{definition}
A category $\C$ with tensors is an \emph{autonomous category} if it
has an internal hom $\lolli$ and a functor $h^\I$ with a natural isomorphism
\[h^\I(X\lolli Y)\;\cong\;\C(X,Y)\]
such that the following diagram (of mostly virtual arrows) commutes:
\begin{diagram}[h=1.5em]
        &&X &\rTo.^\cong&\I\tn X&\rTo.^{t\tn X}& Y\tn X\\
        &\ruTo.^s\\
        \I && && && \dTo>\cong\\
        &\rdTo._t\\
        &&Y&\rTo._\cong&\I\tn Y&\rTo._{s\tn Y}&X\tn Y
\end{diagram}
\end{definition}
This diagram corresponds to diagram \pref{II} of the previous
appendix, translated into promonoidal style.

%%\subsection*{$\ast$-autonomous categories without units}

%%We will define autonomous (SMC) and *-autonomous
%%categories not to have units by default. This spares us from
%%having *-autonomous categories without units with units.

%%From now on $\C$ denotes a (small) category. We denote the composition
%%of two maps $f,g$ by either $gf$ or $g \circ f$, depending on
%%readability; the order is the standard (functional, as opposed to
%%diagrammatic) order. Given $X\in\C$, we will write either $X$ or $1_X$
%%to represent the identity map on it, according to readability. We use
%%the standard notation for the covariant representable functor
%%associated with $X$, i.e., $h^X = \text{Hom}_\C(X,-)$, and $h_X$ for
%%the contravariant representable $\text{Hom}_\C(-,X)$.

%%The arguments in the following section need familiarity with Yoneda's
%%Lemma: given a functor $F : \C \to \textbf{Set}$ there is a natural
%%bijective correspondence between $F(X)$ and the set of natural
%%transformations $h^X \to F$.

%\paragraph*{Definition} A category $\C$ \emph{has tensors} if it is
%equipped with a bifunctor $-\tn-$ with the usual associativity and
%symmetry isomorphisms $$\begin{array}{rl}
%{}\textsf{assoc}_{A,B,C}:& A \tn(B \tn C) \t (A \tn B) \tn C\\
%{}\textsf{twist}_{A,B : & A \tn B \to B \tn A
%\end{array}$$
%that obey the usual ``pentagon'' and ``hexagon'' (see
% [15\dcomment{Add reference!}, p.158,p.180]).  Note that we do not ask
% for a unit in that definition.  Nonetheless the ``coherence'' theorem
% for symmetric monoidal categories [14\dcomment{Add reference!}] does
% also hold in our case, or more precisely everything in it that does
% not deal with units. In particular, we can simply write $A \tn B \tn
% C \tn D$ for $((A \tn B) \tn C) \tn D$ or $A \tn ((B \tn C) \tn D)$,
% or even $(B \tn D) \tn (A \tn C)$, because there is a uniquely
% defined isomorphism between any two of them.  If it exists, we denote
% the usual right adjoint to tensoring as $(-) \lolli (-)$ and it
% defines the usual bivariant bifunctor.  We will denote the ``internal
% representable functor'' defined by $X$ as $H^X = X \lolli (-) : \C
% \to \C$. The following two natural isomorphisms are trivial but
% important, and they are natural in both $X$ and $Y$:
% \begin{equation}H^XH^Y \cong H^{X\tn Y}\text{ \;\;\;and\;\;\; }
% h^XH^Y \cong h^{X\tn Y}.\end{equation} It is very well known that a
% functor $\C \to \textbf{Set}$ can be profitably seen as a
% ``generalized object'' of $\C$; we call such a thing a virtual object
% of $\C$ and we emphasize this fact by writing it as $h^\A$, which is
% a functor, and would be the representable functor associated to the
% object $\A$ if the latter only existed. Given $X \in \C$, maps $\A
% \to X$ should morally be in bijective correspondence with natural
% transformations $h^X \to h^\A$, and the latter are truly in bijective
% correspondence with elements of $h^\A(X)$ and this allows us to write
% an $s \in h^\A(X)$ as
%\[
%       \A \rTo.^s X
%\]
%In general a dotted arrow will mean that at least one of the source or
%target is virtual, and it is to be interpreted as a reverse-direction
%natural transformation between the corresponding functors. For
%example, given $f : X \to Y$ and $t = (h^{\A}f)(s)$, we can write this
%as a commutative diagram
%\begin{diagram}[h=1em,w=1.5em]
%       &&\A\\
%       &\ldTo.^s && \rdTo.^t\\
%       X && \rTo_f && Y
%\end{diagram},
%which justifies the notation $t = f \circ s$, or simply $t = fs$.  But
%we have to be very careful on how to extend the $\tn,\lolli$ structure
%to virtuals. At least one thing works: given a virtual object $\A$ and
%a real one $X$ we can define a virtual object $A\tn X$, by
%composing their ``representables'' (the reader should check that this
%makes perfect sense, by plugging an object of $\C$ in the functors):
%$h^{\A\tn X} = h^\A H^X$.


%%% Diagram 3
%\begin{diagram}
%       h^{X'} &\rTo^{h^f}& h^X\\
%       \dTo<{h^\lambda X'}>\cong && \dTo<\cong>{h^\lambda X}\\
%       h^\I H^{X'} &\rTo_{h^\I H^f}& h^\I H^X
%\end{diagram}

%%% Diagram 4
%\begin{diagram}
%       \I\tn X\tn Y & \rTo^\cong& \I\tn Y\tn X\\
%       \dTo.<{\lambda_X\tn Y}&&\dTo.>{\lambda_Y\tn X}\\
%       X\tn Y &\rTo_\cong& Y\tn X
%\end{diagram}

%%% Diagram 5
%\begin{diagram}[h=1.5em]
%       &&X &\rTo.^\cong&\I\tn X&\rTo.^{t\tn X}& Y\tn X\\
%       &\ruTo.^s\\
%       \I && && && \dTo>\cong\\
%       &\rdTo._t\\
%       &&Y&\rTo._\cong&\I\tn Y&\rTo._{s\tn Y}&X\tn Y
%\end{diagram}

\bibliography{cs}
\end{document}
