%&format-test
%!TEX TS-program = tex
%\documentclass{robinthesisdraft}
%\usepackage{robincs,thesisdefs,xr}
\externaldocument[Bicats:]{bicats}
\externaldocument[Psmon:]{pseudomonoids}

\title{Monoidal Bicategories}
\begin{document}
\maketitle

\begin{definition}\label{def-monbicat}
	A monoidal bicategory $\B$ is a bicategory equipped with
	a unit object $\I\in\B$, a pseudo-functor
	\[
		\tn: \B\x\B \to \B,
	\]
	pseudo-natural equivalences $a$, $l$ and $r$ with components
	\[\begin{array}l
		a_{A,B,C}: A\tn (B\tn C) \to (A\tn B)\tn C,\\
		l_A: \I\tn A\to A,\\
		r_A: A\tn \I\to A,
	\end{array}\]
	and invertible modifications $\pi$, $\mu$, $L$ and $R$ with components
	\begin{diagram}
	  A\tensor \bigl(B\tensor (C\tensor D)\bigr)
	  &\rTo^{a_{A,B,C\tn D}}&(A\tensor B)\tensor (C\tensor D)
	  &\rTo^{a_{A\tn B,C,D}} & \bigl((A\tensor B)\tensor C\bigl)\tensor D
	  \\
	  &\rdTo[snake=-1em](1,2)<{A\tn a_{B,C,D}}
	  &\Downarrow\pi_{A,B,C,D}
	  & \ruTo[snake=1em](1,2)>{a_{A,B,C}\tn D}
	  \\
	  & \spleft{A\tensor\big((B\tensor C)\tensor D\big)}
	  & \rTo_{a_{A,B\tn C,D}}
	  & \spright{\bigl(A\tensor(B\tensor C)\bigr)\tensor D}
	\end{diagram}
%
	\begin{diagram}
		A\tn(\I\tn C) &\rTo^{a_{A,\I,C}}&(A\tn \I)\tn C\\
		&\rdTo[snake=-1ex](1,2)<{A\tn l_C}
			\raise1ex\hbox{$\begin{array}c\Rightarrow\\[-5pt]\mu_{A,C}\end{array}$}%
			\ldTo[snake=1ex](1,2)>{r_A\tn C}\\
		&A\tn C
	\end{diagram}
\[
	\begin{diagram}
		\I\tn(B\tn C) &\rTo^{a_{\I,B,C}}&(\I\tn B)\tn C\\
		&\rdTo[snake=-1ex](1,2)<{l_{B\tn C}}
			\raise1ex\hbox{$\begin{array}c\Rightarrow\\[-5pt]L_{B,C}\end{array}$}%
			\ldTo[snake=1ex](1,2)>{l_B\tn C}\\
		&B\tn C
	\end{diagram}
%
	\hskip 3em
%
	\begin{diagram}
		A\tn(B\tn \I) &\rTo^{a_{A,B,\I}}&(A\tn B)\tn \I\\
		&\rdTo[snake=-1ex](1,2)<{A\tn r_B}
			\raise1ex\hbox{$\begin{array}c\Rightarrow\\[-5pt]R_{A,B}\end{array}$}%
			\ldTo[snake=1ex](1,2)>{r_{A\tn B}}\\
		&A\tn B
	\end{diagram}
\]
such that for all $A$, $B$, $C$, $D$ and $E$ in $\B$, the condition
shown in Fig.~\ref{fig-4coc} holds,
\begin{sidewaysfigure}
\begin{diagram}
	  &&
	  (A\tn B)\tensor \bigl(C\tensor (D\tensor E)\bigr)
	  &\rTo[rightshortfall=4.5em]^{a}&\spboth{((A\tn B)\tensor C)\tensor (D\tensor E)}
	  &\rTo[leftshortfall=4.5em]^{a} & \bigl(((A\tn B)\tensor C)\tensor D\bigl)\tensor E
	  \\
	  &\ruTo^a&
	  &\rdTo[snake=-1em](1,2)<{(A\tn B)\tn a}
	  &\spboth{\Downarrow\pi_{(A\tn B),C,D,E}}
	  & \ruTo[snake=1em](1,2)>{a\tn E}
	  && \spboth{((A\tn(B\tn C))\tn D)\tn E} \luTo[snake=1.5em](1,1)^{(a\tn D)\tn E}
	  \\
	  A\tn(B\tn(C\tn(D\tn E)))
	  && \Downarrow a^{-1}
	  & {(A\tn B)\tensor\big((C\tensor D)\tensor E\big)}
	  & \rTo_{a}
	  &{\bigl((A\tn B)\tensor(C\tensor D)\bigr)\tensor E}
	  &\begin{array}c\To\\[-4pt]\pi_{A,B,C,D}\tn E\end{array}
	  & \uTo>{a\tn E}
	  \\
	  & \rdTo_{A\tn(B\tn a)}
	  & \ruTo(1,2)^a
	  &&\spboth{\Searrow\pi_{A,B,(C\tn D),E}}
	  &&\luTo(1,2)>{a\tn E}
	  &\spboth{(A\tn((B\tn C)\tn D))\tn E}
	  \\
	  &&{A\tn(B\tn((C\tn D)\tn E))}
	  & \rTo[rightshortfall=4.5em]_{A\tn a} & \spboth{A\tn((B\tn(C\tn D))\tn E)}
	  & \rTo[leftshortfall=4.5em]_a & (A\tn(B\tn(C\tn D)))\tn E
	   \ruTo[snake=1.5em](1,1)_{(A\tn a)\tn E}
\end{diagram}
must be equal to
\begin{diagram}
	&(A\tn B)\tn(C\tn(D\tn E))
	& \rTo^a & ((A\tn B)\tn C)\tn(D\tn E)
	& \rTo^{a} & (((A\tn B)\tn C)\tn D)\tn E
	\\
	\ruTo(1,3)^a &&\raise-2em\spleft{\Searrow\pi_{A,B,C,(D\tn E)}}
	& \uTo<{a\tn(D\tn E)} & \Searrow a
	& \uTo>{(a\tn D)\tn E}
	\\
	&&& (A\tn(B\tn C))\tn(D\tn E) & \rTo^a
	& ((A\tn(B\tn C))\tn D)\tn E
	\\
	A\tn(B\tn(C\tn(D\tn E)))
	& \rTo[rightshortfall=4.5em]^{A\tn a}
	& \spboth{A\tn((B\tn C)\tn(D\tn E))} \ruTo[snake=-1.5em](1,1)^a
	&& \Searrow\pi_{A,(B\tn C),D,E} & \uTo>{a\tn E}
	\\
	&\rdTo(1,3)_{A\tn(B\tn a)}
	&& A\tn(((B\tn C)\tn D)\tn E) \rdTo[snake=-1.5em](1,1)_{A\tn a}
	& \rTo_a & (A\tn((B\tn C)\tn D))\tn E
	\\
	&&\raise2em\spleft{\Downarrow A\tn\pi_{B,C,D,E}}
	&\uTo>{A\tn(a\tn E)} & \Searrow a & \uTo>{(A\tn a)\tn E}
	\\
	& A\tn(B\tn((C\tn D)\tn E)) & \rTo_{A\tn a} & A\tn((B\tn(C\tn D)\tn E)
	& \rTo_a & (A\tn(B\tn(C\tn D)))\tn E
\end{diagram}
\caption{The associativity axiom used in the definition of monoidal bicategory (sometimes
	called the \textit{non-abelian 4-cocycle condition}).
}\label{fig-4coc}
\end{sidewaysfigure}
and for all $A$, $B$ and $C$, the following two conditions hold:
\begin{diagram}
	&&A\tn(B\tn C) & \rTo^{a_{A,B,C}} & (A\tn B)\tn C
	\\
	&\ruTo^{A\tn l_{B\tn C}}_{\Searrow \mu_{A,B,C}}
	&\uTo>{r_A\tn (B\tn C)} && \uTo>{(r_A\tn B)\tn C}<{\Searrow a_{r_A,B,C}\hskip2em}
	\\
	A\tensor \bigl(\I\tensor (B\tensor C)\bigr)
	 &\rTo^{a_{A,\I,B\tn C}}&(A\tensor \I)\tensor (B\tensor C)
	&\rTo^{a_{A\tn \I,B,C}} & \bigl((A\tensor \I)\tensor B\bigl)\tensor C
	\\
	&\rdTo[snake=-1em](1,2)<{A\tn a_{\I,B,C}}
	&\Downarrow\pi_{A,\I,B,C}
	& \ruTo[snake=1em](1,2)>{a_{A,\I,B}\tn C}
	\\
	& \spleft{A\tensor\big((\I\tensor B)\tensor C\big)}
	& \rTo_{a_{A,\I\tn B,C}}
	& \spright{\bigl(A\tensor(\I\tensor B)\bigr)\tensor C}
\end{diagram}
is equal to
\begin{diagram}
	&&A\tn(B\tn C) & \rTo^{a_{A,B,C}} & (A\tn B)\tn C
	\\
	&\ruTo^{A\tn l_{B\tn C}} &&&& \luTo^{(r_A\tn B)\tn C}
	\\
	A\tn(\I\tn(B\tn C)) & \spright{\Searrow A\tn L_{B,C}} & \uTo[snake=1.5em]>{A\tn(l_B\tn C)}
	& \Searrow a_{A,l_B,C} & \uTo[snake=-1.5em]<{(A\tn l_B)\tn C}
	& \spleft{\begin{array}c\To\\[-4pt]\mu_{A,B}\tn C\end{array}}
	& ((A\tn \I)\tn B)\tn C
	\\
	&\rdTo_{A\tn a_{\I,B,C}}
	&&&& \ruTo_{a_{A,\I,B}\tn C}
	\\
	&&A\tn((\I\tn B)\tn C) & \rTo_{a_{A,\I\tn B,C}} & (A\tn(\I\tn B))\tn C
\end{diagram}
and
\begin{diagram}
	A\tn(B\tn C) & \rTo^{a_{A,B,C}} & (A\tn B)\tn C
	\\
	\uTo<{A\tn (B\tn l_C)}>{\hskip3em\Searrow a_{A,B,l_C}}
	&& \uTo<{(A\tn B)\tn l_C}>{\quad\begin{array}c\To\\[-4pt]\mu_{(A\tn B),C}\end{array}}
	& \luTo^{r_{A\tn B}\tn C}
	\\
	A\tn (B\tn (\I\tn C))
	&\rTo^{a_{A,B,\I\tn C}}
	& (A\tn B)\tn (\I\tn C)
	&\rTo_{a_{(A\tn B),\I,C}}& ((A\tn B)\tn \I)\tn C
	\\
	&\rdTo[snake=-1em](1,2)<{A\tn a_{B,\I,C}}
	&\Downarrow\pi_{A,\I,B,C}
	& \ruTo[snake=1em](1,2)>{a_{A,\I,B}\tn C}
	\\
	& \spleft{A\tn\big((B\tn \I)\tn C\big)}
	& \rTo_{a_{A,B\tn \I,C}}
	& \spright{\bigl(A\tn(B\tn \I)\bigr)\tn C}
\end{diagram}
is equal to
\begin{diagram}
	&&A\tn(B\tn C) & \rTo^{a_{A,B,C}} & (A\tn B)\tn C
	\\
	&\ruTo^{A\tn (B\tn l_C)} &&&& \luTo^{r_{A\tn B}\tn C}
	\\
	A\tn(B\tn(\I\tn C)) & \spright{\Searrow A\tn \mu_{B,C}} & \uTo[snake=1.5em]>{A\tn(r_B\tn C)}
	& \Searrow a_{A,r_B,C} & \uTo[snake=-1.5em]<{(A\tn r_B)\tn C}
	& \spleft{\begin{array}c\To\\[-4pt]R_{A,B}\tn C\end{array}}
	& ((A\tn B)\tn \I)\tn C
	\\
	&\rdTo_{A\tn a_{B,\I,C}}
	&&&& \ruTo_{a_{A,B,\I}\tn C}
	\\
	&&A\tn((B\tn \I)\tn C) & \rTo_{a_{A,B\tn \I,C}} & (A\tn(B\tn \I))\tn C
\end{diagram}
\end{definition}
This is not \emph{quite} the most general possible definition, since we have
merely specified an object $\I$ rather than a pseudo-functor $1\to\B$. But since
every pseudo-functor is equivalent to a normal one, there is no essential loss of
generality.

When we have occasion to refer explicitly to the equivalence-inverse of
$a$, $l$ or $r$, we shall denote it as $a'$, $l'$ or $r'$. Furthermore,
we shall assume where necessary that we have \emph{adjoint} equivalences
$a\dashv a'$, $l\dashv l'$ etc.

When working in
a monoidal bicategory, we extend our convention of not explicitly naming
structural isomorphisms to the isomorphisms representing the pseudo-functoriality
of tensor. Instead we mark them with the symbol $\sim$. Note that there
will usually be some implicit structural isomorphisms too:
for example, given 1-cells $f:A\to B$ and $g: C\to D$, the diagram
\begin{diagram}
	A\tn B & \rTo^{A\tn g} & A\tn D\\
	\dTo<{f\tn B} & \sim & \dTo>{f\tn D}\\
	C\tn B & \rTo_{C\tn g} & C\tn D
\end{diagram}
indicates the 2-cell
\begin{mspill}
	(f\tn D)\o(A\tn g) \rTo^{\tn_{(f,1),(1,g)}} (f\o 1)\tn(1\o g) \rTo^{\r_f\tn\l_g}f.g
		\rTo^{\l_f^{-1}\tn\r_g^{-1}} (1.f)\tn(g.1) \rTo^{\tn_{(1,f),(g,1)}^{-1}} (C\tn g)\o(f\tn B).  
\end{mspill}

\begin{remark}\label{rem-defining-L} % defining L in terms of the other data
	Notice that the first unit equation -- since all its cells are invertible -- allows
	$A\tn L_{B,C}$ to be expressed in terms of $\pi$, $\mu$ and $a$. In particular
	$\I\tn L_{B,C}$ may be so expressed
	which, since the pseudofunctor $\I\tn-$ is equivalent, via $l$, to the identity,
	allows $L_{B,C}$ to be expressed in terms of $\pi$, $\mu$, $a$ and the implicit
	data that make $l$ an equivalence. Furthermore it is not hard to see that the
	components $L_{A,B}$ thus defined constitute a modification. Therefore the
	modification $L$ may be defined in terms of the other data. It is perhaps
	tempting to conclude that $L$ is redundant, but there
	is a subtlety: the modification $L$ so defined does not necessarily -- at least as far
	as I can tell -- satisfy the necessary equation. Thus we \emph{could} suppress $L$
	(and $R$ too, since its definition is symmetrical to that of $L$) from our
	definition, but only at the expense of introducing a new and complicated
	equation involving the other data.
\end{remark}

\section{Monoidal pseudo-functors and transformations}
\begin{definition} % monoidal pseudo-functor
	A \emph{monoidal pseudofunctor} $F: \B\to\BC$,
	between monoidal bicategories $\B$ and $\BC$, consists of a pseudofunctor
	equipped with:
	\begin{itemize}
	\item a 1-cell $F^\tn_\I: \I\to F\I$,
	\item a pseudo-natural transformation $F^\tn$ with components
	\[
		F^\tn_{A,B}: FA\tn FB \to F(A\tn B)
	\]
	\item an invertible modification $F^a$ with components
	\begin{diagram}
		FA\tn(FB\tn FC) & \rTo^{a_{FA,FB,FC}} & (FA\tn FB)\tn FC\\
		\dTo<{FA\tn F^\tn_{B,C}} && \dTo>{F^\tn_{A,B}\tn FC}\\
		FA\tn F(B\tn C) &\Nearrow F^a_{A,B,C}& F(A\tn B)\tn FC\\
		\dTo<{F^\tn_{A,B\tn C}} && \dTo>{F^\tn_{A\tn B,C}}\\
		F(A\tn(B\tn C)) & \rTo_{F(a_{A,B,C})} & F((A\tn B)\tn C)
	\end{diagram}
	\end{itemize}
	\item invertible modifications $F^l$ and $F^r$ with components
	\[
	\begin{diagram}
		\I\tn FA & \rTo^{F^\tn_\I\tn FA} & F\I \tn FA\\
		\dTo<{l_{FA}} & \begin{array}c\Rightarrow\\F^l_{A}\end{array} & \dTo>{F^\tn_{\I,A}}\\
		FA & \lTo_{F(l_{A})} & F(\I\tn A)
	\end{diagram}
	\qquad
	\begin{diagram}
		FA\tn \I & \rTo^{FA\tn F^\tn_\I} & FA \tn F\I\\
		\dTo<{r_{FA}} & \begin{array}c\Rightarrow\\F^r_{A}\end{array} & \dTo>{F^\tn_{A,\I}}\\
		FA & \lTo_{F(r_{A})} & F(A\tn \I)
	\end{diagram}
	\]
	\item satisfying the equation shown in Figs.~\ref{fig-monpsf-1}
		and~\ref{fig-monpsf-2}, and
	\begin{sidewaysfigure}
		\begin{diagram}
			%&&
			FA\tn(FB\tn(FC\tn FD))
			& \rTo^{FA\tn(FB\tn F^\tn_{C,D})} & FA\tn(FB\tn F(C\tn D))
			& \rTo^{FA\tn F^\tn_{B,C\tn D}} & FA\tn F(B\tn(C\tn D))
			& \rTo^{F^\tn_{A,B\tn(C\tn D)}} & \rnode{t}{F(A\tn(B\tn(C\tn D)))}
			\\
			%&%\ldTo(2,4)^{a_{FA,FB,(FC\tn FD)}}
			%&
			\dTo<{FA\tn a_{FB,FC,FD}}
			&& \Swarrow FA\tn F^a_{B,C,D}
			&& \dTo<{FA\tn F(a_{B,C,D})}
			& \Swarrow (F^\tn_{A,a_{B,C,D}})^{-1}
			& \dTo>{F(A\tn a_{B,C,D})}
			\\
			%&&
			FA\tn((FB\tn FC)\tn FD)
			& \rTo^{FA\tn(F^\tn_{B,C}\tn FD)} & FA\tn(F(B\tn C)\tn FD)
			& \rTo^{FA\tn F^\tn_{B\tn C, D}} & FA\tn F((B\tn C)\tn D)
			& \rTo^{F^\tn_{A,(B\tn C)\tn D}} & F(A\tn((B\tn C)\tn D)
		%	\\
		%	%&&
		%	&&&& \dTo>{F^\tn_{A,(B\tn C)\tn D}}
			\\
			%\spleft{\rnode{l}{(FA\tn FB)\tn(FC\tn FD)}}
			%& \spright{\begin{array}c\To\\\pi_{FA,FB,FC,FD}\end{array}}
			%&
			\dTo<{a_{FA,FB\tn FC, FD}}
			& {\Swarrow \a_{FA,F^\tn_{B,C}, FD}}
			& \dTo>{a_{FA,F(B\tn C), FD}}
			&& \Swarrow F^a_{A,B\tn C,D}
			%& F(A\tn((B\tn C)\tn D))
			&& \dTo<{F(a_{A,B\tn C,D})}
			& \spleft{\begin{array}c\Leftarrow\\ F(\pi_{A,B,C,D})\end{array}}
			& \spright{\rnode{r}{F((A\tn B)\tn(C\tn D))}}
		%	\\
		%	%&%\rdTo(2,4)_{a_{(FA\tn FB),FC,FD}}
		%	%&
		%	&&&& \dTo>{F(a_{A,(B\tn C),D})}
			\\
			%&&
			(FA\tn(FB\tn FC))\tn FD
			& \rTo_{(FA\tn F^\tn_{B,C})\tn FD} & (FA\tn F(B\tn C))\tn FD
			& \rTo_{F^\tn_{A,B\tn C}\tn FD} & F(A\tn(B\tn C))\tn FD
			& \rTo_{F^\tn_{A\tn(B\tn C), D}} & F((A\tn(B\tn C))\tn D)
			\\
			%&&
			\dTo<{a_{FA,FB,FC}\tn FD}
			&& \Swarrow F^a_{A,B,C}\tn FD
			&& \dTo<{F(a_{A,B,C})\tn FD}
			& \Swarrow (F^\tn_{a_{A,B,C},D})^{-1}
			& \dTo>{F(a_{A,B,C}\tn D)}
			\\
			%&&
			((FA\tn FB)\tn FC)\tn FD
			& \rTo_{(F^\tn_{A,B}\tn FC)\tn FD} & (F(A\tn B)\tn FC)\tn FD
			& \rTo_{F^\tn_{A\tn B,C}\tn FD} & F((A\tn B)\tn C)\tn FD
			& \rTo_{F^\tn_{(A\tn B)\tn C, D}} & \rnode{b}{F(((A\tn B)\tn C)\tn D)}
			\nccurve[angleA=0,angleB=90]{->}tr\Aput{F(a_{A,B,C\tn D})}
			\nccurve[angleA=270,angleB=0]{->}rb\Aput{F(a_{A,B,C\tn D})}
		\end{diagram}
		\caption{Left-hand side of an equation used in the definition of monoidal pseudo-functor:
		for all $A$, $B$, $C$, $D\in\B$, this pasting must be equal to the one shown
		in Fig.~\ref{fig-monpsf-2}.}\label{fig-monpsf-1}
	\end{sidewaysfigure}
	\begin{sidewaysfigure}
		\begin{diagram}
			&&
			\rnode{1}{FA\tn(FB\tn(FC\tn FD))}
			& \rTo^{FA\tn(FB\tn F^\tn_{C,D})} & FA\tn(FB\tn F(C\tn D))
			& \rTo^{FA\tn F^\tn_{B,C\tn D}} & FA\tn F(B\tn(C\tn D))
			& \rTo^{F^\tn_{A,B\tn(C\tn D)}} & \rnode{t}{F(A\tn(B\tn(C\tn D)))}
			\\
			&&
			& \Swarrow a_{FA,FB,F(C\tn D)}
			& \dTo>{a_{FA,FB,F(C\tn D)}}
			\\
			\rnode{2}{FA\tn((FB\tn FC)\tn FD)}
			&& \dTo>{a_{FA,FB,FC\tn FD}}
			&& (FA\tn FB)\tn F(C\tn D)
			&& \Swarrow F^a_{A,B,C\tn D}
			&& \dTo>{F(a_{A,B,C\tn D})}
			\\
			&&
			& \ruTo^{(FA\tn FB)\tn F^\tn_{C,D}}
			&& \rdTo^{F^\tn_{A,B}\tn F(C\tn D)}
			\\
			\dTo<{a_{FA,FB\tn FC, FD}}
			& \spboth{\begin{array}c\Leftarrow\\\pi_{FA,FB,FC,FD}\end{array}}
			& \spright{(FA\tn FB)\tn(FC\tn FD)}
			&& \cong
			&& F(A\tn B) \tn F(C\tn D)
			& \rTo^{F^\tn_{A\tn B,C\tn D}}
			& F((A\tn B)\tn(C\tn D))
			\\
			&&
			& \rdTo_{F^\tn_{A,B}\tn(FC\tn FD)}
			&& \ruTo_{F(A\tn B)\tn F^\tn_{C,D}}
			\\
			\rnode{3}{(FA\tn(FB\tn FC))\tn FD}
			&& \dTo>{a_{FA\tn FB, FC, FD}}
			&& F(A\tn B)\tn(FC\tn FD)
			&& \Swarrow F^a_{A\tn B,C,D}
			&& \dTo>{F(a_{A\tn B,C,D})}
			\\
			&&
			& \Swarrow a_{F^\tn_{A,B},FC,FD}
			& \dTo>{a_{F(A\tn B),FC,FD}}
			\\
			&&
			\rnode{4}{((FA\tn FB)\tn FC)\tn FD}
			& \rTo_{(F^\tn_{A,B}\tn FC)\tn FD} & (F(A\tn B)\tn FC)\tn FD
			& \rTo_{F^\tn_{A\tn B,C}\tn FD} & F((A\tn B)\tn C)\tn FD
			& \rTo_{F^\tn_{(A\tn B)\tn C, D}} & \rnode{b}{F(((A\tn B)\tn C)\tn D)}
			\nccurve[angleA=180,angleB=90]{->}12\Aput{FA\tn a_{FB,FC,FD}}
			\nccurve[angleA=270,angleB=180]{->}34\Aput{a_{FA,FB,FC}\tn FD}
		\end{diagram}
		\caption{Right-hand side of the equation: this pasting must be equal to the one
		shown in Fig.~\ref{fig-monpsf-1}}\label{fig-monpsf-2}
	\end{sidewaysfigure}
	also such that for all $A$, $B\in\B$ the pasting
	\begin{diagram}
		&&
		\rnode{1}{FA\tn(\I\tn FB)}
		& \rTo^{a_{FA,\I,FB}} & (FA\tn \I)\tn FB
		\\
		&&
		\dTo<{FA\tn(F^\tn_\I\tn FB)}
		& \Nearrow a_{FA,F^\tn_\I,FB}
		& \dTo>{(FA\tn F^\tn_\I)\tn FB}
		\\
		&&
		FA\tn(F\I\tn FB)
		& \rTo_{a_{FA,F\I,FB}} & (FA\tn F\I)\tn FB
		& \rTo^{F^\tn_{A,\I}\tn FB} & F(A\tn \I)\tn FB
		\\
		&\begin{array}c\Rightarrow\\FA\tn F^l_B\end{array}
		& \dTo>{FA\tn F^\tn_{\I,B}}
		&& \Nearrow F^a_{A,\I,B}
		&& \dTo>{F^\tn_{A\tn \I, B}}
		\\
		&&
		FA\tn F(\I\tn B)
		& \rTo^{F^\tn_{A,\I,B}} & F(A\tn(\I\tn B))
		& \rTo^{F(a_{A,\I,B})} & F((A\tn \I)\tn B)
		\\
		&&\dTo<{FA\tn FB}
		& \Nearrow F^\tn_{A,l_B}
		& \dTo<{F(A\tn l_B)}>{\begin{array}c\To\\[-4pt]F(\mu_{A,B})\end{array}}
		& \ldTo_{F(r_A\tn B)}
		\\
		&&
		\rnode{2}{FA\tn FB}
		& \rTo_{F^\tn_{A,B}} & F(A\tn B)
		\nccurve[angle=180,ncurv=1]{->}12\Bput{FA\tn l_{FB}}
	\end{diagram}
	is equal to
	\begin{diagram}
		FA\tn(\I\tn FB)
		& \rTo^{a_{FA,\I,FB}} & (FA\tn \I)\tn FB
		& \rTo^{(FA\tn F^\tn_\I)\tn FB} & (FA\tn F\I)\tn FB
		\\
		& \raise1ex\hbox{$\begin{array}c\Rightarrow\\[-5pt]\mu_{FA,FB}\end{array}$}
		\rdTo(1,2)_{FA\tn l_{FB}}
		\ldTo(1,2)>{r_{FA}\tn FB}
		&& \begin{array}c\To\\ F^r_A\tn FB\end{array}
		& \dTo>{F^\tn_{A,\I}\tn FB}
		\\
		& FA\tn FB
		& \lTo_{F(r_A)\tn FB}
		&& F(A\tn \I)\tn FB
		\\
		& \dTo<{F^\tn_{A,B}}
		&& \Searrow F^\tn_{r_A,B}
		& \dTo>{F^\tn_{A\tn \I, B}}
		\\
		& F(A\tn B)
		&& \lTo_{F(r_A\tn B)}
		& F((A\tn \I)\tn B) 
	\end{diagram}
\end{definition}
\begin{definition} % monoidal pseudo-natural transformation
	A \emph{monoidal pseudo-natural transformation} $\gamma: F\To G: \B\to\BC$
	between monoidal pseudo-functors $F$ and $G$ is a pseudo-natural transformation
	equipped with an invertible 2-cell
	\[
		\gamma^\tn_\I: \gamma_\I\o F^\tn_\I \To G^\tn_\I
	\]
	and an invertible modification with components
	\begin{diagram}
		FA\tn FB & \rTo^{F^\tn_{A,B}} & F(A\tn B)
		\\
		\dTo<{\gamma_A \tn \gamma_B} & \Swarrow\gamma^\tn_{A,B}
			& \dTo>{\gamma_{A\tn B}}
		\\
		GA\tn GB & \rTo_{G^\tn_{A,B}} & G(A\tn B) 
	\end{diagram}
	such that for all $A$, $B$, $C\in\B$, the pasting
	\begin{diagram}
		&
		FA \tn(FB\tn FC)
		& \rTo^{FA\tn F^\tn_{B,C}} & FA\tn F(B\tn C)
		& \rTo^{F^\tn_{A,B\tn C}} & F(A\tn(B\tn C))
		\\
		&
		\dTo<{a_{FA,FB,FC}}
		&& \Swarrow F^a_{A,B,C}
		&& \dTo>{F(a_{A,B,C})}
		\\
		&
		\rnode{FAFBFC}{(FA\tn FB)\tn FC}
		& \rTo^{F^\tn_{A,B}\tn FC} & \rnode{FABFC}{F(A\tn B)\tn FC}
		& \rTo^{F^\tn_{A\tn B,C}} & F((A\tn B)\tn C)
		\\
		&
		\dTo<{(FA\tn FB)\tn\gamma_C}
		&\hbox to 0pt{\hss$\cong$\qquad}
		& \dTo<{F(A\tn B)\tn\gamma_C}>{\enskip\cong}
		\\
		&
		(FA\tn FB)\tn GC
		& \rTo_{F^\tn_{A,B}\tn GC} & F(A\tn B)\tn GC
		& \raise-1em\spright{\Swarrow \gamma^\tn_{A\tn B,C}}
		& \dTo>{\gamma_{A\tn B,C}}
		\\
		&
		\dTo<{(\gamma_A\tn\gamma_B)\tn GC}
		& \raise-.5em\spleft{\Swarrow\gamma^\tn_{A,B}\tn GC}
		& \dTo[snake=.5em]<{\gamma_{A\tn B}\tn GC}
		\\
		&
		\rnode{GAGBGC}{(GA\tn GB)\tn GC}
		& \rTo_{G^\tn_{A,B}\tn GC} & \rnode{GABGC}{G(A\tn B)\tn GC}
		& \rTo_{G^\tn_{A\tn B, C}} & G((A\tn B)\tn C)
		%
		\nccurve[angle=180,ncurv=1]{->}{FAFBFC}{GAGBGC}
			\Bput{\gamma_A\tn(\gamma_B\tn\gamma_C)}
		\nccurve[angleA=-45,angleB=45,ncurv=.9]{->}{FABFC}{GABGC}
			\aput(.4){\gamma_{A\tn B}\tn\gamma_C}
	\end{diagram}
	is equal to
	\begin{diagram}
		&&
		\rnode{1}{FA\tn(FB\tn FC)}
		& \rTo^{FA\tn F^\tn_{B,C}} & \rnode{4}{FA\tn F(B\tn C)}
		& \rTo^{F^\tn_{A,B\tn C}} & \spright{F(A\tn(B\tn C))}
		\\
		&&
		\dTo[snake=.5em]>{FA\tn(\gamma_B\tn\gamma_C)}
		& \raise-1em\spboth{\Swarrow FA\tn\gamma^\tn_{B,C}}
		& \dTo<{FA\tn\gamma_{B\tn C}}>{\enskip\cong}
		&&& \rdTo(2,3)^{F(a_{A,B,C})}
		\\
		&&
		FA\tn(GB\tn GC)
		& \rTo_{FA\tn G^\tn_{B,C}} & FA\tn G(B\tn C)
		&& \dTo[snake=-1em]<{\gamma_{A\tn(B\tn C)}}
		\\
		&&
		\dTo[snake=-1em]>{\gamma_A\tn(GB\tn GC)}
		&\cong& \dTo[snake=1em]<{\gamma_A\tn G(B\tn C)}
		& \spboth{\Swarrow \gamma^\tn_{A,B\tn C}}
		&& \begin{array}c\Leftarrow\\[-4pt]\gamma_{\alpha_{A,B,C}}\end{array}
		& F((A\tn B)\tn C)
		\\
		&&
		\rnode{2}{GA\tn(GB\tn GC)}
		& \rTo_{GA\tn G^\tn_{B,C}} & \rnode{5}{GA\tn G(B\tn C)}
		& \rTo_{G^\tn_{A,B\tn C}} & \spright{G(A\tn(B\tn C))}
		& \ldTo(2,3)_{\gamma_{(A\tn B)\tn C}}
		\\
		& \Swarrow a_{\gamma_A,\gamma_B,\gamma_C}
		& \dTo>{a_{GA,GB,GC}}
		&& \Swarrow G^a_{A,B,C}
		&& \dTo<{G(a_{A,B,C})}
		\\
		\rnode{3}{(FA\tn FB)\tn FC} & \rTo_{(\gamma_A\tn\gamma_B)\tn\gamma_C}
		& (GA\tn GB)\tn GC
		& \rTo_{G^\tn_{A,B}\tn GC} & G(A\tn B)\tn GC
		& \rTo_{G^\tn_{A\tn B,C}} & \spright{G((A\tn B)\tn C)}
		%
		\nccurve[angleA=-170,angleB=170]{->}12
			\bput{90}{\gamma_A\tn(\gamma_B\tn\gamma_C)}
			\Aput{\quad\cong}
		\nccurve[angleA=180,angleB=90]{->}13
			\aput(.7){a_{FA,FB,FC}}
		\nccurve[angleA=-45,angleB=45,ncurv=.9]{->}45
			\aput(.4){\gamma_A\tn\gamma_{B\tn C}}
	\end{diagram}
	and for all $A\in\B$, the pasting
	\begin{diagram}
		&& \rnode{1}{FA\tn \I}
		& \rTo^{FA\tn F^\tn_\I} & \rnode{FAFI}{FA\tn F\I}
		\\
		&& \dTo<{r_{FA}} & \begin{array}c\To\\F^r_A\end{array}
		& \dTo>{F^\tn_{A,\I}}
		\\
		\rnode{2}{GA\tn \I}
		& \begin{array}c\To\\[-4pt]r_{\gamma_A}\end{array}
		& FA & \lTo_{F(r_A)} & F(A\tn \I)
		& \begin{array}c\To\\[-4pt]\gamma^\tn_{A,\I}\end{array}
		& \rnode{GAGI}{GA\tn G\I}
		\\
		&& \dTo<{\gamma_A}
		& \Nearrow\gamma_{r_A} & \dTo>{\gamma_{A\tn \I}}
		\\
		&& \rnode{3}{GA} & \lTo_{G(r_A)} & \rnode{GAI}{G(A\tn \I)}
		%
		\nccurve[angleA=180,angleB=90]{->}12
			\Bput{\gamma_A\tn \I}
		\nccurve[angleA=-90,angleB=180]{->}23
			\Bput{r_{GA}}
		\nccurve[angleA=0,angleB=90]{->}{FAFI}{GAGI}
			\Aput{\gamma_A\tn\gamma_\I}
		\nccurve[angleA=-90,angleB=0]{->}{GAGI}{GAI}
			\Aput{G^\tn_{A,\I}}
	\end{diagram}
	is equal to
	\begin{diagram}
		FA\tn \I & \rTo^{FA\tn F^\tn_\I} & FA\tn F\I
		\\
		\dTo<{\gamma_A\tn \I}
		& \cong & \dTo<{\gamma_A\tn F\I}>{\quad\cong}
		& \rdTo^{\gamma_A\tn\gamma_\I}
		\\
		\rnode{GAI}{GA\tn \I}
		& \rTo_{GA\tn F^\tn_\I} & GA\tn F\I
		& \rTo_{GA\tn\gamma_\I} & \rnode{GAGI}{GA\tn G\I}
		\\
		&& \raise1.5em\spboth{\Uparrow GA\tn\gamma^\tn_\I}
		\\
		\dTo[snake=1.5em]<{r_{GA}}
		&& \begin{array}c\To\\[-4pt]G^r_A\end{array}
		&& \dTo[snake=1.5em]>{G^\tn_{A,\I}}
		\\
		GA && \lTo_{G(r_A)} && G(A\tn \I)
		\ncarc[arcangle=-35]{->}{GAI}{GAGI}\Bput{GA\tn G^\tn_\I}
	\end{diagram}
	and
	\begin{diagram}
		&& \rnode{1}{\I\tn FA}
		& \rTo^{F^\tn_\I\tn FA} & \rnode{FIFA}{F\I\tn FA}
		\\
		&& \dTo<{l_{FA}} & \begin{array}c\To\\F^l_A\end{array}
		& \dTo>{F^\tn_{\I,A}}
		\\
		\rnode{2}{\I\tn GA}
		& \begin{array}c\To\\[-4pt]l_{\gamma_A}\end{array}
		& FA & \lTo_{F(l_A)} & F(\I\tn A)
		& \begin{array}c\To\\[-4pt]\gamma^\tn_{\I,A}\end{array}
		& \rnode{GIGA}{G\I\tn GA}
		\\
		&& \dTo<{\gamma_A}
		& \Nearrow\gamma_{l_A} & \dTo>{\gamma_{\I\tn A}}
		\\
		&& \rnode{3}{GA} & \lTo_{G(l_A)} & \rnode{GIA}{G(\I\tn A)}
		%
		\nccurve[angleA=180,angleB=90]{->}12
			\Bput{\I\tn \gamma_A}
		\nccurve[angleA=-90,angleB=180]{->}23
			\Bput{l_{GA}}
		\nccurve[angleA=0,angleB=90]{->}{FIFA}{GIGA}
			\Aput{\gamma_\I\tn\gamma_A}
		\nccurve[angleA=-90,angleB=0]{->}{GIGA}{GIA}
			\Aput{G^\tn_{\I,A}}
	\end{diagram}
	is equal to
	\begin{diagram}
		\I\tn FA & \rTo^{F^\tn_\I\tn FA} & F\I\tn FA
		\\
		\dTo<{\I\tn\gamma_A}
		& \cong & \dTo<{F\I\tn\gamma_A}>{\quad\cong}
		& \rdTo^{\gamma_I\tn\gamma_A}
		\\
		\rnode{IGA}{\I\tn GA}
		& \rTo_{F^\tn_\I\tn GA} & F\I\tn GA
		& \rTo_{\gamma_\I\tn GA} & \rnode{GIGA}{G\I\tn GA}
		\\
		&& \raise1.5em\spboth{\Uparrow\gamma^\tn_\I\tn GA}
		\\
		\dTo[snake=1.5em]<{l_{GA}}
		&& \begin{array}c\To\\[-4pt]G^l_A\end{array}
		&& \dTo[snake=1.5em]>{G^\tn_{\I,A}}
		\\
		GA && \lTo_{G(l_A)} && G(\I\tn A)
		\ncarc[arcangle=-35]{->}{IGA}{GIGA}\Bput{G^\tn_\I\tn GA}
	\end{diagram}
\end{definition}
\begin{definition} % monoidal modification
	A \emph{monoidal modification} $m:\gamma\Tto\delta: F\To G$, between monoidal
	pseudo-natural transformations $F$ and $G$, is a modification with the property that
	\[
		\begin{diagram}[h=2em]
			&& \rnode{FI}{F\I}
			\\
			& \ruTo^{F^\tn_\I}
			\\
			\I & \Swarrow\delta^\tn_\I & \dTo<{\delta_\I}
			\\
			& \rdTo_{G^\tn_\I}
			\\
			&& \rnode{GI}{G\I}
			\nccurve[angle=0]{->}{FI}{GI}\Aput{\gamma_\I}
			\Bput{\begin{array}c\Leftarrow\\[-4pt]m_\I\end{array}}
		\end{diagram}
		\hskip 5em=\quad
		\gamma^\tn_\I
	\]
	and
	\[
	\begin{diagram}
		& \rnode{FAFB}{FA\tn FB} & \rTo^{F^\tn_{A,B}} & F(A\tn B)
		\\
		\spright{\mkern-12mu\begin{array}c\Leftarrow\\[-4pt]m_A\tn m_B\end{array}}
		& \dTo[snake=-1em]>{\gamma_A\tn\gamma_B}
		& \raise1em\spboth{\Swarrow\gamma^\tn_{A,B}}
		& \dTo>{\gamma_{A\tn B}}
		\\
		& \rnode{GAGB}{GA\tn GB}
		& \rTo_{G^\tn_{A,B}}
		& G(A\tn B)
		%
		\nccurve[angle=180,ncurv=1]{->}{FAFB}{GAGB}
			\Bput{\delta_A\tn\delta_B}
	\end{diagram}
	\quad=\quad
	\begin{diagram}
		FA\tn FB & \rTo^{F^\tn_{A,B}} & \rnode{FAB}{F(A\tn B)}
		\\
		\dTo<{\delta_A\tn\delta_B}
		& \raise-1em\spboth{\Swarrow\delta^\tn_{A,B}}
		& \dTo[snake=1em]<{\delta_{A\tn B}}
		& \spleft{\begin{array}c\Leftarrow\\[-4pt]m_{A\tn B}\end{array}}
		\\
		\rnode{GAGB}{GA\tn GB}
		& \rTo_{G^\tn_{A,B}}
		& \rnode{GAB}{G(A\tn B)}
		%
		\nccurve[angle=0,ncurv=1]{->}{FAB}{GAB}
			\Aput{\gamma_{A\tn B}}
	\end{diagram}
	\]
\end{definition}

\section{On coherence}\label{s-coherence}
The coherence theorem of \citet{GPS} implies that every monoidal bicategory
is monoidally biequivalent to a Gray monoid:
\begin{definition}
A \emph{Gray monoid} is a monoidal
bicategory $\B$ in which:
\begin{itemize}
\item the underlying bicategory $\B$ is a 2-category,
\item given composable pairs $f$, $g$ and $h$, $k$ of 1-cells,
	if either $f$ or $k$ is an identity then the structural 2-cell
	\[
		(f\tn h)\o(g\tn k) \To (f\o g)\tn(h\o k)
	\]
	is an identity,
\item the structural equivalences $a$, $l$ and $r$ are identities.
\end{itemize}
\end{definition}
%
The second condition here is the most mysterious. It means that
for every object $A\in\B$, the pseudofunctors $A\tn-$ and $-\tn A$
are 2-functors, and that furthermore the tensor product $f\tn g$ of
1-cells $f:A\to C$ and $g:B\to D$ is equal to the composite
\[
	A\tn B \rTo^{f\tn B} C\tn B \rTo^{C\tn g} C\tn D.
\]
When working in a Gray monoid, it will often be convenient to decompose
tensor products of 1-cells in this way. Then the only structural
2-cells are of the form
\begin{diagram}
	A\tn B & \rTo^{f\tn B} & C\tn B \\
	\dTo<{A\tn g} & \sim & \dTo>{C\tn g} \\
	A\tn D & \rTo_{f\tn D} & C\tn D,
\end{diagram}
or composites thereof. We shall label them merely with the symbol
$\sim$ (as above), since there is no possible ambiguity.

Furthermore, \citet{GurskiThesis} shows that every \emph{free} monoidal bicategory
is monoidally biequivalent to a free Gray monoid on the same generators.
In particular, Gurski shows that the \emph{canonical} monoidal pseudofunctor
from the free monoidal bicategory to the free Gray monoid is a monoidal
biequivalence.
%
These theorems mean that it is often possible to prove a theorem
about monoidal bicategories in two stages:
\begin{enumerate}
	\item Prove the theorem for Gray monoids,
	\item Use a coherence theorem to deduce that the result
		holds for all monoidal bicategories.
\end{enumerate}
%
Gurski's theorem is especially
useful here, because it gives a general argument for the second stage
that applies in an important class of cases. In particular it applies
to problems of the following form: some data (objects, 1-cells,
2-cells) are given, subject to axioms that take the form of
equations between 2-cells. New structures are constructed in terms
of the given data, and it's desired to show that some equations
must hold between constructed 2-cells. (The chapter on pseudomonoids
contains several theorems that follow this pattern, since a pseudomonoid
is defined in terms of data and axioms as outlined above.) This kind of
question is 
essentially a question about the free monoidal bicategory generated
by the data and axioms, since an equation holds in every such monoidal
bicategory if and only if it holds in the free one. Now, if the equations
can be proved for Gray monoids, then in particular they hold in the
appropriate free Gray monoid. Gurski's theorem then implies that they
hold in the free monoidal bicategory, hence in every monoidal bicategory.

Sadly there are other cases in which it's desirable to work in a Gray
monoid to keep the diagrams manageable, to which this argument does
not apply. The external version of Cayley's theorem for pseudomonoids
(chapter~\chCayley) is such a case; there we'll have to work a little harder
to establish the second stage of the argument.

Gurski's theorem also justifies the labour-saving device of \emph{stating}
equations (between 2-cells) in the Gray monoid setting: since the monoidal
pseudofunctor from the free monoidal bicategory to the free Gray monoid is
a monoidal biequivalence, it has a biequi\-valence-inverse that has the effect
of inserting structural equivalences and 2-cells where necessary. This
inverse is not uniquely determined -- it's unique only up to pseudonatural
equivalence -- but of course any such choice will preserve equality
of 2-cells. So, if an equation is stated and proved in the Gray monoid
setting then:
\begin{enumerate}
	\item it's always possible to insert structural equivalences and
		2-cells to reinterpret the equation in the general setting,
	\item however this is done, the reinterpreted equation holds.
\end{enumerate}

So much for theorems. A practice adopted by many authors
\citep[e.g.][]{MonBicat,HDA1} is to state \emph{definitions} for
Gray monoids rather than general monoidal bicategories. This has
the undeniable advantage that it makes the definitions simpler
to state and use. The disadvantage is that there isn't generally a \emph{unique}
way to reinterpret such a definition for general monoidal bicategories.
To be sure, all such reinterpretations will be interconvertible, but
there is nevertheless an unfortunate ambiguity. We shall avoid this
problem by adopting a hybrid strategy: we'll describe the \emph{data}
(pseudonatural transformations, modifications) in the general setting,
but state our \emph{axioms} for Gray monoids. As explained above, there is
never any ambiguity in stating an equation between 2-cells in the Gray monoid
setting.

\section{Braided monoidal bicategories}\label{s-braiding}
The history of attempts to define the concept of braided monoidal bicategory highlights
the difficulties inherent in even finding the correct definition. The
first definition was given by \citet{KV}, with some errors and omissions.
The errors and some omissions were pointed out by \citet{CarmodyThesis}
and \citet{HDA1} -- presumably independently, since neither cites the
other -- though \citeauthor{HDA1} go further than \citeauthor{CarmodyThesis}
and give their own definition of braiding for a Gray monoid. This
definition includes an additional axiom (our third axiom below), which they
attribute to \citet{Breen-ator}.
%
\Citet{HDA1} do not give any axioms relating the unit object to the
braiding. This is mentioned in their Section~5(1) as an issue that
remains to be resolved. Subsequently \citet{GeneralizedCenters}
noticed that this omission causes an error in \citeauthor{HDA1}'s
`center' construction, and showed how it could be fixed by adding
six axioms for the unit. \Citet{MonBicat} also gave a definition of
braiding with just one axiom for the unit, and despite the superficial
differences this axiomatisation is equivalent to
\citeauthor{GeneralizedCenters}'s. (It turns out that four of Crans's
six unit axioms are redundant.)

These definitions both take the unit to be strict -- in terms of
our definition below, they take the 1-cells $s_{A,\I}$ and $s_{\I,A}$
and the 2-cells $U_{A|\I}$ and $U_{\I|A}$ to be \emph{identities}.
The justification for such a restriction is presumably the reasonable
expectation that a coherence theorem for tetracategories would
show every braided monoidal bicategory to be suitably equivalent
to one with such a strict unit. Since an expectation, however
reasonable, is not a proof, we have opted to eschew the modest
simplification that such a restriction brings.
%
Thus the definition below is conceivably the first that
includes all the structure known to be necessary at its natural level
of strictness, though we do not claim that any new insight was needed
to formulate it. (It is not unimaginable that further axioms should yet
be found wanting, though this does not seem especially likely.)
%
\begin{definition}\label{def-braiding}
	A \defn{braiding} for a monoidal bicategory $\B$
	consists of a pseudo-natural equivalence $s$ with 1-cell components
	\[
		s_{A,B}: A\tn B \to B\tn A,
	\]
	invertible modifications $S_{-|-,-}$ and $S_{-,-|-}$ with components
	\begin{diagram}
		 A\tn (B\tn C) &\rTo^{a} & (A\tn B)\tn C
		   & \rTo^{s_{A\tn B,C}} & C\tn(A\tn B) \\
		 \dTo<{A\tn s_{B,C}} && \Arr\Swarrow{S_{A,B|C}} && \dTo>{a}\\
		 A\tn(C\tn B) & \rTo_{a} & (A\tn C)\tn B
		   & \rTo_{s_{A,C}\tn B} & (C\tn A)\tn B,
	\end{diagram}
	%
	\begin{diagram}
		 (A\tn B)\tn C &\rTo^{a'} & A\tn(B\tn C)
		   & \rTo^{s_{A,B\tn C}} & (B\tn C)\tn A \\
		 \dTo<{s_{A,B}\tn C} && \Arr\Nearrow{S_{A|B,C}} && \dTo>{a'}\\
		 (B\tn A)\tn C & \rTo_{a'} & B\tn(A\tn C)
		   & \rTo_{B\tn s_{A,C}} & B\tn(C\tn A),
	\end{diagram}
	and invertible modifications $U_{\I|-}$ and $U_{-|\I}$ with components
	\[
	\begin{diagram}[h=2em]
		\I\tn A & \rTo^{s_{\I,A}} & A\tn \I \\
		&\raise 1em\hbox{$\mathop\Leftarrow\limits_{U_{\I|A}}$} \rdTo(1,2)_{l_A} \ldTo(1,2)_{r_A}\\
		&A
	\end{diagram}
	\hskip 3em
	\begin{diagram}[h=2em]
		A\tn \I & \rTo^{s_{A,\I}} & \I\tn A \\
		&\raise 1em\hbox{$\mathop\Rightarrow\limits_{U_{A|\I}}$} \rdTo(1,2)_{r_A} \ldTo(1,2)_{l_A}\\
		&A,
	\end{diagram}
	\]
	subject to various axioms. As explained in section~\ref{s-coherence},
	we state the axioms in the Gray monoid setting. There are four axioms
	that relate the $S$ modifications to each other:
	\[
	\hskip-1em
	\begin{diagram}[h=4em,labelstyle=\scriptstyle]
		A\tn B\tn C\tn D & \rTo^{s_{A\tn B\tn C,D}} & D\tn A\tn B\tn C \\
		\dTo<{A\tn B\tn s_{C,D}} & \rdTo(2,2)[hug]_{A\tn s_{B\tn C,D}}
			\raise 1.5em\hbox to 0pt{$\Downarrow\scriptstyle S_{A,B\tn C|D}$\hss}
			\raise-2em\hbox to 0pt{\hss$\Swarrow\scriptstyle A\tn S_{B,C|D}$}
			& \uTo>{s_{A,D}\tn B\tn C} \\
		A\tn B\tn D\tn C &\rTo_{A\tn s_{B,D}\tn C} & A\tn D\tn B\tn C
	\end{diagram}
	=
	\begin{diagram}[h=4em,labelstyle=\scriptstyle]
		A\tn B\tn C\tn D & \rTo^{s_{A\tn B\tn C,D}} & D\tn A\tn B\tn C \\
		\dTo<{A\tn B\tn s_{C,D}} & \ruTo(2,2)[hug]_{s_{A\tn B,D}\tn C}
			\raise 1.5em\hbox to 0pt{\hss$\Downarrow\scriptstyle S_{A\tn B,C|D}$}
			\raise-2em\hbox to 0pt{$\Searrow\scriptstyle S_{A,B|D}\tn C$\hss}
			& \uTo>{s_{A,D}\tn B\tn C} \\
		A\tn B\tn D\tn C &\rTo_{A\tn s_{B,D}\tn C} & A\tn D\tn B\tn C
	\end{diagram}
	\]
	\[
	\hskip-1em
	\begin{diagram}[h=4em,labelstyle=\scriptstyle]
		A\tn B\tn C\tn D & \rTo^{s_{A\,B\tn C\tn D}} & B\tn C\tn D\tn A \\
		\dTo<{s_{A,B}\tn C\tn D} & \rdTo(2,2)[hug]_{s_{A,B\tn C}\tn D}
			\raise 1.5em\hbox to 0pt{$\Uparrow\scriptstyle S_{A|B\tn C,D}$\hss}
			\raise-2em\hbox to 0pt{\hss$\Nearrow\scriptstyle S_{A|B,C}\tn D$}
			& \uTo>{B\tn C\tn s_{A,D}} \\
		B\tn A\tn C\tn D &\rTo_{B\tn s_{A,C}\tn D} & B\tn C\tn A\tn D
	\end{diagram}
	=
	\begin{diagram}[h=4em,labelstyle=\scriptstyle]
		A\tn B\tn C\tn D & \rTo^{s_{A\,B\tn C\tn D}} & B\tn C\tn D\tn A \\
		\dTo<{s_{A,B}\tn C\tn D} & \ruTo(2,2)[hug]_{B\tn s_{A,C\tn D}}
			\raise 1.5em\hbox to 0pt{\hss$\Uparrow\scriptstyle S_{A|B,C\tn D}$}
			\raise-2em\hbox to 0pt{$\Nwarrow\scriptstyle B\tn S_{A|C,D}$\hss}
			& \uTo>{B\tn C\tn s_{A,D}} \\
		B\tn A\tn C\tn D &\rTo_{B\tn s_{A,C}\tn D} & B\tn C\tn A\tn D
	\end{diagram}
	\]
	\[\hskip-1em\begin{array}{l}
	\hskip-2em
	\begin{diagram}[hug,labelstyle=\scriptstyle,w=4em,tight]
		A\tn B\tn C\tn D &&& \rTo^{\displaystyle A\tn s_{B,C\tn D}}
			&&& A\tn C\tn D\tn B \\
		&\rdTo^{A\tn s_{B,C}\tn D} && \Uparrow{\scriptstyle A\tn S_{B|C,D}}
			&& \ruTo(4,2)^{A\tn C\tn s_{B,D}} \ldTo(2,2)_{s_{A,C}\tn B\tn D} \\
		\dTo<{\displaystyle s_{A\tn B,C}\tn D}
			& \hbox to 0pt{\hss$\mathop\Rightarrow\limits_{S_{A,B|C}\tn D}$}
			& A\tn C\tn B\tn D & \sim & C\tn A\tn D\tn B
			& \hbox to 0pt{$\mathop\Rightarrow\limits_{S_{A|C,D}\tn B}$\hss}
			& \dTo>{\displaystyle s_{A,C\tn D}\tn B} \\
		&\ldTo_{s_{A,C}\tn B\tn D} && \ruTo(4,2)_{C\tn A\tn s_{B,D}}
			& \Uparrow{\scriptstyle C \tn S_{A,B|D}}
			& \rdTo^{C \tn s_{A,D}\tn B} \\
		C\tn A\tn B\tn D &&& \rTo_{C \tn s_{A\tn B,D}} &&& C\tn D\tn A\tn B
	\end{diagram}
	\\[8em]
	\multicolumn 1r{
	= \qquad \begin{diagram}[h=4em]
		A\tn B\tn C\tn D & \rTo^{A\tn s_{B,C\tn D}} & A\tn C\tn D\tn B \\
		\dTo<{s_{A\tn B,C}\tn D}
			& \rdTo(2,2)[hug]_{s_{A\tn B,C\tn D}}
			\raise 1.5em\hbox to 0pt{$\Nearrow\scriptstyle S_{A,B|C\tn D}$\hss}
			\raise-2em\hbox to 0pt{\hss$\Nearrow\scriptstyle S_{A\tn B|C,D}$}
			& \dTo>{s_{A,C\tn D}\tn B} \\
			C\tn A\tn B\tn D & \rTo_{C \tn s_{A\tn B,D}} & C\tn D\tn A\tn B
	\end{diagram}
	}
	\end{array}\]
	\[
		\begin{diagram}[hug,s=2.5em,tight]
			A\tn B\tn C && \rTo^{s_{A\tn B,C}} && C\tn A\tn B \\
			& \rdTo_{A\tn s_{B,C}}
				&\raise 1em\hbox{$\Arr\Downarrow{\scriptstyle S_{A,B|C}}$}
				& \ruTo_{s_{A,C}\tn B} \\
			\dTo<{s_{A,B\tn C}} && A\tn C\tn B & \Arr\Swarrow{\scriptstyle S_{A|C,B}}
				& \dTo>{C\tn s_{A,B}} \\
			&\Arr\Swarrow{s_{A,s_{B,C}}} && \rdTo_{s_{A,C\tn B}} \\
			B\tn C\tn A&&\rTo_{s_{B,C}\tn A} && C\tn B\tn A
		\end{diagram}
		\quad=\quad
		\begin{diagram}[hug,s=2.5em,tight]
			A\tn B\tn C && \rTo^{s_{A\tn B,C}} && C\tn A\tn B \\
			& \rdTo_{s_{A,B}\tn C} &&\Swarrow{s_{s_{A,B},C}^{-1}} \\
			\dTo<{s_{A,B\tn C}}
				&\hbox to 0pt{\hss$\mathop\Leftarrow\limits_{S_{A|B,C}}$}
				& B\tn A\tn C && \dTo>{C\tn s_{A,B}} \\
			&\ldTo^{B\tn s_{A,C}}
				&\mathop\Leftarrow\limits_{S_{B,A|C}}
				& \rdTo_{s_{B\tn A,C}} \\
			B\tn C\tn A&&\rTo_{s_{B,C}\tn A} && C\tn B\tn A
		\end{diagram}
	\]
	and two axioms that relate the $U$ and $S$ modifications:
	the 2-cells pictured below should each be equal to the identity on $s_{A,B}$:
	\[\begin{array}{c@{\hskip4em}c}
		\begin{diagram}[hug,s=2.5em]
			A\tn \I\tn B && \rTo^{s_{A\tn \I,B}} && B\tn A\tn \I \\
			&\rdTo_{A\tn s_{\I,B}} & \Downarrow{\scriptstyle S_{A,\I|B}}
				& \ruTo_{s_{A,B}\tn \I} \\
			\dTo<1 & \hbox to 0pt{\hss$\mathop\Leftarrow\limits_{A\tn U_{\I|B}}$}
				& A\tn B\tn \I && \dTo>1 \\
			&\ldTo_{1} \\
			A\tn B && \rTo_{s_{A,B}} && B\tn A
		\end{diagram}
		&
		\begin{diagram}[hug,s=2.5em]
			A\tn \I\tn B && \rTo^{s_{A,\I\tn B}} && \I\tn B\tn A \\
			&\rdTo_{s_{A,\I}\tn B} & \Uparrow{\scriptstyle S_{A|\I,B}}
				& \ruTo_{\I\tn s_{A,B}} \\
			\dTo<1 & \hbox to 0pt{\hss$\mathop\Rightarrow\limits_{U_{A|\I}\tn B}$}
				& \I\tn A\tn B && \dTo>1 \\
			&\ldTo_{1} \\
			A\tn B && \rTo_{s_{A,B}} && B\tn A
		\end{diagram}
	\end{array}\]
\end{definition}

\subsection{The duality of the definition}
Since $s$ is a pseudo-natural equivalence, it has an adjoint
equivalence-inverse $s'$. We shall assume that such an $s'$
has been chosen: by definition it has 1-cell components
\[
	s'_{A,B}: B\tn A \to A\tn B,
\]
and by Prop.~\chref{Bicats}{prop-adj-1} each 2-cell component $s'_{f,g}$
is the inverse of the mate of $s_{f,g}$. Now we may define
a pseudo-natural equivalence $s^*$ with 1-cell components
$s^*_{A,B} = s'_{A,B}$ and 2-cell components $s^{*}_{f,g} = s'_{g,f}$.
The duality is expressed by the following:
\begin{propn}
	The pseudo-natural transformation $s^{*}$ can be made into
	a braiding in the following way.
	\begin{itemize}
		\item $S^{*}_{A|B,C}$ is defined to be the mate of $S_{B,C|A}$
			with respect to the adjoint equivalences
			$(s_{A,C}\tn B)\o a\o(A\tn s_{B,C})$
			and $a\o(s_{A\tn B,C})\o a$,
		\item $S^{*}_{A,B|C}$ is defined to be the mate of $S_{C|A,B}$
			with respect to the adjoint equivalences
			$a'\o(s_{A,B\tn C})\o a'$ and $(B\tn s_{A,C})\o a'\o (s_{A,B}\tn C)$;
			%
			Note that, by Prop.~\chref{Bicats}{prop-adjeq-mate-dual}, the
			left and right mates are equal in this case and the previous one.
		\item $U^{*}_{\I|A}$ is defined to be the right mate of $U_{A|\I}$
			with respect to the adjoint equivalences $s_{A,\I}$ and $1_{A}$,
		\item $U^{*}_{A|\I}$ is defined to be the left mate of $U_{\I|A}$
			with respect to the adjoint equivalences $s_{\I,A}$ and $1_{A}$.
	\end{itemize}
\end{propn}
\begin{proof}
	The first two axioms for $S$ are duals of each other: taking mates in
	one of them yields the other for $S^{*}$. The third axiom is self-dual,
	in that taking mates gives the corresponding equation for $S^{*}$.
	The fourth axiom is also self-dual, using the fact that the mate of
	$s_{A,s_{B,C}}$ is the inverse of ${s^*_{s^*_{C,B},A}}$. Finally, the
	two unit axioms are duals of each other.
\end{proof}

This symmetry can sometimes spare us a certain amount of repetition,
in Section~\chref{Psmon}{s-braided-facts}, for example.

\subsection{The unit axioms}
The first thing to say about the unit axioms is that we have here
another instance of the phenomenon discussed in Remark~\ref{rem-defining-L}:
the unit axioms show that the 2-cells $U_{\I|A}$ and $U_{A,\I}$ are
definable in terms of the other data. However, the $U$ cells defined
in this way do not themselves necessarily satisfy the unit axioms!
So these cells are redundant data in a sense, but if we were to
eliminate them, we should instead have to impose rather unnatural-looking
axioms involving $S_{A,\I|B}$ and $S_{A|\I,B}$.

One could write down several other natural conditions on the unit
cells. These turn out to be derivable from the two conditions we
have. In particular:
\begin{propn}\label{prop-braiding-unit}
	In any braided Gray monoid, the 2-cells
	\[
		\begin{diagram}[hug,s=4em]
			\I\tn A\tn B & \rTo^{\I\tn s_{A,B}} & \I\tn B\tn A \\
			\dTo<{s_{\I\tn A,B}} & \ldTo_{s_{\I,B}\tn A}
				\raise 2em\hbox to 0pt{\hss$\mathop\Rightarrow\limits_{S_{\I,A|B}}$}
				\raise-2em\hbox to 0pt{$\mathop\Rightarrow\limits_{U_{\I|B}\tn A}$\hss}
				& \dTo>{1}\\
			B\tn \I\tn A & \rTo_{1} & B\tn A
		\end{diagram}
		\qquad\mbox{and}\qquad
		\begin{diagram}[hug,s=4em]
			A\tn B\tn \I & \rTo^{s_{A,B}\tn \I} & B\tn A\tn \I \\
			\dTo<{s_{A,B\tn \I}} & \ldTo_{B\tn s_{A,\I}}
				\raise 2em\hbox to 0pt{\hss$\mathop\Leftarrow\limits_{S_{A|B,\I}}$}
				\raise-2em\hbox to 0pt{$\mathop\Leftarrow\limits_{B\tn U_{A|\I}}$\hss}
				& \dTo>{1}\\
			B\tn \I\tn A & \rTo_{1} & B\tn A
		\end{diagram}
	\]
	are both identities.
\end{propn}
\begin{proof}
	By duality, it suffices to prove one of the two: we'll prove
	the second one.
	If we set $B=\I$ in our first axiom for $S$, we obtain
	\[
	\hskip-1em
	\begin{diagram}[h=4em,labelstyle=\scriptstyle]
		A\tn \I\tn C\tn D & \rTo^{s_{A\tn \I\tn C,D}} & D\tn A\tn \I\tn C \\
		\dTo<{A\tn \I\tn s_{C,D}} & \rdTo(2,2)[hug]_{A\tn s_{\I\tn C,D}}
			\raise 1.5em\hbox to 0pt{$\Downarrow\scriptstyle S_{A,\I\tn C|D}$\hss}
			\raise-2em\hbox to 0pt{\hss$\Swarrow\scriptstyle A\tn S_{\I,C|D}$}
			& \uTo>{s_{A,D}\tn \I\tn C} \\
		A\tn \I\tn D\tn C &\rTo_{A\tn s_{\I,D}\tn C} & A\tn D\tn \I\tn C
	\end{diagram}
	=
	\begin{diagram}[h=4em,labelstyle=\scriptstyle]
		A\tn \I\tn C\tn D & \rTo^{s_{A\tn \I\tn C,D}} & D\tn A\tn \I\tn C \\
		\dTo<{A\tn \I\tn s_{C,D}} & \ruTo(2,2)[hug]_{s_{A\tn \I,D}\tn C}
			\raise 1.5em\hbox to 0pt{\hss$\Downarrow\scriptstyle S_{A\tn \I,C|D}$}
			\raise-2em\hbox to 0pt{$\Searrow\scriptstyle S_{A,\I|D}\tn C$\hss}
			& \uTo>{s_{A,D}\tn \I\tn C} \\
		A\tn \I\tn D\tn C &\rTo_{A\tn s_{\I,D}\tn C} & A\tn D\tn \I\tn C
	\end{diagram}
	\]
	Cancelling the invertible 2-cell $S_{A,C|D}$, we have
	\[
	\hskip-1em
	\begin{diagram}[w=5em,labelstyle=\scriptstyle,tight]]
		A\tn \I\tn C\tn D & & D\tn A\tn \I\tn C \\
		\dTo<{A\tn \I\tn s_{C,D}} & \rdTo(2,2)[hug]^{A\tn s_{\I\tn C,D}}
			\raise-2em\hbox to 0pt{\hss$\Swarrow\scriptstyle A\tn S_{\I,C|D}$}
			& \uTo>{s_{A,D}\tn \I\tn C} \\
		A\tn \I\tn D\tn C &\rTo_{A\tn s_{\I,D}\tn C} & A\tn D\tn \I\tn C
	\end{diagram}
	=
	\begin{diagram}[w=5em,labelstyle=\scriptstyle,tight]]
		A\tn \I\tn C\tn D && D\tn A\tn \I\tn C \\
		\dTo<{A\tn \I\tn s_{C,D}} & \ruTo(2,2)[hug]^{s_{A\tn \I,D}\tn C}
			\raise-2em\hbox to 0pt{$\Searrow\scriptstyle S_{A,\I|D}\tn C$\hss}
			& \uTo>{s_{A,D}\tn \I\tn C} \\
		A\tn \I\tn D\tn C &\rTo_{A\tn s_{\I,D}\tn C} & A\tn D\tn \I\tn C
	\end{diagram}
	\]
	Thus
	\begin{diagram}[w=5em,labelstyle=\scriptstyle,tight]
		A\tn \I\tn C\tn D & & D\tn A\tn \I\tn C \\
		\dTo<{A\tn \I\tn s_{C,D}} & \rdTo(2,2)[hug]^{A\tn s_{\I\tn C,D}}
			\raise-1em\hbox to 0pt{\hss$\Swarrow\scriptstyle A\tn S_{\I,C|D}$}
			& \uTo[snake=1em]<{s_{A,D}\tn \I\tn C}
			%%
			&\rdTo^{1}
			%%
			\\
		A\tn \I\tn D\tn C &\rTo_{A\tn s_{\I,D}\tn C} & A\tn D\tn \I\tn C
		%%
		& & D\tn A\tn C \\
		& \rdTo_{1} \raise1em\rlap{$\Swarrow\scriptstyle  A\tn U_{\I|D}\tn C$}
			& \dTo>1 & \ruTo_{s_{A,D}\tn C} \\
		& & A\tn D\tn C
	\end{diagram}
	is equal to
	\begin{diagram}[w=5em,labelstyle=\scriptstyle,tight]
		A\tn \I\tn C\tn D && D\tn A\tn \I\tn C \\
		\dTo<{A\tn \I\tn s_{C,D}} & \ruTo(2,2)[hug]^{s_{A\tn \I,D}\tn C}
			\raise-1.5em\hbox to 0pt{$\Searrow\scriptstyle S_{A,\I|D}\tn C$\hss}
			& \uTo<{s_{A,D}\tn \I\tn C}
			%%
			&\rdTo^{1}
			%%
			\\
		A\tn \I\tn D\tn C &\rTo_{A\tn s_{\I,D}\tn C} & A\tn D\tn \I\tn C
		%%
		& & D\tn A\tn C \\
		& \rdTo_{1} \raise1em\rlap{$\Swarrow\scriptstyle  A\tn U_{\I|D}\tn C$}
			& \dTo>1 & \ruTo_{s_{A,D}\tn C} \\
		& & A\tn D\tn C
	\end{diagram}
	which, by our first unit axiom, is the identity.
	Cancelling the equivalence $s_{A,D}\tn C$, we find that
	\begin{diagram}[h=3.5em,labelstyle=\scriptstyle]
		A\tn \I\tn C\tn D \\
		\dTo<{A\tn \I\tn s_{C,D}} & \rdTo(2,2)[hug]^{A\tn s_{\I\tn C,D}}
			\raise-1.5em\hbox to 0pt{\hss$\Swarrow\scriptstyle A\tn S_{\I,C|D}$}
			\\
		A\tn \I\tn D\tn C &\rTo_{A\tn s_{\I,D}\tn C} & A\tn D\tn \I\tn C \\
		%%
		& \rdTo_{1} \raise1em\rlap{$\Swarrow\scriptstyle A\tn U_{\I|D}\tn C$}
			& \dTo>1 \\
		& & A\tn D\tn C
	\end{diagram}
	is the identity, and setting $A=\I$ yields the claim. XYZ
\end{proof}

\bibliography{cs}
\end{document}