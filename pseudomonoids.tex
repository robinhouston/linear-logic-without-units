%!TEX TS-program = latex
\documentclass{robinthesis}

\begin{thesischapter}{Psmon}{Pseudomonoids}
\renewcommand\ss{\mathfrak{s}}
A monoid, in a monoidal category, consists of an object $A$ equipped
with a unit $u: I\to A$ and a multiplication $m: A\tn A\to A$, satisfying
the obvious unit and associativity axioms. In a monoidal \emph{bicategory},
the corresponding notion is that of a pseudomonoid, where the unit and
associativity laws hold, not on the nose, but up to coherent isomorphism.
The primeval example is of course that of monoidal categories, which are
pseudomonoids in the monoidal bicategory $\Cat$; though for present
purposes we are particularly interested in promonoidal categories,
which are pseudomonoids in $\Prof$.

Little has been published about pseudomonoids per se, though the
definition is given by \citet{MonBicat}. However, the
equivalent notion of \emph{pseudomonad} (in a tricategory or Gray-category)
has received more attention:
from \citet{MarmolejoPseudomonads, MarmolejoDistributive, MarmolejoDistributiveII, LackPseudomonads, TanakaThesis},
among others.

\begin{definition} % definition of pseudomonoid
	A pseudomonoid $\C$ in a monoidal bicategory $\B$ is a normal
	pseudofunctor $1\to\B$. More concretely, it consists of an object $\C$,
	1-cells
	\[\begin{array}l
		J: \I\to\C,\\
		P:\C\tn\C\to\C,
	\end{array}\]
	and invertible 2-cells
	\[\begin{array}{c@{\qquad}c}
		\multicolumn2c{\begin{diagram}
			\C\tn(\C\tn\C) && \rTo^{a_{\C,\C,\C}} && (\C\tn\C)\tn\C
			\\
			\dTo<{\C\tn P}
			&& \begin{array}c\To\\[-4pt]\aa\end{array}
			&& \dTo>{P\tn\C}
			\\
			\C\tn\C & \rTo_{P} & \C & \lTo_{P} & \C\tn\C
		\end{diagram}}
		\\[6em]
		\begin{diagram}[vtrianglewidth=1.5em,tight]
			\I\tn\C &&\rTo^{J\tn\C}&&\C\tn\C\\
			&\rdTo[snake=-1ex]<{l_\C}
				&\raise1ex\hbox{$\begin{array}c\Rightarrow\\[-5pt]\ll\end{array}$}%
				&\ldTo[snake=1ex]>{P}\\
			&&\C
		\end{diagram}
		&
		\begin{diagram}[vtrianglewidth=1.5em,tight]
			\C\tn \I &&\rTo^{\C\tn J} && \C\tn\C\\
			&\rdTo[snake=-1ex]<{r_\C}
				&\raise1ex\hbox{$\begin{array}c\Rightarrow\\[-5pt]\rr\end{array}$}%
				&\ldTo[snake=1ex]>{P}\\
			&&\C
		\end{diagram}
	\end{array}\]
	subject to the two equations below (stated in the Gray monoid setting).
	Since the 2-cells are assumed to be invertible, we shall permit ourselves
	to omit the arrow. Also, here and elsewhere, we write $\C^{2}$ for $\C\tn\C$.
	\begin{equation}\label{eq-lra}
	\begin{diagram}
		\C\tn\I\tn\C\\
		\dTo<1 &\hbox to0pt{\hss$\C\tn\ll$}\rdTo(2,1)^{\C\tn J\tn\C} & \C\tn\C\tn\C\\
		\C\tn\C & \ldTo(2,1)_{\C\tn P} &\dTo>{P\tn\C}\\
		\dTo<P & \raise1.5em\hbox{$\aa$}& \C\tn\C\\
		\C&\ldTo(2,1)_P
	\end{diagram}
	\qquad=\qquad
	\begin{diagram}
		\C\tn\I\tn\C\\
		\dTo<1 &\hbox to0pt{\hss$\rr\tn\C$}\rdTo(2,1)^{\C\tn J\tn\C} & \C\tn\C\tn\C\\
		\C\tn\C & \ldTo(2,1)_{P\tn\C}\\
		\dTo<P\\
		\C
	\end{diagram}
	\end{equation}
	and
	\begin{equation}\label{eq-aa}
		\begin{diagram}[s=2.2em,labelstyle=\scriptstyle,tight]
			&&\C^3\\
			&\ruTo^{\C^2\tn P}&\dTo[snake=-5pt]<{P\tn\C}&\rdTo^{\C\tn P}\\
			\C^4 &\sim& \C^2 &\mathop{\Leftarrow}\limits_{\;\;\;\aa}& \C^2\\
			\dTo<{P\tn\C^2}&\ruTo_{\C\tn P} && \rdTo_P & \dTo>{P}\\
			\C^3 && \Arr\Downarrow\aa && \C\\
			&\rdTo_{P\tn\C}&&\ruTo>{P}\\
			&&\C^2
		\end{diagram}
		\qquad=\qquad
		\begin{diagram}[s=2.2em,labelstyle=\scriptstyle,tight]
			&&\C^3\\
			&\ruTo^{\C^2\tn P}&&\rdTo^{\C\tn P}\\
			\C^4 && \Arr\Downarrow{\C\tn \aa} && \C^2\\
			\dTo<{P\tn\C^2}&\rdTo^{\C\tn P\tn\C} && \ruTo^{\C\tn P} & \dTo>{P}\\
			\C^3 &\mathop{\Leftarrow}\limits_{\;\;\;\aa\tn\C}& \C^3 &\Arr\Swarrow{\scriptstyle\!\!\!\aa}& \C\\
			&\rdTo_{P\tn\C}&\dTo[snake=5pt]>{\!\!P\tn\C}&\ruTo>{P}\\
			&&\C^2
		\end{diagram}
	\end{equation}
	% Note that equation (\ref{eq-lra}) is equivalent to
	% \[
	% \begin{diagram}
	% 	\C\tn\I\tn\C\\
	% 	\dTo<1 &\hbox to0pt{\hss$\C\tn\ll$}\rdTo(2,1)^{\C\tn J\tn\C} & \C\tn\C\tn\C\\
	% 	\C\tn\C & \ldTo(2,1)_{\C\tn P}\\
	% 	\dTo<P\\
	% 	\C
	% \end{diagram}
	% \qquad=\qquad
	% \begin{diagram}
	% 	\C\tn\I\tn\C\\
	% 	\dTo<1 &\hbox to0pt{\hss$\rr\tn\C$}\rdTo(2,1)^{\C\tn J\tn\C} & \C\tn\C\tn\C\\
	% 	\C\tn\C & \ldTo(2,1)_{P\tn\C} &\dTo>{\C\tn P}\\
	% 	\dTo<P & \raise1.5em\hbox{$\aa$}& \C\tn\C\\
	% 	\C&\ldTo(2,1)_P
	% \end{diagram}
	% \]
	% by composing with the inverse of $\aa$. We shall often use it in this form, without
	% further remark. Similar variations of other equations may also be used without
	% drawing attention to the fact.
\end{definition}
%
The first of these equations corresponds to the triangle axiom relating
$\alpha$, $\lambda$ and $\rho$, and the second to Mac Lane's pentagon
axiom.

\section{Some facts about pseudomonoids}
If we express the results of \cite{KellyML} in the language
of general pseudomonoids, we obtain the three equations below.
In the following sections, we shall prove that they hold in
general, by showing how Kelly's argument can be applied,
via the calculus of components, to any pseudomonoid $\C$.
\begin{equation}\label{eq-lla}
\begin{diagram}
	\I\tn\C^2 & \rTo^{J\tn\C^2} &\C^3& \rTo^{\C\tn P} & \C^2\\
	&\rdTo(2,2)<1\raise1em\hbox to0pt{$\ll\tn\C$\hss} & \dTo[snake=.5em]>{P\tn\C} &\aa&\dTo>P\\
	&&\C^2 &\rTo_P &\C
\end{diagram}
\quad=\quad
\begin{diagram}
	\I\tn\C^2 &\rTo^{J\tn\C^2} & \C^3\\
	\dTo<{\I\tn P} &\sim & \dTo>{\C\tn P}\\
	\I\tn\C &\rTo^{J\tn\C} & \C^2\\
	&\rdTo(2,2)_1\raise0.5em\hbox to0pt{\hskip0.5em$\ll$\hss} &\dTo>P\\
	&&\C
\end{diagram}
\end{equation}
\begin{equation}\label{eq-rra}
\begin{diagram}
	\C^2\tn\I & \rTo^{\C^2\tn J} &\C^3& \rTo^{P\tn\C} & \C^2\\
	&\rdTo(2,2)<1\raise1em\hbox to0pt{$\C\tn\rr$\hss} & \dTo[snake=.5em]>{\C\tn P} &\aa&\dTo>P\\
	&&\C^2 &\rTo_P &\C
\end{diagram}
\quad=\quad
\begin{diagram}
	\C^2\tn\I &\rTo^{\C^2\tn J} & \C^3\\
	\dTo<{P\tn\I} &\sim & \dTo>{P\tn\C}\\
	\C\tn\I &\rTo^{\C\tn J} & \C^2\\
	&\rdTo(2,2)_1\raise0.5em\hbox to0pt{\hskip0.5em$\rr$\hss} &\dTo>P\\
	&&\C
\end{diagram}
\end{equation}
\begin{equation}\label{eq-lr}
\begin{diagram}
	\I\tn\I\\
	\dTo<{J\tn\I} & \rdTo(2,1)^{\I\tn J} & \I\tn\C\\
	\C\tn\I &\raise1em\hbox{$\sim$}&\dTo>{J\tn\C}\\
	\dTo<1 & \hbox to0pt{\hss$\rr$\hskip0.5em}\rdTo(2,1)^{\C\tn J}& \C\tn\C\\
	\C&\ldTo(2,1)_P
\end{diagram}
\quad=\quad
\begin{diagram}
	\I\tn\I\\
	\dTo<{J\tn\I} & \rdTo(2,1)^{\I\tn J} & \I\tn\C\\
	\C\tn\I &\ldTo(2,3)^1&\dTo>{J\tn\C}\\
	\dTo<1 & \raise1em\hbox to0pt{\hskip1em$\ll$\hss}& \C\tn\C\\
	\C&\ldTo(2,1)_P
\end{diagram}
\end{equation}
%
% In fact, it turns out that only equation (\ref{eq-lla}) is needed to prove the embedding theorem.
% For the sake of completeness, however, we shall prove all three.
%
\citet[section~3.4]{LackThesis} describes an interesting geometrical way to
prove these equations, using certain four-dimensional diagrams.\footnote{
	Lack is working in the slightly more general
	context of enriched bicategories: a pseudomonoid is an enriched
	bicategory with one object.}
\citet[][Proposition~8.1]{MarmolejoPseudomonads} gives a more down-to-earth version of the argument, using pasting diagrams. Instead we will show how they follow from Kelly's proof for ordinary monoidal categories, by the calculus of components.

\section{Pseudomonoids via the calculus of components}
Next we see how the language of components may be used to reason about pseudomonoids.
We will define the theory $\M$ of pseudomonoids, and show that reasoning
in the formal language corresponds precisely to the usual modes of reasoning
about monoidal categories, and that a model of $\M$ is precisely a pseudomonoid.
The theory concerns
a single object $\C$, so $\M_{0} = \{\C\}$. There are two basic 1-cells,
$J:\I\to\C$ and $P:\C\tn\C\to\C$, so formally we have $\M(;\C) = \{J\}$
and $\M(\C,\C;\C):=\{P\}$. There are six basic 2-cells, corresponding to
$\aa$, $\ll$, $\rr$ and their inverses. The 2-cell $\aa$ goes from
\[
	A\in\C, B\in\C, C\in\C \proves P(A,P(B,C))\in\C
\]
to
\[
	A\in\C, B\in\C, C\in\C \proves P(P(A,B),C)\in\C;
\]
formally, for every three distinct names $A$, $B$ and $C$ we have
\[\M^{(A\in\C,B\in\C,C\in\C)}_{\C}[P(A,P(B,C)),P(P(A,B),C)] = \{\aa\}.\]
To make the notation appear more familiar, we shall write $A\ast B$ to
mean $P(A,B)$, and $I$ to mean $J()$. Thus $\aa$ is a 2-cell with components
\[
	\aa_{A,B,C}: A\ast (B\ast C) \to (A\ast B)\ast C
\]
for $A$, $B$, $C\in\C$.
In fact we want $\aa$ to be an invertible 2-cell, which formally
means that there is another 2-cell $\aa^{-1}$ with components
\[
	\aa^{-1}_{A,B,C}: (A\ast B)\ast C \to A\ast (B\ast C)
\]
such that $\aa_{A,B,C}\cdot\aa^{-1}_{A,B,C}$ and
$\aa^{-1}_{A,B,C}\cdot\aa_{A,B,C}$ are identities.
In terms of the formal definition of the theory $\M$, this means
\[\M^\Gamma_{\C}[P(P(A,B),C),P(A,P(B,C))] = \{\aa^{-1}\},\]
\[(\Gamma,\C,P(P(A,B),C),P(P(A,B),C),\aa^{-1}_{A,B,C}\cdot\aa_{A,B,C},
1_{P(P(A,B),C)})\in\M_{=},\]
\[(\Gamma,\C,P(A,P(B,C)),P(A,P(B,C)),\aa_{A,B,C}\cdot\aa^{-1}_{A,B,C},
1_{P(A,P(B,C))})\in\M_{=},\]
where $\Gamma=(A\in\C,B\in\C,C\in\C)$.

In the same way, we want $\ll$ to be an invertible 2-cell with
components
\[
	\ll_{A}: I\ast A \to A
\]
for $A\in\C$, and $\rr$ to be an invertible 2-cell with
components
\[
	\rr_{A}: A\ast I \to A
\]
for $A\in\C$. (Formally, the theory $\M$ contains 2-cells $\ll$,
$\ll^{-1}$, $\rr$ and $\rr^{-1}$, and four equations expressing
that $\ll^{-1}$ is inverse to $\ll$ and $\rr^{-1}$ is inverse to $\rr$.)

Finally, we have the pentagon and triangle axioms. Consider the
pentagon:
\begin{mspill}\begin{diagram}%\dlabel{pentagon}
  A\ast (B\ast (C\ast D))
	  &\rTo^{\aa_{A,B,C\ast D}}&(A\ast B)\ast (C\ast D)
	  &\rTo^{\aa_{A\ast B,C,D}} & \bigl(((A\ast B)\ast C)\ast D\bigr)
  \\
  &\rdTo[snake=-1em](1,2)<{A\ast \aa_{B,C,D}} &%\dnum[pentagon]
	  & \ruTo[snake=1em](1,2)>{\aa_{A,B,C}\ast D}
  \\
  & \spleft{A\ast ((B\ast C)\ast D)}
	  & \rTo_{\aa_{A,B\ast C,D}}
	  & \spright{(A\ast (B\ast C))\ast D}
\end{diagram}\end{mspill}
Of course, this diagram is just a convenient way of writing the equation
\[
	\aa_{A\ast B,C,D} \cdot \aa_{A,B\ast C,D}
	=
	P(\aa_{A,B,C},D) \cdot \aa_{A,B\ast C,D} \cdot (A\ast\aa_{B,C,D})
\]
which we take as a formal equation of $\M$. Thus any model of $\M$
must satisfy
\[
	\semint{\aa_{A\ast B,C,D} \cdot \aa_{A,B\ast C,D}}
	=
	\semint{P(\aa_{A,B,C},D) \cdot \aa_{A,B\ast C,D} \cdot (A\ast\aa_{B,C,D})}.
\]
Similarly the triangle:
\begin{diagram}
        A\ast (I\ast C) &\rTo^{\aa_{A,I,C}}& (A\ast I)\ast C\\
        &\rdTo[snake=-1ex](1,2)<{A\ast\ll_C}\ldTo[snake=1ex](1,2)>{\rr_A\ast C}\\
        &A\ast C
\end{diagram}
represents the equation
\[
	(\rr_A\ast C) \cdot \aa_{A,I,C}  =  A\ast\ll_C,
\]
hence a model of $\M$ must also satisfy
\[
	\semint{(\rr_A\ast C) \cdot \aa_{A,I,C}}  =  \semint{A\ast\ll_C}.
\]
%
Now, let us consider an interpretation of $\M$ in a Gray monoid $\B$. We shall
omit $v()$, writing just $\C$ instead of $v(\C)$, and $P$ instead of $v(P)$, etc.
So an interpretation of $\M$ consists of an object $\C$,
1-cells $J: \I \to \C$ and $P:\C\tn\C\to\C$,
and invertible 2-cells
\[\begin{array}{c@{\qquad}c}
	\multicolumn2c{	\begin{diagram}[s=2.2em,tight]
		&& \C^2\\
		&\ruTo^{\C\tn P} && \rdTo^P\\
		\C^3 && \Arr\Downarrow\aa && \C\\
		&\rdTo_{P\tn\C}&&\ruTo>{P}\\
		&&\C^2
	\end{diagram}
}
	\\[6em]
	\begin{diagram}[vtrianglewidth=1.5em,tight]
		\C\tn\C &&\lTo^{J\tn\C}&&\I\tn\C\\
		&\rdTo[snake=-1ex]<{P}
			&\raise1ex\hbox{$\begin{array}c\Rightarrow\\[-5pt]\ll\end{array}$}%
			&\ldTo[snake=1ex]>{1}\\
		&&\C
	\end{diagram}
	&
	\begin{diagram}[vtrianglewidth=1.5em,tight]
		\C\tn \C &&\lTo^{\C\tn J} && \C\tn\I\\
		&\rdTo[snake=-1ex]<{P}
			&\raise1ex\hbox{$\begin{array}c\Rightarrow\\[-5pt]\rr\end{array}$}%
			&\ldTo[snake=1ex]>{1}\\
		&&\C
	\end{diagram}
\end{array}\]
where we have written $\C^{2}$ as an abbreviation for $\C\tn\C$, etc.,
subject to equations corresponding to the pentagon and triangle conditions.
%
Let us first consider the pentagon equation
\[
	\semint{\aa_{A\ast B,C,D} \cdot \aa_{A,B,C\ast D}}
	=
	\semint{(\aa_{A,B,C}\ast D) \cdot \aa_{A,B\ast C,D} \cdot (A\ast\aa_{B,C,D})}.
\]
equivalently
\[
	\semint{\aa_{A\ast B,C,D}} \cdot \semint{\aa_{A,B,C\ast D}}
	=
	\semint{(\aa_{A,B,C}\ast D)} \cdot \semint{\aa_{A,P(B,C),D}} \cdot \semint{P(A,\aa_{B,C,D})}.
\]
By definition, $\semint{\aa_{A,B, C\ast D}}: \semint{A\ast(B\ast(C\ast D))}
	\To \semint{(A\ast B)\ast(C\ast D)}$ is
\begin{diagram}
	\semint{A\ast(B\ast(C\ast D))} && \semint{(A\ast B)\ast(C\ast D)} \\
	\dTo<{\norm^{-1}} && \uTo>{\norm} \\ 
	\semint{A\ast(B\ast X)} \o (\C\tn\C\tn P)
	& \rTo_{\aa \o (\C\tn\C\tn P)}
	& \semint{(A\ast B)\ast X} \o (\C\tn\C\tn P)
\end{diagram}
On the left we have
\begin{mmulti}
	\semint{
	A\ast(B\ast(C\ast D))
	}
	\\= P \o (\C\tn\semint{
	B\ast(C\ast D)
	})
	\\= P \o (\C\tn(
		P \o (\C\tn\semint{
			C\ast D
		}
	)))
	\\= P \o (\C\tn(
		P \o (\C\tn P)
	))
	\\= P \o (\C\tn P) \o (\C\tn\C\tn P)
\end{mmulti}
and the $\norm^{-1}$ map is just the identity. The reason is
essentially that every occurrence of $P$ has one argument equal
to a constant. Thus the right-hand side is more interesting: we have
\[
	\semint{
	(A\ast B)\ast(C\ast D)
	}
	=
	P \o (P\tn P)
\]
and the $\norm$ map is the isomorphism
\[
	P \o (P\tn\C) \o (\C\tn\C\tn P) \to P \o (P\tn P).
\]
So this 2-cell is
{\def\rnode#1#2{%
	\tikz[baseline=(#1.base),inner sep=0pt,outer sep=3pt]
		\node(#1){\mathsurround=0pt$\displaystyle#2$};
	}
\begin{diagram}[s=2.2em]
	&& && \C^2\\
	&& &\ruTo^{\C\tn P} && \rdTo^P\\
	\rnode{left}{\C^{4}}&\rTo^{\C\tn \C\tn P}& \C^3 && \Arr\Downarrow\aa && \C\\
	&& &\rdTo[hug]_{P\tn\C}&&\ruTo>{P}\\
	&& &&\rnode{bottom}{\C^2}
	%
	\begin{tikzpicture}[overlay]
		\path[->] (left) edge [out=-45, in=180, out looseness=0.7]
			node [left, pos=0.6] {$P\tn P$} node [above=1em, pos=0.5] {$\sim$} (bottom);
	\end{tikzpicture}
\end{diagram}
In a similar way, one may calculate that
\[
	\semint{\aa_{P(A,B),C,D}}
\]
is equal to
\begin{diagram}[s=2.2em]
	&& && \rnode{top}{\C^2}\\
	&& &\ruTo[hug]^{\C\tn P} && \rdTo^P\\
	\rnode{left}{\C^{4}}&\rTo_{P\tn\C\tn\C}& \C^3 && \Arr\Downarrow\aa && \C\\
	&& &\rdTo_{P\tn\C}&&\ruTo>{P}\\
	&& &&\rnode{bottom}{\C^2}
	%
	\begin{tikzpicture}[overlay]
		\path[->] (left) edge [in = 180, out looseness=0.7]
			node [left, pos=0.6] {$P\tn P$} node [below=1em, pos=0.5] {$\sim$} (top);
	\end{tikzpicture}
\end{diagram}}
Note that, according to the definition of Gray monoid, in fact
\[
	P\tn P = (\C\tn P)\o(P\tn\C\tn\C),
\]
and this interchange cell is the identity.
Thus the composite
\[
	\semint{\aa_{A\ast B,C,D} \cdot \aa_{A,B,C\ast D}}
	= \semint{\aa_{A\ast B,C,D}} \cdot \semint{\aa_{A,B,C\ast D}}
\]
is
\begin{diagram}[s=2.2em,labelstyle=\scriptstyle,tight]
	&&\C^3\\
	&\ruTo^{\C^2\tn P}&\dTo[snake=-5pt]<{P\tn\C}&\rdTo^{\C\tn P}\\
	\C^4 &\sim& \C^2 &\mathop{\Leftarrow}\limits_{\;\;\;\aa}& \C^2\\
	\dTo<{P\tn\C^2}&\ruTo[hug]_{\C\tn P} && \rdTo_P & \dTo>{P}\\
	\C^3 && \Arr\Downarrow\aa && \C\\
	&\rdTo_{P\tn\C}&&\ruTo>{P}\\
	&&\C^2
\end{diagram}
%
Calculating
\[
	\semint{(\aa_{A,B,C}\ast D) \cdot \aa_{A,B\ast C,D} \cdot (A\ast\aa_{B,C,D})}.
\]
is less interesting, since all the $\norm$ maps are identities, so
it is just
\begin{diagram}[s=2.2em,labelstyle=\scriptstyle,tight]
	&&\C^3\\
	&\ruTo^{\C^2\tn P}&&\rdTo^{\C\tn P}\\
	\C^4 && \Arr\Downarrow{\C\tn \aa} && \C^2\\
	\dTo<{P\tn\C^2}&\rdTo[hug]^{\C\tn P\tn\C} && \ruTo[hug]^{\C\tn P} & \dTo>{P}\\
	\C^3 &\mathop{\Leftarrow}\limits_{\;\;\;\aa\tn\C}& \C^3 &\Arr\Swarrow{\scriptstyle\!\!\!\aa}& \C\\
	&\rdTo_{P\tn\C}&\dTo[snake=5pt]>{\!\!P\tn\C}&\ruTo>{P}\\
	&&\C^2
\end{diagram}
Thus the pentagon axiom, in a Gray monoid, is precisely
equation~\pref{eq-aa}.
% \begin{equation}\label{eq-aa}
% 	\begin{diagram}[s=2.2em,labelstyle=\scriptstyle,tight]
% 		&&\C^3\\
% 		&\ruTo^{\C^2\tn P}&\dTo[snake=-5pt]<{P\tn\C}&\rdTo^{\C\tn P}\\
% 		\C^4 &\sim& \C^2 &\mathop{\Leftarrow}\limits_{\;\;\;\aa}& \C^2\\
% 		\dTo<{P\tn\C^2}&\ruTo[hug]_{\C\tn P} && \rdTo_P & \dTo>{P}\\
% 		\C^3 && \Arr\Downarrow\aa && \C\\
% 		&\rdTo_{P\tn\C}&&\ruTo>{P}\\
% 		&&\C^2
% 	\end{diagram}
% 	\qquad=\qquad
% 	\begin{diagram}[s=2.2em,labelstyle=\scriptstyle,tight]
% 		&&\C^3\\
% 		&\ruTo^{\C^2\tn P}&&\rdTo^{\C\tn P}\\
% 		\C^4 && \Arr\Downarrow{\C\tn \aa} && \C^2\\
% 		\dTo<{P\tn\C^2}&\rdTo[hug]^{\C\tn P\tn\C} && \ruTo[hug]^{\C\tn P} & \dTo>{P}\\
% 		\C^3 &\mathop{\Leftarrow}\limits_{\;\;\;\aa\tn\C}& \C^3 &\Arr\Swarrow{\scriptstyle\!\!\!\aa}& \C\\
% 		&\rdTo_{P\tn\C}&\dTo[snake=5pt]>{\!\!P\tn\C}&\ruTo>{P}\\
% 		&&\C^2
% 	\end{diagram}
% \end{equation}
In a similar way, the triangle axiom corresponds
to equation~\pref{eq-lra}.
% \begin{equation}
% 	\def\rnode#1#2{%
% 	\tikz[baseline=(#1.base),inner sep=0pt,outer sep=3pt]
% 		\node(#1){\mathsurround=0pt$\displaystyle#2$};
% 	}
% \begin{diagram}[s=2.2em]
% 	& & \rnode{CIC}{\C\tn\I\tn\C} \\
% 	& & \dTo<{\C\tn J\tn\C} \\
% 	& & \C^{3} \\
% 	& \ldTo^{\C\tn P} && \rdTo[hug]^{P\tn\C} \\
% 	\C^{2} && \Right_\aa && \rnode{CC}{\C^{2}} \\
% 	& \rdTo_{P} && \ldTo_{P} \\
% 	& & \C
% 	\begin{tikzpicture}[overlay]
% 		\path[->] (CIC) edge [out=0, in=90, out looseness=0.7]
% 			node [right, pos=0.6] {$1$} node [left=1em, pos=0.5] {$\Right_{\rr\tn\C}$} (CC);
% 	\end{tikzpicture}
% \end{diagram}
% \qquad=\qquad
% \begin{diagram}[s=2.2em]
% 	& & \rnode{CIC}{\C\tn\I\tn\C} \\
% 	& \ldTo^{\C\tn J\tn\C} \\
% 	\C^{3} && \llap{$\Right_{\rr\tn\C}$} \\
% 	& \rdTo_{\C\tn P} \\
% 	& & \rnode{CC}{\C^{2}} \\
% 	& & \dTo<{P} \\
% 	& & \C 
% 	\begin{tikzpicture}[overlay]
% 		\path[->] (CIC) edge [out=-45, in=45, out looseness=0.7]
% 			node [right, pos=0.6] {$1$} (CC);
% 	\end{tikzpicture}
% \end{diagram}
% \end{equation}
%
The interpretation in a general monoidal bicategory works in a
similar way; the monoidal biequivalence $\mathbf{e}'$ causes structural
1-cells and 2-cells to be inserted where necessary. Let us consider
the 2-cell $\aa$: its interpretation needs to be a 2-cell
\[
	\semint{A\ast (B\ast C)} \to \semint{(A\ast B)\ast C},
\]
and $\semint{A\ast (B\ast C)}$ is simply
\[
	\C\tn(\C\tn\C)
	\rTo^{\C\tn P}
	\C\tn\C
	\rTo^{P}
	\C
\]
whereas $\semint{(A\ast B)\ast C}$ is
\[
	\C\tn(\C\tn\C)
	\rTo^{a_{\C,\C,\C}}
	(\C\tn\C)\tn\C
	\rTo^{P\tn\C}
	\C\tn\C
	\rTo^{P}
	\C,
\]
hence $\aa$ should be a 2-cell
\begin{diagram}
		\C\tn(\C\tn\C)
		& \rTo^{\C\tn P} & \C\tn\C
		\\
		\dTo<{a_{\C,\C,\C}}
		& \Swarrow\aa
		&& \rdTo^{P}
		\\
		(\C\tn\C)\tn\C
		& \rTo_{P\tn\C} & \C\tn\C
		& \rTo_{P} & \C
\end{diagram}
The axioms also sport structural 2-cells. For example, the pentagon equation becomes
the requirement that
{\def\rnode#1#2{%
	\tikz[baseline=(#1.base),inner sep=0pt,outer sep=3pt]
		\node(#1){\mathsurround=0pt$\displaystyle#2$};
	}
\begin{mspill}\begin{diagram}
	\C\tn(\C\tn(\C\tn\C))
	& \rTo^{\C\tn(\C\tn P)} & \C\tn(\C\tn\C)
	& \rTo^{\C\tn P} & \C\tn\C
	\\
	\dTo[snake=-1.5em]<{a_{\C,\C,\C\tn\C}}
	& \Swarrow a_{\C,\C,P}
	& \dTo>{a_{\C,\C,\C}}
	& \Swarrow\aa
	&& \rdTo(2,3)^P
	\\
	&& (\C\tn\C)\tn\C
	\\
	(\C\tn\C)\tn(\C\tn\C)
	& \ruTo(2,1)^{(\C\tn\C)\tn P}
	& \cong
	& \rdTo(2,1)^{P\tn\C}
	& \C\tn\C
	& \rTo_P & \C
	\\
	\dTo[snake=-1.5em]<{a_{\C\tn\C,\C,\C}}
	& \rdTo(2,1)_{P\tn(\C\tn\C)}
	&\C\tn(\C\tn\C)
	& \ruTo(2,1)_{\C\tn P}
	&& \ruTo(2,3)_P
	\\
	& \Swarrow a_{P,\C,\C}
	& \dTo>{a_{\C,\C,\C}}
	& \Swarrow\aa
	\\
	((\C\tn\C)\tn\C)\tn\C
	& \rTo_{(P\tn\C)\tn\C} & (\C\tn\C)\tn\C
	& \rTo_{P\tn\C} & \C\tn\C
\end{diagram}\end{mspill}
must be equal to
\begin{diagram}
	&& \rnode{t}{\C\tn(\C\tn(\C\tn\C))}
	& \rTo^{\C\tn(\C\tn P)} & \C\tn(\C\tn\C)
	\\
	& %\ldTo(2,3)
	& \dTo<{\C\tn a_{\C,\C,\C}}
	& \Swarrow\C\tn \aa
	&& \rdTo^{\C\tn P}
	\\
	&& \C\tn ((\C\tn\C)\tn\C)
	& \rTo_{\C\tn(P\tn\C)} & \C\tn(\C\tn\C)
	& \rTo_{\C\tn P} & \C\tn\C
	\\
	\rnode{l}{(\C\tn\C)\tn(\C\tn\C)} & \Left_{\pi_{A,B,C,D}^{-1}}
	& \dTo<{a_{\C,\C\tn\C,\C}}
	& \Swarrow a_{\C,P,\C}
	& \dTo<{a_{\C,\C,\C}}
	& \Swarrow\aa
	&& \rdTo^{P}
	\\
	& %\rdTo(2,3)
	& (\C\tn(\C\tn\C))\tn\C
	& \rTo_{(\C\tn P)\tn\C}
	& (\C\tn\C)\tn\C
	& \rTo_{P\tn\C}
	& \C\tn\C & \rTo_P & \C
	\\
	&& \dTo<{a_{\C,\C,\C}\tn\C}
	& \Downarrow\aa\tn\C
	&& \ruTo_{P\tn\C}
	\\
	&& \rnode{b}{((\C\tn\C)\tn\C)\tn\C} & \rTo_{(P\tn\C)\tn\C} & (\C\tn\C)\tn\C
	\begin{tikzpicture}[overlay]
		\path[->] (t) edge [out=180, in=90]
			node [left] {$a_{\C,\C,\C\tn\C}$} (l);
		\path[->] (l) edge [out=-90, in=180]
			node [left] {$a_{\C\tn\C,\C,\C}$} (b); 
	\end{tikzpicture}
\end{diagram}
and the triangle equation becomes the requirement that
\begin{diagram}
	\C\tn(\I\tn\C) & \rTo^{a_{\C,\I,\C}} & \rnode{t}{(\C\tn\I)\tn\C} \\
	\dTo<{\C\tn(J\tn\C)} & \Arr\Nearrow{a_{\C,J,\C}} & \dTo>{(\C\tn J)\tn\C} \\
	\C\tn(\C\tn\C) & \rTo_{a_{\C,\C,\C}} & (\C\tn\C)\tn\C & \Right_{\rr\tn\C} \\
	\dTo<{\C\tn P} & \Arr\Nearrow{\aa} && \rdTo_{P\tn\C} \\
	\C\tn\C & \rTo_{P} & \C & \lTo_{P} & \rnode{b}{\C\tn\C}
	\begin{tikzpicture}[overlay]
		\path[->] (t) edge [out=0, in=90]
			node [right] {$r_{\C}\tn\C$} (b);
	\end{tikzpicture}
\end{diagram}
be equal to
\begin{diagram}[h=2.2em]
	& & \rnode{CIC}{\C\tn(\I\tn\C)} \\
	& \ldTo^{\C\tn (J\tn\C)} && \rdTo^{a_{\C,\I,\C}} \\
	\C^{3} & \rlap{$\Right_{\rr\tn\C}$} & \dTo~{\C\tn l_{\C}}
		& \llap{$\Right_{\mu_{\C,\C}}$} & (\C\tn\I)\tn\C\\
	& \rdTo_{\C\tn P} && \ldTo_{r_{\C}\tn\C} \\
	& & \rnode{CC}{\C\tn\C} \\
	& & \dTo<{P} \\
	& & \C 
	% \begin{tikzpicture}[overlay]
	% 	\path[->] (CIC) edge [out=-45, in=45, out looseness=0.7]
	% 		node [right, pos=0.6] {$\C\tn l_{\C}$} (CC);
	% \end{tikzpicture}
\end{diagram}}
(In fact the `raw' version of the equation, as it emerges from the
interpretation, has $\mu^{-1}$ on the left-hand side, rather than
$\mu$ on the right as we have written.)

\section{Calculating in the theory of pseudomonoids}
Now we may use the language to prove various facts about
pseudomonoids, essentially using the formal interpretation of the
language as a translation tool that allows us to transfer
proofs from the familiar setting of monoidal categories.
As a first example, consider the simple fact that
\[
	\ll_{I\ast A} = I\ast\ll_{A}: I\ast(I\ast A) \to I\ast A.
\]
The usual proof runs as follows: since $\ll$ is a natural transformation,
we have a naturality square
\begin{diagram}
	I\ast (I\ast A) & \rTo^{I\ast\ll_{A}} & I\ast A \\
	\dTo<{\ll_{I\ast A}} &\natural& \dTo>{\ll_{A}} \\
	I\ast A & \rTo_{\ll_{A}} & A
\end{diagram}
which, since $\ll_{A}$ is invertible, implies the claim.
The formal proof in our language is precisely the same:
the naturality square is an instance of the naturality
axiom; we then compose with $\ll^{-1}_{A}$ (using the axiom
that composition preserves equality), then use our
axiom relating $\ll$ and $\ll^{-1}$ to derive
\[
	1_{A} \cdot \ll_{I\ast A} = 1_{A} \cdot I\ast \ll_{A},
\]
and finally use the identity axiom (and symmetry and transitivity)
to conclude that
\[
	\ll_{I\ast A} = I\ast\ll_{A}: I\ast(I\ast A) \to I\ast A
\]
as required. In a Gray monoid, this shows that
{\def\rnode#1#2{%
	\tikz[baseline=(#1.base),inner sep=0pt,outer sep=3pt]
		\node(#1){\mathsurround=0pt$\displaystyle#2$};
	}
\begin{diagram}[h=1.5em]
	\\\ \\
	&&& \sim\\
	\rnode{l}{\C} & \rTo_{J\tn\C} & \C\tn\C & \rTo_{P} & \rnode{l2}{\C} & \rTo_{J\tn\C}
		& \rnode{r}{\C\tn\C} & \rTo_{P} & \rnode{r2}{\C} \\
	&&&&&& \Downarrow\ll \\
	\begin{tikzpicture}[overlay]
		\path[->] (l) edge [out=30, in=150]
			node [above] {$J\tn(P\o(J\tn\C))$} (r);
		\path[->] (l2) edge [out=-60, in=-120]
			node [below] {$1$} (r2);
	\end{tikzpicture}
\end{diagram}
is equal to
\begin{diagram}[h=1.5em]
	\\\ \\
	&&& \sim\\
	\rnode{l}{\C} & \rTo_{J\tn\C} & \C\tn\C & \rTo_{P} & \rnode{r2}{\C} & \rTo_{J\tn\C}
		& \rnode{r}{\C\tn\C} & \rTo_{P} & \C \\
	&& \Downarrow\ll \\
	\begin{tikzpicture}[overlay]
		\path[->] (l) edge [out=30, in=150]
			node [above] {$J\tn(P\o(J\tn\C))$} (r);
		\path[->] (l) edge [out=-60, in=-120]
			node [below] {$1$} (r2);
	\end{tikzpicture}
\end{diagram}
Since we may cancel the common $\norm$ cell, this can equivalently
be stated as:
\begin{diagram}[h=1.5em]
	\rnode{l}{\C} & \rTo_{J\tn\C} & \C\tn\C & \rTo_{P} & \rnode{l2}{\C} & \rTo_{J\tn\C}
		& \rnode{r}{\C\tn\C} & \rTo_{P} & \rnode{r2}{\C} \\
	&&&&&& \Downarrow\ll \\
	\begin{tikzpicture}[overlay]
		\path[->] (l2) edge [out=-60, in=-120]
			node [below] {$1$} (r2);
	\end{tikzpicture}
\end{diagram}
is equal to
\begin{diagram}[h=1.5em]
	\rnode{l}{\C} & \rTo_{J\tn\C} & \C\tn\C & \rTo_{P} & \rnode{r2}{\C} & \rTo_{J\tn\C}
		& \rnode{r}{\C\tn\C} & \rTo_{P} & \C \\
	&& \Downarrow\ll \\
	\begin{tikzpicture}[overlay]
		\path[->] (l) edge [out=-60, in=-120]
			node [below] {$1$} (r2);
	\end{tikzpicture}
\end{diagram}}
For a more substantial application, consider the triangle \pref{eq-lla}:
\begin{diagram}[vtriangleheight=3em]
	I\ast(A\ast B) && \rTo^{\aa_{I,A,B}} && (I\ast A)\ast B \\
	& \rdTo_{\ll_{A\ast B}} && \ldTo_{\ll_{A}\ast B} \\
	&& A\ast B
\end{diagram}
The proof of this, as given by \citet{KellyML} for monoidal categories,
runs as follows. Consider the diagram
\begin{mspill}\begin{diagram}[midvshaft,hug]
	\bigl(A,(I,(C,D))\bigr)
		&\rTo^{\aa_{A,I,(C\ast D)}} & (A\ast I)\ast (C\ast D)
		&\rTo^{\aa_{(A\ast I),C,D}} & ((A\ast I)\ast C)\ast D
	\\
	& \rdTo(1,2)^{(A,\ll_{(C,D)})} \ldTo(1,2)_{\rr_{A}\ast (C,D)}
		& \rlap{\qquad$\natural$} &\ldTo(1,2)^{(\rr_{A}\ast C)\ast D}
	\\
		& A\ast (C\ast D) & \rTo_{\aa_{A,C,D}} & (A\ast C)\ast D
	\\
	\dTo<{A\ast\aa_{I,C,D}}>{\qquad?}
		\ruTo(1,2)_{A\ast(\ll_{C}\ast D)}
		&& \natural &&
		\luTo(1,2)_{(A\ast\ll_{C})\ast D}
		\uTo>{\aa_{A,I,C}\ast D}
	\\
	A\ast((I\ast C)\ast D)
		&& \rTo_{\aa_{A,I\ast C,D}}
		&& (A\ast(I\ast C))\ast D
\end{diagram}\end{mspill}
The outside commutes by the pentagon axiom.
The quadrilaterals commute by naturality, and the unmarked triangles
by the triangle axiom. Since $\aa_{A,C,D}$ is invertible, it follows
that the triangle marked `$?$' commutes.
Now set $A = I$, and use the naturality and invertibility of $\ll$ to
conclude that the required triangle commutes.
It's easy to see that all this reasoning is formalisable in the language,
hence
\[
	\semint{\aa_{I, A, B} \cdot \ll_{A\ast B}}
	=
	\semint{\ll_{A}\ast B}
\]
in any model.
By a dual argument,
\[
	\semint{\rr_{A\ast B} \cdot \aa_{A,B,I}} = \semint{A\ast\rr_{B}}
\]
as well. Also, we can show that $\semint{\ll_{I}} = \semint{\rr_{I}}$,
as follows:
\[
	(\ll_{I}\ast A) \cdot \aa_{I,I,A} = \ll_{I\ast A}
\]
by the triangle we proved above, which is equal to $I\ast \ll_{A}$
by the naturality argument at the start of this section, which in
turn is equal to $(\rr_{I}\ast A) \cdot \aa_{I,I,A}$ by the
triangle axiom. Since $\aa_{I,I,A}$ is invertible, we have
that $\ll_{J()}\ast A = \rr_{J()}\ast A$. Then use the invertibility
and naturality of $\rr$ to conclude $\ll_{I} = \rr_{I}$, as
required.

Again, this reasoning may easily be formalised in the language,
and as promised the language allows us to transfer proofs from
the setting of ordinary monoidal categories to that of arbitrary
pseudomonoids.

\section{Braided pseudomonoids}\label{s-braided}
In the context of a braided monoidal bicategory $\B$, we can define
what it means to have a braiding on a pseudomonoid in $\B$. Observe
that it is not possible to define \emph{symmetric} pseudomonoids
in this setting: there is simply no way to express the desired
equation. To define symmetric pseudomonoids in general, one needs
some additional structure on the braiding of the monoidal bicategory:
this additional structure is called a \emph{syllepsis}, and consists
of an invertible modification between the identity transformation
on the tensor pseudofunctor and the transformation with components
\[
	A\tn B \rTo^{s_{A,B}} B\tn A \rTo^{s_{B,A}} A\tn B,
\]
subject to coherence conditions. However, all the general facts
that we need concerning symmetric promonoidal categories are true
more generally in the braided case, hence we have no need to
consider symmetry explicitly in the abstract setting. 
\begin{definition} % braided pseudomonoid
	Let $\C$ be a pseudomonoid in the braided monoidal bicategory $\B$.
	A \defn{braiding} for $\C$ is a 2-cell $\ss$:
	\begin{diagram}
		\C\tn\C &\rTo^{s_{{\C,\C}}}&\C\tn\C\\
		&\rdTo[snake=-1ex](1,2)<{P}
			\raise1ex\hbox{$\begin{array}c\Rightarrow\\[-5pt]\ss\end{array}$}%
			\ldTo[snake=1ex](1,2)>{P}\\
		&\C
	\end{diagram}
	subject to two equations, which (in a Gray monoid) are as follows:
	\begin{equation}\label{eq-sa-left}
		\begin{array}{l}
		\begin{diagram}
			&&\rnode{tl}{\C^{3}} & \lTo^{s_{\C,\C^{2}}} & \rnode{tr}{\C^{3}} \\
			&&\dTo<{P\tn \C} & s_{\C,P} & \dTo>{\C\tn P} \\
			&&\C^{2} & \lTo^{{s_{\C,\C}}} & \C^{2} \\
			&\aa && \raise 1em\hbox{$\ss$} \rdTo(1,2)_{P} \ldTo(1,2)_{P} && \aa \\
			\rnode{bl}{\C^{2}} && \rTo_{P} & \C & \lTo_{P} && \rnode{br}{\C^{2}}
			%
			\nccurve[angleA=180,angleB=90]{->}{tl}{bl}\Bput{\C\tn P}
			\nccurve[angleA=0,angleB=90]{->}{tr}{br}\Aput{P\tn \C}
		\end{diagram}
		\\
		\multicolumn 1r{\quad=\quad
		\begin{diagram}
			&&\rnode{tl}{\C^{3}} & \lTo^{s_{\C,\C^{2}}} & \rnode{tr}{\C^{3}} \\
			&&& \raise 1em\hbox{$S_{\C|\C,\C}$}
				\luTo(1,2)_{\C\tn s_{\C,\C}}
				\ldTo(1,2)_{s_{\C,\C}\tn\C} \\
			&\C\tn\ss && \C^{3} && \ss\tn\C \\
			&&\ldTo(3,2)_{\C\tn P} & \aa & \rdTo(3,2)_{P\tn \C} \\
			\rnode{bl}{\C^{2}} && \rTo_{P} & \C & \lTo_{P} && \rnode{br}{\C^{2}}
			%
			\nccurve[angleA=180,angleB=90]{->}{tl}{bl}\Bput{\C\tn P}
			\nccurve[angleA=0,angleB=90]{->}{tr}{br}\Aput{P\tn \C}
		\end{diagram}}
		\end{array}
	\end{equation}
	and
	\begin{equation}\label{eq-sa-right}
		\begin{array}{l}
		\begin{diagram}
			&&\rnode{tl}{\C^{3}} & \rTo^{s_{\C^{2},\C}} & \rnode{tr}{\C^{3}} \\
			&&\dTo<{P\tn \C} & s_{P,\C} & \dTo>{\C\tn P} \\
			&&\C^{2} & \rTo^{{s_{\C,\C}}} & \C^{2} \\
			&\aa && \raise 1em\hbox{$\ss$} \rdTo(1,2)_{P} \ldTo(1,2)_{P} && \aa \\
			\rnode{bl}{\C^{2}} && \rTo_{P} & \C & \lTo_{P} && \rnode{br}{\C^{2}}
			%
			\nccurve[angleA=180,angleB=90]{->}{tl}{bl}\Bput{\C\tn P}
			\nccurve[angleA=0,angleB=90]{->}{tr}{br}\Aput{P\tn \C}
		\end{diagram}
		\\
		\multicolumn 1r{\quad=\quad
		\begin{diagram}
			&&\rnode{tl}{\C^{3}} & \rTo^{s_{\C^{2},\C}} & \rnode{tr}{\C^{3}} \\
			&&& \raise 1em\hbox{$S_{\C,\C|\C}$}
				\rdTo(1,2)_{\C\tn s_{\C,\C}}
				\ruTo(1,2)_{s_{\C,\C}\tn\C} \\
			&\C\tn\ss && \C^{3} && \ss\tn\C \\
			&&\ldTo(3,2)_{\C\tn P} & \aa & \rdTo(3,2)_{P\tn \C} \\
			\rnode{bl}{\C^{2}} && \rTo_{P} & \C & \lTo_{P} && \rnode{br}{\C^{2}}
			%
			\nccurve[angleA=180,angleB=90]{->}{tl}{bl}\Bput{\C\tn P}
			\nccurve[angleA=0,angleB=90]{->}{tr}{br}\Aput{P\tn \C}
		\end{diagram}}
		\end{array}
	\end{equation}
\end{definition}
\begin{definition}
	A \emph{braided pseudomonoid} is a pseudomonoid
	equipped with a braiding.
\end{definition}
%
Observe that, if $\ss$ is a braiding for $\C$ with respect to the
monoidal bicategory braiding $s$, then the inverse of the right mate of $\ss$,
with respect to $s_{\C,\C}$ and $1_{\C}$,
is a braiding with respect to $s^{*}$. We shall denote this dual braiding
as $\ss^{*}$. (Note that, by Lemma~\chref{Bicats}{lemma-adjeq-twisted},
$\ss^{*}$ is also the right mate of the inverse of $\ss$.)

\section{Another approach to braided pseudomonoids}\label{s-braided-facts}
This section concerns the equation
\begin{equation}\label{eq-lrs}
	\begin{diagram}[s=2.5em,tight]
		&&\I\tn \C \\
		&\ldTo^{J\tn \C} && \luTo^{s_{\C,\I}}\\
		\C^{2} & \hbox to0pt{\hskip 4pt$\ll$\hss} & \dTo>1
			& \hskip-4pt U_{\C|\I}
			& \C\tn \I \\
		&\rdTo_{P} && \ldTo_{1} \\
		&&\C
	\end{diagram}
	\qquad=\qquad
	\begin{diagram}[s=2.5em,tight]
		\rnode{CC}{\C^{2}} & \lTo^{J\tn\C}& \I\tn \C &
			\lTo^{s_{\C,\I}}& \rnode{CI}{\C\tn \I} \\
		&\luTo_{s_{\C,\C}} &s_{\C,J}& \ldTo_{\C\tn J} \\
		&&\C^{2} \\
		&\raise 2em\hbox{$\ss$} &\dTo>P& \raise 2em\hbox{$\rr$} \\
		&&\rnode{C}{\C}
		%
		\ncarc[arcangle=-45]{->}{CC}{C}\Bput{P}
		\ncarc[arcangle=45]{->}{CI}{C}\Aput{1}
	\end{diagram}
\end{equation}
We'll first show that this equation holds in every braided pseudomonoid,
and then we'll show that, in the presence of axioms \pref{eq-aa}, \pref{eq-sa-left}
and \pref{eq-sa-right}, the equations \pref{eq-lla} and \pref{eq-lrs} together
imply \pref{eq-lra}. This gives a useful alternative axiomatisation of braided
pseudomonoids.

In the case of ordinary braided monoidal categories, this equation corresponds
to the triangle:
\begin{diagram}[vtrianglewidth=1em]
	I\tn A && \rTo^{\sigma_{I,A}} && A\tn I \\
	&\rdTo_{\lambda_{A}} && \ldTo_{\rho_{A}} \\
	&&A
\end{diagram}
and the alternative axiomatisation consists of the ordinary pentagon
and hexagon equations together with this triangle and the triangle
\begin{diagram}[vtrianglewidth=1em]
	I\tn(A\tn B) && \rTo^{\alpha} && (I\tn A)\tn B \\
	&\rdTo[snake=-1ex]_{\lambda_{A\tn B}}
		&& \ldTo[snake=1ex]_{\lambda_{A}\tn B} \\
	&&A\tn B
\end{diagram}
The advantage of this axiomatisation, in the general case just
as in the case of ordinary monoidal categories, is that $\rho$ appears
just once; hence it may be eliminated from the data and defined
in terms of $\sigma$ and $\lambda$.

In fact this can be proved quite easily using the braided extension
of the calculus of components that was mentioned briefly in Section~\chref{Language}{s-braided}. Sadly time constraints have made
it impossible to incorporate this improvement in adequate detail, so
instead we give a direct proof using pasting diagrams. This may serve
at least to indicate what a dramatic simplification is made possible
by component-based reasoning.

\begin{remark}
	Notice that, if we can prove that this equation holds of every braided
	pseudomonoid, then in particular it holds of the dual braiding $\ss^{*}$,
	so we have
	\[
		\begin{diagram}[s=2.5em,tight]
			&&\I\tn \C \\
			&\ldTo^{J\tn \C} && \luTo^{s^{*}_{\C,\I}}\\
			\C^{2} & \hbox to0pt{\hskip 4pt$\ll$\hss} & \dTo>1
				& \hskip-4pt U^{*}_{\C|\I}
				& \C\tn \I \\
			&\rdTo_{P} && \ldTo_{1} \\
			&&\C
		\end{diagram}
		\qquad=\qquad
		\begin{diagram}[s=2.5em,tight]
			\rnode{CC}{\C^{2}} & \lTo^{J\tn\C}& \I\tn \C &
				\lTo^{s^{*}_{\C,\I}}& \rnode{CI}{\C\tn \I} \\
			&\luTo_{s^{*}_{\C,\C}} & s^{*}_{\C,J} & \ldTo_{\C\tn J} \\
			&&\C^{2} \\
			&\raise 2em\hbox{$\ss^{*}$} &\dTo>P& \raise 2em\hbox{$\rr$} \\
			&&\rnode{C}{\C}
			%
			\ncarc[arcangle=-45]{->}{CC}{C}\Bput{P}
			\ncarc[arcangle=45]{->}{CI}{C}\Aput{1}
		\end{diagram}
	\]
	Taking mates then gives
	\begin{equation}\label{eq-lrs'}
		\begin{diagram}[s=2.5em,tight]
			&&\I\tn \C \\
			&\ldTo^{J\tn \C} && \rdTo^{s_{\I,\C}}\\
			\C^{2} & \hbox to0pt{\hskip 4pt$\ll$\hss} & \dTo>1
				& \hskip-4pt U_{\I|\C}
				& \C\tn \I \\
			&\rdTo_{P} && \ldTo_{1} \\
			&&\C
		\end{diagram}
		\qquad=\qquad
		\begin{diagram}[s=2.5em,tight]
			\rnode{CC}{\C^{2}} & \lTo^{J\tn\C}& \I\tn \C &
				\rTo^{s_{\I,\C}}& \rnode{CI}{\C\tn \I} \\
			&\rdTo_{s_{\C,\C}} &s_{J,\C}& \ldTo_{\C\tn J} \\
			&&\C^{2} \\
			&\raise 2em\hbox{$\ss$} &\dTo>P& \raise 2em\hbox{$\rr$} \\
			&&\rnode{C}{\C}
			%
			\ncarc[arcangle=-45]{->}{CC}{C}\Bput{P}
			\ncarc[arcangle=45]{->}{CI}{C}\Aput{1}
		\end{diagram}
	\end{equation}
	So a proof of \pref{eq-lrs} will also establish \pref{eq-lrs'}. This is
	an example of how the duality principle can be used to establish facts
	about the braiding $\ss$, not only about the dual braiding $\ss^{*}$.
\end{remark}

These equations generalise the one of \citet[Prop.~2.1, part~1]{BTC},
and indeed the essence of the lengthy argument here is contained in the
two-line sketch proof therein. The proof is originally due to \citet{KellyML}
-- that he considered a symmetry, rather than a braiding, does not affect
the proof.

At the end of the proof, we shall need to appeal to the following lemma.
It corresponds to the fact that, in a monoidal category,
the functors $I\tn-$ and $-\tn I$ are faithful.
\begin{lemma}\label{lemma-faithful}
	Let $A$ be some object of $\B$, let $f$, $g: A\to\C$
	and let $\gamma$, $\delta: f\To g$. If
	\[\hbox{\vrule height 3em depth 2em width 0pt}
	\begin{diagram}
		\rnode{CA}{\C\times A} & \Downarrow{\scriptstyle\C\times\gamma} & \rnode{CC}{\C\times\C} &\rTo^P &\C
		\ncarc{->}{CA}{CC}\Aput{\C\times f}
		\ncarc{<-}{CC}{CA}\Aput{\C\times g}
	\end{diagram}
	\quad=\quad
	\begin{diagram}
		\rnode{CA}{\C\times A} & \Downarrow{\scriptstyle\C\times\delta} & \rnode{CC}{\C\times\C} &\rTo^P &\C
		\ncarc{->}{CA}{CC}\Aput{\C\times f}
		\ncarc{<-}{CC}{CA}\Aput{\C\times g}
	\end{diagram}
	\]
	then $\gamma = \delta$.
	%
	Dually, if
	\[\hbox{\vrule height 3em depth 2em width 0pt}
	\begin{diagram}
		\rnode{AC}{A\times\C} & \Downarrow{\scriptstyle\gamma\times\C} & \rnode{CC}{\C\times\C} &\rTo^P &\C
		\ncarc{->}{AC}{CC}\Aput{f\times\C}
		\ncarc{<-}{CC}{AC}\Aput{g\times\C}
	\end{diagram}
	\quad=\quad
	\begin{diagram}
		\rnode{AC}{A\times\C} & \Downarrow{\scriptstyle\delta\times\C} & \rnode{CC}{\C\times\C} &\rTo^P &\C
		\ncarc{->}{AC}{CC}\Aput{f\times\C}
		\ncarc{<-}{CC}{AC}\Aput{g\times\C}
	\end{diagram}
	\]
	then $\gamma=\delta$.
\end{lemma}
\begin{proof}
	Easy, via the calculus of components.
\end{proof}


\begin{propn}\label{prop-lrs}
	Equation~\pref{eq-lrs} holds in any braided pseudomonoid,
	hence (by the discussion above) so does~\pref{eq-lrs'}.
\end{propn}
\begin{proof}
	\diagramstyle[hug]
	Consider the 2-cell
	\begin{diagram}[hug,s=3em,tight]
		&&\C\tn \I\tn \C && \lTo^{s_{\C,\C\tn \I}} && \C^{2}\tn \I \\
		&\ldTo[nohug,snake=-1ex]^{\C\tn J\tn \C}
			&& \luTo_{\C\tn s_{\C,\I}}
			& S_{\C|\C,\I}
			& \ldTo_{s_{\C,\C}\tn \I} \\
		\C^{3} & \C\tn\ll & \dTo<1 & \hskip-1em\C\tn U_{\C|\I} & \C^{2}\tn \I
			&& \dTo>1 \\
		&\rdTo[nohug,snake=-1ex]_{\C\tn P} &&\ldTo_{1} \\
		&& \C^{2} && \lTo^{s_{\C,\C}} && \C^{2} \\
		&&& \rdTo[nohug]_{P} & \raise 1em\hbox{$\ss$} & \ldTo[nohug]_{P} \\
		&&&& \C
	\end{diagram}
	By Proposition~\chref{MonBicats}{prop-braiding-unit-1}, this is equal to
	\begin{diagram}[nohug,s=2em,tight]
		&&\C\tn \I\tn \C && \lTo^{s_{\C,\C\tn \I}} && \C^{2}\tn \I \\
		&\ldTo[snake=-1ex]^{\C\tn J\tn \C}
			&&
			&
			& \\
		\C^{3} & \C\tn\ll & \dTo>1 &&
			&& \dTo>1 \\
		&\rdTo[snake=-1ex]_{\C\tn P} \\
		&& \C^{2} && \lTo^{s_{\C,\C}} && \C^{2} \\
		&&& \rdTo[nohug]_{P} & \raise 1em\hbox{$\ss$} & \ldTo[nohug]_{P} \\
		&&&& \C
	\end{diagram}
 	which is equal, by equation \pref{eq-lra}, to
	\begin{diagram}[nohug,s=2em,tight]
		&&\C\tn \I\tn \C && \lTo^{s_{\C,\C\tn \I}} && \C^{2}\tn \I \\
		&\ldTo[snake=-1ex]^{\C\tn J\tn \C}
			&&
			&
			& \\
		\C^{3} & \rr\tn\C & \dTo>1 &&
			&& \dTo>1 \\
		&\rdTo[hug]_{P\tn\C} \\
		\dTo<{\C\tn P} && \C^{2} && \lTo^{s_{\C,\C}} && \C^{2} \\
		&\aa && \rdTo[nohug]_{P} & \raise 1em\hbox{$\ss$} & \ldTo[nohug]_{P} \\
		\C^{2} &&\rTo_{P}&& \C
	\end{diagram}
	Since $s$ is pseudo-natural, this is equal to
	\begin{diagram}
		&&\C\tn \I\tn \C & \lTo^{s_{\C,\C\tn \I}} & \C^{2}\tn \I \\
		&\ldTo^{\C\tn J\tn \C}
			& s_{\C,\C\tn J}
			& \ldTo^{\C^{2}\tn J}
			& \\
		\C^{3} & \lTo^{s_{\C,\C^{2}}} & \C^{3} & \C\tn\rr
			& \dTo>1 \\
		&\rdTo_{P\tn\C} & s_{\C,P} & \rdTo_{\C\tn P} \\
		\dTo<{\C\tn P} && \C^{2} & \lTo^{s_{\C,\C}} & \C^{2} \\
		&\aa && \rdTo[nohug](1,2)_{P} \raise 1em\hbox{$\ss$} \ldTo[nohug](1,2)_{P} \\
		\C^{2} &&\rTo_{P}& \C
	\end{diagram}
	which, by equation \pref{eq-rra}, equals
	\begin{diagram}
		&&&&\C\tn \I\tn \C&\lTo^{s_{\C,\C\tn \I}} & \rnode{CCI}{\C^{2}\tn \I} \\
		&&&\ldTo^{\C\tn J\tn \C} & s_{\C,\C\tn J} & \ldTo_{\C^{2}\tn J} \\
		&&\rnode{tl}{\C^{3}} & \lTo^{s_{\C,\C^{2}}} & \rnode{tr}{\C^{3}}
			&& \hbox to0pt{\quad$\sim$\hss} \\
		&&\dTo<{P\tn \C} & s_{\C,P} & \dTo>{\C\tn P} \\
		&&\C^{2} & \lTo^{{s_{\C,\C}}} & \C^{2} &&&& \rnode{CI}{\C\tn \I} \\
		&\aa && \raise 1em\hbox{$\ss$} \rdTo[nohug](1,2)_{P} \ldTo[nohug](1,2)_{P}
			&& \aa && \ldTo_{\C\tn J}\\
		\rnode{bl}{\C^{2}} && \rTo_{P} & \rnode{C}{\C} & \lTo_{P} && \rnode{br}{\C^{2}} \\
		&&&&&&\rr \\
		%
		\nccurve[angleA=180,angleB=90]{->}{tl}{bl}\Bput{\C\tn P}
		\nccurve[angleA=0,angleB=90]{->}{tr}{br}\Aput{P\tn \C}
		\nccurve[angleA=-20,angleB=90]{->}{CCI}{CI}\Aput{P\tn \I}
		\nccurve[angleA=270,angleB=-45,ncurv=1.5]{->}{CI}{C}\Aput{1}
	\end{diagram}
	By \pref{eq-sa-left}, this is equal to
	\begin{diagram}
		&&&&\C\tn \I\tn \C&\lTo^{s_{\C,\C\tn \I}} & \rnode{CCI}{\C^{2}\tn \I} \\
		&&&\ldTo^{\C\tn J\tn \C} & s_{\C,\C\tn J} & \ldTo_{\C^{2}\tn J} \\
		%
		&&\rnode{tl}{\C^{3}} & \lTo^{s_{\C,\C^{2}}} & \rnode{tr}{\C^{3}}
			&& \hbox to0pt{\quad$\sim$\hss} \\
		&&& \raise 1em\hbox{$S_{\C|\C,\C}$}
			\luTo[nohug](1,2)_{\C\tn s_{\C,\C}}
			\ldTo[nohug](1,2)_{s_{\C,\C}\tn\C} \\
		&\C\tn\ss && \C^{3} && \ss\tn\C &&& \rnode{CI}{\C\tn \I} \\
		&&\ldTo(3,2)_{\C\tn P} & \aa & \rdTo(3,2)_{P\tn \C}
				&&& \ldTo_{\C\tn J}\\
		\rnode{bl}{\C^{2}} && \rTo_{P} & \rnode{C}{\C} & \lTo_{P}
			&& \rnode{br}{\C^{2}} \\
		&&&&&&\rr \\
		%
		\nccurve[angleA=180,angleB=90]{->}{tl}{bl}\Bput{\C\tn P}
		\nccurve[angleA=0,angleB=90]{->}{tr}{br}\Aput{P\tn \C}
		\nccurve[angleA=-20,angleB=90]{->}{CCI}{CI}\Aput{P\tn \I}
		\nccurve[angleA=270,angleB=-45,ncurv=1.5]{->}{CI}{C}\Aput{1}
	\end{diagram}
	which, since $\tn$ is a pseudo-functor, is equal to
	\begin{diagram}
		&&&&\C\tn \I\tn \C&\lTo^{s_{\C,\C\tn \I}} & \rnode{CCI}{\C^{2}\tn \I} \\
		&&&\ldTo^{\C\tn J\tn \C} & s_{\C,\C\tn J} & \ldTo_{\C^{2}\tn J}
		 	& \dTo>{s_{\C,\C}\tn\I} \\
		%
		&&\rnode{tl}{\C^{3}} & \lTo^{s_{\C,\C^{2}}} & \rnode{tr}{\C^{3}}
			& \sim & \C^{2}\tn \I & \ss\tn\I \\
		&&& \raise 1em\hbox{$S_{\C|\C,\C}$}
			\luTo[nohug](1,2)_{\C\tn s_{\C,\C}}
			\ldTo(1,2)_{s_{\C,\C}\tn\C}
			&& \ldTo(3,2)_{\C^{2}\tn J}
			&& \rdTo_{P\tn \I} \\
		&\C\tn\ss && \C^{3} &&& \hbox to 0pt{\hss$\sim$\quad} && \rnode{CI}{\C\tn \I} \\
		&&\ldTo(3,2)_{\C\tn P} & \aa & \rdTo(3,2)_{P\tn \C}
				&&& \ldTo_{\C\tn J}\\
		\rnode{bl}{\C^{2}} && \rTo_{P} & \rnode{C}{\C} & \lTo_{P}
			&& \rnode{br}{\C^{2}} \\
		&&&&&&\rr \\
		%
		\nccurve[angleA=180,angleB=90]{->}{tl}{bl}\Bput{\C\tn P}
		\nccurve[angleA=-20,angleB=90]{->}{CCI}{CI}\Aput{P\tn \I}
		\nccurve[angleA=270,angleB=-45,ncurv=1.5]{->}{CI}{C}\Aput{1}
	\end{diagram}
	Since $S$ is a modification, this in turn is equal to
	\begin{diagram}[s=4em,tight]
		&& \rnode{CCC}{\C^{3}} & \lTo^{\C\tn J\tn \C} & \C\tn \I\tn \C
			& \lTo^{s_{\C,\C\tn \I}} & \rnode{CCI}{\C^{2}\tn \I} \\
		&\raise -1em\rlap{$\C\tn\ss$} & \uTo[snake=-1.5em]>{\C\tn s_{\C,\C}}
			& \C\tn s_{J,\C}
			& \uTo[snake=1.5em]<{\C\tn s_{\C,\I}}
			& \raise1em\llap{$S_{\C|\C,\I}$}\ldTo_{s_{\C,\C}\tn\I} \\
		\rnode{CC}{\C^{2}} & \lTo_{\C\tn P} & \C^{3} & \lTo_{\C^{2}\tn J}
			& \C^{2}\tn \I & \ss\tn\I \\
		\dTo<{P} & \aa & \dTo<{P\tn \C} & \sim & \dTo<{P\tn \I} \\
		\rnode{C}{\C} & \lTo_{P} & \C^{2} & \lTo_{\C\tn J} & \rnode{CI}{\C\tn \I} \\
		&&\raise1.6em\hbox{$\rr$}
		%
		\ncarc[arcangle=80,ncurv=.7]{->}{CI}{C} \Aput{1}
		\ncarc[arcangle=-30]{->}{CCC}{CC} \Bput{\C\tn P}
		\ncarc[arcangle=45]{->}{CCI}{CI} \Aput{P\tn \I}
	\end{diagram}
	which by \pref{eq-rra} is equal to
	\begin{diagram}[s=4em,tight]
		&& \rnode{CCC}{\C^{3}} & \lTo^{\C\tn J\tn \C} & \C\tn \I\tn \C
			& \lTo^{s_{\C,\C\tn \I}} & \rnode{CCI}{\C^{2}\tn \I} \\
		&\raise -1em\rlap{$\C\tn\ss$} & \uTo[snake=-1em]>{\C\tn s_{\C,\C}} & \C\tn s_{J,\C}
			& \uTo[snake=1em]<{\C\tn s_{\C,\I}}
			& \raise1em\llap{$S_{\C|\C,\I}$}\ldTo_{s_{\C,\C}\tn\I} \\
		\rnode{CC}{\C^{2}} & \lTo_{\C\tn P} & \C^{3} & \lTo_{\C^{2}\tn J}
			& \rnode{mid}{\C^{2}\tn \I} & \ss\tn\I \\
		\dTo<{P} & & \raise2em\hbox{$\C\tn\rr$} & & \dTo>{P\tn \I} \\
		\rnode{C}{\C} & & \lTo_{1} & & \rnode{CI}{\C\tn \I} \\
		%
		\ncarc[arcangle=80,ncurv=.7,offset=-3pt]{->}{mid}{CC} \aput(0.517){1}
		\ncarc[arcangle=-30]{->}{CCC}{CC} \Bput{\C\tn P}
		\ncarc[arcangle=45]{->}{CCI}{CI} \Aput{P\tn \I}
	\end{diagram}
	If we compare this with the diagram we began with, and
	cancel the invertible 2-cells $S_{\C|\C,\I}$ and $\ss=\ss\tn\I$,
	we have that
	\[
	\begin{diagram}[s=3.2em,tight]
		&&\C\tn \I\tn \C && \C^{2}\tn \I \\
		&\ldTo^{\C\tn J\tn \C}
			&& \luTo^{\C\tn s_{\C,\I}}
			& \dTo_{s_{\C,\C}\tn \I} \\
		\C^{3} & \C\tn\ll & \dTo>1 & \hbox to 3em{\hss$\C\tn U_{\C|\I}$}
			& \C^{2}\tn \I \\
		&\rdTo_{\C\tn P} &&\ldTo[nohug]_{1} \\
		&& \C^{2} \\
		&& \dTo[nohug]_{P} \\
		&& \C
	\end{diagram}
	\qquad=\qquad
	\begin{diagram}
		&&\C\tn\I\tn \C && \C^{2}\tn \I\\
		&\ldTo^{\C\tn J\tn \C} && \luTo^{\C\tn s_{\C,\I}} & \dTo>{s_{\C,\C}\tn\I} \\
		\rnode{CCC}{\C^{3}} && \C\tn s_{\C,J} && \rnode{CCI}{\C^{2}\tn \I} \\
		&\luTo_{\C\tn s_{\C,\C}} && \ldTo_{\C^{2}\tn J} \\
		&&\C^{3} \\
		&\raise 2em\hbox{$\C\tn\ss$} &\dTo>{\C\tn P}& \raise 2em\hbox{$\C\tn\rr$} \\
		&&\rnode{CC}{\C^{2}} \\
		&&\dTo>P \\
		&&\C
		%
		\ncarc[arcangle=-45]{->}{CCC}{CC}\Bput{\C\tn P}
		\ncarc[arcangle=45]{->}{CCI}{CC}\Aput{\C\tn P}
	\end{diagram}
	\]
	and since $s_{\C,\C}$ is an equivalence, it follows that
	\[
	\begin{diagram}[s=3.2em,tight]
		&&\C\tn \I\tn \C \\
		&\ldTo^{\C\tn J\tn \C}
			&& \luTo^{\C\tn s_{\C,\I}} \\
		\C^{3} & \C\tn\ll & \dTo>1 & \hbox to 3em{\hss$\C\tn U_{\C|\I}$}
			& \C^{2}\tn \I \\
		&\rdTo_{\C\tn P} &&\ldTo[nohug]_{1} \\
		&& \C^{2} \\
		&& \dTo[nohug]_{P} \\
		&& \C
	\end{diagram}
	\qquad=\qquad
	\begin{diagram}
		\rnode{CCC}{\C^{3}} &\lTo^{\C\tn J\tn \C}& \C\tn s_{\C,J}
			& \lTo^{\C\tn s_{\C,\I}} & \rnode{CCI}{\C^{2}\tn \I} \\
		&\luTo_{\C\tn s_{\C,\C}} & \raise1ex\hbox{$\C\tn s_{\C,J}$}
			& \ldTo_{\C^{2}\tn J} \\
		&&\C^{3} \\
		&\raise 2em\hbox{$\C\tn\ss$} &\dTo>{\C\tn P}& \raise 2em\hbox{$\C\tn\rr$} \\
		&&\rnode{CC}{\C^{2}} \\
		&&\dTo>P \\
		&&\C
		%
		\ncarc[arcangle=-45]{->}{CCC}{CC}\Bput{\C\tn P}
		\ncarc[arcangle=45]{->}{CCI}{CC}\Aput{\C\tn P}
	\end{diagram}
	\]
	Now Lemma~\ref{lemma-faithful} yields the claim.
\end{proof}

\begin{propn}
	Let the object $\C$, the 1-cells $P$ and $J$,
	and the 2-cells $\aa$, $\ll$, $\rr$ and $\ss$ be given, as in
	the definition of braided pseudomonoid. Suppose that equations \pref{eq-aa},
	\pref{eq-sa-left}, \pref{eq-sa-right} and \pref{eq-lla} are satisfied.
	If~\pref{eq-lrs} or~\pref{eq-lrs'}
	is satisfied then the structure is indeed a braided pseudomonoid.
	
	It follows that a braided pseudomonoid may be defined using just
	the 2-cells $\aa$, $\ll$, and $\ss$, subject to equations
	\pref{eq-aa}, \pref{eq-sa-left}, \pref{eq-sa-right} and~\pref{eq-lla}.
	The 2-cell $\rr$ can be defined, if necessary, using equation~\pref{eq-lrs}
	or~\pref{eq-lrs'}.
\end{propn}
\begin{proof}
	We shall assume equations~\pref{eq-sa-left}, \pref{eq-lla} and~\pref{eq-lrs},
	and derive~\pref{eq-lra}.
	%
	Consider the 2-cell
	\begin{diagram}
		&&\I\tn \C^{2} & \lTo^{s_{\C,\I\tn \C}} & \rnode{CIC}{\C\tn \I\tn \C} \\
		&&\dTo<{J\tn \C^{2}} & s_{\C,J\tn\C} & \dTo>{\C\tn J\tn \C} \\
		%
		&&\rnode{tl}{\C^{3}} & \lTo^{s_{\C,\C^{2}}} & \rnode{tr}{\C^{3}}
			&& \hbox to 2em{\hss$\rr\tn\C$} \\
		&&\dTo<{P\tn \C} & s_{\C,P} & \dTo>{\C\tn P} \\
		&&\C^{2} & \lTo^{{s_{\C,\C}}} & \C^{2} \\
		&\aa && \raise 1em\hbox{$\ss$} \rdTo[nohug](1,2)_{P} \ldTo[nohug](1,2)_{P}
			&& \aa\\
		\rnode{bl}{\C^{2}} && \rTo_{P} & \rnode{C}{\C} & \lTo_{P} && \rnode{br}{\C^{2}} \\
		%
		\nccurve[angleA=180,angleB=90]{->}{tl}{bl}\Bput{\C\tn P}
		\nccurve[angleA=0,angleB=90]{->}{tr}{br}\Aput{P\tn \C}
		\nccurve[angleA=0,angleB=45,ncurv=1]{->}{CIC}{br}\Aput{1}
	\end{diagram}
	By equation~\pref{eq-sa-left}, this is equal to
	\begin{diagram}
		&&\I\tn \C^{2} & \lTo^{s_{\C,\I\tn \C}} & \rnode{CIC}{\C\tn \I\tn \C} \\
		&&\dTo<{J\tn \C^{2}} & s_{\C,J\tn\C} & \dTo>{\C\tn J\tn \C} \\
		%
		&&\rnode{tl}{\C^{3}} & \lTo^{s_{\C,\C^{2}}} & \rnode{tr}{\C^{3}}
			&& \hbox to 2em{\hss$\rr\tn\C$} \\
		&&& \raise 1em\hbox{$S_{\C|\C,\C}$}
			\luTo[nohug](1,2)_{\C\tn s_{\C,\C}}
			\ldTo[nohug](1,2)_{s_{\C,\C}\tn\C} \\
		&\C\tn\ss && \C^{3} && \ss\tn\C \\
		&&\ldTo(3,2)_{\C\tn P} & \aa & \rdTo(3,2)_{P\tn \C} \\
		\rnode{bl}{\C^{2}} && \rTo_{P} & \rnode{C}{\C} & \lTo_{P}
			&& \rnode{br}{\C^{2}} \\
		%
		\nccurve[angleA=180,angleB=90]{->}{tl}{bl}\Bput{\C\tn P}
		\nccurve[angleA=0,angleB=90]{->}{tr}{br}\Aput{P\tn \C}
		\nccurve[angleA=0,angleB=45,ncurv=1]{->}{CIC}{br}\Aput{1}
	\end{diagram}
	which, since $S$ is a modification, is equal to
	\begin{diagram}[hug]
		\I\tn \C^{2} && \lTo^{s_{\C,\I\tn \C}} && \rnode{CIC}{\C\tn \I\tn \C} \\
		\dTo<{J\tn \C^{2}}&\luTo_{\I\times s_{\C,\C}}
			& \raise 1em\hbox{$S_{\C|\I,\C}$}
			& \ldTo_{s_{\C,\I}\tn\C}
			& \dTo>{\C\tn J\tn \C}\\
		\C^{3} & \sim & \I\tn \C^{2} & s_{\C,J}\tn\C & \C^{2}\\
		&\luTo_{\C\tn s_{\C,\C}} & \dTo[snake=1ex]>{J\tn \C^{2}} & \ldTo_{s_{\C,\C}\tn\C}
			&&\rr\tn\C \\
		\dTo<{\C\tn P} & \C\tn\ss & \C^{3} & \ss\tn\C & \dTo>{P\tn\C} \\
		&\ldTo_{\C\tn P} &\aa& \rdTo_{P\tn \C} \\
		\C^{2} &\rTo_{P}& \C& \lTo_{P}& \rnode{CC}{\C^{2}} \\
		%
		\nccurve[angleA=0,angleB=0]{->}{CIC}{CC} \Aput{1}
	\end{diagram}
	By equation~\pref{eq-lrs}, this is
	\begin{diagram}[hug,midvshaft]
		\I\tn \C^{2} &&  \raise-1em\hbox to0pt{\hskip5em$S_{\C|\I,\C}$\hss} \lTo^{s_{\C,\I\tn \C}}
			&& \C\tn \I\tn \C \\
		&\luTo(2,1)_{\I\tn s_{\C,\C}} & \I\tn \C^{2} & \ldTo(2,1)_{s_{\C,\I}\tn\C} 
			\raise -1em\hbox to 0pt{\hss\quad$U_{\C|\I}\tn\C$\hss} \\
		\dTo<{J\tn \C^{2}} & \sim & \dTo<{J\tn \C^{2}} & \raise-1em\llap{$\ll\tn\C$}
			\rdTo[nohug]^{1} & \dTo>{1} \\
		\C^{3} & \lTo^{\C\tn s_{\C,\C}} & \C^{3} & \rTo_{P\tn \C} & \C^{2} \\
		& \rdTo[nohug]_{\C\tn P} \raise1em\rlap{$\C\tn\ss$} & \dTo>{\C\tn P} & \aa & \dTo>P \\
		&& \C^{2} & \rTo_{P} & \C
	\end{diagram}
	which, by \pref{eq-lla}, is equal to
	\begin{diagram}[hug,midvshaft]
		&&\I\tn \C^{2} &&  \raise-1em\hbox to0pt{\hskip3.5em$S_{\C|\I,\C}$\hss} \lTo^{s_{\C,\I\tn \C}}
			&& \C\tn \I\tn \C \\
		&\ldTo(2,3)<{J\tn \C^{2}} & &\luTo(2,1)_{\I\tn s_{\C,\C}} & \I\tn \C^{2} & \ldTo(2,1)_{s_{\C,\I}\tn\C} 
			\raise -1em\hbox to 0pt{\hss\quad$U_{\C|\I}\tn\C$\hss} \\
		&& \sim & \ldTo<{J\tn \C^{2}} & \dTo[snake=-1ex]<{\I\tn P}
			& \rdTo[nohug]_{1} & \dTo>{1} \\
		\C^{3} & \lTo^{\C\tn s_{\C,\C}} & \C^{3} & \sim & \I\tn \C & & \C^{2} \\
		& \rdTo[nohug]_{\C\tn P} \raise1em\rlap{$\C\tn\ss$} & \dTo>{\C\tn P}
			& \ldTo_{J\tn \C} & \raise -1em\hbox{$\ll$} & \rdTo[nohug]_{1} & \dTo>P \\
		&& \C^{2} && \rTo_{P} && \C
	\end{diagram}
	which, since the $\sim$ cells are natural, is equal to
	\begin{diagram}[hug,midvshaft]
		&&\rnode{ICC}{\I\tn \C^{2}} &&  \raise-1em\hbox to0pt{\hskip3.5em$S_{\C|\I,\C}$\hss} \lTo^{s_{\C,\I\tn \C}}
			&& \C\tn \I\tn \C \\
		&\ldTo(2,3)<{J\tn \C^{2}} & &\luTo(2,1)_{\I\tn s_{\C,\C}}
			\raise -1.5em\hbox{$\I\tn\ss$}& \I\tn \C^{2} & \ldTo(2,1)_{s_{\C,\I}\tn\C} 
			\raise -1em\hbox to 0pt{\hss\quad$U_{\C|\I}\tn\C$\hss} \\
		& & & & \dTo[snake=-1ex]<{\I\tn P}
			& \rdTo[nohug]_{1} & \dTo>{1} \\
		\C^{3} & & \sim &  & \rnode{IC}{\I\tn \C} & & \C^{2} \\
		& \rdTo[nohug]_{\C\tn P} &
			& \ldTo_{J\tn \C} & \raise -1em\hbox{$\ll$} & \rdTo[nohug]_{1} & \dTo>P \\
		&& \C^{2} && \rTo_{P} && \C
		%
		\ncarc[arcangle=-40,ncurv=.6]{->}{ICC}{IC} \Bput{\I\tn P}
	\end{diagram}
	By~\pref{eq-lla}, this is equal to
	\begin{diagram}[hug,midvshaft]
		&&\rnode{ICC}{\I\tn \C^{2}} &&  \raise-1em\hbox to0pt{\hskip3.5em$S_{\C|\I,\C}$\hss} \lTo^{s_{\C,\I\tn \C}}
			&& \C\tn \I\tn \C \\
		& & &\luTo(2,1)_{\I\tn s_{\C,\C}}
			\raise -1.5em\hbox{$\I\tn\ss$}& \I\tn \C^{2} & \ldTo(2,1)_{s_{\C,\I}\tn\C} 
			\raise -1em\hbox to 0pt{\hss\quad$U_{\C|\I}\tn\C$\hss} \\
		& & & & \dTo[snake=-1ex]<{\I\tn P}
			& \rdTo[nohug]_{1} & \dTo>{1} \\
		& \ll\tn\C & \dTo<1 &  & \rnode{IC}{\I\tn \C} & & \C^{2} \\
		& & & & & \rdTo[nohug]_{1} & \dTo>P \\
		\rnode{CCC}{\C^{3}} & \rTo^{P\tn \C} & \C^{2} && \rTo_{P} && \C \\
		& \rdTo[nohug](3,2)_{\C\tn P} && \aa && \ruTo[nohug](3,2)_{P} \\
		&&& \C^{2}
		%
		\ncarc[arcangle=-40,ncurv=.6,offsetA=3pt]{->}{ICC}{IC} \bput{-60}(.65){\I\tn P}
		\ncarc[arcangle=-30]{->}{ICC}{CCC} \Bput{J\tn \C^{2}}
	\end{diagram}
	which equals
	\begin{diagram}
		&&& \raise -1em\hbox{$$} &\rnode{CIC}{\C\tn \I\tn \C} \\
		\rnode{ICC}{\I\tn \C^{2}} & \lTo^{\I\tn s_{\C,\C}} & \I\tn \C^{2}
			& \ldTo[hug,snake=-6pt](2,1)^{s_{\C,\I}\tn\C}
			\raise -1em\hbox to2em{$U_{\C|\I}\tn\C$\hss} & \dTo[midvshaft]>1 \\
		\dTo<{J\tn \C^{2}} & \raise-1em\llap{$\ll\tn\C$} \rdTo^{1} && \rdTo_{1} \\
		\C^{3} & \rTo_{P\tn \C} & \C^{2} & \lTo^{s_{\C,\C}} & \C^{2} \\
		\dTo<{\C\tn P} & \rlap{$\aa$} && \rdTo_{P} \raise.8em\rlap{\hskip1.5em$\ss$}
			& \dTo>P \\
		\C^{2} && \rTo_{P} && \C
		%
		\ncarc[arcangle=-30]{->}{CIC}{ICC} \Bput{s_{\C,\I\tn \C}}
		 	\Aput{\raise-4pt\hbox{$S_{\C|\I,\C}$}}
	\end{diagram}
	By one of the unit axioms in the definition of braiding for
	a monoidal bicategory, this is
	\begin{diagram}
		\rnode{ICC}{\I\tn \C^{2}} & & \lTo^{s_{\C,\I\tn\C}}
			& & \C\tn \I\tn \C \\
		\dTo<{J\tn \C^{2}} & \raise-1em\llap{$\ll\tn\C$} \rdTo^{1} &&& \dTo>1 \\
		\C^{3} & \rTo_{P\tn \C} & \C^{2} & \lTo^{s_{\C,\C}} & \C^{2} \\
		\dTo<{\C\tn P} & \rlap{$\aa$} && \rdTo_{P} \raise.8em\rlap{\hskip1.5em$\ss$}
			& \dTo>P \\
		\C^{2} && \rTo_{P} && \C
	\end{diagram}
	which, since $s$ is pseudo-natural, is equal to
	\begin{diagram}
		\I\tn \C^{2} &&\lTo^{s_{\C,\I\tn \C}} && \C\tn \I\tn \C \\
		& s_{\C,J\tn \C} && \ldTo[hug]^{\C\tn J\tn \C} \\
		\dTo<{J\tn \C^{2}} && \C^{3} & \rlap{$\C\tn\ll$} & \dTo>1 \\
		&\ldTo[hug]^{s_{\C,\C^{2}}} & s_{\C,P} & \rdTo[hug]^{\C\tn P} \\
		\C^{3} & \rTo_{P\tn \C} & \C^{2} & \lTo^{s_{\C,\C}} & \C^{2} \\
		\dTo<{\C\tn P} & \rlap{$\aa$} && \rdTo_{P} \raise.8em\rlap{\hskip1.5em$\ss$}
			& \dTo>P \\
		\C^{2} && \rTo_{P} && \C
	\end{diagram}
	If we now compare this with the 2-cell we began with, and cancel the common
	invertible 2-cell
	\begin{diagram}
		&&\I\tn \C^{2} & \lTo^{s_{\C,\I\tn \C}} & \rnode{CIC}{\C\tn \I\tn \C} \\
		&&\dTo<{J\tn \C^{2}} & s_{\C,J\tn\C} & \dTo>{\C\tn J\tn \C} \\
		%
		&&\rnode{tl}{\C^{3}} & \lTo^{s_{\C,\C^{2}}} & \rnode{tr}{\C^{3}}\\
		&&\dTo<{P\tn \C} & s_{\C,P} & \dTo>{\C\tn P} \\
		&&\C^{2} & \lTo^{{s_{\C,\C}}} & \C^{2} \\
		&\aa && \raise 1em\hbox{$\ss$} \rdTo[nohug](1,2)_{P} \ldTo[nohug](1,2)_{P} \\
		\rnode{bl}{\C^{2}} && \rTo_{P} & \rnode{C}{\C} \\
		%
		\nccurve[angleA=180,angleB=90]{->}{tl}{bl}\Bput{\C\tn P}
	\end{diagram}
	we obtain equation~\pref{eq-lra}, as claimed.
\end{proof}

\end{thesischapter}