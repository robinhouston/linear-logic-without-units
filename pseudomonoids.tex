%!TEX TS-program = latex
\documentclass{robinthesisdraft}
\usepackage{robincs,thesisdefs,xr}
\externaldocument[Bicats:]{bicats}
\externaldocument[MonBicats:]{mon-bicats}

\title{Pseudomonoids}
\begin{document}
\maketitle

A monoid, in a monoidal category, consists of an object $A$ equipped
with a unit $u: I\to A$ and a multiplication $m: A\tn A\to A$, satisfying
the obvious unit and associativity axioms. In a monoidal \emph{bicategory},
the corresponding notion is that of a pseudomonoid, where the unit and
associativity laws hold, not on the nose, but up to coherent isomorphism.
The primeval example is of course that of monoidal categories, which are
pseudomonoids in the monoidal bicategory $\Cat$; though for present
purposes we are particularly interested in promonoidal categories,
which are pseudomonoids in $\Prof$.

Though the notion of pseudomonoid is fairly well known (at least in
certain circles) no published account goes
much further than stating the definition \citep{MonBicat}.
So without further ado, here is the definition.

\begin{definition} % definition of pseudomonoid
	A pseudomonoid $\C$ in a monoidal bicategory $\B$ is a normal
	pseudofunctor $1\to\B$. More concretely, it consists of an object $\C$,
	1-cells
	\[\begin{array}l
		J: I\to\C,\\
		P:\C\tn\C\to\C,
	\end{array}\]
	and 2-cells
	\[\begin{array}{c@{\qquad}c}
		\multicolumn2c{\begin{diagram}
			\C\tn(\C\tn\C) && \rTo^{a_{\C,\C,\C}} && (\C\tn\C)\tn\C
			\\
			\dTo<{\C\tn P}
			&& \begin{array}c\To\\[-4pt]\aa\end{array}
			&& \dTo>{P\tn\C}
			\\
			\C\tn\C & \rTo_{P} & \C & \lTo_{P} & \C\tn\C
		\end{diagram}}
		\\[6em]
		\begin{diagram}
			I\tn\C &\rTo^{J\tn\C}&\C\tn\C\\
			&\rdTo[snake=-1ex](1,2)<{l_\C}
				\raise1ex\hbox{$\begin{array}c\Rightarrow\\[-5pt]\ll\end{array}$}%
				\ldTo[snake=1ex](1,2)>{P}\\
			&\C
		\end{diagram}
		&
		\begin{diagram}
			\C\tn I &\rTo^{\C\tn J}&\C\tn\C\\
			&\rdTo[snake=-1ex](1,2)<{r_\C}
				\raise1ex\hbox{$\begin{array}c\Rightarrow\\[-5pt]\rr\end{array}$}%
				\ldTo[snake=1ex](1,2)>{P}\\
			&\C
		\end{diagram}
	\end{array}\]
	subject to the two equations below (stated in the Gray monoid setting).
	Since the 2-cells are assumed to be invertible, we shall permit ourselves
	to omit the arrow. Also, here and elsewhere, we write $\C^{2}$ for $\C\tn\C$.
	\begin{equation}\label{eq-lra}
	\begin{diagram}
		\C\tn\I\tn\C\\
		\dTo<1 &\hbox to0pt{\hss$\C\tn\ll$}\rdTo(2,1)^{\C\tn J\tn\C} & \C\tn\C\tn\C\\
		\C\tn\C & \ldTo(2,1)_{\C\tn P} &\dTo>{P\tn\C}\\
		\dTo<P & \raise1.5em\hbox{$\aa$}& \C\tn\C\\
		\C&\ldTo(2,1)_P
	\end{diagram}
	\qquad=\qquad
	\begin{diagram}
		\C\tn\I\tn\C\\
		\dTo<1 &\hbox to0pt{\hss$\rr\tn\C$}\rdTo(2,1)^{\C\tn J\tn\C} & \C\tn\C\tn\C\\
		\C\tn\C & \ldTo(2,1)_{P\tn\C}\\
		\dTo<P\\
		\C
	\end{diagram}
	\end{equation}
	and
	\begin{equation}\label{eq-aa}
		\begin{diagram}[s=2.2em,labelstyle=\scriptstyle,tight]
			&&\C^3\\
			&\ruTo^{\C^2\tn P}&\dTo[snake=-5pt]<{P\tn\C}&\rdTo^{\C\tn P}\\
			\C^4 &\sim& \C^2 &\mathop{\Leftarrow}\limits_{\;\;\;\aa}& \C^2\\
			\dTo<{P\tn\C^2}&\ruTo_{\C\tn P} && \rdTo_P & \dTo>{P}\\
			\C^3 && \Arr\Downarrow\aa && \C\\
			&\rdTo_{P\tn\C}&&\ruTo>{P}\\
			&&\C^2
		\end{diagram}
		\qquad=\qquad
		\begin{diagram}[s=2.2em,labelstyle=\scriptstyle,tight]
			&&\C^3\\
			&\ruTo^{\C^2\tn P}&&\rdTo^{\C\tn P}\\
			\C^4 && \Arr\Downarrow{\C\tn \aa} && \C^2\\
			\dTo<{P\tn\C^2}&\rdTo^{\C\tn P\tn\C} && \ruTo^{\C\tn P} & \dTo>{P}\\
			\C^3 &\mathop{\Leftarrow}\limits_{\;\;\;\aa\tn\C}& \C^3 &\Arr\Swarrow{\scriptstyle\!\!\!\aa}& \C\\
			&\rdTo_{P\tn\C}&\dTo[snake=5pt]>{\!\!P\tn\C}&\ruTo>{P}\\
			&&\C^2
		\end{diagram}
	\end{equation}
	Note that equation (\ref{eq-lra}) is equivalent to
	\[
	\begin{diagram}
		\C\tn\I\tn\C\\
		\dTo<1 &\hbox to0pt{\hss$\C\tn\ll$}\rdTo(2,1)^{\C\tn J\tn\C} & \C\tn\C\tn\C\\
		\C\tn\C & \ldTo(2,1)_{\C\tn P}\\
		\dTo<P\\
		\C
	\end{diagram}
	\qquad=\qquad
	\begin{diagram}
		\C\tn\I\tn\C\\
		\dTo<1 &\hbox to0pt{\hss$\rr\tn\C$}\rdTo(2,1)^{\C\tn J\tn\C} & \C\tn\C\tn\C\\
		\C\tn\C & \ldTo(2,1)_{P\tn\C} &\dTo>{\C\tn P}\\
		\dTo<P & \raise1.5em\hbox{$\aa$}& \C\tn\C\\
		\C&\ldTo(2,1)_P
	\end{diagram}
	\]
	by composing with the inverse of $\aa$. We shall often use it in this form, without
	further remark. Similar variations of other equations may also be used without
	drawing attention to the fact.
\end{definition}
%
The first of these equations corresponds to Mac Lane's pentagon
axiom, and the second corresponds to the triangle axiom relating
$\alpha$, $\lambda$ and $\rho$.

\section{Some facts about pseudomonoids}
This section is essentially a translation of \cite{KellyML} into the language
of general pseudomonoids. In particular, we shall show that the following
three equations hold of any pseudomonoid $\C$.
\begin{equation}\label{eq-lla}
\begin{diagram}
	\I\tn\C^2 & \rTo^{J\tn\C^2} &\C^3& \rTo^{\C\tn P} & \C^2\\
	&\rdTo(2,2)<1\raise1em\hbox to0pt{$\ll\tn\C$\hss} & \dTo[snake=.5em]>{P\tn\C} &\aa&\dTo>P\\
	&&\C^2 &\rTo_P &\C
\end{diagram}
\quad=\quad
\begin{diagram}
	\I\tn\C^2 &\rTo^{J\tn\C^2} & \C^3\\
	\dTo<{\I\tn P} &\sim & \dTo>{\C\tn P}\\
	\I\tn\C &\rTo^{J\tn\C} & \C^2\\
	&\rdTo(2,2)_1\raise0.5em\hbox to0pt{\hskip0.5em$\ll$\hss} &\dTo>P\\
	&&\C
\end{diagram}
\end{equation}
\begin{equation}\label{eq-rra}
\begin{diagram}
	\C^2\tn\I & \rTo^{\C^2\tn J} &\C^3& \rTo^{P\tn\C} & \C^2\\
	&\rdTo(2,2)<1\raise1em\hbox to0pt{$\C\tn\rr$\hss} & \dTo[snake=.5em]>{\C\tn P} &\aa&\dTo>P\\
	&&\C^2 &\rTo_P &\C
\end{diagram}
\quad=\quad
\begin{diagram}
	\C^2\tn\I &\rTo^{\C^2\tn J} & \C^3\\
	\dTo<{P\tn\I} &\sim & \dTo>{P\tn\C}\\
	\C\tn\I &\rTo^{\C\tn J} & \C^2\\
	&\rdTo(2,2)_1\raise0.5em\hbox to0pt{\hskip0.5em$\rr$\hss} &\dTo>P\\
	&&\C
\end{diagram}
\end{equation}
\begin{equation}\label{eq-lr}
\begin{diagram}
	\I\tn\I\\
	\dTo<{J\tn\I} & \rdTo(2,1)^{\I\tn J} & \I\tn\C\\
	\C\tn\I &\raise1em\hbox{$\sim$}&\dTo>{J\tn\C}\\
	\dTo<1 & \hbox to0pt{\hss$\rr$\hskip0.5em}\rdTo(2,1)^{\C\tn J}& \C\tn\C\\
	\C&\ldTo(2,1)_P
\end{diagram}
\quad=\quad
\begin{diagram}
	\I\tn\I\\
	\dTo<{J\tn\I} & \rdTo(2,1)^{\I\tn J} & \I\tn\C\\
	\C\tn\I &\ldTo(2,3)^1&\dTo>{J\tn\C}\\
	\dTo<1 & \raise1em\hbox to0pt{\hskip1em$\ll$\hss}& \C\tn\C\\
	\C&\ldTo(2,1)_P
\end{diagram}
\end{equation}
%
% In fact, it turns out that only equation (\ref{eq-lla}) is needed to prove the embedding theorem.
% For the sake of completeness, however, we shall prove all three.
%
\citet[section~3.4]{LackThesis} describes an interesting geometrical way to
prove these equations, using certain four-dimensional diagrams.\footnote{
	Lack is working in the slightly more general
	context of enriched bicategories: a pseudomonoid is an enriched bicategory
	with one object.}
Here we give a more down-to-earth version of the argument, where we confine
our diagrams to the two dimensions of the page. The reader is thereby spared
the trouble of attempting to visualise a four-dimensional hypercube, at the
expense of having to follow sometimes lengthy sequences of equations between
two-dimensional diagrams.

The following lemma will be very useful. It corresponds to the fact that, in a monoidal category,
the functors $I\tn-$ and $-\tn I$ are faithful.
\begin{lemma}\label{lemma-faithful}
	Let $A$ be some object of $\B$, let $f$, $g: A\to\C$
	and let $\gamma$, $\delta: f\To g$. If
	\[\hbox{\vrule height 3em depth 2em width 0pt}
	\begin{diagram}
		\rnode{CA}{\C\tn A} & \Downarrow{\scriptstyle\C\tn\gamma} & \rnode{CC}{\C\tn\C} &\rTo^P &\C
		\ncarc{->}{CA}{CC}\Aput{\C\tn f}
		\ncarc{<-}{CC}{CA}\Aput{\C\tn g}
	\end{diagram}
	\quad=\quad
	\begin{diagram}
		\rnode{CA}{\C\tn A} & \Downarrow{\scriptstyle\C\tn\delta} & \rnode{CC}{\C\tn\C} &\rTo^P &\C
		\ncarc{->}{CA}{CC}\Aput{\C\tn f}
		\ncarc{<-}{CC}{CA}\Aput{\C\tn g}
	\end{diagram}
	\]
	then $\gamma = \delta$.
	%
	Dually, if
	\[\hbox{\vrule height 3em depth 2em width 0pt}
	\begin{diagram}
		\rnode{AC}{A\tn\C} & \Downarrow{\scriptstyle\gamma\tn\C} & \rnode{CC}{\C\tn\C} &\rTo^P &\C
		\ncarc{->}{AC}{CC}\Aput{f\tn\C}
		\ncarc{<-}{CC}{AC}\Aput{g\tn\C}
	\end{diagram}
	\quad=\quad
	\begin{diagram}
		\rnode{AC}{A\tn\C} & \Downarrow{\scriptstyle\delta\tn\C} & \rnode{CC}{\C\tn\C} &\rTo^P &\C
		\ncarc{->}{AC}{CC}\Aput{f\tn\C}
		\ncarc{<-}{CC}{AC}\Aput{g\tn\C}
	\end{diagram}
	\]
	then $\gamma=\delta$.
\end{lemma}
\begin{proof}
	For the first part, suppose
	\[\hbox{\vrule height 3em depth 2em width 0pt}
	\begin{diagram}
		\rnode{CA}{\C\tn A} & \Downarrow{\scriptstyle\C\tn\gamma} & \rnode{CC}{\C\tn\C} &\rTo^P &\C
		\ncarc{->}{CA}{CC}\Aput{\C\tn f}
		\ncarc{<-}{CC}{CA}\Aput{\C\tn g}
	\end{diagram}
	\quad=\quad
	\begin{diagram}
		\rnode{CA}{\C\tn A} & \Downarrow{\scriptstyle\C\tn\delta} & \rnode{CC}{\C\tn\C} &\rTo^P &\C.
		\ncarc{->}{CA}{CC}\Aput{\C\tn f}
		\ncarc{<-}{CC}{CA}\Aput{\C\tn g}
	\end{diagram}
	\]
	Then
	\[\begin{array}{rcl}
		\gamma &=&
			\begin{diagram}
			&&\rnode{top}{\I\tn\C}\\
			&&\sim\\
			\rnode{IA}{\I\tn A} & \rTo^{J\tn A} & \rnode{CA}{\C\tn A}
				& \Downarrow{\scriptstyle\C\tn\gamma} & \rnode{CC}{\C\tn\C}
				\raise2em\hbox to0pt{\hskip.5em$\ll$\hss}
				\raise-2em\hbox to0pt{\hskip.5em$\ll$\hss}
				&\rTo^P & \rnode{C}{\C}\\
			&&\sim\\
			&&\rnode{bot}{\I\tn\C}
			\ncarc{->}{CA}{CC}\Aput{\C\tn f}
			\ncarc{<-}{CC}{CA}\Aput{\C\tn g}
			\ncarc{->}{IA}{top}\Aput{\I\tn f}
			\ncarc{->}{top}{CC}\aput{-53}(0.6){J\tn\C}
			\ncarc{->}{top}{C}\Aput{1}
			\ncarc{<-}{C}{bot}\Aput{1}
			\ncarc{<-}{CC}{bot}\aput{53}(0.4){J\tn\C}
			\ncarc{<-}{bot}{IA}\Aput{\I\tn g}
			\end{diagram}
		\\[7em]
		&=&
			\begin{diagram}
			&&\rnode{top}{\I\tn\C}\\
			&&\sim\\
			\rnode{IA}{\I\tn A} & \rTo^{J\tn A} & \rnode{CA}{\C\tn A}
				& \Downarrow{\scriptstyle\C\tn\delta} & \rnode{CC}{\C\tn\C}
				\raise2em\hbox to0pt{\hskip.5em$\ll$\hss}
				\raise-2em\hbox to0pt{\hskip.5em$\ll$\hss}
				&\rTo^P & \rnode{C}{\C}\\
			&&\sim\\
			&&\rnode{bot}{\I\tn\C}
			\ncarc{->}{CA}{CC}\Aput{\C\tn f}
			\ncarc{<-}{CC}{CA}\Aput{\C\tn g}
			\ncarc{->}{IA}{top}\Aput{\I\tn f}
			\ncarc{->}{top}{CC}\aput{-53}(0.6){J\tn\C}
			\ncarc{->}{top}{C}\Aput{1}
			\ncarc{<-}{C}{bot}\Aput{1}
			\ncarc{<-}{CC}{bot}\aput{53}(0.4){J\tn\C}
			\ncarc{<-}{bot}{IA}\Aput{\I\tn g}
			\end{diagram}
		\\
		&=&\delta.
	\end{array}\]
	The second part is proved similarly.
\end{proof}
%
\begin{propn} % Equations (\ref{eq-lla}) and (\ref{eq-rra}) hold
	Equations (\ref{eq-lla}) and (\ref{eq-rra}) hold, of any pseudomonoid $\C$.
\end{propn}
\begin{proof}
	Consider the 2-cell
	\begin{diagram}
	\C\tn\I\tn\C^2 & \rTo^{\C\tn J\tn\C^2} &\C^4& \rTo^{\C^2\tn P} & \C^3\\
	&\rdTo(2,2)<1\raise1em\hbox to0pt{$\C\tn\ll\tn\C$\hss}
		& \dTo[snake=.8em]>{\C\tn P\tn\C} &\C\tn\aa&\dTo>{\C\tn P}\\
	&&\C^3 &\rTo_{\C\tn P} &\C^2\\
	&&\dTo<{P\tn\C} & \aa & \dTo>P\\
	&&\C^2 & \rTo_P & \C
	\end{diagram}
	%
	By equation (\ref{eq-lra}), this is equal to
	\begin{diagram}
	\C\tn\I\tn\C^2 & \rTo^{\C\tn J\tn\C^2} &\C^4& \rTo^{\C^2\tn P} & \C^3\\
	\dTo<1&\raise1em\hbox to0pt{\hss$\rr\tn\C^2$}\ldTo[hug]_{P\tn\C^2}
		& \dTo[snake=.8em]>{\C\tn P\tn\C} &\C\tn\aa&\dTo>{\C\tn P}\\
	\C^3&\aa\tn\C&\C^3 &\rTo_{\C\tn P} &\C^2\\
	&\rdTo_{P\tn\C}&\dTo<{P\tn\C} & \aa & \dTo>P\\
	&&\C^2 & \rTo_P & \C
	\end{diagram}
	%
	which, by (\ref{eq-aa}), is equal to
	\begin{diagram}[tight,w=3.5em]
	\C\tn\I\tn\C^2 & \rTo^{\C\tn J\tn\C^2} &\C^4& \rTo^{\C^2\tn P} & \C^3\\
	\dTo<1 &\ldTo(2,2)_{P\tn\C^2}\raise1em\hbox to0pt{\hss$\rr\tn\C^2$}&\sim
		&\ldTo^{P\tn\C} &\dTo>{\C\tn P}\\
	\C^3 &\rTo_{\C\tn P} & \C^2 &\aa &\C^2\\
	&\rdTo_{P\tn\C} &\aa&\rdTo_P&\dTo>P\\
	&&\C^2 & \rTo_P & \C
	\end{diagram}
	%
	Pasting on a structural 2-cell and its inverse gives
	\begin{diagram}[tight,w=4em]
	&&\C^4\\
	&\ruTo^{\C\tn J\tn\C^2} &\sim&\rdTo^{\C^2\tn P}\\
	\C\tn\I\tn\C^2&\rTo^{\C\tn\I\tn P} &\C\tn\I\tn\C &\rTo_{\C\tn J\tn\C} &\rnode{x}{\C^3}\\
	&\rdTo[snake=1em]^{\C\tn J\tn\C^2} &&\sim\\
	\dTo<1 &\rr&\C^4& \rTo^{\C^2\tn P} & \rnode{y}{\C^3}\\
	&\ldTo(2,2)_{P\tn\C^2}&\sim
		&\ldTo^{P\tn\C} &\dTo>{\C\tn P}\\
	\C^3 &\rTo_{\C\tn P} & \C^2 &\aa &\C^2\\
	&\rdTo_{P\tn\C} &\aa&\rdTo_P&\dTo>P\\
	&&\C^2 & \rTo_P & \C
	\ncline[doubleline=true,doublesep=2pt]{-}xy
	\end{diagram}
	which, since the structural 2-cells behave naturally, is equal to
	\begin{diagram}[tight,w=4em]
	&&\C^4\\
	&\ruTo^{\C\tn J\tn\C^2} &\sim&\rdTo^{\C^2\tn P}\\
	\C\tn\I\tn\C^2&\rTo^{\C\tn\I\tn P} &\C\tn\I\tn\C &\rTo_{\C\tn J\tn\C} &\rnode{x}{\C^3}\\
	&&&\raise-1em\hbox{$\rr\tn\C$}\\
	\dTo<1 &&\dTo<1&& \rnode{y}{\C^3}\\
	&&&\ldTo^{P\tn\C} &\dTo>{\C\tn P}\\
	\C^3 &\rTo_{\C\tn P} & \C^2 &\aa &\C^2\\
	&\rdTo_{P\tn\C} &\aa&\rdTo_P&\dTo>P\\
	&&\C^2 & \rTo_P & \C\hbox{\hskip1pt.\hss}
	\ncline[doubleline=true,doublesep=2pt]{-}xy
	\end{diagram}
	%
	This diagram can be redrawn as
	\begin{diagram}
	&&\rnode{CICC}{\C\tn\I\tn\C^2} &\rTo^{\C\tn J\tn\C^2} &\C^4\\
	&&\dTo<{\C\tn\I\tn P} &\sim& \dTo>{\C^2\tn P}\\
	&&\C\tn\I\tn\C &\rTo^{\C\tn J\tn\C} &\rnode{x}{\C^3}\\
	&&\dTo<1&\rr\tn\C\\
	\rnode{CCC}{\C^3}&\rTo^{\C\tn P}&\C^2 &\lTo^{P\tn\C} &\rnode{y}{\C^3}\\
	\dTo<{P\tn\C} &\aa & \dTo>P & \aa & \dTo>{\C\tn P}\\
	\C^2 &\rTo_P & \C &\lTo_P & \C^2
	\ncline[doubleline=true,doublesep=2pt]{-}xy
	\nccurve[angleA=180,angleB=90]{->}{CICC}{CCC}\Bput{1}
	\end{diagram}
	%
	which by equation (\ref{eq-lra}) is equal to
	\begin{diagram}
	&&\rnode{CICC}{\C\tn\I\tn\C^2} &\rTo^{\C\tn J\tn\C^2} &\C^4\\
	&&\dTo<{\C\tn\I\tn P} &\sim& \dTo>{\C^2\tn P}\\
	&&\C\tn\I\tn\C &\rTo^{\C\tn J\tn\C} &\rnode{x}{\C^3}\\
	&&\dTo<1&\C\tn\ll\\
	\rnode{CCC}{\C^3}&\rTo^{\C\tn P}&\C^2 &\lTo^{\C\tn P} &\rnode{y}{\C^3}\\
	\dTo<{P\tn\C} &\aa & \dTo>P\\
	\C^2 &\rTo_P & \C\hbox to0pt{\hskip1pt.\hss}
	\ncline[doubleline=true,doublesep=2pt]{-}xy
	\nccurve[angleA=180,angleB=90]{->}{CICC}{CCC}\Bput{1}
	\end{diagram}
	%
	Comparing this with the diagram we started with, and cancelling $\aa$, we have
	\[
	\begin{diagram}
	\C\tn\I\tn\C^2 & \rTo^{\C\tn J\tn\C^2} &\C^4& \rTo^{\C^2\tn P} & \C^3\\
	&\rdTo(2,2)<1\raise1em\hbox to0pt{$\C\tn\ll\tn\C$\hss}
		& \dTo[snake=.8em]>{\C\tn P\tn\C} &\C\tn\aa&\dTo>{\C\tn P}\\
	&&\C^3 &\rTo_{\C\tn P} &\C^2\\
	&&&& \dTo>P\\
	&&&& \C
	\end{diagram}
	\quad=\quad
	\begin{diagram}
	\C\tn\I\tn\C^2 &\rTo^{\C\tn J\tn\C^2} &\C^4\\
	\dTo<{\C\tn\I\tn P} &\sim& \dTo>{\C^2\tn P}\\
	\C\tn\I\tn\C &\rTo^{\C\tn J\tn\C} &\rnode{x}{\C^3}\\
	\dTo<1&\C\tn\ll\\
	\C^2 &\lTo^{\C\tn P} &\rnode{y}{\C^3}\\
	\dTo>P\\
	\C
	\ncline[doubleline=true,doublesep=2pt]{-}xy
	\end{diagram}
	\]
	whence Lemma~\ref{lemma-faithful} allows us to deduce equation (\ref{eq-lla}), as required.
	Equation (\ref{eq-rra}) follows by symmetry.
\end{proof}
%
The next lemma is the general form of the fact that $I\tn\lambda_A = \lambda_{I\tn A}$
in a monoidal category.
\begin{lemma}\label{lemma-lambda}
The following equation holds of any pseudomonoid $\C$:
\begin{equation}\label{eq-lambda}
\hskip-2em
\begin{diagram}
	\I\tn\I\tn\C & \rTo^{J\tn\I\tn\C} & \rnode{CIC}{\C\tn\I\tn\C}\\
	&&\dTo>{\C\tn J\tn\C}\\
	&\hbox to0pt{\hskip1em$\C\tn\ll$\hss}&\C^3\\
	&&\dTo>{\C\tn P}\\
	&&\rnode{CC}{\C^2}\\
	&&\dTo>P\\
	&&\C
	\ncarc[arcangle=-70]{->}{CIC}{CC}\Bput{1}
\end{diagram}
=
\begin{diagram}
	\I\tn\I\tn\C & \rTo^{J\tn\I\tn\C} & \C\tn\I\tn\C\\
	\dTo<{\I\tn J\tn\C} & \sim & \dTo>{\hbox to0pt{$\C\tn J\tn\C$\hss}}\\
	\I\tn\C^2 & \rTo_{J\tn\C^2}& \C^3\\
	\dTo<{\I\tn P} & \sim & \dTo>{\C\tn P}\\
	\I\tn\C & \rTo^{J\tn \C} & \C^2\\
	&\rdTo(2,2)_1\raise1em\hbox to 0pt{$\ll$\hss} & \dTo>P\\
	&&\C
\end{diagram}
\end{equation}
\end{lemma}
\begin{proof}
	Since the structural 2-cells behave naturally,
	\begin{diagram}
		\I\tn\I\tn\C & \rTo^{J\tn\I\tn\C} & \rnode{CIC}{\C\tn\I\tn\C}\\
		&&\dTo>{\C\tn J\tn\C}\\
		\dTo<1&\hbox to0pt{\hskip1em$\C\tn\ll$\hss}&\C^3\\
		&&\dTo>{\C\tn P}\\
		\I\tn\C&\rTo^{J\tn\C}&\rnode{CC}{\C^2}\\
		&\rdTo(2,2)_1\raise1em\hbox to0pt{\hskip1em$\ll$\hss}&\dTo>P\\
		&&\C
		\ncarc[arcangle=-70]{->}{CIC}{CC}\Bput{1}
	\end{diagram}
	is equal to
	\begin{diagram}
		&\rnode{IIC}{\I\tn\I\tn\C} & \rTo^{J\tn\I\tn\C} & \C\tn\I\tn\C\\
		&\dTo<{\I\tn J\tn\C} & \sim & \dTo>{\C\tn J\tn\C}\\
		\ll&\rnode{IC}{\I\tn\C^2} & \rTo_{J\tn\C^2}& \C^3\\
		&\dTo<{\I\tn P} & \sim & \dTo>{\C\tn P}\\
		&\rnode{IC}{\I\tn\C} & \rTo^{J\tn \C} & \C^2\\
		&&\rdTo(2,2)_1\raise1em\hbox to 0pt{\hskip1em$\ll$\hss} & \dTo>P\\
		&&&\C
		\ncarc[arcangle=-70,ncurv=1]{->}{IIC}{IC}\Bput{1}
	\end{diagram}
	Vertically composing with the inverse of $\ll$ gives the claimed equation.
\end{proof}
%
\begin{propn}
	Equation (\ref{eq-lr}) holds of any pseudomonoid $\C$.
\end{propn}
\begin{proof}
	\[\begin{array}{rcl}
	\begin{diagram}
		&\I\tn\I\tn\C \\
		&\dTo<{J\tn\I\tn\C} \\
		&\rnode{CIC}{\C\tn\I\tn\C}\\
		&\dTo>{\C\tn J\tn\C} \\
		\rr\tn\C & \C^3 &\rTo^{\C\tn P} & \C^2 \\
		&\dTo>{P\tn\C} &\aa& \dTo>P \\
		&\rnode{CC}{\C^2} &\rTo_{P} & \C
		\ncarc[arcangle=-70,ncurv=1]{->}{CIC}{CC}\Bput{1}
	\end{diagram}
	&\quad=\quad&
	\begin{diagram}
		&\I\tn\I\tn\C \\
		&\dTo<{J\tn\I\tn\C} \\
		&\rnode{CIC}{\C\tn\I\tn\C}\\
		&\dTo>{\C\tn J\tn\C} \\
		\C\tn\ll & \C^3 \\
		&\dTo>{\C\tn P} \\
		&\rnode{CC}{\C^2} &\rTo_{P} & \C
		\ncarc[arcangle=-70,ncurv=1]{->}{CIC}{CC}\Bput{1}
	\end{diagram}
	\\[11em]
	&=&
	\begin{diagram}
		\I\tn\I\tn\C & \rTo^{J\tn\I\tn\C} & \C\tn\I\tn\C\\
		\dTo<{\I\tn J\tn\C} & \sim & \dTo>{\C\tn J\tn\C}\\
		\I\tn\C^2 & \rTo_{J\tn\C^2}& \C^3\\
		\dTo<{\I\tn P} & \sim & \dTo>{\C\tn P}\\
		\I\tn\C & \rTo^{J\tn \C} & \C^2\\
		&\rdTo(2,2)_1\raise1em\hbox to 0pt{$\ll$\hss} & \dTo>P\\
		&&\C
	\end{diagram}
	\\[10em]
	&=&
	\begin{diagram}
		\I\tn\I\tn\C & \rTo^{J\tn\I\tn\C} & \C\tn\I\tn\C\\
		\dTo<{\I\tn J\tn\C} & \sim & \dTo>{\C\tn J\tn\C} \\
		\I\tn\C^2 & \rTo^{J\tn\C^2} & \rnode{x}{\C^3}\\
		\dTo<1 & \ll\tn\C \\
		\C^2 & \lTo^{P\tn\C} & \rnode{y}{\C^3} \\
		\dTo<P & \aa & \dTo>{\C\tn P} \\
		\C & \lTo_P & \C^2
		\ncline[doubleline=true,doublesep=2pt]{-}xy
	\end{diagram}
	\end{array}\]
	These equalities hold by, respectively, equations (\ref{eq-lra}), (\ref{eq-lambda}), and (\ref{eq-lla}).
	Now vertically compose with the inverse of $\aa$, and apply Lemma~\ref{lemma-faithful}
	to deduce the claim.
\end{proof}

\section{Braided pseudomonoids}
\begin{definition} % braided pseudomonoid
	Let $\C$ be a pseudomonoid in the braided monoidal bicategory $\B$.
	A \defn{braiding} for $\C$ is a 2-cell $\ss$:
	\begin{diagram}
		\C\tn\C &\rTo^{s_{{\C,\C}}}&\C\tn\C\\
		&\rdTo[snake=-1ex](1,2)<{P}
			\raise1ex\hbox{$\begin{array}c\Rightarrow\\[-5pt]\ss\end{array}$}%
			\ldTo[snake=1ex](1,2)>{P}\\
		&\C
	\end{diagram}
	subject to two equations, which (in a Gray monoid) are as follows:
	\begin{equation}\label{eq-sa-left}
		\begin{array}{l}
		\begin{diagram}
			&&\rnode{tl}{\C^{3}} & \lTo^{s_{\C,\C^{2}}} & \rnode{tr}{\C^{3}} \\
			&&\dTo<{P\tn \C} & s_{\C,P} & \dTo>{\C\tn P} \\
			&&\C^{2} & \lTo^{{s_{\C,\C}}} & \C^{2} \\
			&\aa && \raise 1em\hbox{$\ss$} \rdTo(1,2)_{P} \ldTo(1,2)_{P} && \aa \\
			\rnode{bl}{\C^{2}} && \rTo_{P} & \C & \lTo_{P} && \rnode{br}{\C^{2}}
			%
			\nccurve[angleA=180,angleB=90]{->}{tl}{bl}\Bput{\C\tn P}
			\nccurve[angleA=0,angleB=90]{->}{tr}{br}\Aput{P\tn \C}
		\end{diagram}
		\\
		\multicolumn 1r{\quad=\quad
		\begin{diagram}
			&&\rnode{tl}{\C^{3}} & \lTo^{s_{\C,\C^{2}}} & \rnode{tr}{\C^{3}} \\
			&&& \raise 1em\hbox{$S_{\C|\C,\C}$}
				\luTo(1,2)_{\C\tn s_{\C,\C}}
				\ldTo(1,2)_{s_{\C,\C}\tn\C} \\
			&\C\tn\ss && \C^{3} && \ss\tn\C \\
			&&\ldTo(3,2)_{\C\tn P} & \aa & \rdTo(3,2)_{P\tn \C} \\
			\rnode{bl}{\C^{2}} && \rTo_{P} & \C & \lTo_{P} && \rnode{br}{\C^{2}}
			%
			\nccurve[angleA=180,angleB=90]{->}{tl}{bl}\Bput{\C\tn P}
			\nccurve[angleA=0,angleB=90]{->}{tr}{br}\Aput{P\tn \C}
		\end{diagram}}
		\end{array}
	\end{equation}
	and
	\begin{equation}\label{eq-sa-right}
		\begin{array}{l}
		\begin{diagram}
			&&\rnode{tl}{\C^{3}} & \rTo^{s_{\C^{2},\C}} & \rnode{tr}{\C^{3}} \\
			&&\dTo<{P\tn \C} & s_{P,\C} & \dTo>{\C\tn P} \\
			&&\C^{2} & \rTo^{{s_{\C,\C}}} & \C^{2} \\
			&\aa && \raise 1em\hbox{$\ss$} \rdTo(1,2)_{P} \ldTo(1,2)_{P} && \aa \\
			\rnode{bl}{\C^{2}} && \rTo_{P} & \C & \lTo_{P} && \rnode{br}{\C^{2}}
			%
			\nccurve[angleA=180,angleB=90]{->}{tl}{bl}\Bput{\C\tn P}
			\nccurve[angleA=0,angleB=90]{->}{tr}{br}\Aput{P\tn \C}
		\end{diagram}
		\\
		\multicolumn 1r{\quad=\quad
		\begin{diagram}
			&&\rnode{tl}{\C^{3}} & \rTo^{s_{\C^{2},\C}} & \rnode{tr}{\C^{3}} \\
			&&& \raise 1em\hbox{$S_{\C,\C|\C}$}
				\rdTo(1,2)_{\C\tn s_{\C,\C}}
				\ruTo(1,2)_{s_{\C,\C}\tn\C} \\
			&\C\tn\ss && \C^{3} && \ss\tn\C \\
			&&\ldTo(3,2)_{\C\tn P} & \aa & \rdTo(3,2)_{P\tn \C} \\
			\rnode{bl}{\C^{2}} && \rTo_{P} & \C & \lTo_{P} && \rnode{br}{\C^{2}}
			%
			\nccurve[angleA=180,angleB=90]{->}{tl}{bl}\Bput{\C\tn P}
			\nccurve[angleA=0,angleB=90]{->}{tr}{br}\Aput{P\tn \C}
		\end{diagram}}
		\end{array}
	\end{equation}
\end{definition}
%
Observe that, if $\ss$ is a braiding for $\C$ with respect to the
monoidal bicategory braiding $s$, then the inverse of the right mate of $\ss$,
with respect to $s_{\C,\C}$ and $1_{\C}$,
is a braiding with respect to $s^{*}$. We shall denote this dual braiding
as $\ss^{*}$. (Note that, by Lemma~\chref{Bicats}{lemma-adjeq-twisted},
$\ss^{*}$ is also the right mate of the inverse of $\ss$.)

\section{Facts about braided pseudomonoids}\label{s-braided-facts}
This section concerns the equation
\begin{equation}\label{eq-lrs}
	\begin{diagram}[s=2.5em,tight]
		&&\I\tn \C \\
		&\ldTo^{J\tn \C} && \luTo^{s_{\C,\I}}\\
		\C^{2} & \hbox to0pt{\hskip 4pt$\ll$\hss} & \dTo>1
			& \hskip-4pt U_{\C|\I}
			& \C\tn \I \\
		&\rdTo_{P} && \ldTo_{1} \\
		&&\C
	\end{diagram}
	\qquad=\qquad
	\begin{diagram}[s=2.5em,tight]
		\rnode{CC}{\C^{2}} & \lTo^{J\tn\C}& \I\tn \C &
			\lTo^{s_{\C,\I}}& \rnode{CI}{\C\tn \I} \\
		&\luTo_{s_{\C,\C}} &s_{\C,J}& \ldTo_{\C\tn J} \\
		&&\C^{2} \\
		&\raise 2em\hbox{$\ss$} &\dTo>P& \raise 2em\hbox{$\rr$} \\
		&&\rnode{C}{\C}
		%
		\ncarc[arcangle=-45]{->}{CC}{C}\Bput{P}
		\ncarc[arcangle=45]{->}{CI}{C}\Aput{1}
	\end{diagram}
\end{equation}
We'll first show that this equation holds in every braided pseudomonoid,
and then we'll show that, in the presence of axioms \pref{eq-aa}, \pref{eq-sa-left}
and \pref{eq-sa-right}, the equations \pref{eq-lla} and \pref{eq-lrs} together
imply \pref{eq-lra}. This gives a useful alternative axiomatisation of braided
pseudomonoids.

\begin{remark}
	Notice that, if we can prove that this equation holds of every braided
	pseudomonoid, then in particular it holds of the dual braiding $\ss^{*}$,
	so we have
	\[
		\begin{diagram}[s=2.5em,tight]
			&&\I\tn \C \\
			&\ldTo^{J\tn \C} && \luTo^{s^{*}_{\C,\I}}\\
			\C^{2} & \hbox to0pt{\hskip 4pt$\ll$\hss} & \dTo>1
				& \hskip-4pt U^{*}_{\C|\I}
				& \C\tn \I \\
			&\rdTo_{P} && \ldTo_{1} \\
			&&\C
		\end{diagram}
		\qquad=\qquad
		\begin{diagram}[s=2.5em,tight]
			\rnode{CC}{\C^{2}} & \lTo^{J\tn\C}& \I\tn \C &
				\lTo^{s^{*}_{\C,\I}}& \rnode{CI}{\C\tn \I} \\
			&\luTo_{s^{*}_{\C,\C}} & s^{*}_{\C,J} & \ldTo_{\C\tn J} \\
			&&\C^{2} \\
			&\raise 2em\hbox{$\ss^{*}$} &\dTo>P& \raise 2em\hbox{$\rr$} \\
			&&\rnode{C}{\C}
			%
			\ncarc[arcangle=-45]{->}{CC}{C}\Bput{P}
			\ncarc[arcangle=45]{->}{CI}{C}\Aput{1}
		\end{diagram}
	\]
	Taking mates then gives
	\begin{equation}\label{eq-lrs'}
		\begin{diagram}[s=2.5em,tight]
			&&\I\tn \C \\
			&\ldTo^{J\tn \C} && \rdTo^{s_{\I,\C}}\\
			\C^{2} & \hbox to0pt{\hskip 4pt$\ll$\hss} & \dTo>1
				& \hskip-4pt U_{\I|\C}
				& \C\tn \I \\
			&\rdTo_{P} && \ldTo_{1} \\
			&&\C
		\end{diagram}
		\qquad=\qquad
		\begin{diagram}[s=2.5em,tight]
			\rnode{CC}{\C^{2}} & \lTo^{J\tn\C}& \I\tn \C &
				\rTo^{s_{\I,\C}}& \rnode{CI}{\C\tn \I} \\
			&\rdTo_{s_{\C,\C}} &s_{J,\C}& \ldTo_{\C\tn J} \\
			&&\C^{2} \\
			&\raise 2em\hbox{$\ss$} &\dTo>P& \raise 2em\hbox{$\rr$} \\
			&&\rnode{C}{\C}
			%
			\ncarc[arcangle=-45]{->}{CC}{C}\Bput{P}
			\ncarc[arcangle=45]{->}{CI}{C}\Aput{1}
		\end{diagram}
	\end{equation}
	So a proof of \pref{eq-lrs} will also establish \pref{eq-lrs'}. This is
	an example of how the duality principle can be used to establish facts
	about the braiding $\ss$, not only about the dual braiding $\ss^{*}$.
\end{remark}

These equations generalise the one of \citet[Prop.~2.1, part~1]{BTC},
and indeed the essence of the lengthy argument here is contained in the
two-line sketch proof therein. The proof is originally due to \citet{KellyML}
-- that he considered a symmetry, rather than a braiding, does not affect
the proof.
\begin{propn}\label{prop-lrs}
	Equation~\pref{eq-lrs} holds in any braided pseudomonoid,
	hence (by the discussion above) so does~\pref{eq-lrs'}.
\end{propn}
\begin{proof}
	\diagramstyle[hug]
	Consider the 2-cell
	\begin{diagram}
		&&\C\tn \I\tn \C && \lTo^{s_{\C,\C\tn \I}} && \C^{2}\tn \I \\
		&\ldTo^{\C\tn J\tn \C}
			&& \luTo_{\C\tn s_{\C,\I}}
			& S_{\C|\C,\I}
			& \ldTo_{s_{\C,\C}\tn \I} \\
		\C^{3} & \C\tn\ll & \dTo>1 & \C\tn U_{\C|\I} & \C^{2}\tn \I
			&& \dTo>1 \\
		&\rdTo_{\C\tn P} &&\ldTo_{1} \\
		&& \C^{2} && \lTo^{s_{\C,\C}} && \C^{2} \\
		&&& \rdTo[nohug]_{P} & \raise 1em\hbox{$\ss$} & \ldTo[nohug]_{P} \\
		&&&& \C
	\end{diagram}
	By Proposition~\chref{MonBicats}{prop-braiding-unit}, this is equal to
	\begin{diagram}
		&&\C\tn \I\tn \C && \lTo^{s_{\C,\C\tn \I}} && \C^{2}\tn \I \\
		&\ldTo^{\C\tn J\tn \C}
			&&
			&
			& \\
		\C^{3} & \C\tn\ll & \dTo>1 &&
			&& \dTo>1 \\
		&\rdTo_{\C\tn P} \\
		&& \C^{2} && \lTo^{s_{\C,\C}} && \C^{2} \\
		&&& \rdTo[nohug]_{P} & \raise 1em\hbox{$\ss$} & \ldTo[nohug]_{P} \\
		&&&& \C
	\end{diagram}
 	which is equal, by equation \pref{eq-lra}, to
	\begin{diagram}
		&&\C\tn \I\tn \C && \lTo^{s_{\C,\C\tn \I}} && \C^{2}\tn \I \\
		&\ldTo^{\C\tn J\tn \C}
			&&
			&
			& \\
		\C^{3} & \rr\tn\C & \dTo>1 &&
			&& \dTo>1 \\
		&\rdTo_{P\tn\C} \\
		\dTo<{\C\tn P} && \C^{2} && \lTo^{s_{\C,\C}} && \C^{2} \\
		&\aa && \rdTo[nohug]_{P} & \raise 1em\hbox{$\ss$} & \ldTo[nohug]_{P} \\
		\C^{2} &&\rTo_{P}&& \C
	\end{diagram}
	Since $s$ is pseudo-natural, this is equal to
	\begin{diagram}
		&&\C\tn \I\tn \C & \lTo^{s_{\C,\C\tn \I}} & \C^{2}\tn \I \\
		&\ldTo^{\C\tn J\tn \C}
			& s_{\C,\C\tn J}
			& \ldTo^{\C^{2}\tn J}
			& \\
		\C^{3} & \lTo^{s_{\C,\C^{2}}} & \C^{3} & \C\tn\rr
			& \dTo>1 \\
		&\rdTo_{P\tn\C} & s_{\C,P} & \rdTo_{\C\tn P} \\
		\dTo<{\C\tn P} && \C^{2} & \lTo^{s_{\C,\C}} & \C^{2} \\
		&\aa && \rdTo[nohug](1,2)_{P} \raise 1em\hbox{$\ss$} \ldTo[nohug](1,2)_{P} \\
		\C^{2} &&\rTo_{P}& \C
	\end{diagram}
	which, by equation \pref{eq-rra}, equals
	\begin{diagram}
		&&&&\C\tn \I\tn \C&\lTo^{s_{\C,\C\tn \I}} & \rnode{CCI}{\C^{2}\tn \I} \\
		&&&\ldTo^{\C\tn J\tn \C} & s_{\C,\C\tn J} & \ldTo_{\C^{2}\tn J} \\
		&&\rnode{tl}{\C^{3}} & \lTo^{s_{\C,\C^{2}}} & \rnode{tr}{\C^{3}}
			&& \hbox to0pt{\quad$\sim$\hss} \\
		&&\dTo<{P\tn \C} & s_{\C,P} & \dTo>{\C\tn P} \\
		&&\C^{2} & \lTo^{{s_{\C,\C}}} & \C^{2} &&&& \rnode{CI}{\C\tn \I} \\
		&\aa && \raise 1em\hbox{$\ss$} \rdTo[nohug](1,2)_{P} \ldTo[nohug](1,2)_{P}
			&& \aa && \ldTo_{\C\tn J}\\
		\rnode{bl}{\C^{2}} && \rTo_{P} & \rnode{C}{\C} & \lTo_{P} && \rnode{br}{\C^{2}} \\
		&&&&&&\rr \\
		%
		\nccurve[angleA=180,angleB=90]{->}{tl}{bl}\Bput{\C\tn P}
		\nccurve[angleA=0,angleB=90]{->}{tr}{br}\Aput{P\tn \C}
		\nccurve[angleA=-20,angleB=90]{->}{CCI}{CI}\Aput{P\tn \I}
		\nccurve[angleA=270,angleB=-45,ncurv=1.5]{->}{CI}{C}\Aput{1}
	\end{diagram}
	By \pref{eq-sa-left}, this is equal to
	\begin{diagram}
		&&&&\C\tn \I\tn \C&\lTo^{s_{\C,\C\tn \I}} & \rnode{CCI}{\C^{2}\tn \I} \\
		&&&\ldTo^{\C\tn J\tn \C} & s_{\C,\C\tn J} & \ldTo_{\C^{2}\tn J} \\
		%
		&&\rnode{tl}{\C^{3}} & \lTo^{s_{\C,\C^{2}}} & \rnode{tr}{\C^{3}}
			&& \hbox to0pt{\quad$\sim$\hss} \\
		&&& \raise 1em\hbox{$S_{\C|\C,\C}$}
			\luTo[nohug](1,2)_{\C\tn s_{\C,\C}}
			\ldTo[nohug](1,2)_{s_{\C,\C}\tn\C} \\
		&\C\tn\ss && \C^{3} && \ss\tn\C &&& \rnode{CI}{\C\tn \I} \\
		&&\ldTo(3,2)_{\C\tn P} & \aa & \rdTo(3,2)_{P\tn \C}
				&&& \ldTo_{\C\tn J}\\
		\rnode{bl}{\C^{2}} && \rTo_{P} & \rnode{C}{\C} & \lTo_{P}
			&& \rnode{br}{\C^{2}} \\
		&&&&&&\rr \\
		%
		\nccurve[angleA=180,angleB=90]{->}{tl}{bl}\Bput{\C\tn P}
		\nccurve[angleA=0,angleB=90]{->}{tr}{br}\Aput{P\tn \C}
		\nccurve[angleA=-20,angleB=90]{->}{CCI}{CI}\Aput{P\tn \I}
		\nccurve[angleA=270,angleB=-45,ncurv=1.5]{->}{CI}{C}\Aput{1}
	\end{diagram}
	which, since $\tn$ is a pseudo-functor, is equal to
	\begin{diagram}
		&&&&\C\tn \I\tn \C&\lTo^{s_{\C,\C\tn \I}} & \rnode{CCI}{\C^{2}\tn \I} \\
		&&&\ldTo^{\C\tn J\tn \C} & s_{\C,\C\tn J} & \ldTo_{\C^{2}\tn J}
		 	& \dTo>{s_{\C,\C}\tn\I} \\
		%
		&&\rnode{tl}{\C^{3}} & \lTo^{s_{\C,\C^{2}}} & \rnode{tr}{\C^{3}}
			& \sim & \C^{2}\tn \I & \ss\tn\I \\
		&&& \raise 1em\hbox{$S_{\C|\C,\C}$}
			\luTo[nohug](1,2)_{\C\tn s_{\C,\C}}
			\ldTo(1,2)_{s_{\C,\C}\tn\C}
			&& \ldTo(3,2)_{\C^{2}\tn J}
			&& \rdTo_{P\tn \I} \\
		&\C\tn\ss && \C^{3} &&& \hbox to 0pt{\hss$\sim$\quad} && \rnode{CI}{\C\tn \I} \\
		&&\ldTo(3,2)_{\C\tn P} & \aa & \rdTo(3,2)_{P\tn \C}
				&&& \ldTo_{\C\tn J}\\
		\rnode{bl}{\C^{2}} && \rTo_{P} & \rnode{C}{\C} & \lTo_{P}
			&& \rnode{br}{\C^{2}} \\
		&&&&&&\rr \\
		%
		\nccurve[angleA=180,angleB=90]{->}{tl}{bl}\Bput{\C\tn P}
		\nccurve[angleA=-20,angleB=90]{->}{CCI}{CI}\Aput{P\tn \I}
		\nccurve[angleA=270,angleB=-45,ncurv=1.5]{->}{CI}{C}\Aput{1}
	\end{diagram}
	Since $S$ is a modification, this in turn is equal to
	\begin{diagram}[s=4em,tight]
		&& \rnode{CCC}{\C^{3}} & \lTo^{\C\tn J\tn \C} & \C\tn \I\tn \C
			& \lTo^{s_{\C,\C\tn \I}} & \rnode{CCI}{\C^{2}\tn \I} \\
		&\raise -1em\rlap{$\C\tn\ss$} & \uTo[snake=-1.5em]>{\C\tn s_{\C,\C}}
			& \C\tn s_{J,\C}
			& \uTo[snake=1.5em]<{\C\tn s_{\C,\I}}
			& \raise1em\llap{$S_{\C|\C,\I}$}\ldTo_{s_{\C,\C}\tn\I} \\
		\rnode{CC}{\C^{2}} & \lTo_{\C\tn P} & \C^{3} & \lTo_{\C^{2}\tn J}
			& \C^{2}\tn \I & \ss\tn\I \\
		\dTo<{P} & \aa & \dTo<{P\tn \C} & \sim & \dTo<{P\tn \I} \\
		\rnode{C}{\C} & \lTo_{P} & \C^{2} & \lTo_{\C\tn J} & \rnode{CI}{\C\tn \I} \\
		&&\raise1.6em\hbox{$\rr$}
		%
		\ncarc[arcangle=80,ncurv=.7]{->}{CI}{C} \Aput{1}
		\ncarc[arcangle=-30]{->}{CCC}{CC} \Bput{\C\tn P}
		\ncarc[arcangle=45]{->}{CCI}{CI} \Aput{P\tn \I}
	\end{diagram}
	which by \pref{eq-rra} is equal to
	\begin{diagram}[s=4em,tight]
		&& \rnode{CCC}{\C^{3}} & \lTo^{\C\tn J\tn \C} & \C\tn \I\tn \C
			& \lTo^{s_{\C,\C\tn \I}} & \rnode{CCI}{\C^{2}\tn \I} \\
		&\raise -1em\rlap{$\C\tn\ss$} & \uTo[snake=-1em]>{\C\tn s_{\C,\C}} & \C\tn s_{J,\C}
			& \uTo[snake=1em]<{\C\tn s_{\C,\I}}
			& \raise1em\llap{$S_{\C|\C,\I}$}\ldTo_{s_{\C,\C}\tn\I} \\
		\rnode{CC}{\C^{2}} & \lTo_{\C\tn P} & \C^{3} & \lTo_{\C^{2}\tn J}
			& \rnode{mid}{\C^{2}\tn \I} & \ss\tn\I \\
		\dTo<{P} & & \raise2em\hbox{$\C\tn\rr$} & & \dTo>{P\tn \I} \\
		\rnode{C}{\C} & & \lTo_{1} & & \rnode{CI}{\C\tn \I} \\
		%
		\ncarc[arcangle=80,ncurv=.7,offset=-3pt]{->}{mid}{CC} \aput(0.517){1}
		\ncarc[arcangle=-30]{->}{CCC}{CC} \Bput{\C\tn P}
		\ncarc[arcangle=45]{->}{CCI}{CI} \Aput{P\tn \I}
	\end{diagram}
	If we compare this with the diagram we began with, and
	cancel the invertible 2-cells $S_{\C|\C,\I}$ and $\ss=\ss\tn\I$,
	we have that
	\[
	\begin{diagram}[s=3.2em,tight]
		&&\C\tn \I\tn \C && \C^{2}\tn \I \\
		&\ldTo^{\C\tn J\tn \C}
			&& \luTo^{\C\tn s_{\C,\I}}
			& \dTo_{s_{\C,\C}\tn \I} \\
		\C^{3} & \C\tn\ll & \dTo>1 & \hbox to 3em{\hss$\C\tn U_{\C|\I}$}
			& \C^{2}\tn \I \\
		&\rdTo_{\C\tn P} &&\ldTo[nohug]_{1} \\
		&& \C^{2} \\
		&& \dTo[nohug]_{P} \\
		&& \C
	\end{diagram}
	\qquad=\qquad
	\begin{diagram}
		&&\C\tn\I\tn \C && \C^{2}\tn \I\\
		&\ldTo^{\C\tn J\tn \C} && \luTo^{\C\tn s_{\C,\I}} & \dTo>{s_{\C,\C}\tn\I} \\
		\rnode{CCC}{\C^{3}} && \C\tn s_{\C,J} && \rnode{CCI}{\C^{2}\tn \I} \\
		&\luTo_{\C\tn s_{\C,\C}} && \ldTo_{\C^{2}\tn J} \\
		&&\C^{3} \\
		&\raise 2em\hbox{$\C\tn\ss$} &\dTo>{\C\tn P}& \raise 2em\hbox{$\C\tn\rr$} \\
		&&\rnode{CC}{\C^{2}} \\
		&&\dTo>P \\
		&&\C
		%
		\ncarc[arcangle=-45]{->}{CCC}{CC}\Bput{\C\tn P}
		\ncarc[arcangle=45]{->}{CCI}{CC}\Aput{\C\tn P}
	\end{diagram}
	\]
	and since $s_{\C,\C}$ is an equivalence, it follows that
	\[
	\begin{diagram}[s=3.2em,tight]
		&&\C\tn \I\tn \C \\
		&\ldTo^{\C\tn J\tn \C}
			&& \luTo^{\C\tn s_{\C,\I}} \\
		\C^{3} & \C\tn\ll & \dTo>1 & \hbox to 3em{\hss$\C\tn U_{\C|\I}$}
			& \C^{2}\tn \I \\
		&\rdTo_{\C\tn P} &&\ldTo[nohug]_{1} \\
		&& \C^{2} \\
		&& \dTo[nohug]_{P} \\
		&& \C
	\end{diagram}
	\qquad=\qquad
	\begin{diagram}
		\rnode{CCC}{\C^{3}} &\lTo^{\C\tn J\tn \C}& \C\tn s_{\C,J}
			& \lTo^{\C\tn s_{\C,\I}} & \rnode{CCI}{\C^{2}\tn \I} \\
		&\luTo_{\C\tn s_{\C,\C}} & \raise1ex\hbox{$\C\tn s_{\C,J}$}
			& \ldTo_{\C^{2}\tn J} \\
		&&\C^{3} \\
		&\raise 2em\hbox{$\C\tn\ss$} &\dTo>{\C\tn P}& \raise 2em\hbox{$\C\tn\rr$} \\
		&&\rnode{CC}{\C^{2}} \\
		&&\dTo>P \\
		&&\C
		%
		\ncarc[arcangle=-45]{->}{CCC}{CC}\Bput{\C\tn P}
		\ncarc[arcangle=45]{->}{CCI}{CC}\Aput{\C\tn P}
	\end{diagram}
	\]
	Now Lemma~\ref{lemma-faithful} yields the claim.
\end{proof}

\begin{propn}
	Let the object $\C$, the 1-cells $P$ and $J$,
	and the 2-cells $\aa$, $\ll$, $\rr$ and $\ss$ be given, as in
	the definition of braided pseudomonoid. Suppose that equations \pref{eq-aa},
	\pref{eq-sa-left}, \pref{eq-sa-right} and \pref{eq-lla} are satisfied.
	If~\pref{eq-lrs} or~\pref{eq-lrs'}
	is satisfied then the structure is indeed a braided pseudomonoid.
	
	It follows that a braided pseudomonoid may be defined using just
	the 2-cells $\aa$, $\ll$, and $\ss$, subject to equations
	\pref{eq-aa}, \pref{eq-sa-left}, \pref{eq-sa-right} and~\pref{eq-lla}.
	The 2-cell $\rr$ can be defined, if necessary, using equation~\pref{eq-lrs}
	or~\pref{eq-lrs'}.
\end{propn}
\begin{proof}
	We shall assume equations~\pref{eq-sa-left}, \pref{eq-lla} and~\pref{eq-lrs},
	and derive equation~\pref{eq-lra}.
	
	Consider the 2-cell
	\begin{diagram}
		&&\I\tn \C^{2} & \lTo^{s_{\C,\I\tn \C}} & \rnode{CIC}{\C\tn \I\tn \C} \\
		&&\dTo<{J\tn \C^{2}} & s_{\C,J\tn\C} & \dTo>{\C\tn J\tn \C} \\
		%
		&&\rnode{tl}{\C^{3}} & \lTo^{s_{\C,\C^{2}}} & \rnode{tr}{\C^{3}}
			&& \hbox to 2em{\hss$\rr\tn\C$} \\
		&&\dTo<{P\tn \C} & s_{\C,P} & \dTo>{\C\tn P} \\
		&&\C^{2} & \lTo^{{s_{\C,\C}}} & \C^{2} \\
		&\aa && \raise 1em\hbox{$\ss$} \rdTo[nohug](1,2)_{P} \ldTo[nohug](1,2)_{P}
			&& \aa\\
		\rnode{bl}{\C^{2}} && \rTo_{P} & \rnode{C}{\C} & \lTo_{P} && \rnode{br}{\C^{2}} \\
		%
		\nccurve[angleA=180,angleB=90]{->}{tl}{bl}\Bput{\C\tn P}
		\nccurve[angleA=0,angleB=90]{->}{tr}{br}\Aput{P\tn \C}
		\nccurve[angleA=0,angleB=45,ncurv=1]{->}{CIC}{br}\Aput{1}
	\end{diagram}
	By equation~\pref{eq-sa-left}, this is equal to
	\begin{diagram}
		&&\I\tn \C^{2} & \lTo^{s_{\C,\I\tn \C}} & \rnode{CIC}{\C\tn \I\tn \C} \\
		&&\dTo<{J\tn \C^{2}} & s_{\C,J\tn\C} & \dTo>{\C\tn J\tn \C} \\
		%
		&&\rnode{tl}{\C^{3}} & \lTo^{s_{\C,\C^{2}}} & \rnode{tr}{\C^{3}}
			&& \hbox to 2em{\hss$\rr\tn\C$} \\
		&&& \raise 1em\hbox{$S_{\C|\C,\C}$}
			\luTo[nohug](1,2)_{\C\tn s_{\C,\C}}
			\ldTo[nohug](1,2)_{s_{\C,\C}\tn\C} \\
		&\C\tn\ss && \C^{3} && \ss\tn\C \\
		&&\ldTo(3,2)_{\C\tn P} & \aa & \rdTo(3,2)_{P\tn \C} \\
		\rnode{bl}{\C^{2}} && \rTo_{P} & \rnode{C}{\C} & \lTo_{P}
			&& \rnode{br}{\C^{2}} \\
		%
		\nccurve[angleA=180,angleB=90]{->}{tl}{bl}\Bput{\C\tn P}
		\nccurve[angleA=0,angleB=90]{->}{tr}{br}\Aput{P\tn \C}
		\nccurve[angleA=0,angleB=45,ncurv=1]{->}{CIC}{br}\Aput{1}
	\end{diagram}
	which, since $S$ is a modification, is equal to
	\begin{diagram}[hug]
		\I\tn \C^{2} && \lTo^{s_{\C,\I\tn \C}} && \rnode{CIC}{\C\tn \I\tn \C} \\
		\dTo<{J\tn \C^{2}}&\luTo_{\I\times s_{\C,\C}}
			& \raise 1em\hbox{$S_{\C|\I,\C}$}
			& \ldTo_{s_{\C,\I}\tn\C}
			& \dTo>{\C\tn J\tn \C}\\
		\C^{3} & \sim & \I\tn \C^{2} & s_{\C,J}\tn\C & \C^{2}\\
		&\luTo_{\C\tn s_{\C,\C}} & \dTo[snake=1ex]>{J\tn \C^{2}} & \ldTo_{s_{\C,\C}\tn\C}
			&&\rr\tn\C \\
		\dTo<{\C\tn P} & \C\tn\ss & \C^{3} & \ss\tn\C & \dTo>{P\tn\C} \\
		&\ldTo_{\C\tn P} &\aa& \rdTo_{P\tn \C} \\
		\C^{2} &\rTo_{P}& \C& \lTo_{P}& \rnode{CC}{\C^{2}} \\
		%
		\nccurve[angleA=0,angleB=0]{->}{CIC}{CC} \Aput{1}
	\end{diagram}
	By equation~\pref{eq-lrs}, this is
	\begin{diagram}[hug,midvshaft]
		\I\tn \C^{2} &&  \raise-1em\hbox to0pt{\hskip5em$S_{\C|\I,\C}$\hss} \lTo^{s_{\C,\I\tn \C}}
			&& \C\tn \I\tn \C \\
		&\luTo(2,1)_{\I\tn s_{\C,\C}} & \I\tn \C^{2} & \ldTo(2,1)_{s_{\C,\I}\tn\C} 
			\raise -1em\hbox to 0pt{\hss\quad$U_{\C|\I}\tn\C$\hss} \\
		\dTo<{J\tn \C^{2}} & \sim & \dTo<{J\tn \C^{2}} & \raise-1em\llap{$\ll\tn\C$}
			\rdTo[nohug]^{1} & \dTo>{1} \\
		\C^{3} & \lTo^{\C\tn s_{\C,\C}} & \C^{3} & \rTo_{P\tn \C} & \C^{2} \\
		& \rdTo[nohug]_{\C\tn P} \raise1em\rlap{$\C\tn\ss$} & \dTo>{\C\tn P} & \aa & \dTo>P \\
		&& \C^{2} & \rTo_{P} & \C
	\end{diagram}
	which, by \pref{eq-lla}, is equal to
	\begin{diagram}[hug,midvshaft]
		&&\I\tn \C^{2} &&  \raise-1em\hbox to0pt{\hskip3.5em$S_{\C|\I,\C}$\hss} \lTo^{s_{\C,\I\tn \C}}
			&& \C\tn \I\tn \C \\
		&\ldTo(2,3)<{J\tn \C^{2}} & &\luTo(2,1)_{\I\tn s_{\C,\C}} & \I\tn \C^{2} & \ldTo(2,1)_{s_{\C,\I}\tn\C} 
			\raise -1em\hbox to 0pt{\hss\quad$U_{\C|\I}\tn\C$\hss} \\
		&& \sim & \ldTo<{J\tn \C^{2}} & \dTo[snake=-1ex]<{\I\tn P}
			& \rdTo[nohug]_{1} & \dTo>{1} \\
		\C^{3} & \lTo^{\C\tn s_{\C,\C}} & \C^{3} & \sim & \I\tn \C & & \C^{2} \\
		& \rdTo[nohug]_{\C\tn P} \raise1em\rlap{$\C\tn\ss$} & \dTo>{\C\tn P}
			& \ldTo_{J\tn \C} & \raise -1em\hbox{$\ll$} & \rdTo[nohug]_{1} & \dTo>P \\
		&& \C^{2} && \rTo_{P} && \C
	\end{diagram}
	which, since the $\sim$ cells are natural, is equal to
	\begin{diagram}[hug,midvshaft]
		&&\rnode{ICC}{\I\tn \C^{2}} &&  \raise-1em\hbox to0pt{\hskip3.5em$S_{\C|\I,\C}$\hss} \lTo^{s_{\C,\I\tn \C}}
			&& \C\tn \I\tn \C \\
		&\ldTo(2,3)<{J\tn \C^{2}} & &\luTo(2,1)_{\I\tn s_{\C,\C}}
			\raise -1.5em\hbox{$\I\tn\ss$}& \I\tn \C^{2} & \ldTo(2,1)_{s_{\C,\I}\tn\C} 
			\raise -1em\hbox to 0pt{\hss\quad$U_{\C|\I}\tn\C$\hss} \\
		& & & & \dTo[snake=-1ex]<{\I\tn P}
			& \rdTo[nohug]_{1} & \dTo>{1} \\
		\C^{3} & & \sim &  & \rnode{IC}{\I\tn \C} & & \C^{2} \\
		& \rdTo[nohug]_{\C\tn P} &
			& \ldTo_{J\tn \C} & \raise -1em\hbox{$\ll$} & \rdTo[nohug]_{1} & \dTo>P \\
		&& \C^{2} && \rTo_{P} && \C
		%
		\ncarc[arcangle=-40,ncurv=.6]{->}{ICC}{IC} \Bput{\I\tn P}
	\end{diagram}
	By~\pref{eq-lla}, this is equal to
	\begin{diagram}[hug,midvshaft]
		&&\rnode{ICC}{\I\tn \C^{2}} &&  \raise-1em\hbox to0pt{\hskip3.5em$S_{\C|\I,\C}$\hss} \lTo^{s_{\C,\I\tn \C}}
			&& \C\tn \I\tn \C \\
		& & &\luTo(2,1)_{\I\tn s_{\C,\C}}
			\raise -1.5em\hbox{$\I\tn\ss$}& \I\tn \C^{2} & \ldTo(2,1)_{s_{\C,\I}\tn\C} 
			\raise -1em\hbox to 0pt{\hss\quad$U_{\C|\I}\tn\C$\hss} \\
		& & & & \dTo[snake=-1ex]<{\I\tn P}
			& \rdTo[nohug]_{1} & \dTo>{1} \\
		& \ll\tn\C & \dTo<1 &  & \rnode{IC}{\I\tn \C} & & \C^{2} \\
		& & & & & \rdTo[nohug]_{1} & \dTo>P \\
		\rnode{CCC}{\C^{3}} & \rTo^{P\tn \C} & \C^{2} && \rTo_{P} && \C \\
		& \rdTo[nohug](3,2)_{\C\tn P} && \aa && \ruTo[nohug](3,2)_{P} \\
		&&& \C^{2}
		%
		\ncarc[arcangle=-40,ncurv=.6,offsetA=3pt]{->}{ICC}{IC} \bput{-60}(.65){\I\tn P}
		\ncarc[arcangle=-30]{->}{ICC}{CCC} \Bput{J\tn \C^{2}}
	\end{diagram}
	which equals
	\begin{diagram}
		&&& \raise -1em\hbox{$$} &\rnode{CIC}{\C\tn \I\tn \C} \\
		\rnode{ICC}{\I\tn \C^{2}} & \lTo^{\I\tn s_{\C,\C}} & \I\tn \C^{2}
			& \ldTo[hug,snake=-6pt](2,1)^{s_{\C,\I}\tn\C}
			\raise -1em\hbox to2em{$U_{\C|\I}\tn\C$\hss} & \dTo[midvshaft]>1 \\
		\dTo<{J\tn \C^{2}} & \raise-1em\llap{$\ll\tn\C$} \rdTo^{1} && \rdTo_{1} \\
		\C^{3} & \rTo_{P\tn \C} & \C^{2} & \lTo^{s_{\C,\C}} & \C^{2} \\
		\dTo<{\C\tn P} & \rlap{$\aa$} && \rdTo_{P} \raise.8em\rlap{\hskip1.5em$\ss$}
			& \dTo>P \\
		\C^{2} && \rTo_{P} && \C
		%
		\ncarc[arcangle=-30]{->}{CIC}{ICC} \Bput{s_{\C,\I\tn \C}}
		 	\Aput{\raise-4pt\hbox{$S_{\C|\I,\C}$}}
	\end{diagram}
	By one of the unit axioms in the definition of pseudomonoid, this is
	\begin{diagram}
		\rnode{ICC}{\I\tn \C^{2}} & & \lTo^{s_{\C,\I\tn\C}}
			& & \C\tn \I\tn \C \\
		\dTo<{J\tn \C^{2}} & \raise-1em\llap{$\ll\tn\C$} \rdTo^{1} &&& \dTo>1 \\
		\C^{3} & \rTo_{P\tn \C} & \C^{2} & \lTo^{s_{\C,\C}} & \C^{2} \\
		\dTo<{\C\tn P} & \rlap{$\aa$} && \rdTo_{P} \raise.8em\rlap{\hskip1.5em$\ss$}
			& \dTo>P \\
		\C^{2} && \rTo_{P} && \C
	\end{diagram}
	which, since $s$ is pseudo-natural, is equal to
	\begin{diagram}
		\I\tn \C^{2} &&\lTo^{s_{\C,\I\tn \C}} && \C\tn \I\tn \C \\
		& s_{\C,J\tn \C} && \ldTo[hug]^{\C\tn J\tn \C} \\
		\dTo<{J\tn \C^{2}} && \C^{3} & \rlap{$\C\tn\ll$} & \dTo>1 \\
		&\ldTo[hug]^{s_{\C,\C^{2}}} & s_{\C,P} & \rdTo[hug]^{\C\tn P} \\
		\C^{3} & \rTo_{P\tn \C} & \C^{2} & \lTo^{s_{\C,\C}} & \C^{2} \\
		\dTo<{\C\tn P} & \rlap{$\aa$} && \rdTo_{P} \raise.8em\rlap{\hskip1.5em$\ss$}
			& \dTo>P \\
		\C^{2} && \rTo_{P} && \C
	\end{diagram}
	If we now compare this with the 2-cell we began with, and cancel the common
	invertible 2-cell
	\begin{diagram}
		&&\I\tn \C^{2} & \lTo^{s_{\C,\I\tn \C}} & \rnode{CIC}{\C\tn \I\tn \C} \\
		&&\dTo<{J\tn \C^{2}} & s_{\C,J\tn\C} & \dTo>{\C\tn J\tn \C} \\
		%
		&&\rnode{tl}{\C^{3}} & \lTo^{s_{\C,\C^{2}}} & \rnode{tr}{\C^{3}}\\
		&&\dTo<{P\tn \C} & s_{\C,P} & \dTo>{\C\tn P} \\
		&&\C^{2} & \lTo^{{s_{\C,\C}}} & \C^{2} \\
		&\aa && \raise 1em\hbox{$\ss$} \rdTo[nohug](1,2)_{P} \ldTo[nohug](1,2)_{P} \\
		\rnode{bl}{\C^{2}} && \rTo_{P} & \rnode{C}{\C} \\
		%
		\nccurve[angleA=180,angleB=90]{->}{tl}{bl}\Bput{\C\tn P}
	\end{diagram}
	we obtain equation~\pref{eq-lra}, as claimed.
\end{proof}

\bibliography{cs}
\end{document}