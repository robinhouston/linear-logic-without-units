%!TEX TS-program = latex
\documentclass{robinthesis}

% In a few places, I used := to define things. Andrea pointed out that
% I didn't do so consistently, so I've changed the instances of := to
% \defeqto, which for now is just an = sign. If I later decide to consistently
% distinguish definitional equality, I'll redefine this and change other
% definitions to use it.
\newcommand\defeqto{=}

\begin{thesischapter}{Bicats}{Bicategories}
Monoidal and promonoidal categories are both instances of the
general notion of a pseudomonoid in a monoidal bicategory. We
shall need some results about promonoidal categories that are
in fact generally true of pseudomonoids in any monoidal bicategory,
and which we should therefore like to prove as such.

Unfortunately the literature on monoidal bicategories is still fairly sparse.
The relevant definitions may be obtained by specialising to the one-object
case those given for tricategories by \citet{GPS}, and this has been done
explicitly in the unpublished dissertation of \citet{CarmodyThesis}.
Also \citet{MonBicat} and \citet{HDA1} have given explicit definitions for
the important special case of Gray-categories. However there is no explicit,
published account of the general notion.

Even the literature on plain bicategories is rather scanty, and although
the situation is improving \citep[see][for example]{LackCompanion} many
fundamental results have no published proof, and there is still a
substantial gap between what is known to the experts and what has been
written down. Neither is notation
yet standardised. For these reasons, we give in this chapter a rapid
but reasonably thorough account of the bicategory theory that we need,
with the occasional digression.

\section{Bicategories: basic definitions}
\begin{definition} % bicategory
	A bicategory $\B$ consists of:
	\begin{itemize}
		\item a set\footnote{Possibly quite a large set: whereas most of ordinary
		category theory can be formalised in a ``one universe'' foundation, it's
		convenient to assume at least two Grothendieck universes for the purposes
		of bicategory theory. (We want to permit e.g.\ the bicategory of large
		categories.) We shall leave these considerations implicit, on the whole.}
		$|\B|$ of objects,
		\item for every pair $A$, $B$ of objects, a hom-category $\B(A,B)$,
			whose objects are called \emph{1-cells} or arrows, and whose morphisms
			are called \emph{2-cells}
		\item for every object $A$, a selected `identity' 1-cell $1_A\in\B(A,A)$,
		\item for every triple $A$,$B$,$C$ of objects, a `horizontal composition' functor
		\[
			\o: \B(B,C)\x\B(A,B) \to \B(A,C),
		\]
		\item for every pair $A$, $B$ of objects, natural isomorphisms
		\[
			\begin{array}{l}
			\l: 1_B\o- \To -: \B(A,B)\to\B(A,B),\\
			\r: -\o 1_A \To -: \B(A,B)\to\B(A,B),
			\end{array}
		\]
		\item for every four objects $A$,$B$,$C$,$D$, a natural isomorphism
		\[
			\a: -\o(-\o-)\To(-\o-)\o-: \B(C,D)\x\B(B,C)\x\B(A,B)\to\B(A,D)
		\]
	\end{itemize}
	subject to two coherence conditions: for all $f\in\B(A,B)$ and $h\in\B(B,C)$,
	the diagram
	\begin{diagram}
		h\o(1_B\o f) &\rTo^{\a_{h,1_B,f}}&(h\o1_B)\o f\\
		&\rdTo[snake=-1ex](1,2)<{h\o \l_f}
			\raise1ex\hbox{$\clr$}%
			\ldTo[snake=1ex](1,2)>{\r_h\o f}\\
		&h\o f,
	\end{diagram}
	commutes in $\B(A,C)$,
	and for all $e\in\B(A,B)$, $f\in\B(B,C)$, $g\in\B(C,D)$, $h\in\B(D,E)$, the diagram
	\begin{diagram}
	  h\o \bigl(g\o (f\o e)\bigr)
	  &\rTo^{\a_{h,g,f\o e}}&(h\o g)\o (f\o e)
	  &\rTo^{\a_{h\o g,f,e}} & \bigl((h\o g)\o f\bigl)\o e
	  \\
	  &\rdTo[snake=-1em](1,2)<{h\o \a_{g,f,e}}
	  &\ca
	  & \ruTo[snake=1em](1,2)>{\a_{h,g,f}\o e}
	  \\
	  & \spleft{h\o\bigl((g\o f)\o e\bigr)}
	  & \rTo_{\a_{h,g\o f,e}}
	  & \spright{\bigl(h\o(g\o f)\bigr)\o e}
	\end{diagram}
	commutes in $\B(A,E)$.
\end{definition}
We shall often omit the subscript on identity 1-cells, writing just $1$
rather than $1_{A}$, when the object can be easily determined from the context.
We also omit the 1-cell subscripts of $\a$, $\l$ and $\r$ from time
to time.
\begin{remark} % on coherence
	A bicategory with just one object may be regarded as a monoidal
	category: the definition simply reduces to the usual definition
	of monoidal category, in that case.
	%
	Furthermore, Mac Lane's coherence theorem for monoidal categories
	\citep{MLCoh} equally well applies to bicategories in general:
	the proof goes through essentially unchanged. \Citet{KellyML}
	shows -- again for monoidal categories -- that just two axioms,
	corresponding to our $\ca$ and $\clr$, suffice for coherence; and
	that proof, too, applies equally well to bicategories in general.
	
	This shows that, between any pair of functors built up from identities
	and composition, there is at most one natural transformation built
	from $\a$, $\l$, $\r$ and their inverses.
	%
	Therefore we shall not usually give names to the `structural isomorphisms'
	$\a$, $\l$, $\r$, their inverses, and composites thereof. We shall instead
	use the symbol `$\cong$' as a generic label for a structural isomorphism.
	Since by coherence there is a unique such isomorphism of each type,
	this practice introduces no ambiguity.
\end{remark}
\begin{remark}\label{rem-abstract-coherence} % on the abstract coherence theorem
	There is another version of the coherence theorem, proven for
	the case of monoidal categories by \citet[section~1]{BTC}, and
	explicitly for bicategories by \citet[chapter~2]{GurskiThesis}%
	\footnote{
		Gurski's account is exceptionally thorough and largely
		self-contained. He does use the bicategorical Yoneda
		lemma without proof, and indeed that proof does not
		appear ever to have been published -- presumably because
		the idea is straightforward, even if the details verge
		on overwhelming. We provide a proof as Prop.~\ref{prop-yoneda}
		below.
	}.
	Anticipating some definitions we have yet to make, this version
	says that the canonical functor from a free bicategory to the
	corresponding free 2-category is a biequivalence. This is a
	powerful result, which gives an honest justification for neglecting
	the structural isomorphisms in many circumstances.
\end{remark}
\begin{remark} % on pasting diagrams
	We shall often describe 2-cells, and equations between them,
	using pasting diagrams. It's important to be clear about how
	such diagrams are to be interpreted, which we'll explain by
	reference to the following example. Let $\sigma$ and $\tau$
	be 2-cells that fit into the diagram
	\begin{diagram}
		A & \rTo^{f} & B & \rTo^{g} & C \\
		\dTo<h & \Arr\Swarrow\sigma & \dTo>k & \Arr\Swarrow\tau & \dTo>l \\
		X & \rTo_{m} & Y & \rTo_{n} & Z
	\end{diagram}
	Firstly, notice that this diagram does not uniquely define a
	2-cell. Instead it defines a \emph{family} of 2-cells, one
	for each bracketing of the source and target edges, e.g.\ for
	the bracketings $l\o(g\o f)$ and $(h\o m)\o n$,
	the pasting diagram describes the 2-cell
	\[
		l\o(g\o f) \rTo^{\a_{l,g,f}} (l\o g)\o f
			\rTo^{\tau\o f} (n\o k)\o f
			\rTo^{\a^{-1}_{n,k,f}} n\o (k\o f)
			\rTo^{n\o\sigma} (h\o m)\o n.
	\]
	This also demonstrates the second subtlety: the associativity
	$(n\o k)\o f \rTo^{\a^{-1}_{n,k,f}} n\o (k\o f)$ must be implicitly
	inserted between $\tau$ and $\sigma$.
	
	Observe also that an \emph{equation} between pasting diagrams may
	be regarded as a family of equations, one for each bracketing of the
	source and target 1-cells. These equations are all equivalent, in the
	sense that each implies the others, so to prove such a family of
	equations it suffices to prove one of them.
	
	These remarks are intended only to help the reader to understand
	what we mean when we draw a pasting diagram or an equation
	involving them. A rigorous treatment of bicategorical pasting
	is given by \citet{VerityThesis}.
\end{remark}

\begin{remark} % on string diagrams
	As well as pasting diagrams, one can also use \emph{string diagrams}
	\citep{GTC, LowDim} to represent and calculate with 2-cells in a
	bicategory. In a string diagram, an object is represented by a region
	of the plane, a 1-cell by a line, and a 2-cell by a node. For example,
	the 2-cell $\gamma: f \To g: A \to B$ is drawn as
	\begin{diagram}
		&& A \\
		\rnode{f}{f} && \circlenode{gamma}{\gamma} && \rnode{g}{g} \\
		&& B
		\ncline[nodesepB=0pt]{-}{f}{gamma}
		\ncline[nodesepA=0pt]{-}{gamma}{g}
	\end{diagram}
	and a 2-cell $\delta: g\o f \To h$ would be drawn as
	\begin{diagram}[h=1.5em]
		\rnode{f}{f} \\
		&&& A \\
		& B & \circlenode{delta}{\delta} && \rnode{h}{h} \\
		&&& C \\
		\rnode{g}{g}
		\ncline[nodesepB=0pt]{-}{f}{delta}
		\ncline[nodesepB=0pt]{-}{g}{delta}
		\ncline[nodesepA=0pt]{-}{delta}{h}
	\end{diagram}
	for $f: A\to B$, $g:B\to C$ and $h:A\to C$.
	Identity 1-cells are not drawn, so a 2-cell $\zeta: 1_{A}\To f: A\to A$
	is drawn
	\begin{diagram}
		&& & A \\
		A && \circlenode{zeta}{\zeta} && \rnode{f}{f}\\
		&& & A
		\ncline[nodesepA=0pt]{-}{zeta}{f}
	\end{diagram}
	Composition of 2-cells corresponds to the pasting together of
	string diagrams along the appropriate edge: the orientation we
	have chosen for our diagrams has the unfortunate effect that
	horizontal composition is represented by vertical pasting, and
	vice versa. However it has the psychological advantage that
	the diagrams read from left to right.
	
	The object labels may usually be omitted, since they can be
	inferred from the types of the 1-cells.
	
	It is the case that geometric manipulations of string diagrams
	always correspond to allowable operations. For example, given
	2-cells $\gamma: f\To 1: A\to A$ and $\delta: 1\To g: A\to A$,
	the diagram
	\begin{diagram}
		\rnode{f}{\strut f} & & \circlenode{gamma}{\strut\gamma} \\
		& \circlenode{delta}{\strut\delta} & & \rnode{g}{\strut g}
		\ncline[nodesepB=0pt]{-}{f}{gamma}
		\ncline[nodesepA=0pt]{-}{delta}{g}
	\end{diagram}
	can be deformed to
	\begin{diagram}
		\rnode{f}{\strut f} & & \circlenode{gamma}{\strut\gamma} & \circlenode{delta}{\strut\delta}
			&& \rnode{g}{\strut g}
		\ncline[nodesepB=0pt]{-}{f}{gamma}
		\ncline[nodesepA=0pt]{-}{delta}{g}
	\end{diagram}
	and then to
	\begin{diagram}
		& \circlenode{delta}{\strut\delta} & & \rnode{g}{\strut g} \\
		\rnode{f}{\strut f} & & \circlenode{gamma}{\strut\gamma}
		\ncline[nodesepB=0pt]{-}{f}{gamma}
		\ncline[nodesepA=0pt]{-}{delta}{g}
	\end{diagram}
	The corresponding sequence of pasting diagrams is
	\vskip4em
	\hbox to \linewidth{
	$\begin{diagram}
		&&\cong \\
		\rnode{1}{A} & \Arr\Downarrow{\scriptstyle\gamma} & \rnode{2}{A}
			& \Arr\Downarrow{\scriptstyle\delta} & \rnode{3}{A} \\
		&&\cong
		\ncarc[arcangle=30]{->} {1}{2} \Aput{f}
		\ncarc[arcangle=-30]{->}{1}{2} \Bput{1}
		\ncarc[arcangle=30]{->} {2}{3} \Aput{1}
		\ncarc[arcangle=-30]{->}{2}{3} \Bput{g}
		\ncarc[arcangle=80,ncurv=1]{->}{1}{3} \Aput{f}
		\ncarc[arcangle=-80,ncurv=1]{->}{1}{3} \Bput{g}
	\end{diagram}$\hfil}
	\vskip1em
	\begin{diagram}[h=1em]
		&\Arr\Downarrow{\scriptstyle\gamma} \\
		\rnode{l}{A} & \rTo^{1} & \rnode{r}{A} \\
		&\Arr\Downarrow{\scriptstyle\delta}
		\ncarc[arcangle=60,ncurv=1]{->}{l}{r} \Aput{f}
		\ncarc[arcangle=-60,ncurv=1]{->}{l}{r} \Bput{g}
	\end{diagram}
	\hbox to \linewidth{\hfil
	$\begin{diagram}
		&&\cong \\
		\rnode{1}{A} & \Arr\Downarrow{\scriptstyle\delta} & \rnode{2}{A}
			& \Arr\Downarrow{\scriptstyle\gamma} & \rnode{3}{A} \\
		&&\cong
		\ncarc[arcangle=30]{->} {1}{2} \Aput{1}
		\ncarc[arcangle=-30]{->}{1}{2} \Bput{g}
		\ncarc[arcangle=30]{->} {2}{3} \Aput{f}
		\ncarc[arcangle=-30]{->}{2}{3} \Bput{1}
		\ncarc[arcangle=80,ncurv=1]{->}{1}{3} \Aput{f}
		\ncarc[arcangle=-80,ncurv=1]{->}{1}{3} \Bput{g}
	\end{diagram}$}
	\vskip4em\noindent
	which is clearly much harder to follow. This example illustrates
	the way that string diagrams leave the unit constraints $\l$
	and $\r$ implicit, making powerful use of coherence. (Of
	course the string diagram formalism -- in common with pasting
	diagrams -- also leaves implicit the associator $\a$, though that
	is not shown in this particular example. But the real power of string
	diagrams comes from the ease with which they handle the identities.)
\end{remark}

\begin{definition} % ^{op} and ^{co}
	For any bicategory $\B$ there is a bicategory $\B\op$ obtained from $\B$
	by reversing the direction of the 1-cells, so $\B\op(A,B) = \B(B,A)$; and also a
	bicategory $\B\co$ obtained by reversing the direction of the 2-cells, so
	$\B\co(A,B) = \B(A,B)\op$.
\end{definition}
\begin{definition} % product of bicats
	Bicategories $\B$ and $\BC$ have a product formed
	in the obvious way: the set of objects is $|\B\x\BC| = |\B|\x|\BC|$,
	the hom-categories are
	$(\B\x\BC)(\langle A,B\rangle,\langle X,Y\rangle) = \B(A,X)\x\BC(B,Y)$, and
	horizontal composition is defined pointwise. This also extends in the obvious
	way to the product of three of more bicategories.
\end{definition}

\begin{definition}\label{def-psfun} % pseudo-functor
	Given bicategories $\B$ and $\BC$, a \emph{pseudo-functor}
	$F:\B\to\BC$ consists of:
	\begin{itemize}
		\item for every object $A\in\B$, an object $FA\in\BC$,
		\item for every pair $A$, $B$ of objects of $\B$, a functor
		\[
			F_{A,B}: \B(A,B)\to\B(FA,FB),
		\]
		\item for every $A\in\B$, an invertible 2-cell $F_A: 1_{FA}\To F(1_A): FA\to FA$,
		\item for every $A$ and $B\in\B$, a natural isomorphism with components
		\[
			F_{g,f}: F(g)\o F(f) \To F(g\o f),
		\]
	\end{itemize}
	such that for every $f:A\to B$ in $\B$, the diagrams below commute in the category $\BC(FA,FB)$,
	\[
	\begin{diagram}
		1_{FB}\o F(f) & \rTo^{F_B\o F(f)} & F(1_B)\o F(f)\\
		\dTo<{\l_{F(f)}} &\cFl&\dTo>{F_{1_B, f}}\\
		F(f) & \lTo_{F(\l_f)} & F(1_B\o f)
	\end{diagram}
	\qquad
	\begin{diagram}
		F(f)\o 1_{FA} & \rTo^{F(f)\o F_A} & F(f)\o F(1_A)\\
		\dTo<{\r_{F(f)}} &\cFr&\dTo>{F_{f,1_A}}\\
		F(f) & \lTo_{F(\r_f)} & F(f\o 1_A)
	\end{diagram}
	\]
	and for every $A\rTo^f B\rTo^g C\rTo^h D$ in $\B$, the diagram
	\begin{diagram}
		F(h)\o(F(g)\o F(f)) & \rTo^{\a_{Fh,Fg,Ff}}& (F(h)\o F(g))\o F(f)\\
		\dTo<{F(h)\o F_{g,f}} && \dTo>{F_{h,g}\o F(f)}\\
		F(h)\o F(g\o f) &\cFa& F(h\o g)\o F(f)\\
		\dTo<{F_{h,(g\o f)}} && \dTo>{F_{(h\o g),f}}\\
		F(h\o(g\o f)) & \rTo_{F(\a_{h,g,f})} & F((h\o g)\o f)
	\end{diagram}
	commutes in the category $\BC(FA,FD)$.
	A pseudo-functor is also known as a \emph{homomorphism of bicategories}.
\end{definition}
\begin{remark}
	There is also a notion of \emph{lax functor} between bicategories,
	defined as above, except that the 2-cells that appear in the
	definition need not be invertible. Lax functors have their uses
	-- for example, a lax functor $1\to\B$ is the same as a monad
	in $\B$ -- but we have no need of them.
\end{remark}

\begin{definition}\label{def-psnat} % pseudo-natural transformation
	Given pseudo-functors $F$, $G:\B\to\BC$, a \emph{pseudo-natural
	transformation} $\gamma: F\To G$ consists of:
	\begin{itemize}
		\item for every $A\in\B$, a 1-cell $\gamma_A: FA\to GA$,
		\item for every $f:A\to B$ in $\B$, an invertible 2-cell
		\begin{diagram}
			FA &\rTo^{\gamma_A} & GA\\
			\dTo<{Ff} & \Nearrow \gamma_f& \dTo>{Gf}\\
			FB & \rTo_{\gamma_B} & GB
		\end{diagram}
		such that this assignment is natural in $f$. Naturality amounts to
		asking that, for every 2-cell $\tau: f\To g: A\to B$, we have
		\[
		\hskip-1em
		\begin{diagram}[size=4em]
			FA & \rTo^{\gamma_A} & \rnode{GA}{GA}\\
			\dTo<{Ff} & \llap{$\Nearrow \gamma_f$}
				& \begin{array}{c}\Rightarrow\\G(\tau)\end{array}\\
			FB & \rTo_{\gamma_B} & \rnode{GB}{GB}
			\ncarc{->}{GA}{GB}\Aput{Gg}
			\ncarc{<-}{GB}{GA}\Aput{Gf}
		\end{diagram}
		\hskip3em=\hskip3em
		\begin{diagram}[size=4em]
			\rnode{FA}{FA} & \rTo^{\gamma_A} & GA\\
			\begin{array}{c}\Rightarrow\\F(\tau)\end{array}
				& \rlap{$\Nearrow \gamma_g$} & \dTo>{Gg}\\
			\rnode{FB}{FB} & \rTo_{\gamma_B} & GB
			\ncarc{->}{FA}{FB}\Aput{Fg}
			\ncarc{<-}{FB}{FA}\Aput{Ff}
		\end{diagram}
		\]
		\item These data must satisfy the \emph{unit condition}: for every $A\in\B$ we have
		\[
		\hskip-1em
			\begin{diagram}[size=4em]
				\rnode{FA}{FA} & \rTo^{\gamma_A} & GA\\
				\begin{array}{c}\Rightarrow\\F_A\end{array}
					& \hbox to 0pt{$\mkern4mu\Nearrow \gamma_{1_A}$} & \dTo>{G(1_A)}\\
				\rnode{FB}{FA} & \rTo_{\gamma_A} & \rnode{GA}{GA}
				\ncarc{->}{FA}{FB}\Aput{F(1_A)}
				\ncarc{<-}{FB}{FA}\Aput{1_{FA}}
				%\nccurve{->}{FA}{GA} \Bput{\gamma_A}
			\end{diagram}
			%
			\hskip3em=\hskip3em
			%
			\begin{diagram}[size=4em]
				FA & \rTo^{\gamma_A} & \rnode{GA}{GA}\\
				\dTo<{1_{FA}} & \llap{$\cong\mkern4mu$}
					& \begin{array}{c}\Rightarrow\\G_A\end{array}\\
				FA & \rTo_{\gamma_A} & \rnode{GB}{GA}
				\ncarc{->}{GA}{GB}\Aput{G(1_A)}
				\ncarc{<-}{GB}{GA}\Aput{1_{GA}}
			\end{diagram}
		\]
		\item and the \emph{composition condition}: for all $A\rTo^f B\rTo^g C$ in $\B$, we have
		\[
		\begin{diagram}[h=2em]
			\rnode{FA}{FA} & \rTo^{Ff}& FB & \rTo^{Fg}& \rnode{FC}{FC}\\
			&&\raise4pt\hbox{$\Downarrow F_{g,f}$}\\
			\dTo<{\gamma_A} &&&& \dTo>{\gamma_C}\\
			&&\Swarrow \gamma_{g\o f}\\
			GA && \rTo_{G(g\o f)} && GC
			\ncarc[arcangle=-60]{->}{FA}{FC}\Bput{F(g\o f)}
		\end{diagram}
		\quad=\quad
		\begin{diagram}[h=2em]
			FA & \rTo^{Ff}& FB & \rTo^{Fg}& FC\\
			&\Swarrow\gamma_f & \dTo>{\gamma_B}&\Swarrow\gamma_g\\
			\dTo<{\gamma_A} && GB && \dTo>{\gamma_C}\\
			&\ruTo^{Gf} & \Downarrow G_{g,f} & \rdTo^{Gg}\\
			GA && \rTo_{G(g\o f)} && GC
		\end{diagram}
		\]
	\end{itemize} 
\end{definition}
\begin{remark}
	Of course there is such a thing as a lax natural transformation,
	where the 2-cells $\gamma_{f}$ need not be invertible. (In
	B{\'e}nabou's original terminology, this would actually be an
	oplax transformation -- his 2-cells point the other way --
	but the direction we use seems to be a more natural and
	useful choice: see \citet[Section~5.7]{LackCompanion} for
	one piece of technical evidence for this assertion.)
	However, it should be noted that the collection of
	lax functors $\B\to\BC$, lax transformations between them,
	and modifications (for which see below) between those does
	\emph{not} form a bicategory, which shows that one must be careful
	what one laxifies. In any case, pseudo-functors and pseudo-natural
	transformations are all that we shall need.
\end{remark}
\begin{definition} % modification
	Given pseudo-natural transformations $\gamma$, $\delta: F\To G:\B\to\BC$,
	a \emph{modification} $m: \gamma\Tto\delta$ consists of: for every
	$A\in\B$, a 2-cell $m_A: \gamma_A\To\delta_A$ such that for every $f:A\to B$
	in $\B$, we have
	\[
		\begin{diagram}
			\rnode{FA}{FA} & \Uparrow m_A & \rnode{GA}{GA}\\
			\dTo<{Ff} & \raise-4pt\hbox{$\Nearrow\gamma_f$} & \dTo>{Gf}\\
			FB & \rTo_{\gamma_B} & GB
			\ncarc{->}{FA}{GA}\Aput{\delta_A}
			\ncarc{<-}{GA}{FA}\Aput{\gamma_A}
		\end{diagram}
		\qquad=\qquad
		\begin{diagram}
			FA & \rTo^{\delta_A} & GA\\
			\dTo<{Ff} & \raise4pt\hbox{$\Nearrow\delta_f$} & \dTo>{Gf}\\
			\rnode{FB}{FB} & \Uparrow m_B & \rnode{GB}{GB}
			\ncarc{->}{FB}{GB}\Aput{\delta_B}
			\ncarc{<-}{GB}{FB}\Aput{\gamma_B}
		\end{diagram}
	\]
\end{definition}

\begin{remark}
	Recall that, given a commutative diagram in an ordinary category,
	if one of the arrows is invertible, and is replaced by its inverse in such
	a way that the resulting diagram still has a single source and a single
	target object, then this resulting diagram also commutes.
	%
	This fact has a two-dimensional analogue, as follows: a pasting equation may be
	viewed as a polyhedron, by gluing the two pasting diagrams together along their
	(common) boundary. If, in this polyhedron, one of the cells (faces) is
	invertible and is replaced
	by its inverse in such a way that there is still a unique way to decompose the
	resulting polyhedron as a pair of pasting diagrams, then this resulting pasting
	equation also holds.
	%
	We shall sometimes use this implicitly, when it is more convenient to use
	such a variant of some equation.
\end{remark}

The usual 2-categorical notions of adjunction, equivalence, monad, etc.\ may
also be defined in a bicategory in the obvious way: a little of the theory
of adjunctions is developed below. Since it will be useful almost
immediately, we give here the definition of equivalence:
\begin{definition}\label{def-equivalence} % equivalence
	An \emph{equivalence} from $A$ to $B$ in a bicategory consists of a pair of 1-cells
	$f: A\to B$ and $g:B\to A$, and a pair of invertible 2-cells $e: 1_A\To g\o f$
	and $e': 1_B\To f\o g$. We also say that the arrow $f$ is an equivalence
	just when there exist $g$, $e$, $e'$ as above.
\end{definition}
In particular, every identity arrow $1_{A}$ is an equivalence, as is any
1-cell isomorphic to an identity. There is also an obvious
notion of isomorphism, but it is not very useful in general. In particular, identity
arrows in a bicategory are generally only equivalences and not isomorphisms,
and an object of a bicategory need not even be isomorphic to itself.

Pseudo-natural transformations compose in the obvious way, as do modifications.
Given any two bicategories $\B$ and $\BC$, there is a \emph{pseudo-functor bicategory}
$\Bicat(\B,\BC)$ whose objects are pseudo-functors $\B\to\BC$,
whose 1-cells are pseudo-natural transformations and whose
2-cells are modifications.\footnote{
	Note that this notation is inconsistent with that of
	\citet{FIB}, which uses $\Bicat(\B,\BC)$ to denote the
	bicategory whose objects are \emph{lax} functors from
	$\B$ to $\BC$.
}
We omit the routine verification that
this is indeed a bicategory, but remark that coherence lifts from
$\BC$. It is significant that if $\BC$ is in fact a 2-category
(i.e.\ its associativity and unit isomorphisms are all identities) then
$\Bicat(\B,\BC)$ is also a 2-category.

A pseudo-functor $F:\B\to\BC$ is said to be a \emph{biequivalence}
if it is a local equivalence and \emph{biessentially surjective} -- i.e.\ every
object of $\BC$ is equivalent to one of the form $FA$. If there is such a
biequivalence then we say that $\B$ is \emph{biequivalent} to $\BC$.
(There is a lot more that could be said about biequivalence, of course\dots)

For any bicategories $\BA$, $\B$, $\BC$, there is a biequivalence
\[
	\Bicat(\BA\x\B, \BC) \simsim \Bicat(\BA, \Bicat(\B,\BC))
\]
defined in the natural way.
% XXXX - give more details here!
% 
% as follows. First on objects: let $F:\BA\x\B\to\BC$, and we'll
% define the corresponding pseudofunctor $F': \BA\to \Bicat(\B,\BC)$.
% For an object $A\in\BA$, let $F'(A)$ be the pseudofunctor $F(A,-)$.
% On 1-cells: let $\gamma: F\To G: \BA\x\B\to \BC$ be a pseudo-%
% natural transformation, and define $\gamma': F'\To G'$ to have
% $\gamma'_A = 

\section{On the identities}\label{s-identities}
It is a trivial observation that, say, a semigroup may have at
most one unit: if $i$ and $j$ are both units then $ij=i$ -- because
$j$ is a unit -- and also $ij=j$ because $i$ is a unit. So $i=j$.
Thus the existence of a unit is a property of a semigroup, rather
than being real additional structure. In higher dimensions, this
phenomenon persists: for example, let
$A$ be an object of the bicategory $\B$, and let $1^\bullet: A\to A$
be equipped with natural isomorphisms $\l^{\bullet}$ and $\r^{\bullet}$
making $1^{\bullet}$ act as an identity. Then there is an isomorphism
\[
	1^{\bullet} \rTo^{(\r^{\bullet}_{1_{A}})^{-1}} 1_{A}\o 1^{\bullet}
		\rTo^{\l_{1^{\bullet}}} 1_{A},
\]
so identities in a bicategory are unique up to isomorphism, which
is as much as one could reasonably expect in the circumstances.

An important, if partially-submerged, theme of the present work
is that the property-likeness of units has some interesting consequences.
One consequence of this has been studied in detail by
\citet{KockUnits,JoyalKockUnits,KockWeakIdentityArrows};
a different aspect is visible in the present work, especially in
Section~\chref{Promon}{s-promon-unit}, where we
find that a braided promonoidal category has a \emph{canonical} unit,
and identify a simple property that holds just when this unit exists.

In the immediate context, we can illustrate some of the phenomena as
follows: the unit conditions in the
definitions of pseudo-functor and pseudo-natur\-al transformation
are, in a sense, redundant. (This is not the case where lax functors
or lax natural transformations are concerned -- the invertibility of
our 2-cells is essential to the argument.)
%
Although there is, to my knowledge, no written account available
of the material of this section, in view of its elementary nature
it is reasonable to suppose that it is known to experts in the
field. (The expert whom I asked declined to comment on the question
of how well-known these results are: make of that what you will.)

It will be convenient to introduce some temporary notation,
that we use only in this section. Let $F:\B\to\BC$ be a
pseudo-functor.
Then given a 1-cell $f:FA\to FB$ in $\BC$, let us write $\r^{F}_{f}$
for the composite
\[
	f\o F(1_{A}) \rTo^{f\o F_{A}^{-1}} f\o 1_{FA}
		\rTo^{\r_{f}} f,
\]
and $\l^{F}_{f}$ for
\[
		F(1_{A})\o f \rTo^{F_{A}^{-1}\o f} 1_{FA}\o f
		\rTo^{\l_{f}} f.
\]
Observe that $\r^{F}$ and $\l^{F}$ inherit the coherence of
$\r$ and $\l$, so that in particular the diagrams
\[
	\begin{diagram}[vtriangleheight=3em]
		h\o (k\o F1) && \rTo^{\a} && (h\o k)\o F1 \\
		&\rdTo[snake=-1ex]<{h\o\r^{F}_{k}} && \ldTo[snake=1ex]>{\r^{F}_{h\o k}} \\
		&&h\o k
	\end{diagram}
	and
	\begin{diagram}[vtriangleheight=3em]
		h\o (F1\o k) && \rTo^{\a} && (h\o F1)\o k \\
		& \rdTo[snake=-1ex]<{h\o\l^{F}_{k}} && \ldTo[snake=1ex]>{\r^{F}_{h}\o k} \\
		&& h\o k
	\end{diagram}
\]
commute for all suitably-typed 1-cells $h$ and $k$.

Now we can demonstrate the promised redundancy.
We shall start with the pseudo-natural transformations, since
the situation there is very simple: the unit condition is quite
redundant:
\begin{propn}\label{prop-psnat-redundant}
	Given pseudo-functors $F$, $G:\B\to\BC$ and the data of
	Definition~\ref{def-psnat}, the composition condition
	implies the unit condition.
\end{propn}
\begin{proof}
	Let $\gamma$ be given as in Definition~\ref{def-psnat},
	and suppose it to satisfy the composition condition.
	%
	Now, for every $f: A\to B$ in $\B$, we have the following
	diagram of 1-cells and 2-cells (with associativities left
	implicit):
	\begin{diagram}
		&&\gamma_{B}\o Ff\o F1 & \rTo^{\gamma_{f}\o F1}
			& Gf\o \gamma_{A}\o F1 \\
		&\ldTo^{\gamma_{B}\o F_{f,1}} \raise-1em\rlap{$\gamma_{B}\o\cFr$}
			& \dTo>{\r^{F}} & \natural & \dTo<{\r^{F}}
			& \rdTo^{Gf\o\gamma_{1}} \\
		\rnode{l}{\gamma_{B}\o F(f\o 1)} & \rTo_{\gamma_{B}\o F(\r_{f})}
			& \gamma_{B}\o Ff & \rTo_{\gamma_{f}}
			& Gf\o\gamma_{A} & \lTo_{\r^{G}\o\gamma_{A}}
			& \rnode{r}{Gf\o G1\o\gamma_{A}} \\
		& & \natural
			& \ruTo[snake=-1em](1,2)<{G(\r_{f})\o\gamma_{A}}
			& & \llap{$\cFr\o\gamma_{A}$} \\
		& & & \rnode{b}{G(f\o 1)\o\gamma_{A}}
		%
		\nccurve[angleA=290,angleB=180,ncurv=.5]{->}{l}{b} \Bput{\gamma_{f\o 1}}
		\nccurve[angleA=250,angleB=0,ncurv=.5]{->}{r}{b} \Aput{G_{f,1}\o\gamma_{A}}
	\end{diagram}
	The marked cells commute for the reasons shown, and the outside
	commutes by the composition condition. Since $\gamma_{f}\o F1$ is
	invertible, it follows that the unlabelled triangle commutes. By
	the observation above, about coherence of $r^{F}$ and $\l^{F}$,
	this triangle is equivalent to
	\begin{diagram}[htrianglewidth=4em,tight]
		Gf\o\gamma_{A}\o F1 & \rTo^{Gf\o \gamma_{1}} & Gf \o G1 \o \gamma_{A} \\
		&\rdTo(1,2)<{Gf\o\r^{F_{\gamma_{A}}}} \ldTo(1,2)>{Gf\o\l^{F}_{\gamma_{A}}}\\
		&Gf\o\gamma_{A}
	\end{diagram}
	so, by letting $f=1$, we conclude that
	\begin{diagram}[htrianglewidth=4em,tight]
		\gamma_{A}\o F1 & \rTo^{\gamma_{1}} & G1 \o \gamma_{A} \\
		&\rdTo(1,2)<{\r^{F}_{\gamma_{A}}} \ldTo(1,2)>{\l^{F}_{\gamma_{A}}}\\
		&\gamma_{A}
	\end{diagram}
	commutes, which is equivalent to the unit condition.
\end{proof}
%
For pseudo-functors, the situation is a little more subtle.
The following notion will be useful, both here
and later in Chapter~\refchapter{Cayley}.
\begin{definition}\label{def-fully-faithful}
	A 1-cell $f: A\to B$ is \emph{(representably) fully-faithful}
	if, for every object $X$, the functor
	$\B(X, f): \B(X, A)\to \B(X, B)$ is fully faithful.
	%
	Concretely, this means that, for every pair of arrows
	$h$, $k: X\to A$, every 2-cell
	\begin{diagram}[w=2em,h=1em,tight]
		&& A \\
		&\ruTo^{h} && \rdTo^{f} \\
		X && \Downarrow && B \\
		&\rdTo_{k} && \ruTo_{f} \\
		&& A
	\end{diagram}
	is equal to
	\begin{diagram}
		\\
		\rnode{X}{X} &\Downarrow\gamma& \rnode{A}{A} & \rTo^{f} & B \\
		%
		\ncarc[arcangle=50,ncurv=1]{->}{X}{A}  \Aput{h}
		\ncarc[arcangle=-50,ncurv=1]{->}{X}{A} \Bput{k}
	\end{diagram}
	for some unique $\gamma$.
	
	Dually, $f$ is \emph{co-fully-faithful} if, for every object $X$,
	the functor $\B(f, X): \B(B, X)\to \B(A, X)$ is fully faithful.
\end{definition}
\begin{remark} % remarks on the definition
	Some remarks on the definition:
	\begin{itemize}
	\item It's easy to check that the fully-faithful 1-cells in
		$\Cat$ are precisely the full and faithful functors. On
		the other hand, the analogous property does not generally
		hold for enriched categories: the $\V$-fully-faithful
		functors do not always coincide with those 1-cells of
		$\V$-$\Cat$ that are representably fully-faithful.
	\item The co-fully-faithful 1-cells of $\Cat$ are characterised
	by \citet{LaxEpis}, who call them `lax epimorphisms'.
	%
	\item Every equivalence is both fully-faithful and
	co-fully-faithful.
\end{itemize}
\end{remark}
\begin{lemma}\label{lemma-psnat-unit-only-one}
	The conditions $\cFl$ and $\cFr$ of Definition~\ref{def-psfun}
	are redundant, in the sense that each implies the other.
\end{lemma}
\begin{proof}
	Let $F:\B\to\BC$ be a pseudo-functor, and let
	\[
		A \rTo^{f} B \rTo^{g} C
	\]
	be 1-cells in $\B$. Consider the following diagram
	in $\BC(FA, FC)$:
	\begin{diagram}[w=4em,h=2em,tight,hug]
		Fg\o(F1\o Ff) && \rTo^{\a} && (Fg\o F1)\o Ff \\
		& \rdTo_{Fg\o\l^{F}} && \ldTo_{\r^{F}\o Ff} \\
		\dTo<{Fg\o F_{1,f}} &\hskip-2em \raise -1ex\hbox{$Fg\o\cFl$}& Fg\o Ff
			& \hskip2em \raise -1ex\hbox{$\cFr\o Ff$} & \dTo>{F_{g,1}\o Ff} \\
		& \ruTo_{Fg\o F(\l)} && \luTo_{F(\r)\o Ff} \\
		Fg\o F(1\o f) && \dTo>{F_{g,f}} && F(g\o 1)\o Ff \\
		& \natural && \natural \\
		\dTo<{F_{g,1\o f}} && F(g\o f) && \dTo>{F_{g\o 1, f}} \\
		&\ruTo^{F(g\o\l)} && \luTo^{F(\r\o f)} \\
		F(g\o(1\o f)) && \rTo_{F(\a)} && F((g\o 1)\o f)
	\end{diagram}
	The upper and lower triangles commute, as does the outside edge.
	Since all the 2-cells in the diagram are invertible, if $\cFl$
	holds then so does $\cFr\o Ff$. Taking $f=1$ and using the fact
	that $F1$ is co-fully-faithful, we conclude that $\cFr$ holds.
	In the other direction, if $\cFr$ holds then so does $Fg\o\cFl$,
	whence taking $g=1$ and using the fact that $F1$ is fully-faithful,
	we conclude that $\cFl$ holds.
\end{proof}
\begin{lemma}\label{lemma-psnat-unit-uq}
	The invertible 2-cells $F_{A}: 1_{FA}\to F(1_{A})$
	of Definition~\ref{def-psfun} are uniquely determined
	by the other data.
\end{lemma}
\begin{proof}
	Let $F: \B\to\BC$ be a pseudo-functor. Condition~$\cFl$
	implies that, for every object $A\in\B$, the 2-cell $F_{A}\o F(1_{A})$
	is equal to
	\[
		1_{FA}\o F(1_{A}) \rTo^{\l_{F(f)}} F(1_{A})
			\rTo^{F(\lambda_{1_{A}}^{-1})} F(1_{A}\o 1_{A})
			\rTo^{F_{1_{A},1_{A}}^{-1}} F(1_{A})\o F(1_{A})
	\]
	(just by taking $f=1_{A}$). Since $F(1_{A})$ is fully-faithful,
	this equation uniquely determines $F_{A}$.
\end{proof}
\begin{propn}
	Let $F: \B\to\BC$ be as in the definition of
	pseudo-functor, though without the 2-cells $F_{A}$.
	These data may be augmented to give a pseudo-functor $F$,
	in a unique way, if and only if $F(1_{A})$ is both
	fully-faithful and co-fully-faithful for each object $A\in\B$,
	if and only if $F(1_{A})$ is \emph{either}
	fully-faithful \emph{or} co-fully-faithful for each object $A\in\B$.
\end{propn}
\begin{proof}
	If we have invertible 2-cells $F_{A}$, then each $F(1_{A})$ is
	isomorphic to the identity, hence an equivalence, so in particular
	is fully-faithful and co-fully-faithful. For the converse,
	take an object $A\in\B$, and suppose that $F(1_{A})$ is co-fully-%
	faithful. Let $F_{A}: 1_{FA}\To F(1_{A})$ be the unique
	2-cell for which $F_{A}\o F(1_{A})$ is equal to the composite
	\[
		1_{FA}\o F(1_{A}) \rTo^{\l} F(1_{A})
			\rTo^{F(\l^{-1})} F(1_{A}\o 1_{A})
			\rTo^{F_{1_{A},1_{A}}^{-1}} F(1_{A})\o F(1_{A}).
	\]
	Since this $F_{A}$ is invertible, $F(1_{A})$ is isomorphic to
	the identity, hence also fully-faithful.
	For any $f: A\to B$,
	we have the following diagram in $\B(FA, FB)$ (with associativities
	left implicit):
	\begin{diagram}
		Ff\o 1\o F1 & \rTo^{Ff\o\l} & \rnode{FfF1}{Ff\o F1}
			& \rTo^{Ff\o F(\l^{-1})} & Ff\o F(1\o 1)
			& \rTo^{Ff\o F_{1,1}^{-1}} & Ff\o F1\o F1 \\
		&& && \dTo<{F_{f,f\o 1}} &\cFa& \dTo>{F_{f,1}\o F1} \\
		&& &\raise 1em\hbox{$\natural$}& F(f\o 1\o 1) & \rTo_{F_{f\o 1,1}^{-1}}
			& F(f\o 1)\o F1 \\
		&& && \dTo<{\begin{array}{r}F(f\o\l)\\=F(\r\o 1)\end{array}}
			&\natural& \dTo>{F(\r)\o F1} \\
		&& && \rnode{Ff1}{F(f\o 1)} & \rTo_{F_{f,1}^{-1}} & Ff\o F1
		%
		\nccurve[angleA=270,angleB=180,ncurv=1]{->}{FfF1}{Ff1} \Bput{F_{f,1}}
	\end{diagram}
	The top row is equal to $Ff\o F_{A}\o F1$ by definition, hence
	this diagram is equivalent to
	\begin{diagram}
		Ff\o 1\o F1 & \rTo^{Ff\o F_{A}\o F1} & Ff\o F1\o F1 \\
		\dTo<{Ff\o\l_{F1}} && \dTo>{F_{f,1}\o F1} \\
		Ff\o F1 & \lTo_{F(\r_{f})\o F1} & F(f\o 1)\o F1
	\end{diagram}
	and since $F1$ is co-fully-faithful, it follows that $\cFl$ holds.
	By Lemma~\ref{lemma-psnat-unit-only-one}, it follows that condition
	$\cFr$ holds too.
	
	If we start instead with the assumption that $F(1_{A})$ is
	fully-faithful, the dual argument applies.
	Finally, the uniqueness is a consequence of Lemma~\ref{lemma-psnat-unit-uq}.
\end{proof}

\section{The bicategorical Yoneda lemma}
The Yoneda lemma for bicategories was first stated by \citet{FIB},
in a long paper that states many basic results without proof. Since
the proof, even if it is in some sense routine, is rather intricate,
we give a detailed account here.
\begin{definition} % representable pseudofunctors
	Let $A$ be an object of the bicategory $\B$. The \emph{covariant
	representable pseudo-functor determined by $A$},
	\[
		\B(A,-): \B\to\Cat,
	\]
	is defined as follows. On an object $X\in\B$, the category $\B(A,X)$ is
	just the the hom-category of the same name. The action
	\[\B(A,-)_{X,Y}: \B(X,Y)\to[\B(A,X),\B(A,Y)]\] is defined to be the
	currying of the composition functor \[\o:\B(X,Y)\x\B(A,X)\to\B(A,Y).\]
	
	For an object $X\in\B$, the unit isomorphism $\B(A,-)_X: 1_{\B(A,X)}\To1_X\o-$
	is defined to be $\l^{-1}$, and for a composable pair
	\[
		X \rTo^f Y \rTo^g Z
	\]
	in $\B$, the isomorphism $\B(A,-)_{g,f}: \B(A,g)\o\B(A,f)\To\B(A,g\o f)$
	is defined to be $\a_{g,f,-}$. I.e.\ for $x\in\B(A,X)$, the component
	$(\B(A,-)_{g,f})_x: g\o(f\o x)\to (g\o f)\o x$ is just $\a_{g,f,x}$.
	
	By duality there is also a \emph{contravariant representable pseudo-functor},
	$\B(-,A) \defeqto \B\op(A,-) : \B\op\to\Cat$.
\end{definition}
\begin{propn}\label{prop-yoneda}
	For any bicategory $\B$, pseudo-functor $F: \B\to\Cat$
	and object $A\in\B$, there is an equivalence of categories
	\[
%		\psi^F_A: FA \simeq \Bicat(\B,\Cat)(\B(A,-), F).
		\psi: FA \simeq \Bicat(\B,\Cat)(\B(A,-), F).
	\]
\end{propn}
\begin{proof}
	Fix $F$ and $A$. We shall define a functor
	\[
		\psi: FA \to \Bicat(\B,\Cat)(\B(A,-), F),
	\]
	and show that it is an equivalence.
	% We omit the subscript and superscript
	% on $\psi$, in a vain effort to keep the notation under control.
	(This gets rather dizzying, so take a deep breath.)
	First we shall define $\psi$ on objects: for any $a\in FA$, we need a
	pseudo-natural transformation $\psi(a): \B(A,-)\To F$. Thus for every
	object $X\in\B$, we must give a functor
	\[
		\psi(a)_X: \B(A,X)\to FX.
	\]
	This functor is defined as follows. On objects: for $f\in\B(A,X)$, let
	$\psi(a)_X(f) \defeqto F(f)(a)$. On morphisms: for $\beta: f\To g: A\to X$,
	let $\psi(a)_X(\beta) \defeqto F(\beta)_a$. (Note that $F(\beta)$ is a natural
	transformation $F(f)\To F(g)$, whose component $F(\beta)_a$ is
	therefore indeed a map $F(f)(a)\to F(g)(a)$.)
	
	Also, for every 1-cell $k: X\to Y$ in $\B$, we must give a natural
	isomorphism
	\[
		\psi(a)_k: \psi(a)_Y\cdot\B(A,k) \To F(k)\cdot\psi(a)_X.
	\]
	For $x\in\B(A,X)$, we define the component
		\[(\psi(a)_k)_x: F(k\o x)(a)\to F(k)(F(x)(a))\]
	to be $(F_{k,x}^{-1})_a$.
	%
	The naturality of $F_{k,x}$ ensures that $\psi(a)_k$ is natural.
	
	This completes the definition of $\psi$ on objects, though we must
	check that $\psi(a)$ is indeed a pseudo-natural transformation.
	%
	That follows from the pseudo-functoriality of $F$, as the reader
	may verify: for $\psi(a)$ to satisfy the unit condition is precisely
	for $F$ to satisfy condition~$\cFl$, and for $\psi(a)$ to satisfy the composition
	condition is precisely for $F$ to satisfy~$\cFa$.
	
	Next we define $\psi$ on morphisms. For each morphism $h:a\to b$ in $FA$,
	we must define a modification
	\(
		\psi(h): \psi(a) \Tto \psi(b),
	\)
	so for each $X\in\B$ we need a natural transformation
	\(
		\psi(h)_X: \psi(a)_X \To \psi(b)_X,
	\)
	which means that for every $f:A\to X$ in $\B$ we require a map
	\(
		(\psi(h)_X)_f: F(f)(a) \to F(f)(b)
	\)
	in the category $FX$. So we define $(\psi(h)_X)_f$ to be the map $F(f)(h)$.
	It's easy to check that this makes $\psi(h)_X$ into a natural transformation.
	%
	To complete the definition of $\psi$, we must
	confirm that $\psi(h)$ is indeed a modification. This amounts to
	checking that, for every $f: A\to X$ and $k:X\to Y$ in $\B$, and every
	$h:a\to b$ in $FA$, the square
	\begin{diagram}
		F(kf)(a) & \rTo^{(F_{k,f}^{-1})_a} & (Fk \cdot Ff)(a)\\
		\dTo<{F(kf)(h)} && \dTo>{(Fk\cdot Ff)(h)}\\
		F(kf)(b) & \rTo_{(F_{k,f}^{-1})_b} & (Fk \cdot Ff)(b)
	\end{diagram}
	commutes, which is of course precisely the naturality of $F_{k,f}^{-1}$.
	
	We have defined $\psi$, and need to check that it is indeed a
	functor. Consider the modification $\psi(1_{a}): \psi(a)\Tto\psi(a)$:
	for $f:A\to X$ we have $(\psi(h)_{X})_{f} = F(f)(1_{a})$, and
	since $F(f)$ is a functor, this is equal to $1_{F(f)(a)}$ as
	required.
	%
	For composition, a similar argument applies: given arrows
	$h:a\to b$ and $j:b\to c$ in $FA$, we have
	$(\psi(jh)_{X})_{f} = F(f)(jh)$; and since $F(f)$ is a functor,
	this is equal to $F(f)(j)\cdot F(f)(h)$ as required.
	
	It remains to show that $\psi$ is an equivalence. We begin by exhibiting
	a local inverse, showing that $\psi$ is full and faithful. Fix objects $a$
	and $b\in FA$. We shall define a function $\psi_{a,b}^{-1}$ from the set
	of modifications $\psi(a)\Tto\psi(b)$ to the set $FA(a,b)$, and show that
	it is inverse to $\psi_{a,b}$. The definition is as follows. For a modification
	$\mu: \psi(a)\Tto\psi(b)$, let $\psi_{a,b}^{-1}(\mu)$ be the composite
	\[
		a = 1_{FA}(a) \rTo^{(F_A)_a} F(1_A)(a)
			\rTo^{(\mu_A)_{1_A}} F(1_A)(b) \rTo^{(F_A^{-1})_b} 1_{FA}(b) = b.
	\]
	It is easy to check that, for any $h: a\to b$, we have $\psi_{a,b}^{-1}(\psi(h))= h$:
	indeed it is immediate from the definition of $\psi$, and the naturality of $F_A$.
	The other direction is more interesting. Fix some $\mu: \psi(a)\Tto\psi(b)$, and
	take $X\in\B$ and $f:A\to X$. We wish to show that 
	$(\psi(\psi_{a,b}^{-1}(\mu))_X)_f$ is equal to $(\mu_X)_f$. Consider the diagram
	\begin{diagram}[h=2em,w=4em]
		&&(Ff.1_{FA})(a)\\
		&\ruTo^= && \rdTo^{Ff((F_A)_a)}\\
		F(f)(a) && \cFr && (Ff.F1_A)(a)\\
		&\rdTo^{F(\r_f^{-1})(a)} && \ruTo^{(F_{f,1_A})_a}\\
		&&F(f.1_A)(a)\\
		\dTo<{(\mu_X)_f} &\natural_{\mu_X}& \dTo>{(\mu_X)_{f.1_A}} &\qquad\S_\mu& \dTo>{(Ff)((\mu_A)_{1_A})}\\
		&&F(f.1_A)(b)\\
		&\ldTo^{F(\r_f)(b)} && \luTo^{(F_{f,1_A})_b}\\
		F(f)(b) && \cFr && (Ff.F1_A)(b)\\
		&\luTo_= && \ldTo_{Ff((F_A^{-1})_b)}\\
		&&(Ff.1_{FA})(b)
	\end{diagram}
	whose regions commute for the reasons marked: $\natural_{\mu_X}$ means that
	the square commutes because $\mu_X$ is natural, and $\S_\mu$ means that the
	square commutes because $\mu$ is a modification.
	
	The composite around the upper, right, and bottom edges is equal to
	$(\psi(\psi_{a,b}^{-1}(\mu))_X)_f$ by definition, which is therefore equal
	to $(\mu_X)_f$ as required. Thus $\psi$ is indeed full and faithful. It remains
	only to show that $\psi$ is essentially surjective on objects. Consider an
	arbitrary pseudo-natural transformation
	\(
		\gamma: \B(A,-)\To F.
	\)
	We intend to show that $\psi(\gamma_A(1_A))$ is isomorphic to $\gamma$.
	For any $X\in\B$ and $f:A\to X$, we have an invertible 2-cell
	\begin{diagram}
		\B(A,A) & \rTo^{\gamma_A} & FA \\
		\dTo<{\B(A,f)} & \Nearrow\gamma_f & \dTo>{Ff}\\
		\B(A,X) & \rTo_{\gamma_X} & FX
	\end{diagram}
	thus an isomorphism
	\[
		\gamma_X(f) \rTo^{\gamma_X(\r_f^{-1})} \gamma_X(f.1_A)
			\rTo^{(\gamma_f)_{1_A}} F(f)(\gamma_A(1_A)) = \psi(\gamma_A(1_A))_X(f).
	\]
	This isomorphism is natural in $f$, since $\r_f$ and $\gamma_f$ both are.
	So, for every $X\in\B$ we have defined a natural isomorphism
	$\gamma_X \To \psi(\gamma_A(1_A))_X$. Finally it remains to check
	that this collection constitutes a modification. Take $k:X\to Y$: we need to
	check the commutativity of the diagram
	\begin{equation}\label{diag-ymod}
	\begin{diagram}
		\gamma_Y(kf) & \rTo^{(\gamma_k)_f} & F(k)(\gamma_X(f))\\
		\dTo<{\gamma_Y(\r_{kf}^{-1})} && \dTo>{Fk(\gamma_X(\r_f^{-1}))}\\
		\gamma_Y((kf)1) && Fk(\gamma_X(f1))\\
		\dTo<{(\gamma_{kf})_1} && \dTo>{Fk((\gamma_f)_1)}\\
		F(kf)(\gamma_A(1)) & \rTo_{(F_{k,f}^{-1})_{\gamma_A(1)}} & (Fk\cdot Ff)(\gamma_A(1))
	\end{diagram}
	\end{equation}
	
	Since $\gamma$ is pseudo-natural, we know that
	\[\hskip-1cm
	\begin{diagram}
		\B(A,A) & \rTo^{\gamma_A} & \rnode{FA}{FA}\\
		\dTo<{\B(A,f)} & \Nearrow\gamma_f & \dTo>{Ff}\\
		\B(A,X) & \rTo_{\gamma_X} & FX\\
		\dTo<{\B(A,k)} & \Nearrow\gamma_k & \dTo>{Fk}\\
		\B(A,Y) & \rTo_{\gamma_Y} & \rnode{FY}{FY}
		\nccurve{->}{FA}{FY}\Aput{F(kf)}
	\end{diagram}
	\hskip4em=
	\begin{diagram}
		&& \rnode{BAA}{\B(A,A)} & \rTo^{\gamma_A} & FA \\
		&\ldTo^{\B(A,f)}\\
		\B(A,X) & \hbox to1em{$\begin{array}c\To\\[-4pt]\a_{k,f,-}\end{array}$\hss}
			&\dTo[snake=1em]>{\B(A,kf)} && \dTo>{F(kf)}\\
		&\rdTo_{\B(A,k)} && \raise1em\hbox{$\Nearrow\gamma_{kf}$} \\
		&& \rnode{BAY}{\B(A,Y)} & \rTo_{\gamma_Y} & FY
		%\nccurve{->}{BAA}{BAY}\Bput{\B(A,kf)}
	\end{diagram}
	\]
	hence in particular that the diagram
	\refstepcounter{equation}
	\begin{diagram}[eqno=\textup(\theequation\textup)]\label{diag-yon1}
		\gamma_Y(k(f1)) & \rTo^{\gamma_k)_{f1}} & Fk(\gamma_X(f1))\\
		&&\dTo>{Fk((\gamma_f)_1)}\\
		\dTo<{\gamma_Y(k)} && (Fk\cdot Ff)(\gamma_A(1))\\
		&&\dTo>{(F_{k,f})_{\gamma_A(1)}}\\
		\gamma_Y((kf)1) & \rTo_{(\gamma_{kf})_1} & F(kf)(\gamma_A(1))
	\end{diagram}
	commutes. Now we have
	\begin{diagram}
		\gamma_Y(kf) && \rTo^{(\gamma_k)_f} && F(k)(\gamma_X(f))\\
		\dTo<{\gamma_Y(\r_{kf}^{-1})} &\rdTo^{\gamma_Y(k\r_f^{-1})}&&& \dTo>{Fk(\gamma_X(\r_f^{-1}))}\\
		\gamma_Y((kf)1) &\rTo_{\gamma_Y(\a_{k,f,1}^{-1})}&\gamma_Y(k(f1))
			&\rTo_{(\gamma_k)_{f1}}& Fk(\gamma_X(f1))\\
		\dTo<{(\gamma_{kf})_1} &&&& \dTo>{Fk((\gamma_f)_1)}\\
		F(kf)(\gamma_A(1)) && \rTo_{(F_{k,f}^{-1})_{\gamma_A(1)}} && (Fk\cdot Ff)(\gamma_A(1))
	\end{diagram}
	where the triangle commutes by coherence, the upper-right quadrilateral by
	naturality of $\gamma_k$ and the lower region by \pref{diag-yon1}.
	Thus diagram \pref{diag-ymod} does indeed commute, and we are done.
\end{proof}
%\begin{remark}
%	The components $\psi^F_A$ may be extended to a pseudo-natural
%	transformation $\psi^F$, once the rhs has been made pseudo-functorial
%	in the appropriate way. But we won't go there for now.
%\end{remark}
\begin{definition}
	For any bicategory $\B$, the \emph{bicategorical Yoneda embedding}
	\[
		Y: \B\op\to\Bicat(\B,\Cat)
	\]
	is defined as follows.
	On objects $A\in\B$, we define $YA \defeqto \B(A,-)$; and
	on hom-categories $\B(A,B)$ we define the component
	\[
		Y_{B,A} : \B\op(A,B) = \B(B,A) \to \Bicat(\B,\Cat)(\B(A,-), \B(B,-))
	\]
	to be $\psi^{\B(B,-)}_A$.
\end{definition}
\begin{corollary}
	The Yoneda embedding is locally an equivalence.
\end{corollary}
\begin{proof}
	Immediate from the definition, by Prop.~\ref{prop-yoneda}.
\end{proof}
\begin{remark}
	Since $\Cat$ is a 2-category, so is $\Bicat(\B\op,\Cat)$.
	And since the Yoneda embedding $Y$ is locally an equivalence, any
	bicategory $\B$ is biequivalent to the full sub-bicategory of
	$\Bicat(\B\op,\Cat)$ determined by the objects $YA$ for $A\in\B$, which
	is of course still a 2-category. Thus any bicategory is biequivalent
	to a 2-category. This is a simple coherence result, which can serve
	as a stepping-stone to more sophisticated coherence theorems
	\citep[Chapter~2]{GurskiThesis}.
\end{remark}

\section{Adjunctions}
\begin{definition} % adjunction in a bicategory
	An \emph{adjunction} in a bicategory $\B$ consists of 1-cells
	$f: A\to B$ and $g: B\to A$, and 2-cells $\eta: 1_A\To gf$ and
	$\e: fg\To 1_B$ with the property that
	\[
	\begin{diagram}
		A & \rTo^{f} & B \\
		\dTo<{1_{A}}>{\raise1em\rlap{$\mkern10mu
			\begin{array}c\Rightarrow\\[-4pt]\eta\end{array}$}}
			& \ldTo[snake=1em]_{g}
			& \dTo>{1_{B}}<{\raise-1.5em\llap{$
				\begin{array}c\Rightarrow\\[-4pt]\e\end{array}\mkern10mu$}}\\
		A & \rTo_{f} & B
	\end{diagram}
	\quad=\quad
	\begin{diagram}
		A & \rTo^{f} & B \\
		\dTo<{1_{A}} & \cong & \dTo>{1_{B}}\\
		A & \rTo_{f} & B
	\end{diagram}
	\]
	and
	\[
	\begin{diagram}
		B & \rTo^{g} & A \\
		\dTo<{1_{B}}>{\raise1em\rlap{$\mkern10mu
			\begin{array}c\Leftarrow\\[-4pt]\e\end{array}$}}
			& \ldTo[snake=1em]_{f}
			& \dTo>{1_{A}}<{\raise-1.5em\llap{$
				\begin{array}c\Leftarrow\\[-4pt]\eta\end{array}\mkern10mu$}}\\
		B & \rTo_{g} & A
	\end{diagram}
	\quad=\quad
	\begin{diagram}
		B & \rTo^{g} & A \\
		\dTo<{1_{B}} & \cong & \dTo>{1_{A}}\\
		B & \rTo_{g} & A
	\end{diagram}
	\]
	We write $f\dashv g$ to indicate that there is such an adjunction, and say that
	$f$ is \emph{left adjoint} to $g$, and $g$ is \emph{right adjoint} to $f$. We
	also write $f\dashv g : A \to B$ to mean that $f\dashv g$ for $f: A\to B$
	and $g: B\to A$.
\end{definition}
\begin{remark} % identity and composition of adjunctions
	There is an \emph{identity adjunction} on every object $A$,
	which is just $1_{A} \dashv 1_{A}$, with the unit and counit
	being structural isomorphisms.
	
	Also, adjunctions may be composed: given adjunctions
	$f\dashv g: A \to B$ and $f'\dashv g': B\to C$, there
	is a composite adjunction $f'\o f \dashv g\o g'$ with
	unit
	\[
	\begin{diagram}
		A & \rTo^{f} & B & \rTo^{f'} & C \\
		\dTo<1 & \raise 2em\llap{$\Right_{\eta}$}
			\ldTo_{g}
			\raise -2em\rlap{$\cong$}
			& \dTo>1
			& \raise 2em\llap{$\Right_{\eta}$}
			\ldTo_{g'} \\
			A & \lTo_{g} & B
	\end{diagram}
	\mbox{and counit}
	\begin{diagram}
		A & \rTo^{f} & B & \rTo^{f'} & C \\
		& \luTo_{g} \raise 2em \rlap{$\Left_{\e}$} & \dTo<1
			& \luTo_{g'} \raise2em\rlap{$\Left_{\e'}$}
			\raise-2em\llap{$\cong$}
			& \dTo>1 \\
		&& B & \lTo_{g'} & B
	\end{diagram}
	\]
%
	Indeed, every bicategory $\B$ has a \emph{bicategory of adjunctions},
	whose objects are the objects of $\B$ and whose 1-cells are adjunctions.
\end{remark}
%
\begin{remark}\label{rem-adj-functor} % Pseudofunctors preserve adjunctions
	Pseudofunctors preserve adjunctions, in the following sense.
	If $F:\B\to\BC$ is a pseudo-functor, and
	$f\dashv g: A\to B$ is an adjunction in $\B$,
	then there is an adjunction $Ff \dashv Fg$ in $\BC$
	with the following unit and counit:
	\[
		\begin{diagram}
			&\rnode{t}{FA} & \rTo^{Ff} & FB \\
			\Right_{F_{A}}\hskip-3em&\dTo>{F1} & \ldTo_{Fg}
				\raise 2em\llap{$\Right_{F\eta}$} \\
			&\rnode{b}{FA}
			%
			\ncarc[arcangle=-80,ncurv=1]{->}tb\Bput{1}
		\end{diagram}
		\hskip3em
		\begin{diagram}
			FA & \lTo^{Fg} & \rnode{t}{FB} \\
			& \rdTo_{Ff}
				 \raise 2em\rlap{$\Right_{F\e}$}
				&\dTo<{F1} & \hskip-3em\Right_{F_{B}^{-1}}\\
			&&\rnode{b}{FB}
			%
			\ncarc[arcangle=80,ncurv=1]{->}tb\Aput{1}
		\end{diagram}
	\]
\end{remark}
\subsection{Mates}
The theory of mates \citep[\S2]{ks1} is a useful tool
for dealing with adjunctions in a bicategory. We shall give
a brief overview here.
\begin{definition}\label{def-mate}
	Let $f \dashv g: A \to B$ and $f' \dashv g': A' \to B'$.
	Given a 2-cell
	\begin{diagram} % sigma
		A & \rTo^{f} & B \\
		\dTo<{h} & \Arr\Nearrow\sigma & \dTo>{k} \\
		A' & \rTo_{f'} & B'
	\end{diagram}
	we may form a 2-cell
	\begin{diagram} % tau
		A & \lTo^{g} & B \\
		\dTo<{h} & \Arr\Searrow\tau & \dTo>{k} \\
		A' & \lTo_{g'} & B'
	\end{diagram}
	as the pasting
	\begin{diagram} % definition of right mate of sigma
	\rnode{A}{A} & \lTo^{g} & \rnode{B}{B}\\
	\dTo<{h}&\rdTo_{f}^{\raise4pt\hbox{$\begin{array}c\To\\[-4pt]\e\end{array}$}}
		& \dTo>1\\
	A'&\begin{array}c\To\\[-4pt]\sigma\end{array}&B\\
	\dTo<1
		& \rdTo^{f'}_{\raise-4pt\hbox{$\begin{array}c\To\\[-4pt]\eta'\end{array}$}}
		& \dTo>{k}\\
	\rnode{A'}{A'} & \lTo_{g'} & \rnode{B'}{B'.}
	%
	\nccurve[angleA=210,angleB=140]{->}{A}{A'}\Bput{h}\Aput{\ \ \ \cong}
	\nccurve[angleA=-30,angleB=30]{->}{B}{B'}\Aput{k}\Bput{\cong\ \ \ }
	\end{diagram}
	We say that $\tau$ is the \emph{right mate} of $\sigma$.
	Conversely, given a 2-cell $\tau$ as above, we may form its
	\emph{left mate} as
	\begin{diagram} % definition of left mate of tau
	\rnode{A}{A} & \rTo^{f} & \rnode{B}{B}\\
	\dTo<{1}&\ldTo_{g}^{\raise4pt\hbox{$\begin{array}c\To\\[-4pt]\eta\end{array}$}}
		& \dTo>k\\
	A&\begin{array}c\To\\[-4pt]\tau\end{array}&B'\\
	\dTo<h
		& \ldTo^{g'}_{\raise-4pt\hbox{$\begin{array}c\To\\[-4pt]\e'\end{array}$}}
		& \dTo>{1}\\
	\rnode{A'}{A'} & \rTo_{f'} & \rnode{B'}{B'}
	%
	\nccurve[angleA=210,angleB=140]{->}{A}{A'}\Bput{h}\Aput{\ \ \ \cong}
	\nccurve[angleA=-30,angleB=30]{->}{B}{B'}\Aput{k}\Bput{\cong\ \ \ }
	\end{diagram}
\end{definition}
%
\begin{propn}\label{prop-mate-inv}
	The `left mate' and `right mate' operations are mutually
	inverse, i.e.\ $\sigma$ is the left mate of $\tau$ if and
	only if $\tau$ is the right mate of $\sigma$.
	In this case we say that \emph{$\sigma$ and $\tau$ are
	mates} (with respect to the adjunctions $f\dashv g$ and $f'\dashv g'$).
\end{propn}
\begin{proof}
	This follows easily from the definition of adjunction and
	the coherence of structural isomorphisms.
\end{proof}
%
Matehood can be characterised in terms of either the units or
counits of the adjunctions.
\begin{propn} % characterisation of mates in terms of units or counits
	The 2-cells $\sigma$ and $\tau$ are mates if and only if
	\begin{equation}\label{eq-mate-eta}
	\begin{diagram}
		\rnode{A}{A}\\
		\dTo<{h}&\rdTo^{f} \\
		A'&\begin{array}c\To\\[-4pt]\sigma\end{array}&B\\
		\dTo<1
			& \rdTo^{f'}_{\raise-4pt\hbox{$\begin{array}c\To\\[-4pt]\eta'\end{array}$}}
			& \dTo>{k}\\
		\rnode{A'}{A'} & \lTo_{g'} & B'
		%
		\nccurve[angleA=220,angleB=140]{->}{A}{A'}\Bput{h}\Aput{\ \ \ \cong}
	\end{diagram}
	\qquad=\qquad
	\begin{diagram}
		& \rnode{A}{A} & \rTo^{f} & \rnode{B}{B}\\
		& \dTo<{1}&\ldTo_{g}^{\raise4pt\hbox{$\begin{array}c\To\\[-4pt]\eta\end{array}$}}
		     & \dTo>k\\
		& A &\begin{array}c\To\\[-4pt]\tau\end{array}&B'\\
		& \dTo<h & \ldTo_{g'} \\
		& \rnode{A'}{A'}
		%
		\nccurve[angleA=220,angleB=140]{->}{A}{A'}\Bput{h}\Aput{\ \ \ \cong}
	\end{diagram}
	\end{equation}
	if and only if
	\begin{equation}\label{eq-mate-eps}
	\begin{diagram}
	A & \lTo^{g} & \rnode{B}{B} &\\
	\dTo<{h}&\rdTo_{f}^{\raise4pt\hbox{$\begin{array}c\To\\[-4pt]\e\end{array}$}}
		& \dTo>1 &\\
	A'&\begin{array}c\To\\[-4pt]\sigma\end{array}&B& \\
	& \rdTo_{f} & \dTo>{k} & \\
	& & \rnode{B'}{B'} &
	%
	\nccurve[angleA=-30,angleB=30]{->}{B}{B'}\Aput{k}\Bput{\cong\ \ \ }
	\end{diagram}
	\qquad=\qquad
	\begin{diagram}
	& & \rnode{B}{B}\\
	& \ldTo^{g} & \dTo>k\\
	A & \begin{array}c\To\\[-4pt]\tau\end{array}&B'\\
	\dTo<h
		& \ldTo^{g'}_{\raise-4pt\hbox{$\begin{array}c\To\\[-4pt]\e'\end{array}$}}
		& \dTo>{1}\\
	A' & \rTo_{f'} & \rnode{B'}{B'}
	%
	\nccurve[angleA=-30,angleB=30]{->}{B}{B'}\Aput{k}\Bput{\cong\ \ \ }
	\end{diagram}
	\hskip4em
	\end{equation}
\end{propn}
\begin{proof}
	By definition we have
	\[
	\begin{diagram} % sigma
		A & \rTo^{f} & B \\
		\dTo<{h} & \Arr\Nearrow\sigma & \dTo>{k} \\
		A' & \rTo_{f'} & B'
	\end{diagram}
	\qquad=\qquad
	\begin{diagram} % definition of left mate of tau
	\rnode{A}{A} & \rTo^{f} & \rnode{B}{B}\\
	\dTo<{1}&\ldTo_{g}^{\raise4pt\hbox{$\begin{array}c\To\\[-4pt]\eta\end{array}$}}
		& \dTo>k\\
	A&\begin{array}c\To\\[-4pt]\tau\end{array}&B'\\
	\dTo<h
		& \ldTo^{g'}_{\raise-4pt\hbox{$\begin{array}c\To\\[-4pt]\e'\end{array}$}}
		& \dTo>{1}\\
	\rnode{A'}{A'} & \rTo_{f'} & \rnode{B'}{B'}
	%
	\nccurve[angleA=210,angleB=140]{->}{A}{A'}\Bput{h}\Aput{\ \ \ \cong}
	\nccurve[angleA=-30,angleB=30]{->}{B}{B'}\Aput{k}\Bput{\cong\ \ \ }
	\end{diagram}
	\]
	Onto both sides, we paste the 2-cell
	\begin{diagram}[h=2em]
		A & \rTo^{h} & A' \\
		&\rdTo_{h} \raise 1em\rlap{$\cong$} & \dTo>1
			& \rdTo^{f'} \raise-1em\llap{$\Right_{\eta'}$} \\
		&& A' & \lTo_{g'} & B'
	\end{diagram}
	along the edge $A \rTo^{h} A' \rTo^{f'}B'$,
	then use coherence (on the right) to deduce \pref{eq-mate-eta}.
	
	Similarly we can deduce \pref{eq-mate-eps} by taking the
	equation displayed above and pasting the 2-cell
	\begin{diagram}[h=2em]
		A & \rTo^{f} & B \\
		&\luTo_{g} \raise 1em\rlap{$\Right_{\eta}$} & \dTo>1
			& \rdTo^{k} \raise-1em\llap{$\cong$} \\
		&& B & \rTo_{k} & B'
	\end{diagram}
	onto both sides along the edge $A\rTo^{f} B\rTo^{k}B'$.
\end{proof}
%
Next we list some useful elementary properties of mates.
\begin{propn}\label{prop-mates}
	Mates have the following properties:
	\begin{enumerate}
		\item Mating is natural in $h$ and $k$, i.e.\ if $\sigma$
			and $\tau$ are mates then so are
			\[
			\begin{diagram} % sigma
				&\rnode{A}{A} & \rTo^{f} & \rnode{B}{B} \\
				\Right_{\gamma}\hskip-2.2em
					& \dTo<{h}
					& \Arr\Nearrow\sigma & \dTo>{k}
					& \hskip-2.3em\Right_{\delta} \\
				&\rnode{A'}{A'} & \rTo_{f'} & \rnode{B'}{B'}
				%
				\ncarc[arcangle=-90,ncurv=1]{->}{A}{A'}\Bput{h'}
				\ncarc[arcangle=90,ncurv=1]{->}{B}{B'}\Aput{k'}
			\end{diagram}
			\qquad\mbox{and}\qquad
			\begin{diagram}
				&\rnode{A}{A} & \lTo^{g} & \rnode{B}{B} \\
				\Right_{\gamma}\hskip-2.2em
					& \dTo<{h}
					& \Arr\Searrow\tau & \dTo>{k}
					& \hskip-2.3em\Right_{\delta} \\
				&\rnode{A'}{A'} & \lTo_{g'} & \rnode{B'}{B'}
				%
				\ncarc[arcangle=-90,ncurv=1]{->}{A}{A'}\Bput{h'}
				\ncarc[arcangle=90,ncurv=1]{->}{B}{B'}\Aput{k'}
			\end{diagram}
			\]
			for all appropriately-typed 2-cells $\gamma$ and $\delta$.
			\label{mate-natural}
		\item Mating preserves horizontal pasting:
			if $\sigma_{1}$ and $\tau_{1}$ are mates with respect to the
			adjunctions $f_{1} \dashv g_{1}$ and $f'_{1} \dashv g'_{1}$,
			and $\sigma_{2}$ and $\tau_{2}$ are mates with respect to
			$f_2 \dashv g_2$ and $f_2' \dashv g_2'$, then
			\begin{diagram} % sigma_{1}, sigma_{2}
				A & \rTo^{f_{1}} & B & \rTo^{f_2} & C\\
				\dTo<{h} & \Arr\Nearrow{\sigma_{1}} & \dTo>{k}
				 	& \Arr\Nearrow{\sigma_2} & \dTo>n\\
				A' & \rTo_{f_{1}'} & B' & \rTo_{f_{2}'} & C'
			\end{diagram}
			and
			\begin{diagram} % tau_{1}, tau_{2}
				A & \lTo^{g_{1}} & B & \lTo^{g_2} & C\\
				\dTo<{h} & \Arr\Searrow{\tau_{1}} & \dTo>{k}
				 	& \Arr\Searrow{\tau_2} & \dTo>n\\
				A' & \lTo_{g_{1}'} & B' & \lTo_{g_{2}'} & C'
			\end{diagram}
			are mates, with respect to the adjunctions
			$f_{2}\o f_{1} \dashv g_{1}\o g_{2}$
			and $f_{2}'\o f_{1}' \dashv g_{1}'\o g_{2}'$.
			\label{mate-horiz}
		\item Mating preserves vertical pasting: if $\sigma$ and $\tau$
			are mates with respect to $f\dashv g$ and $f'\dashv g'$,
			and $\sigma'$ and $\tau'$
			are mates with respect to $f'\dashv g'$ and $f''\dashv g''$,
			then
			\[
			\begin{diagram} % sigma
				A & \rTo^{f} & B \\
				\dTo<{h} & \Arr\Nearrow\sigma & \dTo>{k} \\
				A' & \rTo_{f'} & B' \\
				\dTo<{h'} &  \Arr\Nearrow{\sigma'} & \dTo>{k'} \\
				A'' & \rTo_{f''} & B''
			\end{diagram}
			\qquad\mbox{and}\qquad
			\begin{diagram} % tau
				A & \lTo^{g} & B \\
				\dTo<{h} & \Arr\Searrow\tau & \dTo>{k} \\
				A' & \lTo_{g'} & B' \\
				\dTo<{h'} & \Arr\Searrow{\tau'} & \dTo>{k'} \\
				A'' & \lTo_{g''} & B''
			\end{diagram}
			\]
			are mates with respect to $f\dashv g$ and $f''\dashv g''$.
			\label{mate-vert}
		\item For every adjunction $f\dashv g: A\to B$, the
			structural isomorphisms
			\[
				\begin{diagram}
					A & \rTo^{f} & B \\
					\dTo<1 & \cong & \dTo>1 \\
					A & \rTo_{f} & B
				\end{diagram}
				\mbox{\qquad and\qquad}
				\begin{diagram}
					A & \lTo^{g} & B \\
					\dTo<1 & \cong & \dTo>1 \\
					A & \lTo_{g} & B
				\end{diagram}
			\]
			are mates.
			\label{mate-structural}
	\end{enumerate}
\end{propn}
We omit the (routine) verification of this proposition,
which requires nothing more than the definitions of mate
and adjunction, and the coherence of the unit isomorphisms.
%
The next proposition is essential for some of our applications of
mates in later chapters.
\begin{propn} % f -| g implies gamma_{f} is mates with gamma_{g}
	Let $\gamma: F \To G: \B \to \BC$ be a pseudo-natural transformation,
	and let $f\dashv g: A\to B$ be an adjunction in $\B$. Then
	\[
		\begin{diagram}
			FA & \rTo^{Ff} & FB \\
			\dTo<{\gamma_{A}} & \Arr\Nearrow{\gamma_{f}^{-1}}
				& \dTo>{\gamma_{B}} \\
			GA & \rTo_{Gf} & GB
		\end{diagram}
		\qquad\mbox{and}\qquad
		\begin{diagram}
			FA & \lTo^{Fg} & FB \\
			\dTo<{\gamma_{A}} & \Arr\Searrow{\gamma_{g}}
				& \dTo>{\gamma_{B}} \\
			GA & \lTo_{Gg} & GB
		\end{diagram}
	\]
	are mates with respect to the adjunctions $Ff\dashv Fg$ and $Gf\dashv Gg$.
\end{propn}
\begin{proof}
	Consider the 2-cell
	\begin{diagram}[h=2em]
		\rnode{t}{FA} & \rTo^{\gamma_{A}} & GA \\
		& \rdTo^{Ff} & \Arr\Uparrow{\gamma_{f}} & \rdTo^{Gf} \\
		\dTo<{F1} & \llap{$\Right_{F\eta}$} & FB
			& \rTo^{\gamma_{B}} & GB \\
		&\ldTo_{Fg} & \Arr\Nearrow{\gamma_{g}} & \ldTo_{Gg} \\
		\rnode{b}{FA} & \rTo_{\gamma_{A}} & GA
		%
		\ncarc[arcangle=-80,ncurv=1]{->}tb\Bput{1}
	\end{diagram}
	Since $\gamma$ is pseudo-natural, by the naturality condition
	in Definition~\ref{def-psnat} this is equal to
	\begin{diagram}[h=1.5em]
		\rnode{t}{FA} & \rTo^{\gamma_{A}} & GA \\
		& & & \rdTo^{Gf} \\
		\dTo>{F1} & \Arr\Nearrow{\gamma_{1_{A}}} & \dTo>{G1}
			& \llap{$\Right_{G\eta}$} & GB \\
		& & & \ldTo_{Gg} \\
		\rnode{b}{FA} & \rTo_{\gamma_{A}} & GA
		%
		\ncarc[arcangle=-80,ncurv=1]{->}tb\Bput{1}\Aput{\ \ \Right_{F_{A}}}
	\end{diagram}
	which, by the unit condition, is equal to
	\begin{diagram}[h=1.5em,w=3em]
		FA & \rTo^{\gamma_{A}} & \rnode{t}{GA} \\
		& & & \rdTo^{Gf} \\
		\dTo<{1} & \llap{$\cong$\quad} & \dTo>G1
			& \llap{$\Right_{G\eta}$} & GB \\
		& & & \ldTo_{Gg} \\
		FA & \rTo_{\gamma_{A}} & \rnode{b}{GA}
		%
		\ncarc[arcangle=-60]{->}tb\Bput{1}\Aput{\ \Right_{G_{A}}}
	\end{diagram}
	If we paste $\gamma_{f}^{-1}$ onto the top of this equation,
	and $\r_{\gamma_{A}}^{-1}$ onto the bottom, then we get
	\[
		\begin{diagram}
		& \rnode{t}{FA} \\
		\rlap{\quad$\Right_{F_{A}}$} & \dTo>{F1}
			& \llap{$\Right_{F\eta}$}\rdTo(2,1)^{Ff} & FB \\
		& \rnode{m}{FA} & \ldTo(2,1)_{Fg} & \dTo>{\gamma_{B}} \\
		\raise 2em\hbox{$\cong$} & \dTo<{\gamma_{A}}
			& \raise 1.5em\hbox{$\Arr\Swarrow{\gamma_{g}}$} & GB \\
		& \rnode{b}{GA} & \ldTo(2,1)_{Gg}
		%
		\ncarc[arcangle=-80,ncurv=1]{->}tm\Bput{1}
		\ncarc[arcangle=-90,ncurv=1]{->}tb\Bput{\gamma_{A}}
		\end{diagram}
		\quad=\hskip5em
		\begin{diagram}
		& \rnode{t}{FA} \\
		\raise0em\hbox{$\cong$}& \dTo<{\gamma_{A}} & \rdTo(2,1)^{Ff} & FB \\
		& \rnode{m}{GA} & \raise 2em\hbox{$\Arr\Nearrow{\gamma_{f}^{-1}}$}
			& \dTo>{\gamma_{B}}\\
		\rlap{\quad$\Right_{G_{A}}$} & \dTo>{G1}
			& \llap{$\Right_{G\eta}$}\rdTo(2,1)^{Gf} & FB \\
		& \rnode{b}{GA} & \ldTo(2,1)_{Fg}
		%
		\ncarc[arcangle=-80,ncurv=1]{->}mb\Bput{1}
		\ncarc[arcangle=-90,ncurv=1]{->}tb\Bput{\gamma_{A}}
		\end{diagram}
	\]
	Thus, by equation~\pref{eq-mate-eta} and Remark~\ref{rem-adj-functor},
	$\gamma_{f}^{-1}$ and $\gamma_{g}$ are indeed mates.
\end{proof}
%
A special case of mating that is sometimes useful occurs when the
vertical 1-cells $h$ and $k$ are identities. In this case we may
omit them entirely, and speak of the left mate, or right mate, of
a 2-cell $f'\To f$, for adjunctions $f\dashv g$, $f'\dashv g': A\to B$.
It should be clear what is meant by this: for
example, to take the right mate of a 2-cell $\gamma: f'\To f$, one
forms the composite
\begin{diagram}[s=4em]
	\rnode{A}{A} & \rTo^{f} & B \\
	\dTo<1 & \mathop\Nearrow\limits_{\gamma} & \dTo>1 \\
	A & \rTo_{f'} & \rnode{B}{B}
	\ncarc[arcangle=30]{->}{A}{B} \Aput{f}
	\ncarc[arcangle=30]{->}{B}{A} \Aput{f'}
\end{diagram}
(where the triangular cells contain unit isomorphisms), and takes its
right mate, say
\begin{diagram}
	A & \lTo^{g} & B \\
	\dTo<1 & \Searrow\tau & \dTo>1 \\
	A & \lTo_{g'} & B
\end{diagram}
and then forms the composite
\vskip2em
\begin{diagram}
	A & \lTo^{g} & \rnode{B}{B} \\
	\dTo<1 & \Arr\Searrow\tau & \dTo>1 \\
	\rnode{A}{A} & \lTo_{g'} & B
	\nccurve[angleA=135,angleB=135,ncurv=2]{->}{B}{A}
		\Bput{g} \Aput{\raise -1em\rlap{\quad$\cong$}}
	\nccurve[angleA=-45,angleB=-45,ncurv=2]{->}{B}{A}
		\Aput{g'} \Bput{\raise 0.5em\llap{$\cong$\quad}}
\end{diagram}
\vskip3em
In general it is necessary to distinguish between left mate and right mate
in this situation, because one cannot tell from the context which is intended.
However, see Prop.~\ref{prop-adjeq-mate-dual} below for a case where they coincide.

\subsection{Adjunctions and mates in terms of string diagrams}
\renewcommand\graphicscale{1}
Adjunctions and mates have a particularly elegant string diagram
representation. Let $f\dashv g: A\to B$ be an adjunction with unit
$\eta$ and counit $\e$. Then the unit and counit are drawn as
\[
	\cdiag{d-adj/eta}
	\qquad\mbox{and}\qquad
	\cdiag{d-adj/eps}
\]
The adjunction axioms say precisely that
\[\cdiag{d-adj/snake1} \qquad\mbox{and}\qquad \cdiag{d-adj/snake2}\]
are both identities. Given a 2-cell $\sigma$:
\[\cdiag{d-adj/sigma}\]
its right mate is
\[\cdiag{d-adj/right-mate-of-sigma}\]
and given a 2-cell $\tau$:
\[\cdiag{d-adj/tau}\]
its left mate is
\[\diag{d-adj/left-mate-of-tau}\]

\subsection{Adjoint pseudo-natural transformations}
Now we turn our attention to adjoint pseudo-natural transformations,
i.e.\ adjunctions in a pseudo-functor bicategory.
\begin{propn}\label{prop-adj-1}
	Let there be given an adjoint pair of pseudo-natural
	transformations
	\[
		\phi \dashv \gamma: F \To G : \B\to\BC
	\]
	with unit $\eta: 1\Tto\gamma\phi$
	and counit $\e: \phi\gamma \Tto 1$.
	Then
	\begin{enumerate}
		\item For every object $A\in\B$, there is an adjunction
		\[
			\phi_{A} \dashv \gamma_{A}
		\]
		with unit $\eta_{A}$ and counit $\e_{A}$.
		\item For every 1-cell $f: A\to B$ in $\B$, the
		2-cells
		\[
			\begin{diagram}
				FA & \rTo^{\phi_{A}} & GA \\
				\dTo<{F(f)} & \Arr\Nearrow{\phi_{f}} & \dTo>{G(f)} \\
				FB & \rTo_{\phi_{B}} & GB
			\end{diagram}
			\mbox{\quad and\quad}
			\begin{diagram}
				FA & \lTo^{\gamma_{A}} & GA \\
				\dTo<{F(f)} & \Arr\Searrow{\gamma_{f}^{-1}} & \dTo>{G(f)} \\
				FB & \lTo_{\gamma_{B}} & GB
			\end{diagram}
		\]
		are mates with respect to the adjunctions $\phi_{A}\dashv\gamma_{A}$
		and $\phi_{B}\dashv\gamma_{B}$.
	\end{enumerate}
\end{propn}
\begin{proof}
	The first part is immediate by definition, so let's consider the second.
	Since $\eta$ is a modification, we know that for every $f:A\to B$,
	\[
	\begin{diagram}
		FA & \rTo^{\phi_A} & GA & \rTo^{\gamma_A} & FA\\
		\dTo<{Ff} & \Nearrow\phi_f & \dTo<{Gf} &\Nearrow\gamma_f & \dTo>{Ff}\\
		\rnode{1}{FB} & \rTo_{\phi_B} & GB & \rTo_{\gamma_B} & \rnode{2}{FB}\\
		\nccurve[angleA=-50,angleB=-130]{->}12\Bput1\Aput{\raise6pt\hbox{$\Uparrow\eta_B$}}
	\end{diagram}
	=
	\begin{diagram}[h=2em]
		&&GA\\
		&\ruTo^{\phi_A} & \Uparrow\eta_A & \rdTo^{\gamma_A}\\
		FA && \rTo_1 && FA\\
		\dTo<{Ff} && \cong && \dTo>{Ff}\\
		FB && \rTo_1 && FB
	\end{diagram}
	\]
	Onto both sides of the equation, we paste the 2-cells
	$\gamma_{f}^{-1}$ and $\l^{-1}: Ff\to 1\o Ff$, yielding
	the equation
	\[
	\begin{diagram}
		\rnode{FA}{FA} & \rTo^{\phi_A} & GA\\
		\dTo<{Ff} & \Nearrow\phi_f & \dTo>{Gf}\\
		FB & \rTo^{\phi_B} & GB\\
		\dTo<1>{\raise6pt\hbox{$\ \begin{array}c\To\\[-4pt]\eta_B\end{array}$}} & \ldTo_{\gamma_B}\\
		\rnode{FB}{FB}
		\nccurve[angleA=210,angleB=140]{->}{FA}{FB}\Bput{Ff}\Aput{\ \ \ \cong}
	\end{diagram}
	\quad=\hskip4em
	\begin{diagram}
		\rnode{FA}{FA}\\
		\dTo<1>{\raise-6pt\hbox{$\ \begin{array}c\To\\[-4pt]\eta_A\end{array}$}} & \rdTo^{\phi_A}\\
		FA & \lTo_{\gamma_A} & GA\\
		\dTo<{Ff} & \Searrow\gamma_f^{-1}&\dTo>{Gf}\\
		\rnode{FB}{FB} & \lTo_{\gamma_B} & GB
		\nccurve[angleA=210,angleB=140]{->}{FA}{FB}\Bput{Ff}\Aput{\ \ \ \cong}
	\end{diagram}
	\]
	which, by \pref{eq-mate-eta}, is what we require.
\end{proof}

\begin{propn}\label{prop-adj-2}
	Let there be given a pseudo-natural transformation
	\[
		\phi: F\To G: \B \to \BC.
	\]
	To give a right adjoint $\gamma: G\To F$ for $\phi$ is to give
	\begin{itemize}
		\item for each $A\in\B$, a 1-cell $\gamma_{A}: GA\to FA$ and
			an adjunction $\phi_{A}\dashv \gamma_{A}$,			
		\item such that for every $f:A\to B$ in $\B$, the mate of $\phi_{f}$
			with respect to the adjunctions $\phi_{A}\dashv \gamma_{A}$
			and $\phi_{B}\dashv \gamma_{B}$ is invertible.
	\end{itemize}
\end{propn}
\begin{proof}
	We have already shown (Prop.~\ref{prop-adj-1}) that every
	right-adjoint pseudo-natural transformation $\gamma$ has these
	properties, so suppose that we have a collection of
	adjunctions $\phi_{A}\dashv \gamma_{A}$ as in the statement,
	each with unit $\eta_{A}$ and counit $\e_{A}$, say.
	%
	For each 1-cell $f: A\to B$ in $\B$, define the 2-cell
	\begin{diagram}
		GA & \rTo^{\gamma_{A}} & FA \\
		\dTo<{Gf} & \Arr\Nearrow{\gamma_{f}} & \dTo>{Ff} \\
		GB & \rTo_{\gamma_{B}} & FB
	\end{diagram}
	to be the inverse of the mate of $\phi_{f}$.
	%
	We shall show that these data constitute a pseudo-natural
	transformation $\gamma$, checking the naturality and composition
	conditions
	of Definition~\ref{def-psnat}. This is essentially a
	matter of writing down the corresponding conditions for
	$\phi$ and taking mates.
	\begin{itemize}
	\item The naturality condition for $\phi$ states that
		\[
		\hskip-1em
		\begin{diagram}[size=4em]
			FA & \rTo^{\phi_A} & \rnode{GA}{GA}\\
			\dTo<{Ff} & \llap{$\Nearrow \phi_f$}
				& \begin{array}{c}\Rightarrow\\G(\tau)\end{array}\\
			FB & \rTo_{\phi_B} & \rnode{GB}{GB}
			\ncarc{->}{GA}{GB}\Aput{Gg}
			\ncarc{<-}{GB}{GA}\Aput{Gf}
		\end{diagram}
		\hskip3em=\hskip3em
		\begin{diagram}[size=4em]
			\rnode{FA}{FA} & \rTo^{\phi_A} & GA\\
			\begin{array}{c}\Rightarrow\\F(\tau)\end{array}
				& \rlap{$\Nearrow \phi_g$} & \dTo>{Gg}\\
			\rnode{FB}{FB} & \rTo_{\phi_B} & GB
			\ncarc{->}{FA}{FB}\Aput{Fg}
			\ncarc{<-}{FB}{FA}\Aput{Ff}
		\end{diagram}
		\]
		Taking mates of both sides, by Prop.~\ref{prop-mates}(\ref{mate-natural})
		we have
		\[
		\hskip-1em
		\begin{diagram}[size=4em]
			FA & \lTo^{\gamma_A} & \rnode{GA}{GA}\\
			\dTo<{Ff} & \llap{$\Nearrow \gamma_f^{-1}$}
				& \begin{array}{c}\Rightarrow\\G(\tau)\end{array}\\
			FB & \lTo_{\gamma_B} & \rnode{GB}{GB}
			\ncarc{->}{GA}{GB}\Aput{Gg}
			\ncarc{<-}{GB}{GA}\Aput{Gf}
		\end{diagram}
		\hskip3em=\hskip3em
		\begin{diagram}[size=4em]
			\rnode{FA}{FA} & \lTo^{\gamma_A} & GA\\
			\begin{array}{c}\Rightarrow\\F(\tau)\end{array}
				& \rlap{$\Nearrow \gamma_g^{-1}$} & \dTo>{Gg}\\
			\rnode{FB}{FB} & \lTo_{\gamma_B} & GB
			\ncarc{->}{FA}{FB}\Aput{Fg}
			\ncarc{<-}{FB}{FA}\Aput{Ff}
		\end{diagram}
		\]
		which may be rearranged into the naturality condition for $\gamma$.
	% \item The unit condition for $\phi$ states that
	% 	\[
	% 	\hskip-1em
	% 		\begin{diagram}[size=4em]
	% 			\rnode{FA}{FA} & \rTo^{\phi_A} & GA\\
	% 			\begin{array}{c}\Rightarrow\\F_A\end{array}
	% 				& \hbox to 0pt{$\mkern4mu\Nearrow \phi_{1_A}$} & \dTo>{G(1_A)}\\
	% 			\rnode{FB}{FA} & \rTo_{\phi_A} & \rnode{GA}{GA}
	% 			\ncarc{->}{FA}{FB}\Aput{F(1_A)}
	% 			\ncarc{<-}{FB}{FA}\Aput{1_{FA}}
	% 			%\nccurve{->}{FA}{GA} \Bput{\phi_A}
	% 		\end{diagram}
	% 		%
	% 		\hskip3em=\hskip3em
	% 		%
	% 		\begin{diagram}[size=4em]
	% 			FA & \rTo^{\phi_A} & \rnode{GA}{GA}\\
	% 			\dTo<{1_{FA}} & \llap{$\cong\mkern4mu$}
	% 				& \begin{array}{c}\Rightarrow\\G_A\end{array}\\
	% 			FA & \rTo_{\phi_A} & \rnode{GB}{GA}
	% 			\ncarc{->}{GA}{GB}\Aput{G(1_A)}
	% 			\ncarc{<-}{GB}{GA}\Aput{1_{GA}}
	% 		\end{diagram}
	% 	\]
	% 	Taking mates, and using Prop.~\ref{prop-mates}(\ref{mate-natural}, \ref{mate-structural}), gives
	% 	\[
	% 	\hskip-1em
	% 		\begin{diagram}[size=4em]
	% 			\rnode{FA}{FA} & \lTo^{\gamma_A} & GA\\
	% 			\begin{array}{c}\Rightarrow\\F_A\end{array}
	% 				& \hbox to 0pt{$\mkern4mu\Nearrow \gamma_{1_A}^{-1}$} & \dTo>{G(1_A)}\\
	% 			\rnode{FB}{FA} & \lTo_{\gamma_A} & \rnode{GA}{GA}
	% 			\ncarc{->}{FA}{FB}\Aput{F(1_A)}
	% 			\ncarc{<-}{FB}{FA}\Aput{1_{FA}}
	% 			%\nccurve{->}{FA}{GA} \Bput{\gamma_A}
	% 		\end{diagram}
	% 		%
	% 		\hskip3em=\hskip3em
	% 		%
	% 		\begin{diagram}[size=4em]
	% 			FA & \lTo^{\gamma_A} & \rnode{GA}{GA}\\
	% 			\dTo<{1_{FA}} & \llap{$\cong\mkern4mu$}
	% 				& \begin{array}{c}\Rightarrow\\G_A\end{array}\\
	% 			FA & \lTo_{\gamma_A} & \rnode{GB}{GA}
	% 			\ncarc{->}{GA}{GB}\Aput{G(1_A)}
	% 			\ncarc{<-}{GB}{GA}\Aput{1_{GA}}
	% 		\end{diagram}
	% 	\]
	% 	which may be rearranged into the unit condition for $\gamma$.
	\item The composition condition for $\phi$ states that
		\[\hskip-1cm
		\begin{diagram}[h=2em]
			\rnode{FA}{FA} & \rTo^{Ff}& FB & \rTo^{Fg}& \rnode{FC}{FC}\\
			&&\raise4pt\hbox{$\Downarrow F_{g,f}$}\\
			\dTo<{\phi_A} &&&& \dTo>{\phi_C}\\
			&&\Swarrow \phi_{g\o f}\\
			GA && \rTo_{G(g\o f)} && GC
			\ncarc[arcangle=-60]{->}{FA}{FC}\Bput{F(g\o f)}
		\end{diagram}
		\quad=\quad
		\begin{diagram}[h=2em]
			FA & \rTo^{Ff}& FB & \rTo^{Fg}& FC\\
			&\Swarrow\phi_f & \dTo>{\phi_B}&\Swarrow\phi_g\\
			\dTo<{\phi_A} && GB && \dTo>{\phi_C}\\
			&\ruTo^{Gf} & \Downarrow G_{g,f} & \rdTo^{Gg}\\
			GA && \rTo_{G(g\o f)} && GC
		\end{diagram}
		\]
		Taking mates, and using Prop.~\ref{prop-mates}(\ref{mate-natural}, \ref{mate-vert}), gives
		\[\hskip-1cm
		\begin{diagram}[h=2em]
			\rnode{FA}{FA} & \rTo^{Ff}& FB & \rTo^{Fg}& \rnode{FC}{FC}\\
			&&\raise4pt\hbox{$\Downarrow F_{g,f}$}\\
			\uTo<{\gamma_A} &&&& \uTo>{\gamma_C}\\
			&&\Searrow \gamma_{g\o f}^{-1}\\
			GA && \rTo_{G(g\o f)} && GC
			\ncarc[arcangle=-60]{->}{FA}{FC}\Bput{F(g\o f)}
		\end{diagram}
		\quad=\quad
		\begin{diagram}[h=2em]
			FA & \rTo^{Ff}& FB & \rTo^{Fg}& FC\\
			&\Searrow\gamma_f^{-1} & \dTo>{\gamma_B}&\Searrow\gamma_g^{-1}\\
			\uTo<{\gamma_A} && GB && \uTo>{\gamma_C}\\
			&\ruTo^{Gf} & \Downarrow G_{g,f} & \rdTo^{Gg}\\
			GA && \rTo_{G(g\o f)} && GC
		\end{diagram}
		\]
		which may be rearranged into the composition condition for $\gamma$.
	\end{itemize}
	%
	It remains to show that each of the collections $\eta_{A}$
	and $\e_{A}$ constitutes a modification. By equation
	\pref{eq-mate-eta}, we know that
	\[
	\begin{diagram}
		\rnode{A}{FA}\\
		\dTo<{Ff}&\rdTo^{\phi_{A}} \\
		FB&\begin{array}c\To\\[-4pt]\phi_{f}\end{array} & GA \\
		\dTo<1
			& \rdTo^{\phi_{B}}_{\raise-4pt\hbox{$\begin{array}c\To\\[-4pt]\eta_B\end{array}$}}
			& \dTo>{Gf}\\
		\rnode{A'}{FB} & \lTo_{\gamma_{B}} & GB
		%
		\nccurve[angleA=220,angleB=140]{->}{A}{A'}\Bput{Ff}\Aput{\ \ \ \cong}
	\end{diagram}
	\qquad=\qquad
	\begin{diagram}
		& \rnode{A}{FA} & \rTo^{\phi_{A}} & \rnode{B}{GA}\\
		& \dTo<{1}&\ldTo_{\gamma_{A}}^{\raise4pt\hbox{$\begin{array}c\To\\[-4pt]\eta_{A}\end{array}$}}
		     & \dTo>Gf\\
		& FA &\begin{array}c\To\\[-4pt]\gamma_{f}^{-1}\end{array} & GB \\
		& \dTo<{Ff} & \ldTo_{\gamma_{B}} \\
		& \rnode{A'}{FB}
		%
		\nccurve[angleA=220,angleB=140]{->}{A}{A'}\Bput{Ff}\Aput{\ \ \ \cong}
	\end{diagram}
	\]
	This can be rearranged to give
	\[
		\begin{diagram}
			FB & \rTo^{Ff} & FA \\
			&\rdTo^{\phi_{B}} & \Arr\Swarrow{\phi_{f}} & \rdTo^{\phi_{A}} \\
			\dTo<1 & \Right_{\eta_{B}} & GB & \rTo_{Gf} & GA \\
			& \ldTo_{\gamma_{A}} & \Arr\Searrow{\gamma_{f}} & \ldTo_{\gamma_{A}} \\
			FB & \lTo_{Ff} & FA
		\end{diagram}
		\quad=\quad
		\begin{diagram}
			FB & \lTo^{Ff} & FA \\
			&&&\rdTo^{\phi_A} \\
			\dTo<1 & \cong & \dTo<1 & \Right_{\eta_{A}} & GA \\
			&&&\ldTo_{\gamma_{A}} \\
			FB & \lTo_{Ff} & FA
		\end{diagram}
	\]
	which shows that $\eta$ is a modification.
	%
	Similarly, we may use equation \pref{eq-mate-eps} to show that
	$\e$ is a modification.
\end{proof}

In the bicategory $\Cat$, adjunctions $F \dashv G: \C\to\D$ are
characterised by the existence of a natural isomorphism
\[
	\D(FA, B) \cong \C(A, GB),
\]
natural in $A$ and $B$. It is interesting to observe that a similar
characterisation exists for adjunctions in an \emph{arbitrary} bicategory,
as the next proposition shows.
\begin{propn} % characterisation of adjunctions in a bicategory
	Let there be given 1-cells $f:A\to B$ and $g:B\to A$ in a bicategory $\B$.
	To give an adjunction $f\dashv g$ is to give, for every $A\lTo^a X\rTo^b B$,
	an isomorphism
	\[
		\phi_{a,b}: \B(X,B)(f\o a, b) \cong \B(X,A)(a, g\o b),
	\]
	natural in the sense that:
	\begin{itemize}
	\item	for every $\sigma: a\To a'$, $\tau: b\To b'$,
		and $\zeta: f\o a' \To b$, we have
		\[
		\begin{diagram}[w=2em]
			&&\rnode{X}{X}\\
			&\ldTo[snake=1em]_{a'}
				& \raise-1em\hbox{$\begin{array}c\To\\[-4pt]\phi_{a',b}(\zeta)\end{array}$}
				& \rdTo[snake=-1em]_b\\
			\rnode{A}{A} && \lTo_{g} && \rnode{B}{B}
			\ncarc[arcangle=-50,ncurv=1]{->}XA\Bput{a}
				\Aput{\Searrow\sigma}
			\ncarc[arcangle=50,ncurv=1]{->}XB\Aput{b'}
				\Bput{\Nearrow\tau}
		\end{diagram}
		\qquad=\qquad
		\begin{diagram}[w=2em]
			&&\rnode{X}{X}\\
			&& \begin{array}c\To\\\phi_{a,b'}(\tau\cdot\zeta\cdot(f\o\sigma))\end{array}
				&\\
			\rnode{A}{A} && \lTo_{g} && \rnode{B}{B}
			\ncarc[arcangle=-30]{->}XA\Bput{a}
			\ncarc[arcangle=30]{->}XB\Aput{b'}
		\end{diagram}
	\]
	\item for every $k:Y\to X$ and $\zeta:fa\To b$, we have
	\[
		\begin{diagram}[w=2em]
			&&Y\\
			&&\dTo>k\\
			&&X\\
			&\ldTo^{a} & \begin{array}c\To\\[-4pt]\phi_{a,b}(\zeta)\end{array} & \rdTo^b\\
			A && \lTo_{g} && B
		\end{diagram}
		\quad=\hskip3em
		\begin{diagram}[w=2em]
			&&Y\\
			&\ldTo^k&&\rdTo^k\\
			X&&&&X\\
			\dTo<{a} && \begin{array}c\To\\[-4pt]\phi_{ak,bk}(\zeta\o k)\end{array} && \dTo>b\\
			A && \lTo_{g} && B
		\end{diagram}
	\]
	\end{itemize}
\end{propn}
\begin{proof}
	By Yoneda, we know that to give an adjunction $f\dashv g$ is to give
	an adjunction $\B(-,f)\dashv\B(-,g)$. By Prop.~\ref{prop-adj-2}, we know that
	to give such an adjunction is to give, for every $X\in\B$, an adjunction
	$\B(X,f)\dashv\B(X,g)$, collectively subject to condition \pref{eq-mate-eta}.
	%
	This is just an ordinary adjunction, which can therefore be given as a
	natural isomorphism
	\[
		\B(X,B)(\B(X,f)(a), b) \cong \B(X,A)(a, \B(X,g)(b))
	\]
	natural in $a\in\B(X,A)$ and $b\in\B(X,B)$, i.e.\ a natural isomorphism
	\[
		\B(X,B)(f\o a, b) \cong \B(X,A)(a, g\o b).
	\]
	%
	This corresponds to the data in the statement of this Proposition,
	subject to our first naturality condition. We shall write the unit
	of this adjunction as \[\eta_a: a\To g\o(f\o a)\] for $a:X\to A$.
	
	It remains to show that the second naturality condition is satisfied just
	when \pref{eq-mate-eta} is. Writing down the concrete interpretation
	of~\pref{eq-mate-eta} in our setting, we find that it holds
	when, for every $a:X\to A$ and $k:Y\to X$, the diagram
	\begin{diagram}
		a\o k & \rTo^{\eta_{a\o k}} & g\o (f\o (a\o k))\\
		\dTo<{\eta_a\o k} && \dTo>{g\o\a_{f,a,k}}\\
		(g\o (f\o a))\o k & \rTo_{\a_{g,f\o a,k}} & g\o((f\o a)\o k)
	\end{diagram}
	commutes in $\B(Y,B)$.
	%
	In pictures, this says that
	\[
		\begin{diagram}[w=1em,h=1.5em]
			&& &&Y\\ \\
			&& &&\dTo>k\\ \\
			&& &&\rnode{X}{X}\\
			&& &\ldTo(4,4)^{a} && \rdTo^a\\
			&& &&\begin{array}c\To\\[-4pt]\eta_a\end{array} &&A\\
			&&&&&&&\rdTo^f\\
			A && && \lTo_{g}&&  && \rnode{B}{B}
			\ncarc[arcangle=70,ncurv=1]{->}XB\Aput{b}\Bput{\Nearrow\zeta}
		\end{diagram}
		\qquad=\qquad
		\begin{diagram}[w=2em,h=1.5em]
			&&Y\\
			&\ldTo(2,4)^k&&\rdTo(2,4)^k\\
			\\
			\\
			X&&&&\rnode{X}{X}\\
			&&&& \dTo>a\\
			\dTo<{a} && \begin{array}c\To\\[-4pt]\eta_{a\o k}\end{array} && A\\
			&&&&\dTo>{f}\\
			A && \lTo_{g} && \rnode{B}{B}
			\ncarc[arcangle=70,ncurv=1]{->}XB\Aput{b}\Bput{\begin{array}c\To\\[-4pt]\zeta\end{array}}
		\end{diagram}
	\]
	%
	And since $\eta_a = \phi_{a,fa}(1_{fa})$ by definition,
	this is equivalent to the particular case of our second naturality condition
	with $b=f\o a$ and $\zeta = 1_{f\o a}$.
	
	But this special case
	implies the general case, by the first naturality condition: for we have
	\[\begin{array}{rl@{\qquad}c@{\qquad}r}
		&\begin{diagram}[w=2em]
			&&Y\\
			&&\dTo>k\\
			&&X\\
			&\ldTo^{a} & \begin{array}c\To\\[-4pt]\phi_{a,b}(\zeta)\end{array} & \rdTo^b\\
			A && \lTo_{g} && B
		\end{diagram}
		&=&
		\begin{diagram}[w=1em,h=1.5em]
			&& &&Y\\ \\
			&& &&\dTo>k\\ \\
			&& &&\rnode{X}{X}\\
			&& &\ldTo(4,4)^{a} && \rdTo^a\\
			&& &&\begin{array}c\To\\[-4pt]\eta_a\end{array} &&A\\
			&&&&&&&\rdTo^f\\
			A && && \lTo_{g}&&  && \rnode{B}{B}
			\ncarc[arcangle=70,ncurv=1]{->}XB\Aput{b}\Bput{\Nearrow\zeta}
		\end{diagram}
		\\[9em]
		=&
		\begin{diagram}[w=2em,h=1.5em]
			&&Y\\
			&\ldTo(2,4)^k&&\rdTo(2,4)^k\\
			\\
			\\
			X&&&&\rnode{X}{X}\\
			&&&& \dTo>b\\
			\dTo<{a} && \begin{array}c\To\\[-4pt]\eta_{a\o k}\end{array} && A\\
			&&&&\dTo>{f}\\
			A && \lTo_{g} && \rnode{B}{B}
			\ncarc[arcangle=70,ncurv=1]{->}XB\Aput{b}\Bput{\begin{array}c\To\\[-4pt]\zeta\end{array}}
		\end{diagram}
		&=&
		\begin{diagram}[w=2em]
			&&Y\\
			&\ldTo^k&&\rdTo^k\\
			X&&&&X\\
			\dTo<{a} && \begin{array}c\To\\[-4pt]\phi_{a\o k,b\o k}(\zeta\o k)\end{array} && \dTo>b\\
			A && \lTo_{g} && B
		\end{diagram}
	\end{array}\]
	as required.
\end{proof}

\begin{remark} % pseudo-adjunctions between bicategories
	In this section we have considered adjunctions \emph{in} a bicategory.
	It is also possible to consider \emph{pseudo-adjunctions between bicategories}:
	to exhibit $G:\BC\to\B$ as right pseudo-adjoint to $F:\B\to\BC$ is to give
	an equivalence $\BC(FA,X)\simeq\B(A, GX)$ pseudo-natural in $A$ and $X$.
	We do not pursue pseudo-adjunctions further here.
\end{remark}

\section{On equivalence}\label{s-equiv}
Recall the definition of equivalence (Definition\ref{def-equivalence}).
\begin{definition} % adjoint equivalence
	An \emph{adjoint equivalence} is an adjunction whose unit and
	counit are invertible.
\end{definition}
\begin{remark}
	If $f \dashv g$ is an adjoint equivalence with unit $\eta$ and counit $\e$,
	then $g\dashv f$ is an adjoint equivalence with unit $\e^{-1}$
	and counit $\eta^{-1}$.
\end{remark}
%
In later chapters, particularly Chapter~\refchapter{Psmon}, we will often consider
mates with respect to adjoint equivalences. Such mating has some special
properties which are crucial for our applications.
%
\begin{lemma}\label{lemma-adjeq-mate}
	If we have mates
	\[
	\begin{diagram} % sigma
		A & \rTo^{f} & B \\
		\dTo<{h} & \Arr\Nearrow\sigma & \dTo>{k} \\
		A' & \rTo_{f'} & B'
	\end{diagram}
	\qquad\mbox{and}\qquad
	\begin{diagram} % tau
		A & \lTo^{g} & B \\
		\dTo<{h} & \Arr\Searrow\tau & \dTo>{k} \\
		A' & \lTo_{g'} & B'
	\end{diagram}
	\]
	with respect to adjoint \emph{equivalences} $f\dashv g$
	and $f'\dashv g'$, then $\sigma$ is invertible if and only
	if $\tau$ is.
\end{lemma}
\begin{proof}
	Immediate from the definition of mate.
\end{proof}
%
\begin{lemma}\label{lemma-adjeq-twisted}
	Given adjoint equivalences $f\dashv g: A\to B$
	and $f'\dashv g': A'\to B'$,
	and an invertible 2-cell
	\begin{diagram} % sigma
		A & \rTo^{f} & B \\
		\dTo<{h} & \Arr\Nearrow\sigma & \dTo>{k} \\
		A' & \rTo_{f'} & B',
	\end{diagram}
	the inverse of the right mate of $\sigma$ is equal to the
	left mate of its inverse.
\end{lemma}
\begin{proof}
	The right mate of $\sigma$ is
	\begin{diagram} % right mate of sigma
	\rnode{A}{A} & \lTo^{g} & \rnode{B}{B}\\
	\dTo<{h}&\rdTo_{f}^{\raise4pt\hbox{$\begin{array}c\To\\[-4pt]\e\end{array}$}}
		& \dTo>1\\
	A'&\begin{array}c\To\\[-4pt]\sigma\end{array}&B\\
	\dTo<1
		& \rdTo^{f'}_{\raise-4pt\hbox{$\begin{array}c\To\\[-4pt]\eta'\end{array}$}}
		& \dTo>{k}\\
	\rnode{A'}{A'} & \lTo_{g'} & \rnode{B'}{B',}
	%
	\nccurve[angleA=210,angleB=140]{->}{A}{A'}\Bput{h}\Aput{\ \ \ \cong}
	\nccurve[angleA=-30,angleB=30]{->}{B}{B'}\Aput{k}\Bput{\cong\ \ \ }
	\end{diagram}
	whose inverse is
	\begin{diagram} % inverse of right mate of sigma
	\rnode{A}{A} & \lTo^{g} & \rnode{B}{B}\\
	\dTo<{h}&\rdTo_{f}^{\raise4pt\hbox{$\begin{array}c\Leftarrow\\[-4pt]\e^{-1}\end{array}$}}
		& \dTo>1\\
	A'&\begin{array}c\Leftarrow\\[-4pt]\sigma^{-1}\end{array}&B\\
	\dTo<1
		& \rdTo^{f'}_{\raise-4pt\hbox{$\begin{array}c\Leftarrow\\[-4pt]\eta'^{-1}\end{array}$}}
		& \dTo>{k}\\
	\rnode{A'}{A'} & \lTo_{g'} & \rnode{B'}{B',}
	%
	\nccurve[angleA=210,angleB=140]{->}{A}{A'}\Bput{h}\Aput{\ \ \ \cong}
	\nccurve[angleA=-30,angleB=30]{->}{B}{B'}\Aput{k}\Bput{\cong\ \ \ }
	\end{diagram}
	which is the left mate of $\sigma^{-1}$.
\end{proof}
%
\begin{propn}\label{prop-adjeq-mate-dual}
	Given adjoint equivalences $f\dashv g: A\to B$
	and $f'\dashv g': A\to B$,
	and an invertible 2-cell
	\vskip0pt
	\begin{diagram}
		\rnode{A}{A} & \Arr\Downarrow\gamma & \rnode{B}{B}
		\ncarc[arcangle=30]{->}{A}{B} \Aput{f}
		\ncarc[arcangle=30]{->}{B}{A} \Aput{f'}
	\end{diagram}
	\vskip2em
	the left mate of $\gamma$ is equal to its right mate.
\end{propn}
\begin{proof}
	The proof is surprisingly intricate, and rather difficult to
	follow unless string diagrams are used. (In fact, this proof is
	the reason that we have introduced string diagrams into this
	chapter.) In string diagram terms, what we have to prove is that
	\[
		\cdiag{d-adj/gamma-up} \qquad=\qquad \cdiag{d-adj/gamma-down}
	\]
	The proof is as follows:
%	\begin{mspill} \cdiag{d-adj/gamma-up} \quad=\quad \cdiag{d-adj/gamma-pf-1} \end{mspill}
	\[ \cdiag{d-adj/gamma-up} \]
	\[ =\quad \cdiag{d-adj/gamma-pf-1} \]
	\[ =\quad \cdiag{d-adj/gamma-pf-2} \]
	\[ =\quad \cdiag{d-adj/gamma-pf-3} \]
	\[ =\quad \cdiag{d-adj/gamma-pf-4} \]
	\[ =\quad \cdiag{d-adj/gamma-pf-5} \]
	\[ =\quad \cdiag{d-adj/gamma-down} \]
\end{proof}
%
\begin{propn}\label{prop-adjeq} % equivalence -> adjoint equivalence with same unit
	If there is an equivalence $(f,g,e,e')$ from $A$ to $B$, then there
	is an adjoint equivalence $f\dashv g$ with unit $e$.
\end{propn}
\begin{proof}
	It is well known\footnote{And easy to prove using
	the ordinary Yoneda lemma.} that this is true in $\Cat$.
	We shall use Yoneda to
	infer that it is therefore true in an arbitrary bicategory. Let there be
	given an equivalence $(f,g,e,e')$ from $A$ to $B$. This induces
	an equivalence from $\B(-,A)$ to $\B(-,B)$ in $\Bicat(\B\op,\Cat)$.
	Thus for every $X\in\B$ we have an adjoint equivalence in $\Cat$
	from $\B(X,A)$ to $\B(X,B)$, with unit $\B(X,e)$. By Prop.~\ref{prop-adj-2}
	and Lemma~\ref{lemma-adjeq-mate}
	this induces an adjoint equivalence in $\Bicat(\B\op,\Cat)$, and Yoneda
	therefore yields the desired adjoint equivalence in $\B$.
\end{proof}
\begin{remark} % elementary proof
	It is possible to give a more elementary proof of the preceding
	Proposition, by directly constructing a counit for the adjoint
	equivalence. This parallels the situation that one frequently
	encounters in ordinary category theory, where there is a choice
	between a concise, perspicuous Yoneda proof and an obscure but
	elementary equational one.
\end{remark}
\begin{remark} % duality (can keep counit instead of unit)
	An adjunction in $\B$ is an adjunction in ${\B\co}\op$, with the
	unit and counit reversed. Thus in the situation of Prop.~\ref{prop-adjeq}
	there is also a (generally different) adjoint equivalence with counit $e'$.
\end{remark}
Just as an ordinary natural transformation is invertible just when all its
components are, so a modification is invertible just when all its components
are. Furthermore a pseudo-natural transformation is an equivalence
in $\Bicat(\B,\BC)$ just when all its components are equivalences in
their respective hom-categories. This latter fact, though unsurprising,
is not altogether trivial to prove -- though we have done the hard work
already.
\begin{propn}\label{prop-pneq}
	Let there be given a pseudo-natural transformation
	\[
		\gamma: F\To G: \B\to\BC.
	\]
	This $\gamma$ is an equivalence in $\Bicat(\B,\BC)$ just
	when for every $A\in\B$ the 1-cell $\gamma_A: FA\to GA$
	is an equivalence.
\end{propn}
\begin{proof}
	Suppose that for every $A$, the component $\gamma_A: FA\to GA$
	is an equivalence. By Prop.~\ref{prop-adjeq} we may suppose that
	there is an arrow $\delta_A: $ and an adjoint equivalence
	$\gamma_A\dashv\delta_A$. Now the claim follows from
	Lemma~\ref{lemma-adjeq-mate} and Prop.~\ref{prop-adj-2}.
\end{proof}

\section{Normal pseudo-functors}
Here we prove a useful coherence-type result about pseudo-functors.
It is certainly well-known, but I am not aware of a published proof.%
%\footnote{The proof herein is hideously ugly: perhaps that is the reason.}
\begin{definition} % normal pseudo-functor
	A pseudo-functor $F:\B\to\BC$ is \emph{normal} if
	$F_A: 1_{FA}\to F(1_A)$ is an identity map for every $A\in\B$.
\end{definition}
\begin{lemma}\label{l-normal}
	Let $F:\B\to\BC$ be a pseudo-functor, and let there be given,
	for all $A$, $B\in\B$, a functor $G_{A,B}: \B(A,B)\to\BC(FA,FB)$
	and a natural isomorphism $\phi_{A,B}: F_{A,B}\To G_{A,B}$.
	
	Then $G$ may be extended to a pseudo-functor that coincides
	with $F$ on objects, such that $\phi$ becomes a pseudo-natural
	equivalence between $F$ and $G$.
\end{lemma}
\begin{proof}
	Define $G$ like $F$ on objects, and let its action on the hom-category
	$\B(A,B)$ be the functor $G_{A,B}$. For an object $A$, let $G_A$ be the composite
	\[
		1_{GA} = 1_{FA} \rTo^{F_A} F(1_A) \rTo^{(\phi_{A,A})_{1_A}} G(1_A),
	\]
	and for a composable pair
	\(
		A\rTo^f B\rTo^g C
	\)
	let $G_{g,f}$ be the composite
	\[
		G(g)\o G(f) \rTo^{(\phi_{B,C}^{-1})_g\o(\phi_{A,B}^{-1})_f} F(g)\o F(f)
			\rTo^{F_{g,f}} F(g\o f) \rTo^{(\phi_{A,C})_{g\o f}} G(g\o f).
	\]
	It is necessary to check that $G$ is indeed a pseudo-functor. For example,
	for condition $\cFl$ take an arrow $f:A\to B$. We have the diagram
	\begin{diagram}[w=6em]
		1\o Gf & \rTo^{F_B\o Gf} & F1\o Gf & \rTo^{\phi_1\o Gf} & G1\o Gf\\
		&\rdTo(0,6)<{\l_{Gf}}\rdTo(1,2)>{1\o\phi_f^{-1}} &&\rdTo(1,2)<{F1\o \phi_f^{-1}}
			\ldTo(1,2)^{\phi_1^{-1}\o\phi_f^{-1}}\ruTo(0,6)<{G_{1,f}}\\
		& 1\o Ff & \rTo_{F_B\o Ff} & F1\o Ff\\
		\rlap{\qquad$\natural_\l$}&\dTo<{\l_{Ff}} &\cFl& \dTo<{F_{1,f}}\\
		&Ff & \rTo_{F(\l_f)} & F(1\o f)\\
		\ldTo(1,2)>{\phi_f} &&\natural_\phi&& \rdTo(1,2)_{\phi_{1\o f}}\\
		Gf && \rTo_{G(\l_f)} && G(1\o f)
	\end{diagram}
	where we have omitted the object subscripts of $\phi$.
	The regions commute for the marked reasons, or functoriality of composition,
	except for the rightmost region which commutes by definition of $G_{1,f}$.
	Thus the outside commutes, and $G$ satisfies condition $\cFl$.
	The other conditions may be checked similarly.
	
	We make $\phi$ into a pseudo-natural transformation by defining $\phi_A$
	to be the identity at $FA=GA$, for every object $A$. For an arrow $f:A\to B$,
	$\phi_f$ is defined to be the pasting
	\begin{diagram}[size=5em]
		\rnode{FA}{FA} & \rTo^{\phi_A = 1} & GA\\
		\dTo<{Ff} &\hskip4pt\Nearrow (\phi_{A,B})_f& \dTo>{Gf}\\
		FA & \rTo_{\phi_B = 1} & \rnode{GB}{GB}
		\ncarc{->}{FA}{GB}\Aput{Gf}
		\ncarc{<-}{GB}{FA}\Aput{Ff}
	\end{diagram}
	It is easy to check that this constitutes a pseudo-natural transformation,
	and its components are identities (hence equivalences), so by
	Prop.~\ref{prop-pneq} it is a pseudo-natural equivalence, as required.
\end{proof}

\begin{propn}\label{prop-normal}
	Every pseudo-functor $F$ is equivalent to a normal pseudo-functor
	that agrees with $F$ on objects, as well as on non-identity 1-cells and
	the 2-cells between them.
\end{propn}
\begin{proof}
	Let there be given a pseudofunctor $F:\B\to\BC$. We shall
	construct an equivalent normal pseudofunctor $G$. By Lemma~\ref{l-normal}
	it suffices to do so for each hom-category separately. The definition
	of $G_{A,B}$ and $\phi_{A,B}$ is by cases, as follows.
	%
	\begin{itemize}
		\item Given objects $A\neq B$, let $G_{A,B} \defeqto F_{A,B}$
			and let $\phi_{A,B}$ be the identity.
		%
		\item For each object $A$, let $G_{A,A}(1) \defeqto 1$, and let
			$G_{A,A}(f) \defeqto F_{A,A}(f)$ for $f \neq 1$.
		\item Given a 2-cell $\beta:f\To g: A\to A$ with $f\neq1\neq g$,
			let $G_{A,A}(\beta) \defeqto \beta$.
		\item Given a 2-cell $\beta:1\To f: A\to A$ with $f\neq 1$,
			let $G_{A,A}(\beta) \defeqto F(\beta)\cdot F_A$.
		\item Given a 2-cell $\gamma:f\To 1: A\to A$ with $f\neq 1$,
			let $G_{A,A}(\gamma) \defeqto F_A^{-1}\cdot F(\gamma)$.
		\item Given a 2-cell $\delta: 1\To 1: A\to A$,
			let $G_{A,A}(\delta) \defeqto F_A^{-1}\cdot F(\delta)\cdot F_A$.
		%
		\item Let $(\phi_{A,A})_1: F(1_A)\to G(1_A) = 1_{GA} = 1_{FA}$ be $F_A^{-1}$,
		\item and let $(\phi_{A,A})_f \defeqto 1$.
	\end{itemize}
	%
	It is straightforward to check (four cases) that this
	makes $\phi_{A,A}$ a natural transformation.
	
	Now we may extend $G$ to a pseudo-functor using Lemma~\ref{l-normal},
	in which $G_A$ is defined to be $(\phi_{A,A})_1\cdot F_A$. Since
	$(\phi_{A,A})_1 = F_A^{-1}$ by definition, $G_A$ is the identity and
	so $G$ is normal, as required.
\end{proof}

\end{thesischapter}