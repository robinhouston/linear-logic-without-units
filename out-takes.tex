%!TEX TS-program = latex
\documentclass{robinthesisdraft}
\usepackage{robincs,thesisdefs,xr}
\externaldocument[Psmon:]{pseudomonoids}

\newenvironment{snippet}[1]{\section{#1}}{}

\title{The ones that got away: from the cutting-room floor}

\begin{document}
\maketitle

This file consists of substantial snippets that have been removed
from the thesis. Each such snippet is kept in a `snippet' environment,
which can be folded in TextMate.

\begin{snippet}{pseudomonoid axioms} % General (not Gray) version of axioms
	\begin{diagram}
		\C\tn(\C\tn(\C\tn\C))
		& \rTo^{\C\tn(\C\tn P)} & \C\tn(\C\tn\C)
		\\
		\dTo<{\C\tn a_{\C,\C,\C}}
		& \Swarrow\C\tn \aa
		&& \rdTo^{\C\tn P}
		\\
		\C\tn ((\C\tn\C)\tn\C)
		& \rTo_{\C\tn(P\tn\C)} & \C\tn(\C\tn\C)
		& \rTo_{\C\tn P} & \C\tn\C
		\\
		\dTo<{a_{\C,\C\tn\C,\C}}
		& \Swarrow a_{\C,P,\C}
		& \dTo<{a_{\C,\C,\C}}
		& \Swarrow\aa
		&& \rdTo^{P}
		\\
		(\C\tn(\C\tn\C))\tn\C
		& \rTo_{(\C\tn P)\tn\C}
		& (\C\tn\C)\tn\C
		& \rTo_{P\tn\C}
		& \C\tn\C & \rTo_P & \C
		\\
		\dTo<{a_{\C,\C,\C}\tn\C}
		& \Downarrow\aa\tn\C
		&& \ruTo_{P\tn\C}
		\\
		((\C\tn\C)\tn\C)\tn\C & \rTo_{(P\tn\C)\tn\C} & (\C\tn\C)\tn\C
	\end{diagram}
	must be equal to
	\begin{mspill}\begin{diagram}
		&& \C\tn(\C\tn(\C\tn\C))
		& \rTo^{\C\tn(\C\tn P)} & \C\tn(\C\tn\C)
		& \rTo^{\C\tn P} & \C\tn\C
		\\
		& \ldTo^{\C\tn a_{\C,\C,\C}}
		& \dTo[snake=-1.5em]<{a_{\C,\C,\C\tn\C}}
		& \Swarrow a_{\C,\C,P}
		& \dTo>{a_{\C,\C,\C}}
		& \Swarrow\aa
		&& \rdTo(2,3)^P
		\\
		\C\tn((\C\tn\C)\tn\C)
		&&&& (\C\tn\C)\tn\C
		\\
		\dTo<{a_{\C,\C\tn\C,\C}}
		& \begin{array}c\Leftarrow\\[-4pt]\pi_{\C,\C,\C,\C}\end{array}
		& (\C\tn\C)\tn(\C\tn\C)
		& \ruTo(2,1)^{(\C\tn\C)\tn P}
		& \cong
		& \rdTo(2,1)^{P\tn\C}
		& \C\tn\C
		& \rTo_P & \C
		\\
		(\C\tn(\C\tn\C))\tn\C
		&& \dTo[snake=-1.5em]<{a_{\C\tn\C,\C,\C}}
		& \rdTo(2,1)_{P\tn(\C\tn\C)}
		&\C\tn(\C\tn\C)
		& \ruTo(2,1)_{\C\tn P}
		&& \ruTo(2,3)_P
		\\
		& \rdTo_{a_{\C,\C,\C}\tn\C}
		&& \Swarrow a_{P,\C,\C}
		& \dTo>{a_{\C,\C,\C}}
		& \Swarrow\aa
		\\
		&&((\C\tn\C)\tn\C)\tn\C
		& \rTo_{(P\tn\C)\tn\C} & (\C\tn\C)\tn\C
		& \rTo_{P\tn\C} & \C\tn\C
	\end{diagram}\end{mspill}
	and
	\begin{mspill}
		\begin{diagram}[w=1em]
			&\rnode{1}{\C\tn(I\tn\C)}
			&& \rTo^{a_{\C,I,\C}} && (\C\tn I)\tn\C
			\\
			& \dTo<{\C\tn(J\tn\C)}
			&& \hbox to0pt{\hss$\Nearrow a_{\C,J,\C}$\hss}
			&& \dTo>{(\C\tn J)\tn\C}
			\\
			\begin{array}c\To\\[-4pt]\C\tn\ll\end{array}
			& \C\tn(\C\tn\C)
			&& \rTo_{a_{\C,\C,\C}}
			&& (\C\tn\C)\tn\C
			\\
			& \dTo<{\C\tn P}
			&& \Nearrow\aa
			&& \dTo>{P\tn\C}
			\\
			& \rnode{2}{\C\tn\C}
			& \rTo_P & \C
			& \lTo_P & \C\tn\C
			%
			\nccurve[angle=180,ncurv=1]{->}12\Bput{\C\tn l_\C}
		\end{diagram}
		=
		\begin{diagram}[w=2em]
			\C\tn(I\tn\C)
			&& \rTo^{a_{\C,I,\C}}
			&& (\C\tn I)\tn\C
			\\
			& \rdTo_{\C\tn l_\C}
			& \begin{array}c\To\\[-4pt]\mu_{\C,\C}\end{array}
			& \ldTo[snake=-1em]_{r_\C\tn\C}
			& \begin{array}c\To\\[-4pt]\rr\tn\C\end{array}
			& \rdTo^{(\C\tn J)\tn\C}
			\\
			&& \C\tn\C
			&&\lTo_{P\tn\C}
			&& (\C\tn\C)\tn\C
			\\
			&&\dTo>P
			\\
			&& \C
		\end{diagram}
	\end{mspill}
\end{snippet}

\begin{snippet}{proof of prop-adj-2} % Old version of the 2nd half - w/o explicit mates
	In the other direction, suppose that for every $A\in\B$ we have an
	arrow $\gamma_A$ right adjoint to $\phi_A$, with the adjunction
	having unit $\eta_A$ and counit $\e_A$ such that \pref{diag-adj1} is invertible.
	%
	We shall make $\gamma$ into a pseudo-natural transformation as follows.
	For every $f: A\to B$ in $\B$, let $\gamma_f$ be the inverse of the pasting
	\begin{diagram}
		\rnode{FA}{FA} & \lTo^{\gamma_A} & \rnode{GA}{GA}\\
		\dTo<{Ff}&\rdTo_{\phi_A}^{\raise4pt\hbox{$\begin{array}c\To\\[-4pt]\e_A\end{array}$}}
			& \dTo>1\\
		FB&\begin{array}c\To\\[-4pt]\phi_f\end{array}&GA\\
		\dTo<1 & \rdTo^{\phi_B}_{\raise-4pt\hbox{$\begin{array}c\To\\[-4pt]\eta_B\end{array}$}}
			& \dTo>{Gf}\\
		\rnode{FB}{FB} & \lTo_{\gamma_B} & \rnode{GB}{GB}
		\nccurve[angleA=210,angleB=140]{->}{FA}{FB}\Bput{Ff}\Aput{\ \ \ \cong}
		\nccurve[angleA=-30,angleB=30]{->}{GA}{GB}\Aput{Gf}\Bput{\cong\ \ \ }
	\end{diagram}
	This is invertible by assumption, and it is clearly natural in $f$. It remains to
	check the coherence conditions. Using naturality of the unit isomorphisms
	and Lemma~\ref{l-psnat} applied to $\phi$, we know that for every $A\in\B$,
	\[
	\hskip-1em
		\begin{diagram}[size=4em]
			FA & \lTo^{\gamma_A} & \rnode{GA}{GA}\\
			\dTo<{F(1)} & \llap{$\Searrow \gamma_{1_A}^{-1}\mkern4mu$}
				& \begin{array}{c}\Rightarrow\\G_A^{-1}\end{array}\\
			FA & \lTo_{\gamma_A} & \rnode{GB}{GA}
			\ncarc{->}{GA}{GB}\Aput{1}
			\ncarc{<-}{GB}{GA}\Aput{G(1)}
		\end{diagram}
		%
		\hskip1em=\hskip3em
		%
		\begin{diagram}[size=4em]
			\rnode{FA}{FA} & \lTo^{\gamma_A} & GA\\
			\begin{array}{c}\Rightarrow\\F_A^{-1}\end{array}
				& \rlap{$\mkern8mu\cong$} & \dTo>{1}\\
			\rnode{FB}{FA} & \lTo_{\gamma_A} & \rnode{GA}{GA}
			\ncarc{->}{FA}{FB}\Aput{1}
			\ncarc{<-}{FB}{FA}\Aput{F(1)}
			%\nccurve{->}{FA}{GA} \Bput{\gamma_A}
		\end{diagram}
	\]
	hence $\gamma$ satisfies the unit condition. The composition condition
	is also easy to verify. In detail: consider a composable pair
	\[
		A\rTo^f B\rTo^g C
	\]
	in $\B$. By definition of adjunction and the coherence conditions, we have
	\[
	\begin{diagram}
		\rnode{FA}{FA} & \lTo^{\gamma_A} & \rnode{GA}{GA}\\
		\dTo<{Ff}&\rdTo_{\phi_A}^{\raise4pt\hbox{$\begin{array}c\To\\[-4pt]\e_A\end{array}$}}
			& \dTo>1\\
		FB&\begin{array}c\To\\[-4pt]\phi_f\end{array}&GA\\
		\dTo<1 & \rdTo^{\phi_B}_{\raise-4pt\hbox{$\begin{array}c\To\\[-4pt]\eta_B\end{array}$}}
			& \dTo>{Gf}\\
		\rnode{FB}{FB} & \lTo_{\gamma_B} & \rnode{GB}{GB}\\
		\dTo<{Fg}&\rdTo_{\phi_B}^{\raise4pt\hbox{$\begin{array}c\To\\[-4pt]\e_B\end{array}$}}
			& \dTo>1\\
		FC&\begin{array}c\To\\[-4pt]\phi_g\end{array}&GB\\
		\dTo<1 & \rdTo^{\phi_C}_{\raise-4pt\hbox{$\begin{array}c\To\\[-4pt]\eta_C\end{array}$}}
			& \dTo>{Gg}\\
		\rnode{FC}{FC} & \lTo_{\gamma_C} & \rnode{GC}{GC}\\
		\nccurve[angleA=210,angleB=140]{->}{FA}{FB}\Bput{Ff}\Aput{\ \ \ \cong}
		\nccurve[angleA=210,angleB=140]{->}{FB}{FC}\Bput{Fg}\Aput{\ \ \ \cong}
		\nccurve[angleA=180,angleB=180]{->}{FA}{FC}\Bput{F(gf)}
			\Aput{\ \ \ \begin{array}c\To\\[-4pt]F_{g,f}^{-1}\end{array}}
		\nccurve[angleA=-30,angleB=30]{->}{GA}{GB}\Aput{Gf}\Bput{\cong\ \ \ }
		\nccurve[angleA=-30,angleB=30]{->}{GB}{GC}\Aput{Gg}\Bput{\cong\ \ \ }
	\end{diagram}
	\hskip4em=\hskip6em
	\begin{diagram}
		\rnode{FA}{FA} & \lTo^{\gamma_A} & \rnode{GA}{GA}\\
		\dTo<{Ff}&\rdTo_{\phi_A}^{\raise4pt\hbox{$\begin{array}c\To\\[-4pt]\e_A\end{array}$}}
			& \dTo>1\\
		\rnode{FB}{FB}&\begin{array}c\To\\[-4pt]\phi_f\end{array}&GA\\
		\dTo<{Fg}&\rdTo_{\phi_B}
			& \dTo>{Gf}\\
		FC&\begin{array}c\To\\[-4pt]\phi_g\end{array}&\rnode{GB}{GB}\\
		\dTo<1 & \rdTo^{\phi_C}_{\raise-4pt\hbox{$\begin{array}c\To\\[-4pt]\eta_C\end{array}$}}
			& \dTo>{Gg}\\
		\rnode{FC}{FC} & \lTo_{\gamma_C} & \rnode{GC}{GC}\\
		\nccurve[angleA=210,angleB=140]{->}{FB}{FC}\Bput{Fg}\Aput{\ \ \ \cong}
		\nccurve[angleA=180,angleB=180]{->}{FA}{FC}\Bput{F(gf)}
			\aput(.4){\ \ \begin{array}c\To\\[-4pt]F_{g,f}^{-1}\end{array}}
		\nccurve[angleA=-30,angleB=30]{->}{GA}{GB}\Aput{Gf}\Bput{\cong\ \ \ }
	\end{diagram}
	\]
	By naturality of the unit isomorphism and the composition condition for $\phi$,
	this is equal to
	\begin{diagram}
		\rnode{FA}{FA} & \lTo^{\gamma_A} & \rnode{GA}{GA}\\
		\dTo<{F(gf)}&\rdTo_{\phi_A}^{\raise4pt\hbox{$\begin{array}c\To\\[-4pt]\e_A\end{array}$}}
			& \dTo>1 & \rdTo^{Gf}_{\cong\ }\\
		FB&\begin{array}c\To\\[-4pt]\phi_{gf}\end{array}&GA&\rTo^{Gf}&\rnode{GB}{GB}\\
		\dTo<1 & \rdTo^{\phi_C}_{\raise-4pt\hbox{$\begin{array}c\To\\[-4pt]\eta_C\end{array}$}}
			& \dTo[snake=-8pt]>{G(gf)}\\
		\rnode{FC}{FC} & \lTo_{\gamma_C} & \rnode{GC}{GC}
		\nccurve[angleA=200,angleB=150]{->}{FA}{FC}\Bput{F(gf)}\Aput{\ \ \ \cong}
		\nccurve[angleA=-90,angleB=0]{->}{GB}{GC}\Aput{Gg}
			\bput(.2){\Nearrow G_{g,f}^{-1}}
	\end{diagram}
	which, again by naturality and coherence of the unit isomorphism, is equal to
	\begin{diagram}
		\rnode{FA}{FA} & \lTo^{\gamma_A} & \rnode{GA}{GA}\\
		\dTo<{F(gf)}&\rdTo_{\phi_A}^{\raise4pt\hbox{$\begin{array}c\To\\[-4pt]\e_A\end{array}$}}
			& \dTo>1 \\
		FB&\begin{array}c\To\\[-4pt]\phi_{gf}\end{array}&GA&
			&\raise-6pt\hbox{$\begin{array}c\To\\[-4pt]G_{g,f}^{-1}\end{array}$}
			&\rnode{GB}{GB}\\
		\dTo<1 & \rdTo^{\phi_C}_{\raise-4pt\hbox{$\begin{array}c\To\\[-4pt]\eta_C\end{array}$}}
			& \dTo>{G(gf)}\\
		\rnode{FC}{FC} & \lTo_{\gamma_C} & \rnode{GC}{GC}
		\nccurve[angleA=200,angleB=150]{->}{FA}{FC}\Bput{F(gf)}\Aput{\ \ \ \cong}
		\nccurve[angleA=-30,angleB=30]{->}{GA}{GC}\aput(.4){G(gf)}\Bput{\cong\ \ \ }
		\nccurve[angleB=90]{->}{GA}{GB}\Aput{Gf}\nccurve[angleA=-90]{->}{GB}{GC}\Aput{Gg}
	\end{diagram}
	which is equal, by definition, to
	\begin{diagram}[h=2em,w=4em]
		FA & \lTo^{\gamma_A} & GA\\
		&&&\rdTo^{Gf}\\
		\dTo<{F(gf)} & \raise-6pt\hbox{$\Nearrow \gamma_{gf}^{-1}$} & \dTo[snake=1em]<{G(gf)}
			& \llap{$\begin{array}c\To\\[-4pt]G_{g,f}^{-1}\end{array}$}
			& GB\\
		&&&\ldTo_{Gg}\\
		FC & \lTo_{\gamma_C}& GC
	\end{diagram}
	It follows that $\gamma$ satisfies the composition condition.
	
	Finally, it is an easy consequence of our definition of $\gamma_f$ that equation
	\pref{diag-eta-mod} is satisfied, hence that $\eta$ is a modification; and a similar
	argument shows that $\e$ is a modification too.
\end{snippet}

\begin{snippet}{all unit axioms for braiding} % Including the redundant ones
	the following two equations must hold
	\[
		\begin{diagram}[hug,s=4em]
			A\tn B\tn \I & \rTo^{A\tn s_{B,\I}} & A\tn \I\tn B \\
			\dTo<1 & \rdTo(2,2)^{s_{A\tn B,\I}}
				\raise 2em\hbox to 0pt{$\Arr\Nearrow{\scriptstyle S_{A,B|\I}}$\hss}
				\raise-1em\hbox to 0pt{\hss$\mathop\Rightarrow\limits_{U_{A\tn B|\I}}$}
				& \dTo>{s_{A,\I}\tn B} \\
			A\tn B & \lTo_{1} & \I\tn A\tn B
		\end{diagram}
		\hskip3em=\hskip-1em
		\begin{diagram}[hug,s=4em]
			A\tn B\tn \I & \rTo^{A\tn s_{B,\I}} & A\tn \I\tn B \\
			\dTo<1 & \ldTo(2,2)^{1}
				\raise-1em\hbox to 0pt{$\Arr\Searrow{\scriptstyle U_{A|\I}\tn B}$\hss}
				\raise 2em\hbox to 0pt{\hss$\mathop\Rightarrow\limits_{A\tn U_{B|\I}}$}
				& \dTo>{s_{A,\I}\tn B} \\
			A\tn B & \lTo_{1} & \I\tn A\tn B
		\end{diagram}
	\]
	\[
		\begin{diagram}[hug,s=4em]
			\I\tn A\tn B & \rTo^{s_{\I,A}\tn B} & A\tn \I\tn B \\
			\dTo<1 & \rdTo(2,2)^{s_{A\tn B,\I}}
				\raise 2em\hbox to 0pt{$\Arr\Swarrow{\scriptstyle S_{\I|A,B}}$\hss}
				\raise-1em\hbox to 0pt{\hss$\mathop\Leftarrow\limits_{U_{\I|A\tn B}}$}
				& \dTo>{A\tn s_{\I,B}} \\
			A\tn B & \lTo_{1} & A\tn B\tn \I
		\end{diagram}
		\hskip3em=\hskip-1em
		\begin{diagram}[hug,s=4em]
			\I\tn A\tn B & \rTo^{s_{\I,A}\tn B} & A\tn \I\tn B \\
			\dTo<1 & \ldTo(2,2)^{1}
				\raise-1em\hbox to 0pt{$\Arr\Nwarrow{\scriptstyle A\tn U_{\I|B}}$\hss}
				\raise 2em\hbox to 0pt{\hss$\mathop\Leftarrow\limits_{U_{\I|A}\tn B}$}
				& \dTo>{A\tn s_{\I,B}} \\
			A\tn B & \lTo_{1} & A\tn B\tn \I
		\end{diagram}
	\]
	and the four 2-cells pictured below should each be equal to the identity on $s_{A,B}$:
	\[\begin{array}{c@{\qquad}c}
		\begin{diagram}[hug,s=4em]
			\I\tn A\tn B & \rTo^{\I\tn s_{A,B}} & \I\tn B\tn A \\
			\dTo<{s_{\I\tn A,B}} & \ldTo_{s_{\I,B}\tn A}
				\raise 2em\hbox to 0pt{\hss$\mathop\Rightarrow\limits_{S_{\I,A|B}}$}
				\raise-2em\hbox to 0pt{$\mathop\Rightarrow\limits_{U_{\I|B}\tn A}$\hss}
				& \dTo>{1}\\
			B\tn \I\tn A & \rTo_{1} & B\tn A
		\end{diagram}
		&
		\begin{diagram}[hug,s=4em]
			A\tn B\tn \I & \rTo^{s_{A,B}\tn \I} & B\tn A\tn \I \\
			\dTo<{s_{A,B\tn \I}} & \ldTo_{B\tn s_{A,\I}}
				\raise 2em\hbox to 0pt{\hss$\mathop\Leftarrow\limits_{S_{A|B,\I}}$}
				\raise-2em\hbox to 0pt{$\mathop\Leftarrow\limits_{B\tn U_{A|\I}}$\hss}
				& \dTo>{1}\\
			B\tn \I\tn A & \rTo_{1} & B\tn A
		\end{diagram}
		\\
		\begin{diagram}[hug,s=2.5em]
			A\tn \I\tn B && \rTo^{s_{A\tn \I,B}} && B\tn A\tn \I \\
			&\rdTo_{A\tn s_{\I,B}} & \Downarrow{\scriptstyle S_{A,\I|B}}
				& \ruTo_{s_{A,B}\tn \I} \\
			\dTo<1 & \hbox to 0pt{\hss$\mathop\Leftarrow\limits_{A\tn U_{\I|B}}$}
				& A\tn B\tn \I && \dTo>1 \\
			&\ldTo_{1} \\
			A\tn B && \rTo_{s_{A,B}} && B\tn A
		\end{diagram}
		&
		\begin{diagram}[hug,s=2.5em]
			A\tn \I\tn B && \rTo^{s_{A,\I\tn B}} && \I\tn B\tn A \\
			&\rdTo_{s_{A,\I}\tn B} & \Uparrow{\scriptstyle S_{A|\I,B}}
				& \ruTo_{\I\tn s_{A,B}} \\
			\dTo<1 & \hbox to 0pt{\hss$\mathop\Rightarrow\limits_{U_{A|\I}\tn B}$}
				& \I\tn A\tn B && \dTo>1 \\
			&\ldTo_{1} \\
			A\tn B && \rTo_{s_{A,B}} && B\tn A
		\end{diagram}
	\end{array}\]
\end{snippet}

\begin{snippet}{old diagrams from Prop.~\chref{Psmon}{prop-lrs}}
	Since $S$ is a modification, this in turn is equal to
	\begin{diagram}[s=4em,tight]
		&&&&\C\tn \I\tn \C & \lTo^{s_{\C,\C\tn \I}} & \rnode{CCI}{\C^{2}\tn \I} \\
		&&&\ldTo^{\C\tn J\tn \C} & \uTo[snake=-1ex]<{\C\tn s_{\C,\I}}
			& \raise 2em\hbox to0pt{\hss$S_{\C|\C,\I}$}
			\ldTo(2,2)_{s_{\C,\C}\tn\I} \\
		&& \rnode{CCC}{\C^{3}} & \C\tn s_{J,\C} & \C^{2}\tn \I
			& \hbox to 0pt{\quad$\ss\tn\I$} \\
		&& \uTo<{\C\tn s_{\C,\C}}
			& \ldTo_{\C^{2}\tn J} && \rdTo_{P\tn \I} \\
		&\C\tn\ss&\C^{3} && \sim && \rnode{CI}{\C\tn \I} \\
		&\ldTo_{\C\tn P} && \rdTo^{P\tn \C} && \ldTo^{\C\tn J} \\
		\rnode{CC}{\C^{2}} && \aa && \C^{2} & \raise-1em\hbox to0pt{\hss$\rr$\hskip2em} \\
		&\rdTo_{P} && \ldTo_{P} \\
		&&\rnode{C}{\C}
		%
		\ncarc[arcangle=-40]{->}{CCC}{CC}\Bput{\C\tn P}
		\ncarc[arcangle=45]{->}{CCI}{CI} \Aput{P\tn \I}
		\ncarc[arcangle=45]{->}{CI}{C}   \Aput{1}
	\end{diagram}
	which by \pref{eq-rra} is equal to
	\begin{diagram}[s=4em,tight]
		&&&&\C\tn \I\tn \C & \lTo^{s_{\C,\C\tn \I}} & \rnode{tr}{\C^{2}\tn \I} \\
		&&&\ldTo^{\C\tn J\tn \C} & \uTo[snake=-1ex]<{\C\tn s_{\C,\I}}
			& \raise 2em\hbox to0pt{\hss$S_{\C|\C,\I}$}
			\ldTo(2,2)_{s_{\C,\C}\tn\I} \\
		&& \rnode{CCC}{\C^{3}} & \C\tn s_{J,\C} & \rnode{CCI}{\C^{2}\tn \I}
			& \hbox to 0pt{\quad$\ss\tn\I$} \\
		&& \uTo<{\C\tn s_{\C,\C}}
			& \ldTo_{\C^{2}\tn J} && \rdTo_{P\tn \I} \\
		&\C\tn\ss&\C^{3} \raise-1.5em\hbox to0pt{$\C\tn\rr$\hss}
			&& && \rnode{CI}{\C\tn \I} \\
		&\ldTo_{\C\tn P} \\
		\rnode{CC}{\C^{2}} \\
		&\rdTo[nohug]_{P} \\
		&&\rnode{C}{\C}
		%
		\ncarc[arcangle=-40]{->}{CCC}{CC}\Bput{\C\tn P}
		\ncarc[arcangle=45]{->}{tr}{CI} \Aput{P\tn \I}
		\ncarc[arcangle=45]{->}{CCI}{CC}   \Aput{1}
		\ncarc[arcangle=45]{->}{CI}{C}   \Aput{1}
	\end{diagram}
\end{snippet}

\begin{snippet}{proof that $\phi_{X}$ is a map of modules}
\begin{lemma}\label{lemma-ModMX-module}
	Let $X$ be a right $M$-module. The set $\Mod_{M}(M,X)$ is
	a right $\V(I,M)$-module, via the function that takes a
	pair $(f, a)$, where $f: M\to X$ is a map of modules and
	$a: I\to M$, and returns the composite
	\[
		M \rTo^{\cong} I\tn M \rTo^{a\tn M} M\tn M \rTo^{m} M \rTo^{f} X.
	\]
\end{lemma}
\begin{proof}
	First, we must show that the composite displayed in the statement
	is indeed a map of modules. Consider the diagram
	\begin{diagram}
		M\tn M & \rTo^{\cong} & I\tn M\tn M & \rTo^{a\tn M\tn M} & M\tn M\tn M
			& \rTo^{m\tn M} & M\tn M & \rTo^{f\tn M} & X\tn M \\
		\dTo<{m} &\natural&\dTo>{I\tn M} & \natural & \dTo>{M\tn m}
			&& \dTo>m && \dTo>x \\
		M &\rTo_{\cong} & I\tn M & \rTo_{a\tn M} & M\tn M & \rTo_{m} & M & \rTo_{f} & X
	\end{diagram}
	From left to right, the first two squares commute by naturality,
	the third by associativity of $M$ and the fourth since $f$ is a
	map of modules. Thus the outer edge commutes, showing as required
	that the composite displayed in the statement is a map of modules.
	
	We must also show that this function determines a module. Take
	$f: M\to X$ and $a$, $b:I\to M$. Then we have
	\begin{diagram}
		I\tn M & \rTo^{b\tn M} & M\tn M & \rTo^{m} & M \\
		\dTo<{\cong} & \natural & \dTo>\cong & \natural & \dTo>{\cong} \\
		I\tn I\tn M & \rTo^{I\tn b\tn M} & I\tn M\tn M & \rTo^{I\tn m} & I\tn M\\
		&\rdTo_{a\tn b\tn M} & \dTo>{a\tn M\tn M} && \dTo>{a\tn M} \\
		&&M\tn M\tn M & \rTo_{M\tn m} & M\tn M & \rTo_{m} & M & \rTo_{f} & X
	\end{diagram}
	The upper squares commute by naturality of the left-unit isomorphism,
	and the lower cells by functoriality of tensor. Thus the outside
	commutes, as required for the first condition of Definition~\ref{def-module-map}.
	The second condition is trivial: we need to show that
	\[
		M \rTo^{\cong} I\tn M \rTo^{e\tn M} M\tn M \rTo^{m} M \rTo^{f} X
	\]
	is equal to $f$, and we have
	\begin{diagram}
		M & \rTo^{\cong} & I\tn M & \rTo^{e\tn M} & M\tn M \\
		&&&\rdTo_{\cong} & \dTo>m \\
		&&&& M &\rTo_{f} & X
	\end{diagram}
	where the triangle commutes by the left-unit law for the monoid $M$,
	and the lower edge is equal to $f$.
\end{proof}
\begin{lemma}\label{lemma-VIX-module}
	Let $X$ be a right $M$-module. The set $\V(I,X)$ is a
	right $\V(I,M)$-module via the function that takes
	the pair $(z, a)$, for $z:I\to X$ and $a:I\to M$,
	to the arrow
	\[
			I \rTo^{\cong} I\tn I \rTo^{z\tn a} X\tn M \rTo^{x} X.
	\]
\end{lemma}
\begin{proof}
	Take $z:I\to X$, and $a$, $b: I\to M$. We must show that
	the composites
	\[
		I\rTo^{\cong} I\tn I\tn I \rTo^{z\tn a\tn b} X\tn M\tn M
			\rTo^{x\tn M} X\tn M \rTo^{x} M
	\]
	and
	\[
		I\rTo^{\cong} I\tn I\tn I \rTo^{z\tn a\tn b} X\tn M\tn M
			\rTo^{X\tn m} X\tn M \rTo^{x} M
	\]
	are equal. But this is an immediate consequence of the fact
	that $X$ is a right $M$-module.
\end{proof}
Now we can state the claim:
\begin{thm}\label{thm-1d-external-plus}
	The map $\phi_{X}$ of Theorem~\ref{thm-1d-external} is a map of
	$\V(I,M)$-modules, with respect to the module structures described
	in Lemmas~\ref{lemma-ModMX-module} and~\ref{lemma-VIX-module}.
\end{thm}
\begin{proof}
	We must show the commutativity of the square
	\begin{diagram}
		\V(I,M)\times\Mod_{M}(M,X) & \rTo^{\V(I,M)\times\phi_{X}} & \V(I,M)\times\V(I,X) \\
		\dTo && \dTo \\
		\Mod_{M}(M,X) & \rTo_{\phi_{X}} & \V(I,X)
	\end{diagram}
	where the vertical arrows are defined by Lemmas~\ref{lemma-ModMX-module}
	and~\ref{lemma-VIX-module} respectively. In other words, we must show
	that, for $a: I\to M$ and a module map $f: M\to X$, the composites
	\[
		I \rTo^{e} M \rTo^{\cong} I\tn M \rTo^{a\tn M} M\tn M \rTo^{m} M \rTo^{f} X
	\]
	and
	\[
		I \rTo^{\cong} I\tn I \rTo^{(f\cdot e)\tn a} X\tn M \rTo^{x} X
	\]
	are equal. We shall do so by showing that they are both equal to $f\cdot a$.
	%
	For the former, we have
	\begin{diagram}
		&& M \\
		& \ruTo^{a} && \rdTo^{\cong} \\
		I & \rTo_{\cong} & I\tn I & \rTo_{a\tn I} & M\tn I \\
		\dTo<e && \dTo<{I\tn e} && \dTo<{M\tn e} & \rdTo^{\cong} \\
		M & \rTo_{\cong} & I\tn M & \rTo_{a\tn M} & M\tn M & \rTo_{m} & M & \rTo_{f} & X
	\end{diagram}
	where the quadrilaterals commute by naturality or functoriality of tensor, and the
	triangle by the right-unit law for the monoid $M$.
	%
	For the latter, we have
	\begin{diagram}
		I &\rTo^{\cong} & I\tn I & \rTo^{(f\cdot e)\tn a} & X\tn M & \rTo^{x} & X \\
		\dTo<a && \dTo>{I\tn a} && \uTo>{f\tn M} && \uTo>f \\
		M & \rTo_{\cong} & \rnode{IM}{I\tn M} & \rTo_{e\tn M} & M\tn M
			& \rTo_{m} & \rnode{M}{M} \\\ 
		%
		\ncarc[arcangle=-45]{->}{IM}{M} \Bput{\cong}
	\end{diagram}
	whose squares commute, from left to right, by the naturality of the
	left-unit isomorphism, the functoriality of tensor, and the fact
	that $f$ is a module morphism. The curved cell commutes by the
	left-unit law for the monoid $M$.
\end{proof}
\end{snippet}

\begin{snippet}{details of Internal Cayley theorem for monoids}
The theorem above is external in the sense that it essentially
concerns $\V(I, M)$, the underlying ordinary monoid of $M$. If
the monoidal category $\V$ is right closed, then we can state
an internal version, purely in terms of arrows in $\V$ itself.
%
So let $\V$ be biclosed, i.e.\ suppose we are given
functors
\[\begin{array}{l}
	\lolli: \V\op\times\V\to\V \\
	\illol: \V\times\V\op\to\V,
\end{array}\]
and natural isomorphisms with components
\[\begin{array}{l}
	\curry_{A,B,C}:  \V(A\tn B, C) \cong \V(A, B\lolli C), \\
	\curry'_{A,B,C}: \V(A\tn B, C) \cong \V(B, C\illol A).
\end{array}\]
We shall refer to both these operations simply as currying;
and we shall be careful always to provide enough context that
it is clear which  of them is intended.
The corresponding evaluation transformations will be written
\[\begin{array}{l}
	\ev^{A}_{B}: (A\lolli B)\tn A \to B, \\
	{\ev'}^{A}_{B}: A\tn(B\illol A) \to B.
\end{array}\]
and the dual coevaluations as
\[\begin{array}{l}
	\coev^{A}_{B}: B \to A\lolli(B\tn A), \\
	{\coev'}^{A}_{B}: B \to (A\tn B)\illol A.
\end{array}\]
This is enough for us to be able to state the theorem:
\begin{thm}[Internal Cayley]\label{thm-1d-internal}
	Let $(X,x)$ be a right $M$-module. Then the diagram
	\begin{diagram}
		X & \rTo^{\curry_{X,M,X}(x)} & M\lolli X & \pile{\rTo^{h} \\ \rTo_{k}}
		 & (M\tn M)\lolli X
	\end{diagram}
	is a coequaliser diagram, where the maps $h$ and $k$ are defined as
	follows. The map $h$ is obtained by currying
	\[
		(M\lolli X)\tn M\tn M \rTo^{\ev^{M}_{X}\tn M} X\tn M \rTo^{x} X,
	\]
	and $k$ is obtained by currying
	\[
		(M\lolli X)\tn M\tn M \rTo^{(M\lolli X)\tn m} (M\lolli X)\tn M \rTo^{\ev^{M}_{X}} X.
	\]
\end{thm}
%
If the proof is to be organised in a sensible way, we need to call
upon some elementary properties of (bi)closed monoidal categories
\citep[Chapter~1]{KellyEnriched}. We shall state these with respect
to the left internal hom $\illol$, though corresponding things are
of course also true of the right internal hom. There is an internal
composition operation
\[
	\comp^{B}_{A,C}: (B\illol A)\tn(C\illol B) \to C\illol A
\]
obtained by currying the composite
\[
	A\tn(B\illol A)\tn(C\illol B) \rTo^{{ev'}^{A}_{B}}
		B\tn(C\illol B) \rTo^{{ev'}^{B}_{C}} C,
\]
and a family of internal unit maps
\[
	u_{A}: I \to A\illol A
\]
obtained by currying the right-unit isomorphism $A\tn I\rTo^{\cong}A$.
The internal composition is associative, and the internal units are
units for internal composition, in the obvious sense.
%
In particular the object $A\illol A$ is a monoid
with multiplication given by internal composition, and unit $u_{A}$.
Below we shall denote this monoid $A^{A}$ for concision.)

Furthermore, for each object $A\in\V$ there is a natural transformation
\[
	t^{A}_{X,Y}: Y\illol X \to (A\tn Y)\illol(A\tn X)
\]
obtained by currying the maps
\[
	A\tn X\tn(Y\illol X) \rTo^{A\tn{\ev'}}^{X}_{Y} A\tn Y
\]
(which is dinatural in $A$, in addition to being natural in $X$ and $Y$).
We call this the \emph{internal tensor}, and it is preserves internal
identities and composition.

For each object $A\in\V$ there is a natural transformation
\[
	h^{A}_{X,Y}: Y\illol X \to (Y\illol A)\illol(X\illol A)
\]
obtained by currying the internal composition, which is again
dinatural in $A$ and preserves internal identities and composition.
Also, there is a natural transformation with components
\[
	d_{A,B,C}: (C\illol B)\tn A \to (C\tn A)\illol B
\]
obtained by currying the map
\[
	B\tn(C\illol B)\tn A \rTo^{{\ev'}^{B}_{C}\tn A} C\tn A.
\]

\begin{lemma}\label{lemma-module-as-monoid-map}
	A map $x: X\tn M\to X$ makes $(X,x)$ into a right $M$-module
	if and only if
	\[
		\curry'_{X,M,X}(x): M\to X\illol X
	\]
	is a map of monoids.
\end{lemma}
\begin{proof}
	Curry the diagrams in Definition~\ref{def-module}.
\end{proof}
%
In conjunction with the observations above, this lemma implies
that, for every $A\in\V$, if $(X,x)$ is a right $M$-module, there
is a natural way to make both $A\tn X$ and $X\illol A$ into right
$M$-modules too. For, since the $t^{A}$ and $H^{A}$ maps preserve
internal identities and composition, in particular
\[
	t^{A}_{X,X}: X\illol X \to (A\tn X)\illol(A\tn X)
\]
and
\[
	h^{A}_{X,X}: X\illol X \to (X\illol A)\illol(X\illol A)
\]
are maps of monoids, which may be composed with $\curry'_{X,M,X}(x)$.
Explicitly, the module structure of $A\tn X$ is given by the map
\[
	A\tn X\tn M \rTo^{A\tn x} A\tn X,
\]
and the module structure of $X\illol A$ is given by
\[
	(X\illol A)\tn M \rTo^{d_{M,A,X}} (A\tn M)\illol A \rTo^{x\illol A} X\illol A.
\]
\begin{lemma}\label{lemma-mod-map-curry}
	Let $(X,x)$ and $(Y,y)$ be right $M$-modules.
	If $f:A\tn X\to Y$ is a map of modules, then so is
	\[
		\curry'(f): X\to Y\illol A.
	\]
\end{lemma}
\begin{proof}
	Consider the diagram
	\begin{diagram}
		&& (A\tn X\illol A)\tn M & \rTo^{(f\illol A)\tn M} & (Y\illol A)\tn M\\
		&\ruTo^{{\coev'}^{A}_{X}\tn M} & \dTo>{d_{M,A,A\tn X}}
			& & \dTo>{d_{M,A,Y}} \\
		X\tn M & \rTo_{{\coev'}^{A}_{X,M}} & A\tn X\tn M \illol A
			& \rTo_{f\tn M\illol A} & Y\tn M\illol A \\
		\dTo<x && \dTo>{A\tn x\illol A} && \dTo>{x\illol A} \\
		X & \rTo_{{\coev'}^{A}_{X}} & A\tn X\illol A
			& \rTo_{f\illol A} & Y\illol A
	\end{diagram}
	The triangle commutes by definition of $d$, the upper square by
	naturality of $d$, the lower-left square by naturality of $\coev'$,
	and the lower-right square since $f$ is a map of modules. Hence
	the outside commutes, as required.
\end{proof}

With this background in place, it is easy to deduce the internal Cayley
Theorem from the external one:
\begin{proof}[Proof of Theorem~\ref{thm-1d-internal}]
	Let $A\in\V$, and $g: A\to M\lolli X$ be such that $hg=kg$.
	If we uncurry $g$, and recurry the resulting $g'$ with respect
	to the left internal hom, we obtain maps
	\[\begin{array}{l}
		g'\phantom{{}'}: A\tn M\to X, \\
		g'': M\to X\illol A.
	\end{array}\]
	The first thing to observe is that $g'$ is a map of modules, for
	we have \begin{diagram}[midvshaft,hug,w=5em]
		& & (M\lolli X)\tn M\tn M \\
		A\tn M\tn M & \ruTo[snake=-1em](2,1)^{f\tn M\tn M}
			&& \rdTo[snake=1em](2,1)^{\ev^{M}_{X}\tn M}
			& X\tn M \\
		& \rdTo[snake=-1em](2,1)_{f\tn M\tn M} & (M\lolli X)\tn M\tn M \\
		\dTo<{A\tn m} && \dTo>{(M\lolli X)\tn M} && \dTo>x \\
		A\tn M & \rTo_{f\tn M} & (M\lolli X)\tn M & \rTo_{\ev^{M}_{X}} & X
	\end{diagram}
	where the quadrilateral commutes by functoriality of tensor, and the
	hexagon commutes since $hg=kg$. Thus, by Lemma~\ref{lemma-mod-map-curry},
	$g''$ is also a map of modules.
	
	By the external Cayley Theorem, $g''$ is thus equal to
	\[
		M \rTo^{\cong} I\tn M \rTo^{f\tn M} (X\illol A)\tn M
			\rTo^{d_{M,A,X}} (X\tn M)\illol A \rTo^{x\illol A} X\illol A
	\]
	for some unique $f: I \to X\illol A$.
	%
	Hence $g'$ is equal to
	\[
		A\tn M \rTo^{f'\tn M} X\tn M \rTo^{x} X
	\]
	for some unique $f': A\to X$ (obtained by uncurrying $f$).
	%
	Thus $g$ is equal to
	\[
		A \rTo^{f'} X \rTo^{\curry_{X,M,X}(x)} M\lolli X
	\]
	for this unique $f'$, which is precisely the universal property of
	the equaliser diagram.
\end{proof}
\begin{remark}
	The external theorem can be obtained as a corollary of the internal
	one, under an additional assumption on $\V$. The additional assumption is
	that the functor $\V(I,-); \V\to\Set$ should preserve equalisers. In
	particular, this is always so if $\V$ has small coproducts, because in that
	case $\V$ is \emph{tensored} \citep[in the sense of][section~2.7]{KellyEnriched}
	as a $\Set$-category, and so the functor $\V(I,-)$ has a left adjoint.
	Then we can obtain the external theorem simply by applying the functor
	$\V(I,-)$ to the equaliser diagram of the internal theorem.
\end{remark}
\end{snippet}

\end{document}