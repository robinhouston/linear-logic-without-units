%!TEX TS-program = latex
\documentclass{robinthesisdraft}
\usepackage{robincs,thesisdefs}

\newarrow{Into} {boldhook}--->

\title{Co-coherence}

\begin{document}
	\maketitle
	
	\section{Saavedra's Observation}
	\noindent
	%
	A monoidal category is a category $\C$ equipped with a tensor
	product $\tn: \C\times\C\to \C$, a unit object $I\in\C$, and
	natural isomorphisms
	\[\begin{array}{l}
		\alpha_{A,B,C}: A\tn(B\tn C)\to (A\tn B)\tn C \\
		\lambda_{A}: I\tn A\to A \\
		\rho_{A}: A\tn I\to A
	\end{array}\]
	subject to the following two coherence conditions:
	\begin{mspill}\begin{diagram}\dlabel{pentagon}
	  A\tn\bigl(B\tn(C\tn D)\bigr)
		  &\rTo^{\alpha_{A,B\tn C,D}}&(A\tn B)\tn (C\tn D)
		  &\rTo^{\alpha_{A\tn B,C,D}} & \bigl((A\tn B)\tn C\bigr)\tn D
	  \\
	  &\rdTo[snake=-1em](1,2)<{A\tn\alpha_{B,C,D}} &\dnum[pentagon]
		  & \ruTo[snake=1em](1,2)>{\alpha_{A,B,C}\tn D}
	  \\
	  & \spleft{A\tn\big((B\tn C)\tn D\big)}
		  & \rTo_{\alpha_{A,B\tn C,D}}
		  & \spright{\bigl(A\tn(B\tn C)\bigr)\tn D}
	\end{diagram}\end{mspill}

	\begin{diagram}\dlabel{AIC}
        A\tn(I\tn C) &\rTo^{\alpha_{A,I,C}}&(A\tn I)\tn C\\
        &\rdTo[snake=-1ex](1,2)<{A\tn\lambda_C}
			\raise1ex\dnum\ldTo[snake=1ex](1,2)>{\rho_A\tn C}\\
        &A\tn C
	\end{diagram}
	which we shall call the \emph{pentagon} and the \emph{triangle},
	for obvious reasons.
	%
	It is a famous fact
	that these conditions are sufficient to ensure that every diagram
	of this sort commutes. In particular,
	\[
		\lambda_{I}=\rho_{I}: I\tn I\to I.
	\]
	In fact this equation was taken as an axiom by \citet{MLCoh}
	in his original definition, and shortly thereafter \citet{KellyML}
	showed how to prove it from the two axioms above.
	%
	Rather less famous, but much easier to prove, is
	\begin{lemma}\label{lemma-lr-uq}
		The natural isomorphisms $\lambda$ and $\rho$ are uniquely
		determined by the single isomorphism $\lambda_{I}=\rho_{I}$.
	\end{lemma}
	\begin{proof}
		Since the functors $I\tn-$ and $-\tn I$ are each isomorphic to
		the identity, they are equivalences, so in particular are
		faithful. Considering the triangle condition with $A=I$ shows
		that $I\tn\lambda_{C}$ is equal to $(\rho_{I}\tn C)\cdot\alpha_{I,I,C}$,
		and since $I\tn-$ is faithful, there is a unique such map $\lambda_{C}$.
		%
		Similarly, considering the triangle condition with $C=I$ shows
		that $\rho_{A}\tn I$ is equal to $(A\tn\lambda_{I})\cdot\alpha_{A,I,I}^{-1}$,
		and since $-\tn I$ is faithful, there is a unique such map $\rho_{A}$.
	\end{proof}
	Furthermore, there is a kind of converse:
	\begin{lemma}\label{lemma-lr-exist}
		If we are given just $\C$, $\tn$, $I$ and $\alpha$, such that
		the pentagon condition holds and the functors $I\tn-$ and $-\tn I$
		are fully faithful, then for every isomorphism $u:I\tn I\to I$
		there exist natural isomorphisms $\lambda$ and $\rho$, satisfying
		the triangle condition, such that $\lambda_{I} = u = \rho_{I}$.
	\end{lemma}
	\begin{proof}
		Define $\lambda_{C}$ to be the unique map such that
		$I\tn\lambda_{C} = (u\tn C)\cdot\alpha_{I,I,C}^{-1}$,
		and define $\rho_{A}$ to be the unique map such that
		$\rho_{A}\tn I = (A\tn u)\cdot\alpha_{A,I,I}$.
		%
		Since $I\tn\lambda_{C}$ is natural in $C$ by construction
		and $I\tn-$ is faithful, it follows that $\lambda$ is indeed
		a natural transformation. Similarly $\rho$ is natural.
		%
		By construction, we know that the diagrams
		\[
		\begin{diagram}
	        I\tn(I\tn C)&\rTo^{\alpha_{I,I,C}}& (I\tn I)\tn C\\
	        &\rdTo[snake=-1ex](1,2)<{I\tn\lambda_C}
				\ldTo[snake=1ex](1,2)>{u\tn C}\\
	        &I\tn C
		\end{diagram}
		\mbox{\qquad and\qquad}
		\begin{diagram}
	        I\tn(I\tn C)&\rTo^{\alpha_{I,I,C}}& (I\tn I)\tn C\\
	        &\rdTo[snake=-1ex](1,2)<{A\tn u}
				\ldTo[snake=1ex](1,2)>{\rho_A\tn I}\\
	        &A\tn I
		\end{diagram}
		\]
		commute. We shall show that these imply the
		triangle condition. The first observation we need is that,
		since $\lambda$ and $\rho$ are natural by the above, in
		particular the squares
		\[
			\begin{diagram}
				I\tn(I\tn A) & \rTo^{\lambda_{I\tn A}} & I\tn A \\
				\dTo<{I\tn\lambda_A} && \dTo>{\lambda_{A}} \\
				I\tn A &\rTo_{\lambda_{A}} & A
			\end{diagram}
			\mbox{\enskip and\qquad}
			\begin{diagram}
				(A\tn I)\tn I &\rTo^{\rho_{A\tn I}} & A\tn I \\
				\dTo<{\rho_{A}\tn I} && \dTo>{\rho_{A}} \\
				A\tn I & \rTo_{\rho_{A}} & A
			\end{diagram}
		\]
		commute. Since $\lambda_{A}$ and $\rho_{A}$ are invertible,
		they are certainly monic, hence $\lambda_{I\tn A}=I\tn\lambda_{A}$
		and $\rho_{A\tn I}=\rho_{A}\tn I$.
		Now, consider the diagram
		\begin{mspill}\begin{diagram}[w=7em,tight]\dlabel{pentagon}
		  A\tn\bigl(I\tn(I\tn B)\bigr)
			  &\rTo^{\alpha_{A,I\tn I,B}}&(A\tn I)\tn (I\tn B)
			  &\rTo^{\alpha_{A\tn I,I,B}} & \bigl((A\tn I)\tn I\bigr)\tn B
		  \\
		  & \rdTo[snake=-1em,crab=-1em,rightshortfall=.85em,leftshortfall=-.6em]%
				(1,4)<{A\tn\alpha_{I,I,B}}
			\rdTo[snake=0.3em](1,2)>{A\tn\lambda_{I\tn B}}
			\rdTo[snake=2em](1,2)>{=A\tn(I\tn\lambda_{B})}
			& & \rdTo[snake=-2em](1,2)<{(A\tn I)\tn\lambda_{B}}
			\ruTo[snake=1em,crab=-1em,leftshortfall=.85em,rightshortfall=-.6em]%
				(1,4)>{\alpha_{A,I,I}\tn B}
			\ldTo(1,2)[snake=0em]<{(\rho_{A}\tn I)\tn B}
			\ldTo(1,2)[snake=-1.5em]<{=\rho_{A\tn I}\tn B}
		  \\
		  &	A\tn(I\tn B) & \rTo_{\alpha_{A,I,B}}
			& (A\tn I)\tn B 
		  \\
		  & \uTo>{A\tn(u\tn B)} & & \uTo<{(A\tn u)\tn B}
		  \\
		  & {A\tn\big((I\tn I)\tn B\big)}
			  & \rTo_{\alpha_{A,I\tn I,B}}
			  & {\bigl(A\tn(I\tn I)\bigr)\tn B}
		\end{diagram}\end{mspill}
		The two quadrilaterals commute by naturality of $\alpha$,
		and the two lower triangles commute by construction. The outside
		commutes by the pentagon condition. Because
		$\alpha_{A,I\tn I,B}$ is invertible, hence epic, it follows
		that the upper triangle must commute, and so the triangle
		condition is established.
		
		All that remains is to show $\lambda_{I}=u=\rho_{I}$. By the
		triangle condition and the definition of $\lambda$, we know that
		$u\tn C=\rho_{I}\tn C$ for all objects $C$. Letting $C=I$,
		the faithfulness of $-\tn I$ gives $u=\rho_{I}$. For
		$\lambda_{I}$, we can either apply a similar argument to the
		definition of $\rho$, or simply note that $\lambda_{I}=\rho_{I}$
		by coherence.
	\end{proof}
	\begin{remark}
		This observation is originally due to \citet{Saavedra}. Some
		of its implications are studied in detail by \citet{KockUnits},
		who emphasises the fact that the assumptions of Lemma~\ref{lemma-lr-exist}
		constitute an alternative axiomatisation of the notion
		of monoidal category. This axiomatisation has the interesting
		property that the definition of the unit structure does not involve
		the associator.
		It is non-algebraic -- the condition on the functors $I\tn-$ and
		$-\tn I$ is not an equational one -- but of course there is an
		\emph{equivalent} algebraic presentation, namely the usual one!
	\end{remark}
	
	\section{Co-coherence}
	Our purpose here is to show that Saavedra's observation implies a
	result that is in a sense the \emph{dual} of a coherence theorem:
	\begin{propn}[Co-coherence of units]
		The inclusion $\mathbf{MonCat} \rInto \mathbf{MonCat}^{-}$
		has a right adjoint whose unit is the identity.
	\end{propn}
	\bibliography{cs}
\end{document}